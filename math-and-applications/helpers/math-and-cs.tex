%\newcommand{\MathFractionStrut}[1]{%
%    \ensuremath{\vphantom{\displaystyle \frac{\displaystyle#1}{\displaystyle#1}}}%
%}
\newcommand{\MathSumStrut}[1]{%
    \ensuremath{\vphantom{\displaystyle \sum_{#1}^{#1}}}%
}

%
% Math Symbols
%

% My implementations of math mode Greek letters where no command is available due to similarity with Latin letters.
% These commands are used where needed to intentionally keep typographically consistent.
\DeclareMathSymbol{\myAlpha  }{\mathalpha}{operators}{"41}
\DeclareMathSymbol{\myBeta   }{\mathalpha}{operators}{"42}
\DeclareMathSymbol{\myEpsilon}{\mathalpha}{operators}{"45}
\DeclareMathSymbol{\myZeta   }{\mathalpha}{operators}{"5A}
\DeclareMathSymbol{\myEta    }{\mathalpha}{operators}{"48}
\DeclareMathSymbol{\myIota   }{\mathalpha}{operators}{"49}
\DeclareMathSymbol{\myKappa  }{\mathalpha}{operators}{"4B}
\DeclareMathSymbol{\myMu     }{\mathalpha}{operators}{"4D}
\DeclareMathSymbol{\myNu     }{\mathalpha}{operators}{"4E}
\DeclareMathSymbol{\myOmicron}{\mathalpha}{operators}{"4F}
\DeclareMathSymbol{\myRho    }{\mathalpha}{operators}{"50}
\DeclareMathSymbol{\myTau    }{\mathalpha}{operators}{"54}
\DeclareMathSymbol{\myChi    }{\mathalpha}{operators}{"58}
\DeclareMathSymbol{\myomicron}{\mathord  }{letters}{"6F}
\DeclareMathSymbol{\myDigamma}{\mathalpha}{operators}{"46} % TODO: Find a better version.
%% My original commands
%\newcommand{\myAlpha  }{\mathrm{A}}
%\newcommand{\myBeta   }{\mathrm{B}}
%\newcommand{\myEpsilon}{\mathrm{E}}
%\newcommand{\myZeta   }{\mathrm{Z}}
%\newcommand{\myEta    }{\mathrm{H}}
%\newcommand{\myIota   }{\mathrm{I}}
%\newcommand{\myKappa  }{\mathrm{K}}
%\newcommand{\myMu     }{\mathrm{M}}
%\newcommand{\myNu     }{\mathrm{N}}
%\newcommand{\myOmicron}{\mathrm{O}}
%\newcommand{\myomicron}{\mathnormal{o}} % I have no idea what this should be, but the other lower-case letters are italicized...
%\newcommand{\myRho    }{\mathrm{P}}
%\newcommand{\myTau    }{\mathrm{T}}
%\newcommand{\myChi    }{\mathrm{X}}

\newcommand\evalat[2]{{\left.#1\right\rvert{}}_{#2}}

\newcommand{\defeq}{\mathbin{\vcentcolon=}} % := symbol

\DeclarePairedDelimiter\ceil{\lceil{}}{\rceil{}}
\DeclarePairedDelimiter\floor{\lfloor{}}{\rfloor{}}

\newcommand{\complexconjugate}[1]{\overline{#1}}

\DeclareMathOperator{\dom}{Dom} % Domain of a function

% Custom versions of the Re() and Im() functions in complex analysis
\DeclareMathOperator{\MyRe}{Re}
\DeclareMathOperator{\MyIm}{Im}
%\DeclareMathOperator{\MyRe}{\Re} % If these are ever used, they don't space well.
%\DeclareMathOperator{\MyIm}{\Im} % Should figure out how to fix this...

% Other common math functions that aren't in-built
\DeclareMathOperator{\sech}{sech}
\DeclareMathOperator{\csch}{csch}
\DeclareMathOperator{\cis}{cis}
\DeclareMathOperator{\FourierTransform}{\mathcal{F}}
\DeclareMathOperator{\LaplaceTransform}{\mathcal{L}}

% Relational algebra functions
\DeclareMathOperator{\relselect}{\sigma}
\DeclareMathOperator{\relproject}{\pi}
\DeclareMathOperator{\relrename}{\rho}
\DeclareMathOperator{\relgroup}{\gamma}

\newcommand\rhs{\textit{RHS}}
\newcommand\lhs{\textit{LHS}}

\newcommand{\Nth}[2]{\ensuremath{{#1}^\textit{#2}}}
\newcommand{\diff}{\ensuremath{\mathrm{d}}}
\newcommand{\parallelsum}{\mathbin{\|}}

\DeclarePairedDelimiter\parens{(}{)}
\DeclarePairedDelimiter\brackets{[}{]}
\DeclarePairedDelimiter\braces{\{}{\}}

\DeclarePairedDelimiter\abs{\lvert{}}{\rvert{}}
\DeclarePairedDelimiter\inte{[}{]}

% Common math structures
%\newcommand{\setdef}[2]{\left\{{#1} \,\middle|\, {#2}\right\}} % Bar version. TODO: If this is used, fix the spacing around bar.
%\newcommand{\setdef}[2]{\left\{{#1} : {#2}\right\}}
\newcommand{\setdef}[2]{\left\{{#1} \mid {#2}\right\}}
\newcommand{\eqsystem}[1]{\left\{\:{#1}\right.}

% The nothing-unit, for use when siunitx gives the "Found prefix part with no unit." error.
% Use this sparingly!
\DeclareSIUnit{\sinounit}{\relax}

% \newcommand*\Let[2]{\State #1 $\gets$ #2}% THERE MUST BE A BETTER WAY!!!

% Electrical Engineering Conventions
\DeclareSIUnit\voltampere{VA} % Apparent Power
\DeclareSIUnit\var{VAR} % Reactive Power

% Adding small text over things at a more comfortable distance.
% TODO: My implementation is actually a hack. I should find a better way of doing it...
%       Though in the process, I may need to rework formatting where this command is being used.
\newcommand{\MathOverLabel}[2]{\overset{\substack{#1\\\phantom{x}}}{#2}}

% Crossing out and replacing a value
\newcommand{\xcancelto}[2]{\underbracket{\xcancel{#2}}_{#1}}

% Fraction in display mode
\newcommand\DisplayFrac[2]{\displaystyle \frac{\displaystyle {#1}}{\displaystyle {#2}}}

%
% Macros for Computer Science: Databases
%

\newcommand{\MarkExtendedRelAlg}{{\color{myred}\scriptsize\textbf{(Extended RA)}}}

%\colorlet{RelAlgSubsectionColor}{myblue!80!white}
\definecolor{RelAlgSubsectionColor}{rgb}{0.65,0.65,0.65}
\newenvironment{RelAlgSubsection}[1]{\CheatsheetSubsection{#1}{RelAlgSubsectionColor}{white}}{\endCheatsheetSubsection}

% Relational algebra null symbol
\newcommand{\relnullvalue}{\mathord{\bot}}
% Preferred null value symbol is a matter of personal taste.
% I prefer the upside-down "T" (\bot), as introduced to me in David Maier's The Theory of Relational Databases.
% Omega is a common greek letter, while \emptyset (or \varnothing) should just denote an empty set (or empty relation).

% Relational Algebra outer join symbols
% (Packages that contain these symbols don't seem to work...)
%     original source: https://tex.stackexchange.com/questions/20740/symbols-for-outer-joins/20749#20749
%     answer original author: egreg (https://tex.stackexchange.com/users/4427/egreg)
\newcommand\TmpOJoin{\setbox0=\hbox{$\bowtie$} \rule[-.02ex]{.25em}{.4pt}\llap{\rule[\ht0]{.25em}{.4pt}}}
\newcommand\relleftouterjoin{\mathbin{\TmpOJoin\mkern-5.8mu\bowtie}}
\newcommand\relrightouterjoin{\mathbin{\bowtie\mkern-5.8mu\TmpOJoin}}
\newcommand\relfullouterjoin{\mathbin{\TmpOJoin\mkern-5.8mu\bowtie\mkern-5.8mu\TmpOJoin}}

