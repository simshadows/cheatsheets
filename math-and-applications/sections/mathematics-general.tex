\subsection{Elementary Algebra}%
\label{sub:algebra}

\begin{multicols}{2}

    %% These are the Exponentiation and Log Identities sections in the original separated form.
    %
    %\CheatsheetEntryFrame{
    %
    %    \CheatsheetEntryTitle{Exponentiation Identities}
    %    \begin{align*}
    %        a^x a^y &= a^{x+y} \\
    %        a^x b^x &= \parens*{ab}^x \\
    %        \parens*{a^x}^y &= a^{xy} \\
    %        a^0 &= 1
    %        %
    %        \\[\abovedisplayskip]
    %        %
    %        a^{-x} &= \frac{1}{a^x} \\
    %        a^{x-y} &= \frac{a^x}{a^y} \\
    %        \parens*{\frac{a}{b}}^x &= \frac{a^x}{b^x}
    %        %
    %        \\[\abovedisplayskip]
    %        %
    %        a^{\frac{1}{2}} &= \sqrt{a} \\
    %        a^{\frac{1}{x}} &= \sqrt[x]{a} \\
    %        a^{\frac{x}{y}} &= \parens*{\sqrt[y]{a}}^x
    %    \end{align*}
    %
    %}
    %
    %\CheatsheetEntryFrame{
    %
    %    \CheatsheetEntryTitle{Log Identities}
    %
    %    \CheatsheetSmallEquationTitle{(Definition)}
    %    \begin{equation*}
    %        x = a^y \quad\iff\quad \log_a{x} = y
    %    \end{equation*}
    %
    %    \CheatsheetSmallEquationTitle{(Inverses)}
    %    \begin{equation*}
    %        \log_a{a^x} = a^{\log_a{x}} = x
    %    \end{equation*}
    %
    %    \CheatsheetSmallEquationTitle{(Change of Base Law)}
    %    \begin{equation*}
    %        \log_a{x} = \frac{\log_u{x}}{\log_u{a}}
    %    \end{equation*}
    %
    %    \vspace*{-\abovedisplayskip}
    %    \begin{align*}
    %        \log_a{xy} &= \log_a{x} + \log_a{y} \\
    %        \log_a{\parens*{\frac{x}{y}}} &= \log_a{x} - \log_a{y} \\
    %        \log_a{x^n} &= n \log_a{x}
    %        %
    %        \\[\abovedisplayskip]
    %        %
    %        \log_a{1} &= 0 \\
    %        \log_a{a} &= 1
    %        %
    %        \\[\abovedisplayskip]
    %        %
    %        \log_a{\parens*{\frac{1}{x}}} &= - \log_a{x}
    %        %
    %        \\[\abovedisplayskip]
    %        %
    %        \log_a{\parens*{\frac{1}{a}}} &= -1 \\
    %        \log_a{\sqrt{a}} &= \frac{1}{2} \\
    %        \log_a{\sqrt[b]{a}} &= \frac{1}{b}
    %    \end{align*}
    %
    %}

    \CheatsheetEntryFrame{

        \CheatsheetEntryTitle{Exponentiation and Logarithm Identities}

        % Hacks to ensure uniform height
        \renewcommand{\W}{\displaystyle} % Simple
        \renewcommand{\X}{\displaystyle \vphantom{\parens*{\sqrt[\frac{I}{I}]{I}}}} % No fractions
        \renewcommand{\Y}{\displaystyle \vphantom{\frac{I}{I}}} % Simple fractions
        \renewcommand{\Z}{\displaystyle \vphantom{\parens*{\parens*{\frac{I}{I}}^I}}} % Complicated fractions


        \begin{center}
        \begin{tabularx}{\textwidth}{Ccc}

            \hline
            \multicolumn{3}{|l|}{{\scriptsize \textbf{Logarithm Definition}}}
                \\
            \multicolumn{1}{|c}{$\X x = a^y$}
                & $\X \iff$
                & \multicolumn{1}{c|}{$\X \log_a{x} = y$}
                \\
            \hline

            && %%%%%%%%%%%%%%%%%%%%%%%%%%%%%%%%%% GAP
                \\ %%%%%%%%%%%%%%%%%%%%%%%%%%%%%% GAP

            $a^{\log_a{x}} = x$
                & $\X \iff$
                & $\X \log_a{a^x} = x$
                \\

            && %%%%%%%%%%%%%%%%%%%%%%%%%%%%%%%%%% GAP
                \\ %%%%%%%%%%%%%%%%%%%%%%%%%%%%%% GAP

            {} % BLANK
                &
                & $\W \overset{\text{\textbf{Change of Base Law}}}{\boxed{\Z \log_a{x} = \frac{\log_u{x}}{\log_u{a}}}}$
                \\

            && %%%%%%%%%%%%%%%%%%%%%%%%%%%%%%%%%% GAP
                \\ %%%%%%%%%%%%%%%%%%%%%%%%%%%%%% GAP

            $\W
                        \begin{aligned}
                            \X a^0 &= 1 \\
                            \X a^1 &= a \\
                            \Y a^{-1} &= \frac{1}{a}
                        \end{aligned}
            $
                &
                    $\W
                        \begin{gathered}
                            \X \longrightarrow \\
                            \X \longrightarrow \\
                            \Y \longrightarrow
                        \end{gathered}
                    $
                &
                    $\W
                        \begin{aligned}
                            \X \log_a{1} &= 0 \\
                            \X \log_a{a} &= 1 \\
                            \Y \log_a{\frac{1}{a}} &= -1
                        \end{aligned}
                    $
                \\

            && %%%%%%%%%%%%%%%%%%%%%%%%%%%%%%%%%% GAP
                \\ %%%%%%%%%%%%%%%%%%%%%%%%%%%%%% GAP

            $\W
                        \begin{aligned}
                            \X a^x a^y         &= a^{x+y} \\
                            \X a^x b^x         &= \parens*{ab}^x \\
                            \X \parens*{a^x}^y &= a^{xy} \\
                        \end{aligned}
            $
                &
                    $\W
                        \begin{gathered}
                            \X \longrightarrow \\
                            \X \\
                            \X \longrightarrow
                        \end{gathered}
                    $
                &
                    $\W
                        \begin{aligned}
                            \X \log_a{xy} &= \log_a{x} + \log_a{y} \\
                            \X & \\
                            \X \log_a{x^n} &= n \log_a{x}
                        \end{aligned}
                    $
                \\

            && %%%%%%%%%%%%%%%%%%%%%%%%%%%%%%%%%% GAP
                \\ %%%%%%%%%%%%%%%%%%%%%%%%%%%%%% GAP

            $\W
                        \begin{aligned}
                            \Z \frac{1}{a^y}   &= a^{-y} \\
                            \Z \frac{a^x}{a^y} &= a^{x-y} \\
                            \Z \frac{a^x}{b^x} &= \parens*{\frac{a}{b}}^x
                        \end{aligned}
            $
                &
                    $\W
                        \begin{gathered}
                            \Z \\
                            \Z \longrightarrow \\
                            \Z
                        \end{gathered}
                    $
                &
                    $\W
                        \begin{aligned}
                            \Z \log_a{\frac{1}{y}} &= -\log_a{y} \\
                            \Z \log_a{\frac{x}{y}} &= \log_a{x} - \log_a{y} \\
                            \Z &
                        \end{aligned}
                    $
                \\

            && %%%%%%%%%%%%%%%%%%%%%%%%%%%%%%%%%% GAP
                \\ %%%%%%%%%%%%%%%%%%%%%%%%%%%%%% GAP

            $\W
                        \begin{aligned}
                            \Y a^{\frac{1}{2}} &= \sqrt{a} \\
                            \Y a^{\frac{1}{y}} &= \sqrt[y]{a} \\
                            \Y a^{\frac{x}{y}} &= \parens*{\sqrt[y]{a}}^x
                        \end{aligned}
            $
                &
                    $\W
                        \begin{gathered}
                            \Y \longrightarrow \\
                            \Y \longrightarrow \\
                            \Y \longrightarrow
                        \end{gathered}
                    $
                &
                    $\W
                        \begin{alignedat}{2}
                            \Y & \log_a{\sqrt{a}}                &&= \frac{1}{2} \\
                            \Y & \log_a{\sqrt[y]{a}}             &&= \frac{1}{y} \\
                            \Y & \log_a{\parens*{\sqrt[y]{a}}^x} &&= \frac{x}{y}
                        \end{alignedat}
                    $
                \\

        \end{tabularx}
        \end{center}

    }

    \CheatsheetEntryFrame{

        \CheatsheetEntryTitle{Factorial}
        \begin{gather*}
            n! = \prod_{k = 1}^n{k} = 1 \times 2 \times 3 \times \dots \times (n-2) \times (n-1) \times n, \\
            \forall n \in \mathbb{Z} : n > 0.
        \end{gather*}

        Factorial of zero:
        \begin{equation*}
            0! = 1
        \end{equation*}

    }

    % \MulticolsBreak % I want a break here...

    \CheatsheetEntryFrame{

        \CheatsheetEntryTitle{Quadratic Formula}
        %\begin{gather*}
        %    x = \frac{-b \pm \sqrt{b^2 - 4ac}}{2a} \\
        %    \intertext{Discriminant:}
        %    \Delta = b^2 - 4ac
        %\end{gather*}
        \begin{gather*}
            x = \frac{-b \pm \sqrt{b^2 - 4ac}}{2a}
            \qquad\quad
            \MathOverLabel{\text{\ul{discriminant}}}{\Delta = b^2 - 4ac}
        \end{gather*}

        \CheatsheetEntryTitle{Quadratic Factorization}
        \begin{equation*}
            x^2 + bx + c = \parens*{x+v} \parens*{x+u}
            \qquad\quad
            %\MathOverLabel{\text{\ul{sums and products}}}{ % Too much space. Not worth it.
                \begin{array}{c}
                    b = u+v \\
                    c = uv
                \end{array}
            %}
        \end{equation*}
        \vspace{-2ex} % Not sure why there's so much whitespace

        \CheatsheetEntryTitle{Completing The Square}
        \begin{equation*}
            x^2 + b x + c = \parens*{x + \xmemphR{\brackets*{\memphR{\frac{b}{2}}}}}^2 + c - \xmemphP{\brackets*{\memphP{\frac{b}{2}}}^2}
        \end{equation*}

    }

    \CheatsheetEntryFrame{

        \CheatsheetEntryTitle{Quadratic/Cubic Identities}
        \begin{align*}
            a^2 - b^2 &= (a + b) (a - b) \\
            a^3 - b^3 &= (a - b) (a^2 + ab + b^2) \\
            a^3 + b^3 &= (a + b) (a^2 - ab + b^2)
        \end{align*}

    }

\end{multicols}
\newpage

\CheatsheetEntryFrame{

    \CheatsheetEntryTitle{Partial Fraction Decomposition}

    A \textit{proper rational function} can be rewritten as a sum of \textit{partial fractions}.

    For each irreducible factor in the denominator, the partial fractions are as follows:
    \renewcommand{\W}{\displaystyle}
    \begin{equation*}
        \begin{array}{ccccc}
            \substack{\text{\ul{irreducible factor in denominator}}} &&&&
                \substack{\text{\ul{partial fractions}}}
                \\[1ex]
            \W \memphR{(ax + b)}^k &
                & \Longrightarrow & \quad &
                %\W \sum_{n=1}^k{\frac{A_n}{\memphR{(ax + b)}^n}}
                \W \frac{A_1}{ax + b} + \frac{A_2}{\parens*{ax + b}^2} + \dots + \frac{A_k}{\parens*{ax + b}^k}
                \\[3ex]
            \W \memphR{(ax^2 + bx + c)}^k &
                & \Longrightarrow & \quad &
                %\W \sum_{n=1}^k{\frac{A_n x + B_n}{\memphR{(ax^2 + bx + c)}^n}}
                \W \frac{A_1 x + B_1}{ax^2 + bx + c} + \frac{A_2 x + B_2}{\parens*{ax^2 + bx + c}^2} + \dots + \frac{A_k x + B_k}{\parens*{ax^2 + bx + c}^k}
                \\
        \end{array}
    \end{equation*}

    For \textit{improper rational functions}, you must first convert it to a \textit{proper rational function}.

}

\begin{multicols}{2}

    \CheatsheetEntryFrame{

        \CheatsheetEntryTitle{todo}

    }

    \MulticolsBreak

    \MulticolsPhantomPlaceholder

\end{multicols}

%%%%%%%%%%%%%%%%%%%%%%%%%%%%%%%%%%%%%%%%%%%%%%%%%%%%%%%%%%%%%%%%%%%%%%%%%%%%%%%%%%%%%%%%%%%%%%%%%%%%
%%%%%%%%%%%%%%%%%%%%%%%%%%%%%%%%%%%%%%%%%%%%%%%%%%%%%%%%%%%%%%%%%%%%%%%%%%%%%%%%%%%%%%%%%%%%%%%%%%%%
%%%%%%%%%%%%%%%%%%%%%%%%%%%%%%%%%%%%%%%%%%%%%%%%%%%%%%%%%%%%%%%%%%%%%%%%%%%%%%%%%%%%%%%%%%%%%%%%%%%%

\newpage
\subsection{Trigonometry and Hyperbolic Functions}%
\label{sub:trigonometry-and-hyperbolic}

\begin{multicols}{2}

    \CheatsheetEntryFrame{

        %\begin{figure}[H]\centering
        %\begin{tikzpicture}[thick]
        %        \coordinate (O) at (0,0);
        %        \coordinate (A) at (4,0);
        %        \coordinate (B) at (0,2);
        %        \draw (O) -- (A) -- (B) -- cycle;
        %\end{tikzpicture}
        %%\caption{caption}
        %%\label{fig:labelname}
        %\end{figure}

        \CheatsheetEntryTitle{Pythagorean Theorem}
        \begin{equation*}
            a^2 = b^2 + c^2
        \end{equation*}

        \CheatsheetEntryTitle{Pythagorean Identities}
        \begin{gather*}
            \sin^2{\theta} + \cos^2{\theta} = 1 \\
            \tan^2{\theta} + 1 = \sec^2{\theta}
                \quad \parens*{\cos{\theta} \ne 0} \\
            \cot^2{\theta} + 1 = \csc^2{\theta}
                \quad \parens*{\sin{\theta} \ne 0}
        \end{gather*}

        \CheatsheetEntryTitle{Complementary Angles}
        \begin{equation*}
            \sin^{-1}{x} + \cos^{-1}{x} = \frac{\pi}{2}
        \end{equation*}

        \CheatsheetEntryTitle{Compound Angles}
        %\begin{align*}
        %    \sin{\parens*{\alpha + \beta}} &= \sin{\alpha} \cos{\beta } + \cos{\alpha} \sin{\beta } \\
        %    \sin{\parens*{\alpha - \beta}} &= \sin{\alpha} \cos{\beta } - \cos{\alpha} \sin{\beta }
        %    %
        %    \\[\abovedisplayskip]
        %    %
        %    \cos{\parens*{\alpha + \beta}} &= \cos{\alpha} \cos{\beta } - \sin{\alpha} \sin{\beta } \\
        %    \cos{\parens*{\alpha - \beta}} &= \cos{\alpha} \cos{\beta } + \sin{\alpha} \sin{\beta }
        %    %
        %    \\[\abovedisplayskip]
        %    %
        %    \tan{\parens*{\alpha + \beta}} &= \frac{\tan{\alpha} + \tan{\beta}}{1 - \tan{\alpha} \tan{\beta}} \\
        %    \tan{\parens*{\alpha - \beta}} &= \frac{\tan{\alpha} - \tan{\beta}}{1 + \tan{\alpha} \tan{\beta}}
        %\end{align*}
        \begin{align*}
            \sin{\parens*{\alpha \pm \beta}} &= \sin{\alpha} \cos{\beta } \pm \cos{\alpha} \sin{\beta } \\
            \cos{\parens*{\alpha \pm \beta}} &= \cos{\alpha} \cos{\beta } \mp \sin{\alpha} \sin{\beta } \\
            \tan{\parens*{\alpha \pm \beta}} &= \frac{\tan{\alpha} \pm \tan{\beta}}{1 \mp \tan{\alpha} \tan{\beta}}
        \end{align*}

    }

    \CheatsheetEntryFrame{

        \CheatsheetEntryTitle{In Terms of Exponentials}

        Derived from Euler's formula.
        \begin{align*}
            \sin{x} = \frac{e^{ix} - e^{-ix}}{2i} \\
            \cos{x} = \frac{e^{ix} + e^{-ix}}{2}
        \end{align*}

    }

    %% Current prototype for this section.
    %\CheatsheetEntryFrame{

    %    \CheatsheetEntryTitle{Exact Values}

    %    %\renewcommand{\W}{\displaystyle \vphantom{\parens*{\parens*{\frac{I}{I}}^I}}} % Huge vertical space
    %    \renewcommand{\W}{\displaystyle \vphantom{\parens*{\frac{\sqrt{I}}{\sqrt{I}}}}}
    %    %\renewcommand{\W}{}
    %    \begin{tabularx}{\textwidth}{|C|C|C|C|C|C|}
    %        \hline
    %        Deg.
    %            & $\W \ang{0}$
    %            & $\W \ang{30}$
    %            & $\W \ang{45}$
    %            & $\W \ang{60}$
    %            & $\W \ang{90}$
    %            \\ \hline
    %        Rad.
    %            & $\W 0$
    %            & $\W \frac{\pi}{6}$
    %            & $\W \frac{\pi}{4}$
    %            & $\W \frac{\pi}{3}$
    %            & $\W \frac{\pi}{2}$
    %            \\ \hline \hline
    %        $\W \sin{\theta}$
    %            & $\W 0$
    %            & $\W \frac{1}{2}$
    %            & $\W \frac{1}{\sqrt{2}}$
    %            & $\W \frac{\sqrt{3}}{2}$
    %            & $\W 1$
    %            \\ \hline
    %        $\W \cos{\theta}$
    %            & $\W 1$
    %            & $\W \frac{\sqrt{3}}{2}$
    %            & $\W \frac{1}{\sqrt{2}}$
    %            & $\W \frac{1}{2}$
    %            & $\W 0$
    %            \\ \hline
    %        $\W \tan{\theta}$
    %            & $\W 0$
    %            & $\W \frac{1}{\sqrt{3}}$
    %            & $\W 1$
    %            & $\W \sqrt{3}$
    %            & \texttt{UNDEF.}
    %            \\ \hline \hline
    %        $\W \csc{\theta}$
    %            & \texttt{UNDEF.}
    %            & $\W 2$
    %            & $\W \sqrt{2}$
    %            & $\W \frac{2}{\sqrt{3}}$
    %            & $\W 1$
    %            \\ \hline
    %        $\W \sec{\theta}$
    %            & $\W 1$
    %            & $\W \frac{2}{\sqrt{3}}$
    %            & $\W \sqrt{2}$
    %            & $\W 2$
    %            & \texttt{UNDEF.}
    %            \\ \hline
    %        $\W \cot{\theta}$
    %            & \texttt{UNDEF.}
    %            & $\W \sqrt{3}$
    %            & $\W 1$
    %            & $\W \frac{1}{\sqrt{3}}$
    %            & $\W 0$
    %            \\ \hline
    %    \end{tabularx}

    %}

    \MulticolsBreak

    \CheatsheetEntryFrame{

        \CheatsheetEntryTitle{Double-Angle Formulae}

        Derived from compound angle formulae.
        \begin{align*}
            \sin{2 \theta} &= 2 \sin{\theta} \cos{\theta} \\
            \cos{2 \theta} &= \cos^2{\theta} - \sin^2{\theta}  \overbracket[1pt][1.75mm]{= 2 \cos^2{\theta} - 1} \\
                &\phantom{{}= \cos^2{\theta} - \sin^2{\theta}} \underbracket[1pt][1.75mm]{= 1 - 2 \sin^2{\theta}}_{
                    %\substack{
                    %    \text{simplified using} \\
                        \sin^2{\theta} + \cos^2{\theta} = 1
                    %}
                }\\
            \tan{2 \theta} &= \frac{2 \tan{\theta}}{1 - \tan^2{\theta}}
        \end{align*}

        \CheatsheetEntryTitle{Half-Angle Formulae}

        Derived from the $\cos{2 \theta}$ compound angle formula.
        \begin{align*}
            \sin^2{\theta} &= \frac{1}{2} - \frac{1}{2} \cos{2 \theta} \\
            \cos^2{\theta} &= \frac{1}{2} + \frac{1}{2} \cos{2 \theta}
        \end{align*}

        \CheatsheetEntryTitle{Products to Sums}

        Derived by adding compound angle formulae.
        \begin{alignat*}{8}
             &2 \sin&&{A} && \cos&&{B} &&= \sin&&{\parens*{A+B}} &&+ \sin&&{\parens*{A-B}} \\
             &2 \cos&&{A} && \sin&&{B} &&= \sin&&{\parens*{A+B}} &&- \sin&&{\parens*{A-B}} \\
             &2 \cos&&{A} && \cos&&{B} &&= \cos&&{\parens*{A+B}} &&+ \cos&&{\parens*{A-B}} \\
            -&2 \sin&&{A} && \sin&&{B} &&= \cos&&{\parens*{A+B}} &&- \cos&&{\parens*{A-B}}
        \end{alignat*}

        \CheatsheetEntryTitle{Sums to Products}

        Derived by reversing Products to Sums.
        \begin{alignat*}{3}
            \sin{S} + \sin{T} &= \phantom{+} 2 \sin&&{\parens*{\frac{S+T}{2}}} \cos&&{\parens*{\frac{S-T}{2}}} \\
            \sin{S} - \sin{T} &= \phantom{+} 2 \cos&&{\parens*{\frac{S+T}{2}}} \sin&&{\parens*{\frac{S-T}{2}}} \\
            \cos{S} + \cos{T} &= \phantom{+} 2 \cos&&{\parens*{\frac{S+T}{2}}} \cos&&{\parens*{\frac{S-T}{2}}} \\
            \cos{S} - \cos{T} &=          -  2 \sin&&{\parens*{\frac{S+T}{2}}} \sin&&{\parens*{\frac{S-T}{2}}}
        \end{alignat*}

    }

    \pagebreak

    \CheatsheetEntryFrame{

        \CheatsheetEntryTitle{Hyperbolic Function Exponential Definitions}
        \begin{align*}
            \sinh{x}
                &= \frac{e^x - e^{-x}}{2}
                = \frac{e^{2x} - 1}{2 e^x}
                = \frac{1 - e^{-2x}}{2 e^{-x}} \\
                %\quad \forall x \in \mathbb{R} \\
            \cosh{x}
                &= \frac{e^x + e^{-x}}{2}
                = \frac{e^{2x} + 1}{2 e^x}
                = \frac{1 + e^{-2x}}{2 e^{-x}} \\
                %\quad \forall x \in \mathbb{R} \\
            \tanh{x}
                &= \frac{\sinh{x}}{\cosh{x}}
                = \frac{e^x - e^{-x}}{e^x + e^{-x}}
                = \frac{e^{2x} - 1}{e^{2x} + 1}
        \end{align*}

    }

    \MulticolsBreak

    \CheatsheetEntryFrame{

        \CheatsheetEntryTitle{Hyperbolic: Difference of Squares} % TODO: Better title?
        \begin{gather*}
            \cosh^2{x} - \sinh^2{x} = 1 \\
            1 - \tanh^2{x} = \sech^2{x} \\
            \coth^2{x} - 1 = \csch^2{x}
        \end{gather*}

        \CheatsheetEntryTitle{Hyperbolic: Sum and Difference Formulae} % TODO: Better title?
        \begin{align*}
            \sinh{\parens*{\alpha \pm \beta}} &= \sinh{\alpha} \cosh{\beta } \pm \cosh{\alpha} \sinh{\beta } \\
            \cosh{\parens*{\alpha \pm \beta}} &= \cosh{\alpha} \cosh{\beta } \pm \sinh{\alpha} \sinh{\beta } \\
            \tanh{\parens*{\alpha \pm \beta}} &= \frac{\tanh{\alpha} \pm \tanh{\beta}}{1 \pm \tanh{\alpha} \tanh{\beta}}
        \end{align*}

        \CheatsheetEntryTitle{Hyperbolic: Double-Angle Formulae} % TODO: Better title?
        \begin{align*}
            \sinh{2x} &= 2 \sinh{x} \cosh{x} \\
            \cosh{2x} &= \cosh^2{x} + \sinh^2{x} \\
            \tanh{2x} &= \frac{2 \tanh{x}}{1 + \tanh^2{x}}
        \end{align*}

    }

\end{multicols}

%%%%%%%%%%%%%%%%%%%%%%%%%%%%%%%%%%%%%%%%%%%%%%%%%%%%%%%%%%%%%%%%%%%%%%%%%%%%%%%%%%%%%%%%%%%%%%%%%%%%
%%%%%%%%%%%%%%%%%%%%%%%%%%%%%%%%%%%%%%%%%%%%%%%%%%%%%%%%%%%%%%%%%%%%%%%%%%%%%%%%%%%%%%%%%%%%%%%%%%%%
%%%%%%%%%%%%%%%%%%%%%%%%%%%%%%%%%%%%%%%%%%%%%%%%%%%%%%%%%%%%%%%%%%%%%%%%%%%%%%%%%%%%%%%%%%%%%%%%%%%%

\newpage
\subsection{Linear Algebra}%
\label{sub:linear-algebra}

\begin{multicols}{2}

    \CheatsheetEntryFrame{

        \CheatsheetEntryTitle{Dot Product} {\tiny (or Scalar Product)}

        In $\mathbb{R}^n$:
        \begin{equation*}
            \mathbf{a} \cdot \mathbf{b} = a_1 b_1 + \cdots + a_n b_n = \sum_{k=0}^n{a_k b_k}
        \end{equation*}

        In $\mathbb{R}^2$ or $\mathbb{R}^3$:
        \begin{equation*}
            \mathbf{a} \cdot \mathbf{b} = \abs*{\mathbf{a}} \abs*{\mathbf{b}} \cos{\theta}
            ,\quad \theta \in [0, \pi]
        \end{equation*}

        Useful geometric properties:
        \begin{itemize}
            \item $\mathbf{a} \cdot \mathbf{a} = \abs*{a}^2$, and hence $\abs*{\mathbf{a}} = \sqrt{\mathbf{a} \cdot \mathbf{a}}$.
            \item Vectors $\mathbf{a}, \mathbf{b} \in \mathbb{R}^n$ are orthogonal if $\mathbf{a} \cdot \mathbf{b} = 0$.
        \end{itemize}

    }

    \CheatsheetEntryFrame{

        \CheatsheetEntryTitle{Cross Product} {\tiny (or Vector Product)}

        The \textit{cross product} is only defined in $\mathbb{R}^3$.
        \begin{equation*}
            \begin{pmatrix}
                a_1 \\
                a_2 \\
                a_3
            \end{pmatrix}
            \times
            \begin{pmatrix}
                b_1 \\
                b_2 \\
                b_3
            \end{pmatrix}
            =
            \begin{pmatrix}
                a_2 b_3 - a_3 b_2 \\
                a_3 b_1 - a_1 b_3 \\
                a_1 b_2 - a_2 b_1
            \end{pmatrix}
        \end{equation*}

        Calculation using determinants:
        \begin{gather*}
                \begin{pmatrix}
                    a_1 \\
                    a_2 \\
                    a_3
                \end{pmatrix}
                \times
                \begin{pmatrix}
                    b_1 \\
                    b_2 \\
                    b_3
                \end{pmatrix}
                =
                \begin{vmatrix}
                    \mathbf{e}_1 & \mathbf{e}_2 & \mathbf{e}_3 \\
                    a_1          & a_2          & a_3          \\
                    b_1          & b_2          & b_3
                \end{vmatrix}
            \\
                =
                \mathbf{e}_1
                \begin{vmatrix}
                    \memphR{a_2} & \memphB{a_3} \\
                    \memphB{b_2} & \memphR{b_3}
                \end{vmatrix}
                -
                \mathbf{e}_2
                \begin{vmatrix}
                    \memphR{a_1} & \memphB{a_3} \\
                    \memphB{b_1} & \memphR{b_3}
                \end{vmatrix}
                +
                \mathbf{e}_3
                \begin{vmatrix}
                    \memphR{a_1} & \memphB{a_2} \\
                    \memphB{b_1} & \memphR{b_2}
                \end{vmatrix}
            \\
                = \mathbf{e}_1 \parens*{\memphR{a_2 b_3} - \memphB{a_3 b_2}}
                - \mathbf{e}_2 \parens*{\memphR{a_1 b_3} - \memphB{a_3 b_1}}
                + \mathbf{e}_3 \parens*{\memphR{a_1 b_2} - \memphB{a_2 b_1}}
        \end{gather*}

        Useful geometric properties:
        \begin{itemize}
            \item Vector $\mathbf{a} \times \mathbf{b}$ is orthogonal to $\mathbf{a}$ and $\mathbf{b}$.
            \item $\abs*{\mathbf{a} \times \mathbf{b}} = \abs*{\mathbf{a}} \abs*{\mathbf{b}} \sin{\theta} = \text{area of a parallelogram}$
        \end{itemize}

    }

    \MulticolsBreak

    \CheatsheetEntryFrame{

        \CheatsheetEntryTitle{Cramer's Rule}

        Consider the following linear system with $n \times n$ invertible matrix $A$:
        \begin{equation*}
            A \mathbf{x} = \mathbf{b}
            ,\qquad A \in M_{nn},\ \mathbf{x} \in \mathbb{R}^n,\ \mathbf{b} \in \mathbb{R}^n
        \end{equation*}

        The system has a unique solution:
        \begin{equation*}
            x_k = \frac{\det{\parens*{B_k}}}{\det{\parens*{A}}}
            ,\qquad \forall k = 1, \dots, n,
        \end{equation*}
        where $B_k$ is the matrix obtained from $A$ by replacing the $\Nth{k}{th}$ column with the vector $\mathbf{b}$.

        \CheatsheetEntryExtraSeparation

        \CheatsheetEntryTitle{Cramer's Rule ($2 \times 2$ Matrix)}
        \begin{equation*}
            \begin{dcases}
                \xmemphB{a_{11}} x + \xmemphB{a_{12}} y = \xmemphR{b_1} \\
                \xmemphB{a_{21}} x + \xmemphB{a_{22}} y = \xmemphR{b_2}
            \end{dcases}
            \quad \Rightarrow \quad
                \xmemphB{
                    \begin{bmatrix}
                        a_{11} & a_{12} \\
                        a_{21} & a_{22}
                    \end{bmatrix}
                }
                \begin{bmatrix}
                    x \\
                    y
                \end{bmatrix}
                =
                \xmemphR{
                    \begin{bmatrix}
                        b_1 \\
                        b_2
                    \end{bmatrix}
                }
        \end{equation*}
        \begin{equation*}
                x
                    = \frac{
                        \xmemphB{
                            \begin{vmatrix}
                                \xmemphR{b_1} & a_{12} \\
                                \xmemphR{b_2} & a_{22}
                            \end{vmatrix}
                        }
                    }{
                        \xmemphB{
                            \begin{vmatrix}
                                a_{11} & a_{12} \\
                                a_{21} & a_{22}
                            \end{vmatrix}
                        }
                    }
                    ,\qquad
                y
                    = \frac{
                        \xmemphB{
                            \begin{vmatrix}
                                a_{11} & \xmemphR{b_1} \\
                                a_{21} & \xmemphR{b_2}
                            \end{vmatrix}
                        }
                    }{
                        \xmemphB{
                            \begin{vmatrix}
                                a_{11} & a_{12} \\
                                a_{21} & a_{22}
                            \end{vmatrix}
                        }
                    }
        \end{equation*}

        \CheatsheetEntryTitle{Cramer's Rule ($3 \times 3$ Matrix)}
        \begin{equation*}
            \begin{dcases}
                \xmemphB{a_{11}} x + \xmemphB{a_{12}} y + \xmemphB{a_{13}} z = \xmemphR{b_1} \\
                \xmemphB{a_{21}} x + \xmemphB{a_{22}} y + \xmemphB{a_{23}} z = \xmemphR{b_2} \\
                \xmemphB{a_{31}} x + \xmemphB{a_{32}} y + \xmemphB{a_{33}} z = \xmemphR{b_3}
            \end{dcases}
        \end{equation*}
        \begin{equation*}
                \xmemphB{
                    \begin{bmatrix}
                        a_{11} & a_{12} & a_{13} \\
                        a_{21} & a_{22} & a_{23} \\
                        a_{31} & a_{32} & a_{33}
                    \end{bmatrix}
                }
                \begin{bmatrix}
                    x \\
                    y \\
                    z
                \end{bmatrix}
                =
                \xmemphR{
                    \begin{bmatrix}
                        b_1 \\
                        b_2 \\
                        b_3
                    \end{bmatrix}
                }
        \end{equation*}
        \begin{equation*}
                x
                    = \frac{
                        \xmemphB{
                            \begin{vmatrix}
                                \xmemphR{b_1} & a_{12} & a_{13} \\
                                \xmemphR{b_2} & a_{22} & a_{23} \\
                                \xmemphR{b_3} & a_{32} & a_{33}
                            \end{vmatrix}
                        }
                    }{
                        \xmemphB{
                            \begin{vmatrix}
                                a_{11} & a_{12} & a_{13} \\
                                a_{21} & a_{22} & a_{23} \\
                                a_{31} & a_{32} & a_{33}
                            \end{vmatrix}
                        }
                    }
                    ,\qquad
                y
                    = \frac{
                        \xmemphB{
                            \begin{vmatrix}
                                a_{11} & \xmemphR{b_1} & a_{13} \\
                                a_{21} & \xmemphR{b_2} & a_{23} \\
                                a_{31} & \xmemphR{b_3} & a_{33}
                            \end{vmatrix}
                        }
                    }{
                        \xmemphB{
                            \begin{vmatrix}
                                a_{11} & a_{12} & a_{13} \\
                                a_{21} & a_{22} & a_{23} \\
                                a_{31} & a_{32} & a_{33}
                            \end{vmatrix}
                        }
                    }
                    ,
        \end{equation*}
        \begin{equation*}
                z
                    = \frac{
                        \xmemphB{
                            \begin{vmatrix}
                                a_{11} & a_{12} & \xmemphR{b_1} \\
                                a_{21} & a_{22} & \xmemphR{b_2} \\
                                a_{31} & a_{32} & \xmemphR{b_3}
                            \end{vmatrix}
                        }
                    }{
                        \xmemphB{
                            \begin{vmatrix}
                                a_{11} & a_{12} & a_{13} \\
                                a_{21} & a_{22} & a_{23} \\
                                a_{31} & a_{32} & a_{33}
                            \end{vmatrix}
                        }
                    }
        \end{equation*}

    }

\end{multicols}

%%%%%%%%%%%%%%%%%%%%%%%%%%%%%%%%%%%%%%%%%%%%%%%%%%%%%%%%%%%%%%%%%%%%%%%%%%%%%%%%%%%%%%%%%%%%%%%%%%%%
%%%%%%%%%%%%%%%%%%%%%%%%%%%%%%%%%%%%%%%%%%%%%%%%%%%%%%%%%%%%%%%%%%%%%%%%%%%%%%%%%%%%%%%%%%%%%%%%%%%%
%%%%%%%%%%%%%%%%%%%%%%%%%%%%%%%%%%%%%%%%%%%%%%%%%%%%%%%%%%%%%%%%%%%%%%%%%%%%%%%%%%%%%%%%%%%%%%%%%%%%

\newpage
\subsection{Single Variable Calculus}%
\label{sub:calculus}

\begin{multicols}{2}

    \CheatsheetEntryFrame{

        \CheatsheetEntryTitle{Derivative: First Principles}
        \begin{equation*}
            \frac{\diff{y}}{\diff{x}}
                %= \lim_{\delta x \to 0}{\frac{\delta y}{\delta x}}
                = \lim_{h \to 0}{\frac{f(x+h) - f(x)}{h}}
                = \lim_{u \to x}{\frac{f(u) - f(x)}{u-x}}
        \end{equation*}

        \CheatsheetEntryTitle{Derivative: Product Rule}
        \begin{gather*}
            \frac{\diff{}}{\diff{x}} \parens*{\memphR{u} \memphG{v}}
                = \memphR{u} \memphG{\frac{\diff{v}}{\diff{x}}}
                    + \memphG{v} \memphR{\frac{\diff{u}}{\diff{x}}}
                \\
                \frac{\diff{}}{\diff{x}} \parens*{\memphR{u} \memphG{v} \memphB{w}}
                = \memphG{v} \memphB{w} \memphR{\frac{\diff{u}}{\diff{x}}}
                    + \memphR{u} \memphB{w} \memphG{\frac{\diff{v}}{\diff{x}}}
                    + \memphR{u} \memphG{v} \memphB{\frac{\diff{w}}{\diff{x}}}
        \end{gather*}

        \CheatsheetEntryTitle{Derivative: Quotient Rule}
        \begin{equation*}
            \frac{\diff{}}{\diff{x}}{\parens*{\frac{\memphR{a}}{\memphG{u}}}}
                =
                    \frac{
                        \memphG{u} \memphR{\frac{\diff{a}}{\diff{x}}} - \memphR{a} \memphG{\frac{\diff{u}}{\diff{x}}}
                    }{
                        \memphG{u}^2
                    }
                =
                    \underbracket{
                        \frac{
                            \memphG{u} \memphR{a'} - \memphR{a} \memphG{u'}
                        }{
                            \memphG{u}^2
                        }
                    }_{\substack{\text{easier form} \\ \text{to remember}}}
        \end{equation*}

        \CheatsheetEntryTitle{Derivative: Chain Rule}
        \begin{equation*}
            \frac{\memphR{\diff{y}}}{\memphG{\diff{x}}}
                = \frac{\memphR{\diff{y}}}{\memphB{\diff{u}}} \times \frac{\memphB{\diff{u}}}{\memphG{\diff{x}}}
        \end{equation*}

    }

    \CheatsheetEntryFrame{

        \CheatsheetEntryTitle{Integration: Reverse Chain Rule}
        \begin{equation*}
            \int{\memphR{f'(x)} \, \memphB{g(}\memphR{f(x)}\memphB{)} \,\diff{x}}
                = \memphB{g'(}\memphR{f(x)}\memphB{)} + C
        \end{equation*}

        \CheatsheetEntryTitle{Integration: Integration By Parts}
        \begin{equation*}
            \int{\memphR{f(x)} \, \memphB{G(x)} \,\diff{x}}
            = \memphR{F(x)} \, \memphB{G(x)} - \int{\memphR{F(x)} \, \memphB{g(x)} \,\diff{x}}
        \end{equation*}

    }

    \MulticolsBreak

    \MulticolsPhantomPlaceholder

\end{multicols}


%%%%%%%%%%%%%%%%%%%%%%%%%%%%%%%%%%%%%%%%%%%%%%%%%%%%%%%%%%%%%%%%%%%%%%%%%%%%%%%%%%%%%%%%%%%%%%%%%%%%
%%%%%%%%%%%%%%%%%%%%%%%%%%%%%%%%%%%%%%%%%%%%%%%%%%%%%%%%%%%%%%%%%%%%%%%%%%%%%%%%%%%%%%%%%%%%%%%%%%%%
%%%%%%%%%%%%%%%%%%%%%%%%%%%%%%%%%%%%%%%%%%%%%%%%%%%%%%%%%%%%%%%%%%%%%%%%%%%%%%%%%%%%%%%%%%%%%%%%%%%%

\newpage
\subsection{Common Integrals}%
\label{sub:common-integrals}

{\color{extranotecolor} \textit{Note: Constant of integration $+C$ and trivial domain restrictions (usually $a \ne 0$) omitted for brevity except in places of particular significance.} }

% idk if this is why there's excessive whitespace, but it helps it look better!
\vspace*{-\parskip} 
\vspace*{-\abovedisplayskip}
%\unskip % Why doesn't this work?

%{\color{extranotecolor}
%    \begin{equation*}
%        \int{x^n \,\diff{x}} = \frac{1}{n+1} x^{n+1}
%        \qquad \xRightarrow{\text{derivative form}} \qquad
%        \frac{\diff{}}{\diff{x}} \parens*{x^n} = n x^{n-1}
%    \end{equation*}
%}
%
\begin{HackEquationLeftAlign}
\begin{alignat*}{2}
    &\int{\parens*{ax + b}^n \,\diff{x}} = \frac{\parens*{ax + b}^{n+1}}{a \parens*{n+1}}
        ,\quad
        &&{\color{extranotecolor}
            %a \ne 0
            n \neq -1
            ;\ x \neq 0
            ,\ \text{if } n < 0
        }
        %&&
        %\qquad
        %{\color{extranotecolor}
        %    \frac{\diff{}}{\diff{x}} \parens*{x^n} = n x^{n-1}
        %}
        %&&
        %\qquad
        %{\color{extranotecolor}
        %    \int{x^n \,\diff{x}} = \frac{1}{n+1} x^{n+1}
        %}
        \\
    &\int{\frac{1}{ax + b} \,\diff{x}} = \frac{1}{a} \ln{\abs{ax + b}} + C
        = \frac{1}{a} \ln{\abs{K(ax + b)}}
        ,\quad
        &&{\color{extranotecolor}
            C = \frac{1}{a} \ln{K}
            %;\ a \ne 0
            ;\ ax + b \ne 0
        }
        %&&
        %\qquad
        %{\color{extranotecolor}
        %    \frac{\diff{}}{\diff{x}} \parens*{\ln{x}} = \frac{1}{x}
        %}
        %&&
        %\qquad
        %{\color{extranotecolor}
        %    \int{\frac{1}{x} \,\diff{x}} = \ln{\abs*{x}}
        %}
\end{alignat*}
\end{HackEquationLeftAlign}%


\vspace*{-\parskip} % idk if this is why there's excessive whitespace, but it helps it look better!
\begin{myminipage}[t]{0.5\linewidth}
    \begin{HackEquationLeftAlign}
    \begin{equation*}
        \int{e^{ax} \,\diff{x}} = \frac{1}{a} e^{ax}
    \end{equation*}
    \end{HackEquationLeftAlign}%
\end{myminipage}%
\begin{myminipage}[t]{0.5\linewidth}
    \begin{HackEquationLeftAlign}
    \begin{equation*}
        \int{a^x \,\diff{x}}
            %= \int{\parens*{e^{\ln{a}}}^x \,\diff{x}} % Derived from change of base law
            = \frac{a^x}{\ln{a}}
            ,\quad
            {\color{extranotecolor}
                a \neq 1
            }
    \end{equation*}
    \end{HackEquationLeftAlign}%
\end{myminipage}%


\begin{myminipage}[t]{0.5\linewidth}
    \begin{HackEquationLeftAlign}
    \begin{alignat*}{3}
        &\int{\cos{ax} \,\diff{x}} &&= && \frac{1}{a} \sin{ax} \\
        &\int{\sin{ax} \,\diff{x}} &&= -&& \frac{1}{a} \cos{ax} \\
        &\int{\sec^2{ax} \,\diff{x}} &&= && \frac{1}{a} \tan{ax} \\
        &\int{\tan{ax} \,\diff{x}} &&= && \frac{1}{a} \ln{\abs{\sec{ax}}}
    \end{alignat*}
    \end{HackEquationLeftAlign}%
\end{myminipage}%
\begin{myminipage}[t]{0.5\linewidth}
    \begin{HackEquationLeftAlign}
    \begin{alignat*}{2}
        &\int{\cosh{ax} \,\diff{x}} &&= \frac{1}{a} \sinh{ax} \\
        &\int{\sinh{ax} \,\diff{x}} &&= \frac{1}{a} \cosh{ax} \\
        &\int{\sech^2{ax} \,\diff{x}} &&= \frac{1}{a} \tanh{ax} \\
        &\int{\tanh{ax} \,\diff{x}} &&= \frac{1}{a} \ln{\parens*{\cosh{ax}}}
    \end{alignat*}
    \end{HackEquationLeftAlign}%
\end{myminipage}


\begin{myminipage}[t]{0.5\linewidth}
    \begin{HackEquationLeftAlign}
    \begin{alignat*}{3}
        &\int{\csc{ax} \cot{ax} \,\diff{x}} &&= -&& \frac{1}{a} \csc{ax} \\
        &\int{\sec{ax} \tan{ax} \,\diff{x}} &&= && \frac{1}{a} \sec{ax} \\
        &\int{\csc^2{ax} \,\diff{x}} &&= -&& \frac{1}{a} \cot{ax}
    \end{alignat*}
    \end{HackEquationLeftAlign}%
\end{myminipage}%
\begin{myminipage}[t]{0.5\linewidth}
    \begin{HackEquationLeftAlign}
    \begin{alignat*}{3}
        &\int{\csch{ax} \coth{ax} \,\diff{x}} &&= -&& \frac{1}{a} \csch{ax} \\
        &\int{\sech{ax} \tanh{ax} \,\diff{x}} &&= -&& \frac{1}{a} \sech{ax} \\
        &\int{\csch^2{ax} \,\diff{x}} &&= -&& \frac{1}{a} \coth{ax}
    \end{alignat*}
    \end{HackEquationLeftAlign}%
\end{myminipage}


\begin{myminipage}[t]{0.5\linewidth}
    \begin{HackEquationLeftAlign}
    \begin{alignat*}{3}
        &\int{\sec{ax} \,\diff{x}} &&= && \frac{1}{a} \ln{\abs{\sec{ax} + \tan{ax}}} \\
        &\int{\csc{ax} \,\diff{x}} &&= -&& \frac{1}{a} \ln{\abs*{\cot{\frac{ax}{2}}}} \\
        & &&= -&& \frac{1}{a} \ln{\abs{\csc{ax} + \cot{ax}}} \\
        &\int{\cot{ax} \,\diff{x}} &&= && \frac{1}{a} \ln{\abs{\sin{ax}}}
    \end{alignat*}
    \end{HackEquationLeftAlign}%
\end{myminipage}%
\begin{myminipage}[t]{0.5\linewidth}
    \begin{HackEquationLeftAlign}
    \begin{alignat*}{3}
        &\int{\sech{ax} \,\diff{x}} &&= \frac{1}{a} \tan^{-1}{\parens*{\sinh{ax}}} \\
        & &&= \frac{2}{a} \tan^{-1}{\parens*{\tanh{\frac{ax}{2}}}} \\
        &\int{\csch{ax} \,\diff{x}} &&= \frac{1}{a} \ln{\abs*{\tanh{\frac{ax}{2}}}} \\
        & &&= \frac{1}{a} \ln{\abs*{\csch{ax} - \coth{ax}}} \\
        &\int{\coth{ax} \,\diff{x}} &&= \frac{1}{a} \ln{\abs*{\sinh{ax}}}
    \end{alignat*}
    \end{HackEquationLeftAlign}%
\end{myminipage}


\begin{myminipage}[t]{0.5\linewidth}
    \begin{HackEquationLeftAlign}
    \begin{alignat*}{3}
        &\int{\frac{1}{x^2 + a^2} \,\diff{x}} &&= \frac{1}{a} \tan^{-1}{\frac{x}{a}}
            && \\
        \vphantom{
            = \frac{1}{a} \coth^{-1}{\frac{x}{a}} + C_2
        } \\
        \vphantom{
            = \frac{1}{2a} \ln{\abs*{\frac{a+x}{a-x}}} + C_3
        } \\
        &\int{\frac{1}{\sqrt{a^2 - x^2}} \,\diff{x}} &&= \hphantom{-} \sin^{-1}{\frac{x}{a}} + C_1
            ,\quad
            &&{\color{extranotecolor}
                a > 0 ,\ -a < x < a
            } \\
        & &&= - \cos^{-1}{\frac{x}{a}} + C_2
            ,\quad
            &&{\color{extranotecolor}
                a > 0 ,\ -a < x < a
            }
    \end{alignat*}
    \end{HackEquationLeftAlign}%
\end{myminipage}%
\begin{myminipage}[t]{0.5\linewidth}
    \begin{HackEquationLeftAlign}
    \begin{alignat*}{3}
        &\int{\frac{1}{a^2 - x^2} \,\diff{x}} &&= \frac{1}{a} \tanh^{-1}{\frac{x}{a}} + C_1
            ,\quad
            &&{\color{extranotecolor}
                a > \abs{x}
            } \\
        & &&= \frac{1}{a} \coth^{-1}{\frac{x}{a}} + C_2
            ,\quad
            &&{\color{extranotecolor}
                \abs*{x} > a > 0
            } \\
        & &&= \frac{1}{2a} \ln{\abs*{\frac{a+x}{a-x}}} + C_3
            ,\quad
            &&{\color{extranotecolor}
                x^2 \neq a^2
            } \\
        &\int{\frac{1}{\sqrt{x^2 + a^2}} \,\diff{x}} &&= \sinh^{-1}{\frac{x}{a}} + C_1
            && \\
        & &&= \ln{\parens*{x + \sqrt{x^2 + a^2}}} + C_2
            && \\
        &\int{\frac{1}{\sqrt{x^2 - a^2}} \,\diff{x}} &&= \cosh^{-1}{\frac{x}{a}} + C_1
            ,\quad
            &&{\color{extranotecolor}
                x > a > 0
            } \\
        & &&= \ln{\parens*{x + \sqrt{x^2 - a^2}}} + C_2
            ,\quad
            &&{\color{extranotecolor}
                x > a > 0
            }
    \end{alignat*}
    \end{HackEquationLeftAlign}%
\end{myminipage}

%%%%%%%%%%%%%%%%%%%%%%%%%%%%%%%%%%%%%%%%%%%%%%%%%%%%%%%%%%%%%%%%%%%%%%%%%%%%%%%%%%%%%%%%%%%%%%%%%%%%
%%%%%%%%%%%%%%%%%%%%%%%%%%%%%%%%%%%%%%%%%%%%%%%%%%%%%%%%%%%%%%%%%%%%%%%%%%%%%%%%%%%%%%%%%%%%%%%%%%%%
%%%%%%%%%%%%%%%%%%%%%%%%%%%%%%%%%%%%%%%%%%%%%%%%%%%%%%%%%%%%%%%%%%%%%%%%%%%%%%%%%%%%%%%%%%%%%%%%%%%%

\newpage
\subsection{Multivariable and Vector Calculus}%
\label{sub:multivariable-calculus}

\begin{multicols}{2}

    \CheatsheetEntryFrame{

        \CheatsheetEntryTitle{Notation of Second Order Partial Derivatives}

        Consider a function of two variables $f(x, y)$.
        \newcommand{\Fun}{\memphG{f}}
        \renewcommand{\X}{\memphBC{x}}
        \renewcommand{\Y}{\memphRC{y}}
        \newcommand{\DFun}{\memphG{\partial{f}}}
        \newcommand{\DX}{\memphBC{\partial{x}}}
        \newcommand{\DY}{\memphRC{\partial{y}}}
        \newcommand{\DsqFun}{\memphG{\partial^2{f}}}
        \newcommand{\DXsq}{\memphBC{\partial{x}^2}}
        \newcommand{\DYsq}{\memphRC{\partial{y}^2}}
        \begin{alignat*}{4}
            \frac{\DsqFun}{\DXsq}
                &= \frac{\partial{}}{\DX} && \parens*{\frac{\DFun}{\DX}} &
                &= (\memphG{f}_{\X})_{\X} &
                &= \memphG{f}_{\X \X}
                \\
            \frac{\DsqFun}{\DYsq}
                &= \frac{\partial{}}{\DY} && \parens*{\frac{\DFun}{\DY}} &
                &= (\memphG{f}_{\Y})_{\Y} &
                &= \memphG{f}_{\Y \Y}
                \\[\EqExtraSkip]
            \frac{\DsqFun}{\DY \, \DX}
                &= \frac{\partial{}}{\DY} && \parens*{\frac{\DFun}{\DX}} &
                &= (\memphG{f}_{\X})_{\Y} &
                &= \memphG{f}_{\X \Y}
                \\
            \frac{\DsqFun}{\DX \, \DY}
                &= \frac{\partial{}}{\DX} && \parens*{\frac{\DFun}{\DY}} &
                &= (\memphG{f}_{\Y})_{\X} &
                &= \memphG{f}_{\Y \X}
        \end{alignat*}

    }

    \CheatsheetEntryFrame{

        \CheatsheetEntryTitle{Equality of Mixed Partials}

        Consider a function of two variables $f(x, y)$.
        % Consider a multivariable function $f(x_1, x_2, \dots, x_n)$.
        
        If $f$ and all of its first and second order partial derivatives are continuous, then:
        \newcommand{\DsqFun}{\memphG{\partial^2{f}}}
        \newcommand{\DX}{\memphBC{\partial{x}}}
        \newcommand{\DY}{\memphRC{\partial{y}}}
        \begin{equation*}
            \frac{\DsqFun}{\DX \, \DY}
            = \frac{\DsqFun}{\DY \, \DX}
        \end{equation*}

        \Todo{This generalizes to higher derivatives. Rewrite this and find a good source about continuity requirements.}

    }
    
\end{multicols}

%%%%%%%%%%%%%%%%%%%%%%%%%%%%%%%%%%%%%%%%%%%%%%%%%%%%%%%%%%%%%%%%%%%%%%%%%%%%%%%%%%%%%%%%%%%%%%%%%%%%
%%%%%%%%%%%%%%%%%%%%%%%%%%%%%%%%%%%%%%%%%%%%%%%%%%%%%%%%%%%%%%%%%%%%%%%%%%%%%%%%%%%%%%%%%%%%%%%%%%%%
%%%%%%%%%%%%%%%%%%%%%%%%%%%%%%%%%%%%%%%%%%%%%%%%%%%%%%%%%%%%%%%%%%%%%%%%%%%%%%%%%%%%%%%%%%%%%%%%%%%%

\newpage
\subsection{Differential Equations}%
\label{sub:diff-eq}

\begin{multicols}{2}

    \CheatsheetEntryFrame{

        \CheatsheetEntryTitle{Separable ODEs}

        Write the ODE in the form:
        \begin{equation*}
            h(y) \,\diff{y} = g(x) \,\diff{x}
        \end{equation*}

        To solve, integrate both sides:
        \begin{equation*}
            H(y) = G(x) + C
        \end{equation*}

    }

    \CheatsheetEntryFrame{

        \CheatsheetEntryTitle{First-Order Linear ODEs}

        Write the ODE in the form:
        \begin{equation*}
            \frac{\diff{y}}{\diff{x}} + f(x) \, y = g(x)
        \end{equation*}

        First, define the \textit{integrating factor} $h$:
        \begin{equation*}
            h(x) = e^{\int{f(x) \,\diff{x}}}
            \Exn{
                \qquad \Longrightarrow \qquad
                h'(x) = f(x) \, h(x)
            }
        \end{equation*}

        Multiply the ODE by $h$ and undo the product rule:
        \begin{gather*}
            h(x) \, \frac{\diff{y}}{\diff{x}} + h(x) \, f(x) \, y = g(x) \, h(x) \\
            \frac{\diff{}}{\diff{x}} \parens*{h(x) \, y} = g(x) \, h(x)
        \end{gather*}

        To solve, integrate both sides:
        \begin{equation*}
            h(x) \, y = \int{g(x) \, h(x) \,\diff{x}}
        \end{equation*}

    }

    \MulticolsBreak

    \CheatsheetEntryFrame{

        \CheatsheetEntryTitle{Exact ODEs}

        Write the ODE in the form:
        \begin{equation*}
            F(x, y) + G(x, y) \frac{\diff{y}}{\diff{x}} = 0
        \end{equation*}

        First prove the ODE is \textit{exact} by showing:
        \begin{equation*}
            \frac{\partial{F}}{\partial{y}}
            = \frac{\partial{G}}{\partial{x}}
            \Exn{
                {} = \underbrace{
                    \frac{\partial^2{H}}{\partial{x} \, \partial{y}}
                    = \frac{\partial^2{H}}{\partial{y} \, \partial{x}}
                }_{\substack{\text{equality of}\\\text{mixed partials}}}
            }
        \end{equation*}
        \NegateBelowDisplaySkip

        Thus, our ODE can be rewritten:
        \begin{equation*}
            H_x(x, y) + H_y(x, y) \frac{\diff{y}}{\diff{x}} = 0
        \end{equation*}

        This is can be simplified by undoing the chain rule:
        \begin{gather*}
            \frac{\partial{H}}{\partial{x}} \frac{\diff{x}}{\diff{x}}
            + \frac{\partial{H}}{\partial{y}} \frac{\diff{y}}{\diff{x}}
            = 0 \\
            \frac{\diff{}}{\diff{x}} \parens*{H \parens*{x, y(x)}} = 0
        \end{gather*}

        Therefore, the solution is:
        \begin{equation*}
            H(x, y) = C
        \end{equation*}
        \NegateBelowDisplaySkip

        such that:
        \begin{alignat*}{3}
            H(x, y) &= \int{H_x(x, y) \,\partial{x}} &&= H_1(x, y) &&+ C_1(y) \\
            H(x, y) &= \int{H_y(x, y) \,\partial{y}} &&= H_2(x, y) &&+ C_2(x)
        \end{alignat*}

    }

    \Todo{Consider adding a section on change of variable.}
    
\end{multicols}
\newpage
\begin{multicols}{2}

    \CheatsheetEntryFrame{

        \CheatsheetEntryTitle{Second-Order Linear ODEs}

        \Todo{Do this!}

    }
    
\end{multicols}

%%%%%%%%%%%%%%%%%%%%%%%%%%%%%%%%%%%%%%%%%%%%%%%%%%%%%%%%%%%%%%%%%%%%%%%%%%%%%%%%%%%%%%%%%%%%%%%%%%%%
%%%%%%%%%%%%%%%%%%%%%%%%%%%%%%%%%%%%%%%%%%%%%%%%%%%%%%%%%%%%%%%%%%%%%%%%%%%%%%%%%%%%%%%%%%%%%%%%%%%%
%%%%%%%%%%%%%%%%%%%%%%%%%%%%%%%%%%%%%%%%%%%%%%%%%%%%%%%%%%%%%%%%%%%%%%%%%%%%%%%%%%%%%%%%%%%%%%%%%%%%

\newpage
\subsection{Complex Analysis}%
\label{sub:complex-analysis}

\begin{multicols}{2}

    \CheatsheetEntryFrame{

        \CheatsheetEntryTitle{Imaginary Unit}
        \begin{equation*}
            i^2 = -1
        \end{equation*}
        %\begin{tabularx}{\textwidth}{CC}
        %    $i^2 = -1$ & $\underset{\substack{\text{(alternative notation for}\\\text{electrical engineering}\\\text{and some other fields)}}}{j^2 = -1}$ \\
        %\end{tabularx}

        Alternative notation for electrical engineering (and some other fields):
        \begin{equation*}
            j^2 = -1
        \end{equation*}

        \CheatsheetEntryTitle{Complex Number Representation}

        The same complex number $z$ can be written:
        \begin{alignat*}{2}
            z
                &= a + ib &
                    \qquad & \text{(Rectangular Form)} \\
                &= r(\cos{\theta} + i \sin{\theta}) &
                    \qquad & \text{(Polar Form)} \\
                &= r e^{i \theta} &
                    \qquad & \text{(Exponential Form)} \\
                &= r \phase{\theta} &
                    \qquad & \text{(Steinmetz Notation)} \\
                &\Exn{= r \cis{\theta},} &
                    \qquad & \Exn{\text{(cis Form) [rarely used]}}
        \end{alignat*}%
        where $a, b, r, \theta \in \mathbb{R}$.

        %The same complex number $z$ can be written:\\[\abovedisplayskip]
        %\begin{center}
        %\begin{tabular}{lc}
        %    Rectangular Form & $a + ib$ \\
        %    Polar Form & $r(\cos{\theta} + i \sin{\theta})$ \\
        %    Exponential Form & $r e^{i \theta}$ \\
        %    Steinmetz Notation \phantom{x} & $r \phase{\theta}$ \\
        %    \Exn{$\cis$ Form {\scriptsize (rarely used)}} & \Exn{$r \cis{\theta}$} \\
        %\end{tabular}
        %\end{center}

        Components:
        \begin{alignat*}{3}
            & \text{Real Part:}            & \MyRe{(z)}  &= a      &&= r \cos{\theta} \phantom{\frac{}{a}} \\
            & \text{Imaginary Part:} \quad & \MyIm{(z)}  &= b      &&= r \sin{\theta} \phantom{\frac{b}{a}} \\
            & \text{Modulus:}              & \abs*{z}    &= r      &&= \sqrt{a^2 + b^2} \phantom{\frac{b}{a}} \\
            & \text{Argument:}             & \arg{(z)}   &= \theta &&= \tan^{-1}{\parens*{\frac{b}{a}}}
        \end{alignat*}

        \CheatsheetEntryTitle{Euler's Formula}
        \begin{equation*}
            e^{i \theta} = \cos{\theta} + i \sin{\theta}, \qquad \forall \theta \in \mathbb{R}
        \end{equation*}

    }

    \CheatsheetEntryFrame{

        \CheatsheetEntryTitle{Complex Conjugate}
        \begin{equation*}
            \complexconjugate{a + ib} = a - ib
        \end{equation*}

        Alternative notation for physics and engineering:
        \begin{equation*}
            \parens*{a + ib}^* = a - ib
        \end{equation*}

        \CheatsheetEntryTitle{Complex Conjugate: $z \complexconjugate{z}$}
        \begin{gather*}
            z \complexconjugate{z} = \abs*{z}^2 \\
            \parens*{a + ib} \complexconjugate{\parens*{a + ib}} = \parens*{a + ib} \parens*{a - ib} = a^2 + b^2
        \end{gather*}
        
    }

    \MulticolsBreak

    \MulticolsPhantomPlaceholder

\end{multicols}

