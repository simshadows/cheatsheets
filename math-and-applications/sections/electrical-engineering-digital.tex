\subsection{Digital Circuits and Logic}%
\label{sub:digital-circuits-and-logic}

\begin{multicols}{2}

    \CheatsheetEntryFrame{

        \renewcommand{\MyReusableFormatting}[5]{ % This will be used for content
            %\vspace*{1ex}
            \begin{minipage}[c]{0.20\columnwidth}
                \centering
                #1
            \end{minipage}%
            \begin{minipage}[c]{0.18\columnwidth}
                \centering
                #2
            \end{minipage}%
            %\hspace{0.04\columnwidth}%
            \begin{minipage}[c]{0.32\columnwidth}
                \centering
                \scalebox{0.8}{%
                \begin{tabular}{#3}
                    \HLineA
                    #4
                    \HLineA
                \end{tabular}
                }%
            \end{minipage}%
            \begin{minipage}[c]{0.30\columnwidth}
            \begin{center}
                \begin{circuitikz}
                    \draw
                        #5
                    ;
                \end{circuitikz}
            \end{center}
            \end{minipage}%
            \vspace*{1ex}
        }
        \renewcommand{\MyReusableFormattingB}{ % This will be used for separators
            {\color{lightgray} \hrule{}}
        }
        \renewcommand{\MyReusableFormattingC}{ % This will be used for special separators
            \MyReusableFormattingB%
            \vspace*{2.5pt}%
            \MyReusableFormattingB
        }

        \renewcommand{\X}{{\cellcolor{myred}\color{white}$\bm{0}$}} % Used for logical low
        \renewcommand{\Y}{{\cellcolor{mygreen}\color{white}$\bm{1}$}} % Used for logical high

        \renewcommand{\W}[1]{{\cellcolor{black}\color{white}$\bm{#1}$}} % Used for headings

        \renewcommand{\VRuleA}{\vrule width 3pt}
        \renewcommand{\HLineA}{\noalign{\hrule height 3pt}}

        \CheatsheetEntryTitle{Logic Gates}

        \MyReusableFormatting{NOT {\footnotesize(Inverter)}}{$\overline{A}$}{!{\VRuleA}c!{\VRuleA}c!{\VRuleA}}{
            \W{A} & \W{O} \\ \HLineA
            \X & \Y \\
            \Y & \X \\
        }{
            (0,0) node[american not port, name=G, /tikz/circuitikz/bipoles/length=1.0cm, /tikz/circuitikz/bipoles/not port/circle width=0.35] {}
            (G.in)  -- ++(-0.4,0)
            (G.out) -- ++( 0.4,0)
        }
        \MyReusableFormattingC

        \MyReusableFormatting{AND}{$A \cdot B$}{!{\VRuleA}cc!{\VRuleA}c!{\VRuleA}}{
            \W{A} & \W{B} & \W{O} \\ \HLineA
            \X & \X & \X \\
            \X & \Y & \X \\
            \Y & \X & \X \\
            \Y & \Y & \Y \\
        }{
            (0,0) node[american and port, name=G] {}
            (G.in 1) -- ++(-0.4,0)
            (G.in 2) -- ++(-0.4,0)
            (G.out)  -- ++( 0.4,0)
        }
        \MyReusableFormattingB

        \MyReusableFormatting{OR}{$A + B$}{!{\VRuleA}cc!{\VRuleA}c!{\VRuleA}}{
            \W{A} & \W{B} & \W{O} \\ \HLineA
            \X & \X & \X \\
            \X & \Y & \Y \\
            \Y & \X & \Y \\
            \Y & \Y & \Y \\
        }{
            (0,0) node[american or port, name=G] {}
            (G.in 1) -- ++(-0.4,0)
            (G.in 2) -- ++(-0.4,0)
            (G.out)  -- ++( 0.4,0)
        }
        \MyReusableFormattingB

        \MyReusableFormatting{XOR}{$A \oplus B$}{!{\VRuleA}cc!{\VRuleA}c!{\VRuleA}}{
            \W{A} & \W{B} & \W{O} \\ \HLineA
            \X & \X & \X \\
            \X & \Y & \Y \\
            \Y & \X & \Y \\
            \Y & \Y & \X \\
        }{
                (0,0) node[american xor port, name=G] {}
                (G.in 1) -- ++(-0.4,0)
                (G.in 2) -- ++(-0.4,0)
                (G.out)  -- ++( 0.4,0)
        }

        \MyReusableFormattingC

        \MyReusableFormatting{NAND}{$\overline{A \cdot B}$}{!{\VRuleA}cc!{\VRuleA}c!{\VRuleA}}{
            \W{A} & \W{B} & \W{O} \\ \HLineA
            \X & \X & \Y \\
            \X & \Y & \Y \\
            \Y & \X & \Y \\
            \Y & \Y & \X \\
        }{
            (0,0) node[american nand port, name=G] {}
            (G.in 1) -- ++(-0.4,0)
            (G.in 2) -- ++(-0.4,0)
            (G.out)  -- ++( 0.4,0)
        }
        \MyReusableFormattingB

        \MyReusableFormatting{NOR}{$\overline{A + B}$}{!{\VRuleA}cc!{\VRuleA}c!{\VRuleA}}{
            \W{A} & \W{B} & \W{O} \\ \HLineA
            \X & \X & \Y \\
            \X & \Y & \X \\
            \Y & \X & \X \\
            \Y & \Y & \X \\
        }{
            (0,0) node[american nor port, name=G] {}
            (G.in 1) -- ++(-0.4,0)
            (G.in 2) -- ++(-0.4,0)
            (G.out)  -- ++( 0.4,0)
        }
        \MyReusableFormattingB

        \MyReusableFormatting{XNOR}{$\overline{A \oplus B}$}{!{\VRuleA}cc!{\VRuleA}c!{\VRuleA}}{
            \W{A} & \W{B} & \W{O} \\ \HLineA
            \X & \X & \Y \\
            \X & \Y & \X \\
            \Y & \X & \X \\
            \Y & \Y & \Y \\
        }{
            (0,0) node[american xnor port, name=G] {}
            (G.in 1) -- ++(-0.4,0)
            (G.in 2) -- ++(-0.4,0)
            (G.out)  -- ++( 0.4,0)
        }

    }

    \MulticolsBreak

    \CheatsheetEntryFrame{

        \CheatsheetEntryTitle{Binary Numbers}

        \Todo{This!}

        \CheatsheetEntryExtraSeparation

        \CheatsheetEntryTitle{Hexadecimal and Octal Numbers}

        \Todo{This!}

    }

    \CheatsheetEntryFrame{

        %\CheatsheetEntryTitle{Commutative Laws}
        %\begin{gather*}
        %    A \cdot B = B \cdot A \\
        %    A + B = B + A
        %\end{gather*}

        %\CheatsheetEntryTitle{Associative Laws}
        %\begin{gather*}
        %    (A \cdot B) \cdot C = A \cdot (B \cdot C) = A \cdot B \cdot C \\
        %    (A + B) + C = A + (B + C) = A + B + C
        %\end{gather*}

        \CheatsheetEntryTitle{Distributive Laws}
        \begin{gather*}
            A \cdot (B + C) = (A \cdot B) + (A \cdot C) \\
            A + (B \cdot C) = (A + B) \cdot (A + C)
        \end{gather*}

        \CheatsheetEntryTitle{Absorption Laws}
        \begin{gather*}
            A + (A \cdot B) = A \\
            A \cdot (A + B) = A \\
            (A \cdot B) + (A \cdot \overline{B}) = A \\
            (A + B) \cdot (A + \overline{B}) = A
        \end{gather*}

        \CheatsheetEntryTitle{De Morgan's Theorem}
        \begin{gather*}
            \overline{A + B} = \overline{A} \cdot \overline{B} \\
            \overline{A \cdot B} = \overline{A} + \overline{B}
        \end{gather*}

    }

\end{multicols}

