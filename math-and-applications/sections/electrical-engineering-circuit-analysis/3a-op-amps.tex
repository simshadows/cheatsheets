\subsection{Amplifier Analysis}%
\label{sub:amplifier-analysis}

\subsubsection{Operational Amplifiers}

\begin{CheatsheetEntryFrame}

    \CheatsheetEntryTitle{Ideal Opamp}

    \begin{multicols}{2}

        \begin{center}
            \phantom{\myul{\textbf{\footnotesize Schematic Symbol}}}

            \begin{circuitikz}
                \draw 
                    (0,0)
                        node[op amp, anchor=-](OA){}
                    (OA.+)
                        -- ++(-0.5,0)
                            node[ocirc, label=left:$v_+$]{}
                    (OA.-)
                        -- ++(-0.5,0)
                            node[ocirc, label=left:$v_-$]{}
                    (OA.out)
                        -- ++(0.2,0)
                            node[ocirc, label=right:$v_o$]{}
                    %% Biasing is probably not needed here.
                    %(OA.up)
                    %    -- ++(0,0.8)
                    %        node[ocirc, label=above:$V_\text{CC}$]{}
                    %(OA.down)
                    %    -- ++(0,-0.8)
                    %        node[ocirc, label=below:$V_\text{EE}$]{}
                    (OA.out)
                        ++(1.2,1.4)
                        node[]{$\displaystyle \boxed{v_o = A \parens{v_+ - v_-}}$}
                ;
                \draw[stealth-, ultra thick, extranotecolor, color=extranotecolor]
                    (OA.-)
                        ++(0,0.4)
                        -- ++(-0.7,0.5)
                        %node[label={above:{\footnotesize \textsc{Inverting Input}}}]{}
                        node[above, align=center, font={\footnotesize \sc}]{Inverting\\Input}
                ;
                \draw[stealth-, ultra thick, extranotecolor, color=extranotecolor]
                    (OA.+)
                        ++(0,-0.4)
                        -- ++(-0.7,-0.5)
                        %node[label={below:[align=left]{\footnotesize \textsc{Non-Inverting\\Input}}}]{}
                        node[below, align=center, font={\footnotesize \sc}]{Non-Inverting\\Input}
                ;
            \end{circuitikz}
        \end{center}

        \MulticolsBreak

        \begin{center}
            \myul{\textbf{\footnotesize Equivalent Model}}

            \begin{circuitikz}
                \draw 
                    (0,0)
                            node[ocirc, label=left:$v_+$]{}
                        to[short, i=$i_+$] ++(2,0)
                            coordinate(TR1)
                        to[R, l=$R_i$, v=$v_i$] ++(0,-2)
                        to[short, i<=$i_-$] ++(-2,0)
                            node[ocirc, label=left:$v_-$]{}
                    (TR1) ++(2,0)
                        to[cvsource, l=$A v_i$] ++(0,-2)
                            node[ground, scale=2]{}
                    (TR1) ++(2,0)
                        to[R, l=$R_o$] ++(2,0)
                        node[ocirc, label=right:$v_o$]{}
                ;
            \end{circuitikz}
        \end{center}

        \MulticolsCleanEnd
    \end{multicols}%
    \MulticolsReduceVspaceAfter%
    \begin{multicols}{2}
        DC-coupled, high-gain differential-input amplifier.

        \myul{\textbf{Golden rules}}:
        \begin{enumerate}
            \item $v_+ = v_-$ \\ %{\footnotesize (\emph{virtual ground})}
                \emph{\footnotesize (``virtual ground"; appies in \myul{negative feedback} only)}
            \item $i_+ = i_- = 0$
        \end{enumerate}

        %\emph{(Golden rule \#1 is also known as ``virtual ground".)}

        \MulticolsBreak

        \myul{\textbf{Some properties}}:
        \begin{enumerate}
            \item $R_i \to \infty$
            \item $R_o = 0$
            \item $A \to \infty$
            \item infinite CMRR \Exn{(common-mode rejection ratio)}
            \item infinite bandwidth
        \end{enumerate}

        \MulticolsCleanEnd
    \end{multicols}
    \MulticolsReduceVspaceAfter%
\end{CheatsheetEntryFrame}

\begin{multicols}{2}
    \begin{CheatsheetEntryFrame}
        \CheatsheetEntryTitle{Opamp DC Imperfection Modeling}
        \bigskip

        \begin{center}
            \begin{circuitikz}
                \draw 
                    (0,0)
                        node[op amp, anchor=-](OA){}
                    (OA.-)
                        -- ++(-0.5,0)
                        -- ++(-2,0)
                            node[ocirc]{}
                    (OA.+)
                        -- ++(-0.5,0)
                        to[V, invert, color=red, l=$\color{red} V_\text{io}$] ++(-2,0)
                            node[ocirc]{}
                    (OA.out)
                        -- ++(0.2,0)
                            node[ocirc]{}
                ;
                \draw[red, color=red]
                    (OA.-)
                        ++(-0.25,0)
                        node[currarrow, label=above:$I_{\text{B}-}$]{}
                    (OA.+)
                        ++(-0.25,0)
                        node[currarrow, label=above:$I_{\text{B}+}$]{}
                ;
            \end{circuitikz}
        \end{center}

        Currents usually specified in terms of $I_\text{B}$ and $I_\text{io}$:
        \begin{gather*}
            \begin{gathered}
                %\overbrace{
                I_\text{B} = \frac{I_{\text{B}+} + I_{\text{B}-}}{2}
                %}^\text{input bias current}
                \\
                %\underbrace{
                I_\text{io} = I_{\text{B}+} - I_{\text{B}-}
                %}_\text{input offset current}
            \end{gathered}
            \qquad
            \quad
            \begin{aligned}
                I_{\text{B}+} &= I_\text{B} + \frac{1}{2} I_\text{io} \\
                I_{\text{B}-} &= I_\text{B} - \frac{1}{2} I_\text{io}
            \end{aligned}
        \end{gather*}

        Calculate worst-case output offset voltage of a circuit using superposition principle:
        \begin{gather*}
            V_\text{os}
            = \abs*{
                V_\text{os} \parens{I_{\text{B}+}}
                + V_\text{os} \parens{I_{\text{B}-}}
            }
            + \abs*{
                V_\text{os} \parens{V_\text{io}}
            }
        \end{gather*}

        \Exn{
            Terminology:
            \begin{pindent}
                $I_\text{B} = \text{input bias current}$ \\
                $I_\text{io} = \text{input offset current}$ \\
                $V_\text{io} = \text{input offset voltage}$
            \end{pindent}
        }
    \end{CheatsheetEntryFrame}

    \begin{CheatsheetEntryFrame}
        \CheatsheetEntryTitle{Opamp Large Signal Characteristics}

        \begin{enumerate}
            \item Output Saturation
            \item Output Current Limit
            \item Slew Rate Limit
        \end{enumerate}
        %\begin{enumerate}
        %    \setcounter{enumi}{1}
        %    \item Output Current Limit
        %\end{enumerate}
        %\begin{enumerate}
        %    \setcounter{enumi}{2}
        %    \item Slew Rate Limit
        %\end{enumerate}

        \Todo{Expand on this?}
    \end{CheatsheetEntryFrame}

\end{multicols}

\newpage

\begin{CheatsheetEntryFrame}

    \newcommand{\TmpOpampConfigRow}[3]{
        %\vspace*{1ex}
        \begin{minipage}[c]{0.40\textwidth}
            \MinipageInheritDocumentFormatting
            \myul{#1}

            #2
        \end{minipage}%
        \begin{minipage}[c]{0.60\textwidth}
        \begin{center}
            \begin{circuitikz}
                #3
            \end{circuitikz}
        \end{center}
        \end{minipage}%
        \vspace*{1ex}
    }
    \newcommand{\TmpOpampSeparator}{
        \bigskip
        {\color{lightgray} \hrule{}}
        \bigskip
    }

    \CheatsheetEntryTitle{Common Opamp Configurations}

    \TmpOpampSeparator

    \begin{minipage}[t]{0.46\textwidth}
        \MinipageInheritDocumentFormatting
        \myul{Inverting Amplifier}

        \begin{center}
        %\scalebox{0.8}{
        \begin{circuitikz}
            \draw 
                % Op-Amp
                (0,0)
                    node[op amp, noinv input down, anchor=out](OA){}
                (OA.+)
                    -| ++(-0.4,-0.5)
                        coordinate(GND)
                    \MyGround{}
                (OA.-)
                    -- ++(-0.4,0)
                        coordinate(A)
                (OA.out)
                    -- ++(0.3,0)
                        coordinate(C)
                    -- ++(0.8,0)
                        node[ocirc, label=right:$v_o$]{}

                % Left Side
                (A)
                    to[R, l_=$R_1$, red, color=red] ++(-3,0)
                        node[ocirc, label=left:$v_i$]{}

                % Feedback
                (C)
                    -- ++(0,1.8)
                    to[R, l=$R_2$, blue, color=blue] ++(-2.8,0)
                    -| (A)

                % Equation
                (OA.+)
                    ++(-2.6,0.2)
                        node[below]{$
                            \displaystyle
                            \boxed{\frac{v_o}{v_i} = \frac{- \color{blue} R_2}{ \color{red} R_1}}
                        $}
            ;
        \end{circuitikz}
        %}
        \end{center}
    \end{minipage}%
    \SoftVSep%
    \begin{minipage}[t]{0.46\textwidth}
        \MinipageInheritDocumentFormatting
        \myul{Non-Inverting Amplifier}

        \begin{center}
        \begin{circuitikz}
            \draw 
                % Op-Amp
                (0,0)
                    node[op amp, noinv input down, anchor=out](OA){}
                (OA.+)
                    -- ++(-0.4,0)
                        node[ocirc, label=left:$v_i$]{}
                        coordinate(GND)
                (OA.-)
                    -- ++(-0.4,0)
                        coordinate(A)
                (OA.out)
                    -- ++(0.3,0)
                        coordinate(C)
                    -- ++(0.8,0)
                        node[ocirc, label=right:$v_o$]{}

                % Left Side
                (A)
                    to[R, red, color=red, l_=$R_1$] ++(-3,0)
                    \MyGround{}

                % Feedback
                (C)
                    -- ++(0,1.8)
                    to[R, blue, color=blue, l=$R_2$] ++(-2.8,0)
                    -| (A)

                % Equation
                (OA.+)
                    ++(-2.6,-0.2)
                        node[below]{$
                            \displaystyle
                            \boxed{\frac{v_o}{v_i} = 1 + \frac{\color{blue} R_2}{\color{red} R_1}}
                        $}
            ;
        \end{circuitikz}
        \end{center}
    \end{minipage}%
    \TmpOpampSeparator

    \begin{minipage}[t]{0.46\textwidth}
        \MinipageInheritDocumentFormatting
        \myul{Unity-Gain Buffer}
        \bigskip

        \begin{center}
        \begin{circuitikz}
            \draw 
                % Op-Amp
                (0,0)
                    node[op amp, noinv input down, anchor=out](OA){}
                (OA.+)
                    -- ++(-0.4,0)
                        node[ocirc, label=left:$v_i$]{}
                        coordinate(GND)
                (OA.-)
                    -- ++(-0.4,0)
                        coordinate(A)
                (OA.out)
                    -- ++(0.3,0)
                        coordinate(C)
                    -- ++(0.8,0)
                        node[ocirc, label=right:$v_o$]{}

                % Feedback
                (C)
                    |- ++(-2.8,1.5)
                    -| (A)

                % Equation
                (OA.out)
                    ++(0.7,-0.9)
                        node[]{$\displaystyle \boxed{v_o = v_i}$}
            ;
        \end{circuitikz}
        \end{center}

        \bigskip
        Useful for its high input impedance and low output impedance.
    \end{minipage}%
    \SoftVSep%
    \begin{minipage}[t]{0.46\textwidth}
        \MinipageInheritDocumentFormatting
        \myul{Differentiator}
        \begin{center}
        \begin{circuitikz}
            \draw 
                % Op-Amp
                (0,0)
                    node[op amp, noinv input down, anchor=out](OA){}
                (OA.+)
                    -| ++(-0.4,-0.5)
                        coordinate(GND)
                    \MyGround{}
                (OA.-)
                    -- ++(-0.4,0)
                        coordinate(A)
                (OA.out)
                    -- ++(0.3,0)
                        coordinate(C)
                    -- ++(0.8,0)
                        node[ocirc, label=right:$v_o$]{}

                % Left Side
                (A)
                    to[C, red, color=red, l_=$C$] ++(-3,0)
                        node[ocirc, label=left:$v_i$]{}

                % Feedback
                (C)
                    -- ++(0,1.8)
                    to[R, blue, color=blue, l_=$R$] ++(-2.8,0)
                    -| (A)

                % Equation
                (OA.+)
                    ++(-2.6,0.2)
                        node[below]{$
                            \displaystyle
                            \boxed{
                            \begin{gathered}
                                \frac{v_o}{\color{blue} R} + {\color{red} C} \frac{\diff{v_i}}{\diff{t}} = 0
                                \\
                                v_o = - {\color{red} C} {\color{blue} R} \frac{\diff{v_i}}{\diff{t}}
                            \end{gathered}
                            }
                        $}
            ;
        \end{circuitikz}
        \end{center}
    \end{minipage}%
    \TmpOpampSeparator

    \TmpOpampConfigRow{Integrator}{
        Without {\color{darkgreen}$R_m$}:
        \begin{gather*}
            \boxed{
                \begin{gathered}
                    \frac{v_i}{\color{red} R}
                    + {\color{blue} C} \frac{\diff{v_o}}{\diff{t}} = 0
                    \\
                    v_o = \frac{-1}{\color{blue} C \color{red} R} \int{v_i \,\diff{t}}
                \end{gathered}
            }
        \end{gather*}

        {\footnotesize
            {\color{darkgreen}$R_m$} is \myul{optional}, but without it, practical implementations quickly saturate (since $v_i$ is never perfectly zero).
        }

        {\footnotesize
            To select {\color{darkgreen}$R_m$}:
        }
        \begin{gather*}
            \boxed{
                \begin{gathered}
                    \abs*{\frac{1}{j \omega \color{blue} C}} \gg {\color{darkgreen} R_m}
                \end{gathered}
            }
        \end{gather*}

        {\footnotesize \Todo{I'm not 100\% sure how $R_m$ works.} }
    }{
        \draw 
            % Op-Amp
            (0,0)
                node[op amp, noinv input down, anchor=out](OA){}
            (OA.+)
                -| ++(-0.4,-0.5)
                    coordinate(GND)
                \MyGround{}
            (OA.-)
                -- ++(-0.4,0)
                    coordinate(A)
            (OA.out)
                -- ++(0.3,0)
                    coordinate(C)
                -- ++(0.8,0)
                    node[ocirc, label=right:$v_o$]{}

            % Left Side
            (A)
                to[R, red, color=red, l_=$R$] ++(-3,0)
                    node[ocirc, label=left:$v_i$]{}

            % Feedback
            (C)
                -- ++(0,1.8)
                    coordinate(D)
                to[C, blue, color=blue, l=$C$] ++(-2.8,0)
                -| (A)

            %% Equation
            %(OA.out)
            %    ++(2.6,2)
            %        node[]{$
            %            \displaystyle
            %            \boxed{
            %                \begin{gathered}
            %                    \abs*{\frac{1}{j \omega C}} \gg R_m
            %                \end{gathered}
            %            }
            %        $}
        ;
        \draw[darkgreen, color=darkgreen]
            % Extra Resistor
            (D)
                -- ++(0,1)
                to[R, l_=$R_m$] ++(-2.8,0)
                -| (D -| A)
        ;
    }
    \TmpOpampSeparator

    \TmpOpampConfigRow{Difference Amplifier}{
        \begin{gather*}
            \boxed{
                v_o = \frac{\color{blue} R_2}{\color{red} R_1} \parens{v_1 - v_2}
            }
        \end{gather*}

        However, the input impedance can be too low for practical use.

        To solve this, we use an \emph{instrumentational amplifier} instead.
    }{
        \draw 
            % Op-Amp
            (0,0)
                node[op amp, noinv input down, anchor=out](OA){}
            (OA.+)
                -- ++(-0.4,0)
                    coordinate(D)
                to[R, blue, color=blue, l_=$R_2$] ++(0,-1.8)
                    coordinate(GND)
                \MyGround{}
            (OA.-)
                -- ++(-0.4,0)
                    coordinate(A)
            (OA.out)
                -- ++(0.3,0)
                    coordinate(C)
                -- ++(0.8,0)
                    node[ocirc, label=right:$v_o$]{}

            % Left Side
            (A)
                to[R, red, color=red, l_=$R_1$] ++(-3,0)
                    node[ocirc, label=left:$v_2$]{}
            (D)
                to[R, red, color=red, l_=$R_1$] ++(-3,0)
                    node[ocirc, label=left:$v_1$]{}

            % Feedback
            (C)
                -- ++(0,1.8)
                to[R, blue, color=blue, l=$R_2$] ++(-2.8,0)
                -| (A)

            %% Equation
            %(OA.out)
            %    ++(3,1.6)
            %        node[]{$\displaystyle \boxed{v_o = \frac{R_2}{R_1} \parens{v_1 - v_2}}$}
        ;
    }
    \TmpOpampSeparator

    \Todo{Write a section for Instrumentational Amplifier}
    
    \Todo{Maybe also add the summing amplifier?}

\end{CheatsheetEntryFrame}

\Todo{Add notes on frequency response?}

