\subsection{Extra Cheatsheets: Miscellaneous}%
\label{sub:extrasecmath-cheatsheet-misc}

\begin{multicols}{2}

    \begin{CheatsheetEntryFrame}

        \CheatsheetEntryTitle{Polynomial}

        A function $f : \mathbb{R} \to \mathbb{R}$ is a \textit{polynomial} of degree $n$ if:
        \begin{gather*}
            f(x) = a_n x^n + a_{n-1} x^{n-1} + \dots + a_2 x^2 + a_1 x + a_0 \\
            \forall x \in \mathbb{R} %\\
            %\intertext{where:}
            %n \in \mathbb{N}, \\
            %a_0, a_1, \dots, a_{n-1}, a_n \in \mathbb{R}, \\
            %a_n \ne 0.
        \end{gather*}
        where:
        \begin{center}
        \begin{tabular}{lc}
            degree: & $n = 0, 1, 2, 3, \dots$ \\
            coefficients: & $a_0, a_1, \dots, a_{n-1}, a_n \in \mathbb{R}$ \\
            leading coefficient: & $a_n \ne 0$ \\
        \end{tabular}
        \end{center}
        \vspace{\parskip}

        The most common polynomials are named:
        \begin{center}
        \begin{tabular}{|c|l|c|}
            \cline{2-3}
            \multicolumn{1}{c|}{} & Name & \multicolumn{1}{l|}{Form} \\ \hline
            Degree $0$ & \textit{constant}  & $a$ \\
            Degree $1$ & \textit{linear}    & $a x   + b$ \\
            Degree $2$ & \textit{quadratic} & $a x^2 + b x   + c$ \\
            Degree $3$ & \textit{cubic}     & $a x^3 + b x^2 + c x   + d$ \\
            Degree $4$ & \textit{quartic}   & $a x^4 + b x^3 + c x^2 + d x + e$ \\ \hline
        \end{tabular}
        \end{center}
        \vspace{\parskip}

        A \textit{monic polynomial} is one where the leading coefficient is $1$.

        \CheatsheetEntryExtraSeparation

        \CheatsheetEntryTitle{Rational Function}

        A function $f$ is a rational function if it can be written in the form:
        \begin{equation*}
            f(x) = \frac{P(x)}{Q(x)}
        \end{equation*}
        where $P$ and $Q$ are polynomial functions, and $Q$ is not the zero function.

        The domain of $f$ excludes zeroes of the denominator:
        \begin{equation*}
            \dom{(f)} = \braces*{x \in \mathbb{R} : Q(x) \ne 0}
        \end{equation*}

    \end{CheatsheetEntryFrame}

    \MulticolsBreak

    \begin{CheatsheetEntryFrame}

        \textit{The fundamental theorem of calculus is so useful that it absolutely doesn't need to be in the main cheatsheet. This must be known intuitively!}

        \CheatsheetEntryExtraSeparation

        \CheatsheetEntryTitle{First Fundamental Theorem of Calculus:}
        
        Let $f$ be a continuous real-valued function defined on $[a, b]$, and let $F$ be defined by:
        \begin{equation*}
            F : [a, b] \to \mathbb{R}
                \qquad F(x) = \int_a^x{f(t) \,\diff{t}} .
        \end{equation*}

        $F$ is continuous on $[a, b]$, differentiable on $(a, b)$, and has a derivative $F'$ given by:
        \begin{equation*}
            F'(x) = f(x)
                \qquad \forall x \in (a, b) .
        \end{equation*}

        \CheatsheetEntryTitle{Second Fundamental Theorem of Calculus:}

        Let $F$ be a real-valued function on $[a, b]$, and $F$ be an antiderivative of $f$ on $[a, b]$. Then:
        \begin{equation*}
            \int_a^b{f(t) \,\diff{t}} = F(b) - F(a)
        \end{equation*}

        \Todo{Look into this more. I'm not 100\% certain on details such as whether the second fundamental theorem requires $f$ to be continuous, and whether or not $f$ being Riemann integrable is significant here.}

    \end{CheatsheetEntryFrame}

\end{multicols}

%%%%%%%%%%%%%%%%%%%%%%%%%%%%%%%%%%%%%%%%%%%%%%%%%%%%%%%%%%%%%%%%%%%%%%%%%%%%%%%%%%%%%%%%%%%%%%%%%%%%
%%%%%%%%%%%%%%%%%%%%%%%%%%%%%%%%%%%%%%%%%%%%%%%%%%%%%%%%%%%%%%%%%%%%%%%%%%%%%%%%%%%%%%%%%%%%%%%%%%%%
%%%%%%%%%%%%%%%%%%%%%%%%%%%%%%%%%%%%%%%%%%%%%%%%%%%%%%%%%%%%%%%%%%%%%%%%%%%%%%%%%%%%%%%%%%%%%%%%%%%%

\newpage
\subsection{Extended Discussion: Miscellaneous}%
\label{sub:extrasecmath-disc-misc}

\begin{multicols}{2}

    \begin{CheatsheetEntryFrame}

        \CheatsheetEntryTitle{Why are monics favoured in this cheatsheet?}

        The non-monic case usually makes forms really dirty to look at, while monics are much cleaner.
        
        For instance, let's first take a look at the monic forms on \textit{quadratic factorization} and \textit{completing the square}:
        \begin{gather*}
            x^2 + bx + c = \parens*{x + v} \parens*{x + u}
                \qquad\quad
                \begin{array}{c}
                    b = u+v \\
                    c = uv
                \end{array}
                \\
            x^2 + bx + c = \parens*{x + \brackets*{\frac{b}{2}}}^2 + c - \brackets*{\frac{b^2}{4}}
        \end{gather*}

        Now, let's look at their non-monic forms:
        \begin{gather*}
            \memphR{a} x^2 + bx + c = \memphR{\frac{1}{a}} \parens*{\memphR{a} x + v} \parens*{\memphR{a} x + u}
                \qquad\quad
                \begin{array}{c}
                    b = u+v \\
                    \memphR{a} c = uv
                \end{array}
                \\
            \memphR{a} x^2 + bx + c = \memphR{a} \parens*{x + \brackets*{\frac{b}{2\memphR{a}}}}^2 + \brackets*{c - \memphR{a} \brackets*{\frac{b}{2 \memphR{a}}}}
        \end{gather*}

        The addition of $\memphR{a}$ for non-monic forms makes them more difficult to remember, while the monic forms have much clearer patterns.

        Though, the monic forms do come with the disadvantage of requiring an extra step to get a monic out of a non-monic:
        \begin{equation*}
            \underbrace{a x^2 + bx + c}_{\text{non-monic}}
                = a \underbrace{\parens*{
                    x^2 + \brackets*{\frac{b}{a}} x + \brackets*{\vphantom{\frac{b}{a}} \frac{c}{a}}
                }}_{\text{monic}}
        \end{equation*}

        For me personally, I would rather remember less information, so this small extra step is worth it.

    \end{CheatsheetEntryFrame}
    
    \begin{CheatsheetEntryFrame}

        \CheatsheetEntryTitle{Completing the Square} {\scriptsize \textsc{(Extended Discussion)}}

        Suppose we have a quadratic function:
        \begin{equation*}
            f(x) = a x^2 + b x + c
        \end{equation*}

        We want to express the same quadratic in the form:
        \begin{equation*}
            f(x) = A(x + B)^2 + C
        \end{equation*}

        We can expand this form, then equate coefficients:
        \begin{gather*}
            f(x) = A x^2 + 2ABx + A B^2 + C \\[\EqExtraSkip]
            A = a \qquad 2AB = b \qquad A B^2 + C = c
        \end{gather*}

        Solving for $A$, $B$, and $C$, we get our function:
        \begin{equation*}
            f(x) = a \parens*{x + \brackets*{\frac{b}{2a}}}^2 + \brackets*{c - \frac{b^2}{4a}}
        \end{equation*}

        Thus, we have our general form for completing the square.

        However, it can be a bit unwieldy. If we instead assume we have a monic (i.e. $a=1$), then:
        \begin{equation*}
            f(x) = \parens*{x + \xmemphR{\brackets*{\frac{b}{2}}}}^2 + c - \xmemphP{\brackets*{\frac{b}{2}}^2}
        \end{equation*}

        This form because easier to memorize since you just remember where $\memphR{\frac{b}{2}}$ and {\color{mypurple} its square} appears.

        Though, if one forgets {\color{mypurple} the square}, you can guess the form of the outside by expanding $\parens*{x + \memphR{\frac{b}{2}}}^2$:
        \begin{align*}
            f(x)
                &= \parens*{x + \xmemphR{\brackets*{\frac{b}{2}}}}^2 + c + \text{something} \\
                &= \parens*{x^2 + 2 \xmemphR{\brackets*{\frac{b}{2}}} + \xmemphP{\brackets*{\frac{b}{2}}^2}} + c + \text{something} \\
            \intertext{This \textit{something} must be such that the constant doesn't change:}
                &= \parens*{x^2 + 2 \xmemphR{\brackets*{\frac{b}{2}}} + \xmemphP{\brackets*{\frac{b}{2}}^2}} + c - \xmemphP{\brackets*{\frac{b}{2}}^2}
        \end{align*}
        

    \end{CheatsheetEntryFrame}

\end{multicols}
