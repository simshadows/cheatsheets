\subsubsection{Power}

\begin{multicols}{2}

    \begin{CheatsheetEntryFrame}

        \CheatsheetEntryTitle{AC Power in the Time Domain}
        
        For some arbitrary load (potentially both resistive and reactive) where:
        \begin{equation*}
            \begin{alignedat}{2}
                v\parens*{t} &= V_m &&\cos\parens*{\omega t + \theta_v}
                \\
                i\parens*{t} &= I_m &&\cos\parens*{\omega t + \theta_i}
            \end{alignedat}
            \qquad\qquad
            \theta = \theta_v - \theta_i
        \end{equation*}
        The instantaneous power can be expressed as:
        \begin{align*}
            p\parens*{t}
            = v\parens*{t} \, i\parens*{t}
            % Fraction with empty numerator. Makes for slightly better vertical space.
            %\vphantom{\frac{}{2}}
            &= \frac{1}{2} V_m I_m \cos\theta
            \, \memphR{[\underbrace{
                \vphantom{\frac{1}{2}}
                1 + \cos{(2 \omega t + 2 \theta_v)}
            }_{\textbf{average value} = 1}]}
            \\
            &\qquad + \frac{1}{2} V_m I_m \sin\theta
            \, \memphB{\underbrace{
                \vphantom{\frac{1}{2}}
                \sin{(2 \omega t + 2 \theta_v)}
            }_{\textbf{average value} = 0}}
            %% Original version below:
            %&= \underbrace{\frac{1}{2} V_m I_m \cos{(\theta)}}_{\text{constant}}
            %+ \underbrace{\frac{1}{2} V_m I_m \cos{(2 \omega t + 2 \theta_v - \theta)}}_{\text{average value} = 0}
        \end{align*}
        %\Exn{\scriptsize \textit{Rarely applied directly, but useful for understanding and context.}}

    \end{CheatsheetEntryFrame}

    \begin{CheatsheetEntryFrame}

        \CheatsheetEntryTitle{AC Power}

        \emph{Power angle} ($\theta$) is the current lag (relative to voltage).
        \begin{center}
        \begin{tikzpicture}[x=2.0cm, y=2.0cm, transform shape]
            \path (0,0) coordinate (Origin);
            \draw[-stealth, line width=1.7pt] (0:0.5) arc (0:-45:0.5);
            \draw (-22.5:0.35) node {$\theta$};
            \draw[-stealth, mypurple, line width=2.0pt, line cap=round] (0,0) -- ++(-45:1) coordinate (IEnd);
            \draw[-stealth, myred,    line width=2.0pt, line cap=round] (0,0) -- ++(  0:1) coordinate (VEnd);
            \draw
                (VEnd)
                ++(0.0,0) node[right, color=myred] {
                    $\mathbf{V}$
                }
                %++(1.6,0) node[align=center, color=myred, font=\scriptsize] {
                %    \textbf{(assuming voltage as reference)} %\\
                %    %\textbf{($\theta_v = 0$ assumed)}
                %}

                %(0.5,0.05) node[above, align=center, color=myred, font=\scriptsize] {\textbf{REFERENCE}}
                
                (IEnd)
                ++(0,-0.05) node[below, color=mypurple] {$\mathbf{I}$}
            ;
            \draw[angle 60 reversed-angle 60, mygreen, line width=1.7pt, line cap=round]
                ( 10:1.6) arc ( 10: 35:1.6) node[above] {\textbf{$\bm{-}$ve $\bm{\theta}$}}
            ;
            \draw[angle 60 reversed-angle 60, myblue,  line width=1.7pt, line cap=round]
                (-10:1.6) arc (-10:-35:1.6) node[below] {\textbf{$\bm{+}$ve $\bm{\theta}$}}
            ;
            \draw
                ( 15:1.7) ++(1.1, 0.00) coordinate (LeadLabel)
                node[above, align=left, color=mygreen, font=\scriptsize] {
                    \textbf{current \ul{leads} the voltage} \\
                    \textbf{$\quad \implies$ \ul{leading} power factor} \\
                    \textbf{$\quad \implies$ \ul{capacitive} load}
                }

                %(LeadLabel |- Origin)
                %node[align=center] {
                %    $\theta = \theta_v - \theta_i$ \\%[0.75ex]
                %    {\scriptsize ($\theta$ is the current lag angle.)} %\\
                %    %$\text{PF} = \cos{\theta}$
                %}

                (-15:1.7) ++(1.1,-0.00)
                node[below, align=left, color=myblue, font=\scriptsize] {
                    \textbf{current \ul{lags} the voltage} \\
                    \textbf{$\quad \implies$ \ul{lagging} power factor} \\
                    \textbf{$\quad \implies$ \ul{inductive} load}
                }
            ;
            \draw
                (0.9,0.4)
                    node[label=above left:$
                        %\begin{gathered}
                        %    \boxed{\text{pf} = \cos \theta}
                        %    \\
                            \boxed{\theta = \theta_v - \theta_i}
                        %\end{gathered}
                    $]{}
            ;
        \end{tikzpicture}
        \end{center}
        \vspace*{-2ex}

        \begin{alignat*}{2}
            &\text{\parbox{3.5em}{\centering\footnotesize{}Power\\Factor}}
            &\qquad\quad \text{pf}
            &= \cos\theta
            = \frac{P}{S}
            \qquad \text{\parbox{7.5em}{\raggedright\footnotesize{}(must express as \emph{leading} or \emph{lagging})}}
            \\[0.7em]
            &\text{\parbox{3.5em}{\centering\footnotesize{}Average\\Power}}
            &\qquad\quad P
            &= \frac{1}{2} V_m I_m \cos{\theta}
            = V_{\text{rms}} I_{\text{rms}} \cos{\theta}
            \\[0.7em]
            &\text{\parbox{3.5em}{\centering\footnotesize{}Reactive\\Power}}
            &\qquad\quad Q
            &= \frac{1}{2} V_m I_m \sin{\theta}
            = V_{\text{rms}} I_{\text{rms}} \sin{\theta}
            \\[0.7em]
            &\text{\parbox{3.5em}{\centering\footnotesize{}Apparent\\Power}}
            &\qquad\quad S
            &= \frac{1}{2} V_m I_m
            = V_{\text{rms}} I_{\text{rms}}
            \\[0.7em]
            &\text{\parbox{3.5em}{\centering\footnotesize{}Complex\\Power}}
            &\qquad\quad \mathbf{S}
            &= \frac{1}{2} \mathbf{V} \mathbf{I}^*
            = P + j Q
            = \frac{1}{2} V_m I_m \phase{\theta}
            \\
            &&
            &= \frac{1}{2} V_m I_m \brackets*{\cos{\theta} + j \sin{\theta}}
            %= \mathbf{V}_{\text{rms}}^{\phantom{*}} \mathbf{I}_{\text{rms}}^* % meh, not important
            %= V_{\text{rms}} I_{\text{rms}} \phase{\theta} % Probably not necessary
        \end{alignat*}

        %{\scriptsize%
        %\begin{tabular}{lcccc}
        %    Inductive Loads
        %        & $\implies$
        %        %& $\phantom{x} \implies \phantom{x}$ % Ghetto Alignment
        %        & $+$ve $\theta$
        %        & $\implies$
        %        %& $\phantom{x} \implies \phantom{x}$ % Ghetto Alignment
        %        & $+$ve $Q$
        %        \\
        %    Capacitive Loads
        %        & $\implies$
        %        & $-$ve $\theta$
        %        & $\implies$
        %        & $-$ve $Q$
        %        \\
        %\end{tabular}
        %}

    \end{CheatsheetEntryFrame}

    \begin{CheatsheetEntryFrame}

        \CheatsheetEntryTitle{Conservation of AC Power}
        \begin{gather*}
            \sum{\mathbf{S}_k} = 0
            \qquad\quad
            \sum{P_k} = 0
            \qquad\quad
            \sum{Q_k} = 0
        \end{gather*}
        {\footnotesize{}\emph{Note: This does NOT hold for apparent power.}}

    \end{CheatsheetEntryFrame}

    \MulticolsBreak

    \begin{CheatsheetEntryFrame}

        \CheatsheetEntryTitle{RMS Value}

        The \textit{RMS} (or effective) value of \myul{any periodic voltage/current} is the equivalent value in DC to deliver the same average power to a load.
        \begin{equation*}
            F_\text{rms} =\sqrt{\frac{1}{T} \int_0^T{\parens*{f(t)}^2 \,\diff{t}}} 
        \end{equation*}

        For sinusoidal waveforms:

        \begin{minipage}{0.5\columnwidth}%
            \begin{equation*}
                V_\text{rms} = \frac{V_m}{\sqrt{2}} \approx 0.707 \, V_m
            \end{equation*}
        \end{minipage}%
        \begin{minipage}{0.5\columnwidth}%
            \begin{equation*}
                I_\text{rms} = \frac{I_m}{\sqrt{2}} \approx 0.707 \, I_m
            \end{equation*}
        \end{minipage}%

        %% I'm not 100% sure on this one. Will need to learn more!
        %% TODO: Look into this section!
        %\vspace{\parskip}%
        %For a waveform produced from a sum of waveforms:
        %\begin{gather*}
        %    f(t) = f_1(t) + f_2(t) + \dots + f_n(t) \\
        %    F_\text{rms}^2 = F_{\text{rms}1}^2 + F_{\text{rms}2}^2 + \dots + F_{\text{rms}n}^2 
        %\end{gather*}

        \CheatsheetEntryExtraSeparation

    \end{CheatsheetEntryFrame}

    \begin{CheatsheetEntryFrame}

        \CheatsheetEntryTitle{Power Triangle}

        \vspace{-3.5ex}%
        \begin{center}
        \begin{tikzpicture}[x=2.0cm, y=2.0cm, transform shape]
            \path
                (0,0) -- (-0.2,0) % Ghetto alignment

                (0,0) coordinate (Origin)
                (30:2.0) coordinate (TopCorner)
                (TopCorner |- Origin) coordinate (BottomCorner)

                (Origin)       -- coordinate[midway] (SMid) (TopCorner)
                (Origin)       -- coordinate[midway] (PMid) (BottomCorner)
                (BottomCorner) -- coordinate[midway] (QMid) (TopCorner)

                (BottomCorner) ++(1.0, 0   ) coordinate (ArrowsMiddle)
                (ArrowsMiddle) ++(0  , 0.15) coordinate (TopArrowStart)
                (ArrowsMiddle) ++(0  ,-0.15) coordinate (BottomArrowStart)
            ;
            \draw[-stealth, line width=3.0pt, line cap=round]
                (Origin) -- (TopCorner);
            \draw[-stealth, line width=1.7pt]
                (0:0.6) arc (0:30:0.6);
            \draw (15:0.45) node {$\theta$};
            \draw[-stealth, color=magenta, line width=3.0pt, line cap=round]
                (BottomCorner) -- (TopCorner);
            \draw[-stealth, color=blue, line width=3.0pt, line cap=round]
                (Origin) -- (BottomCorner);
            \draw[angle 60 reversed-angle 60, mygreen, line width=2.0pt, line cap=round]
                (BottomArrowStart) -- ++(0,-0.8) node[below, align=left] {
                    \textbf{$\bm{-}$ve $\bm{\theta}$} \\
                    \textbf{$\bm{-}$ve $\bm{Q}$}
                }
            ;
            \draw[angle 60 reversed-angle 60, myblue,  line width=2.0pt, line cap=round]
                (TopArrowStart) -- ++(0,0.8) node[above, align=left] {
                    \textbf{$\bm{+}$ve $\bm{\theta}$} \\
                    \textbf{$\bm{+}$ve $\bm{Q}$}
                }
            ;
            \draw[mygreen] (BottomArrowStart) ++(0.7,-0.4) node[align=center] {%
                \textbf{Capacitive}\\\textbf{Load}%
            };
            \draw[myblue] (TopArrowStart) ++(0.7, 0.4) node[align=center] {%
                \textbf{Inductive}\\\textbf{Load}%
            };
            \draw
                (SMid)
                ++(120:0.16) node {$\mathbf{S}$}
                %++(120:0.25) node[align=center, rotate=30, font=\scriptsize] {%
                %    \textsc{Complex Power}%
                %} % and Apparent Power?
            ;
            \draw[blue]
                (PMid)
                ++(-90:0.16) node {$P$}
                %++(-90:0.22) node[align=center, rotate=0, font=\scriptsize] {%
                %    \textsc{Average Power}%
                %}
            ;
            \draw[magenta]
                (QMid)
                ++(  0:0.16) node {$Q$}
                %++(  0:0.25) node[align=center, rotate=-90, font=\scriptsize] {%
                %    \textsc{Reactive Power}%
                %}
            ;
            \draw
                (BottomCorner)
                    ++(-0.7,-0.3)
                        node[label=below:{$
                            \boxed{
                                \hphantom{x} % Ghetto Whitespace
                                \begin{aligned}
                                    \vphantom{\sqrt{P^2}} \mathbf{S}
                                    &= P + jQ
                                    \\
                                    \vphantom{\sqrt{P^2}} S
                                    &= \abs*{\mathbf{S}} = \sqrt{P^2 + Q^2}
                                    \\
                                    \vphantom{\sqrt{P^2}} P
                                    &= \MyRe{\mathbf{S}} = S \cos{\theta}
                                    \\
                                    \vphantom{\sqrt{P^2}} Q
                                    &= \MyIm{\mathbf{S}} = S \sin{\theta}
                                \end{aligned}
                                \hphantom{x} % Ghetto Whitespace
                            }
                        $}]{}
            ;
        \end{tikzpicture}
        \end{center}

    \end{CheatsheetEntryFrame}

    \begin{CheatsheetEntryFrame}

        \CheatsheetEntryTitle{Maximum Average Power Transfer}
        \smallskip

        \begin{minipage}[c]{0.6\columnwidth}
            \centering%
            \begin{circuitikz}
                \draw
                    (0,0)
                        to[V, l=$\mathbf{V}_{th}$, invert] (0,1.6)
                        to[generic, l=$\mathbf{Z}_{th}$, -o] ++(2.2,0)
                        to[short]     ++( 0.4,0)
                        to[generic, l=$\mathbf{Z}_L$] ++(0,-1.6)
                        to[short]     ++(-0.4,0)
                        to[short, o-]     (0,0)
                ;
            \end{circuitikz}
        \end{minipage}%
        \begin{minipage}[c]{0.4\columnwidth}
            \centering%
            Maximum average power is transferred to $\mathbf{Z}_L$ if:
            \begin{equation*}
                \mathbf{Z}_L = \mathbf{Z}_{th}^*
            \end{equation*}
        \end{minipage}

    \end{CheatsheetEntryFrame}

    \MulticolsCleanEnd
\end{multicols}%
\begin{multicols}{2}

    \begin{CheatsheetEntryFrame}

        \CheatsheetEntryTitle{Power Delivered to a Load Impedance}
        %
        %Power delivered to impedance $\mathbf{Z} = R + jX$:
        \begin{center}
        \begin{circuitikz}
            \draw
                (0,0)
                to[V, l=$\mathbf{V}$, invert] ++(0,4.4)
                to[short, i=$\mathbf{I}$] ++(1.7,0)
                -- ++(0,-0.3)
                to[R, l=$R$, v=$\mathbf{V}_R$] ++(0,-1.8)
                -- ++(0,-0.1)
                    coordinate (TMP_Mid)
                -- ++(0,-0.1)
                to[generic, l=$jX$, v=$\mathbf{V}_X$] ++(0,-1.8)
                -- ++(0,-0.3)
                -- (0,0)
            ;
            \draw
                (TMP_Mid)
                    ++(1.1,0)
                    node[label=right:{$
                        \begin{gathered}
                        \begin{aligned}
                            P &= I_{\text{rms}}^2 R \\
                            Q &= I_{\text{rms}}^2 X
                        \end{aligned}
                        \\[3ex]
                        \begin{aligned}
                            P
                            &= \frac{V_{\text{Rrms}}^2}{R}
                            = V_{\text{rms}}^2 \frac{R}{R^2 + X^2}
                            \\[2ex]
                            Q
                            &= \frac{V_{\text{Xrms}}^2}{X}
                            = V_{\text{rms}}^2 \frac{X}{R^2 + X^2}
                        \end{aligned}
                        \end{gathered}
                    $}]{}
            ;
        \end{circuitikz}
        \end{center}

        %\CheatsheetEntryExtraSeparation

        %\CheatsheetEntryTitle{Power Delivered to a Load Admittance}
        %% This section is still being worked on.

        %\Exn{\scriptsize \textit{(Hint: These expressions can be found by converting $R$ and $X$ to $G$ and $B$ from the previous section.)}}

        %Power delivered to admittance $\mathbf{Y} = G + jB$:

        %%\TwoColumnsMinipages[0.275]{
        %\TwoColumnsMinipages[0.38]{
        %    \raggedright
        %    \begin{circuitikz}
        %        \draw
        %            (0,0)
        %            -- ++(0,0.3)
        %            to[V, l=$\mathbf{V}$, invert] ++(0,1.8)
        %            -- ++(0,0.3)
        %            to[short, i=$\mathbf{I}$] ++(1,0) coordinate (YTL)
        %            -- ++(1,0) coordinate (GTop)
        %            -- ++(1.2,0)
        %            to[generic, l=$jB$, i>_=$\mathbf{I}_B$] ++(0,-2.4) coordinate (YBR)
        %            -- ++(-1.2,0) coordinate (GBottom)
        %            -- (0,0)

        %            (GTop)
        %            to[generic, l=$G$, i>_=$\mathbf{I}_G$] (GBottom)
        %        ;
        %        \begin{scope}[on background layer]
        %            \draw[lightgray, fill=verylightgray, line width=3.0pt]
        %                (YTL) ++(0, 0.4) coordinate (BoxTL)
        %                (YBR) ++(1,-0.4) coordinate (BoxBR)
        %                (BoxBR) rectangle (BoxTL)
        %            ;
        %            \path
        %                (BoxTL -| BoxBR) -- coordinate[midway] (BoxMR) (BoxBR) % Calculate midright point
        %            ;
        %            \draw
        %                (BoxMR) ++(0.1,0) node[right] {$\mathbf{Y}$}
        %            ;
        %            % Below is the old version of the impedance label.
        %            %\draw
        %            %    (BoxTR) ++(0.2,0) coordinate (BoxTRRef)
        %            %    (BoxTRRef |- ZBottom) coordinate (BoxBRRef)
        %            %;
        %            %\draw
        %            %    (BoxTRRef)
        %            %    -- ++(0.3,0) coordinate (ZLabelHorizontalRef)
        %            %    |- (BoxBRRef)

        %            %    (ZLabelHorizontalRef |- ZMid) ++(0.1,0) node[right] {$\mathbf{Z}$}
        %            %;
        %        \end{scope}
        %    \end{circuitikz}%
        %}{%
        %    \phantom{x}
        %    \begin{align*}
        %        P &= \phantom{-} V_{\text{rms}}^2 G \\
        %        Q &= -V_{\text{rms}}^2 B
        %    \end{align*}
        %    %\phantom{x} \\[0mm] % Ghetto alignment
        %    %\phantom{x} \\[0mm]
        %    %\phantom{x} \\[0mm]
        %    \phantom{x}
        %}%
        %\begin{alignat*}{2}
        %    P &= \frac{I_{\text{Rrms}}^2}{R} & \qquad\quad
        %        Q &= \frac{I_{\text{Xrms}}^2}{X} \\
        %    &= I_{\text{rms}}^2 \frac{G}{G^2 + B^2} & \qquad\quad
        %        &= I_{\text{rms}}^2 \frac{B}{G^2 + B^2}
        %\end{alignat*}

        %%\begin{align*}
        %%    P &= \phantom{-} V_{\text{rms}}^2 G \\
        %%    Q &= -V_{\text{rms}}^2 B
        %%\end{align*}
        %%\begin{align*}
        %%    P &= TODO \\
        %%    &= V_{\text{rms}}^2 \frac{R}{R^2 + X^2} \\[2ex]
        %%    Q &= TODO \\
        %%    &= V_{\text{rms}}^2 \frac{X}{R^2 + X^2}
        %%\end{align*}

        %\CheatsheetEntryExtraSeparation

    \end{CheatsheetEntryFrame}

    \begin{CheatsheetEntryFrame}

        \CheatsheetEntryTitle{Parseval's Theorem}
        \begin{equation*}
            W_{\qty{1}{\ohm}}
            = \int_{-\infty}^\infty f^2\parens*{t} \,\diff{t}
            = \frac{1}{2 \pi} \int_{-\infty}^\infty{
                \abs*{F\parens*{\omega}}^2 \,\diff{\omega}
            }
        \end{equation*}
        where $F\parens*{\omega} = \FourierTransform\braces*{f\parens*{t}}$ (a Fourier transform).

        \Todo{What does this mean? What are the implications?}

    \end{CheatsheetEntryFrame}

    \begin{CheatsheetEntryFrame}

        \CheatsheetEntryTitle{Power Factor Correction}

        \Todo{Write notes on this?}

    \end{CheatsheetEntryFrame}

    \begin{CheatsheetEntryFrame}

        \CheatsheetEntryTitle{Power Gain (in Decibels)}

        \begin{center}
        \begin{circuitikz}
            \path
                (0,0)     coordinate (BoxBLref)
                ++(2.0,0) coordinate (BoxBRref)

                (BoxBLref)            ++(0, -0.5) coordinate (BoxBL)
                (BoxBRref) ++(0, 1.5) ++(0,  0.5) coordinate (BoxTR)
            ;
            \draw
                (BoxBLref)
                to[short, -o] ++(-1.2,0)
                ++(-1.0, 0)
                to[open, v_<=$V_i$] ++(0, 1.5)
                ++(1.0, 0)
                to[short, o-, i=$I_i$] ++( 1.2,0)

                (BoxBRref)
                -- ++( 1.2,0)
                to[R, l=$R_l$] ++(0, 1.5)
                to[short, i<_=$I_o$] ++(-1.2,0)

                (BoxBRref)
                ++( 1.2,0)
                ++(1.0, 0)
                to[open, v<=$V_o$] ++(0, 1.5)
                ++(-1.0, 0)
            ;
            \draw
                (BoxBL) rectangle (BoxTR)
            ;
            %\draw
            %    ($(BoxTR)!0.5!(BoxBL)$) coordinate (BoxMid)
            %    (BoxMid) node[align=center] {some label}
            %;
            \draw[stealth-, line width=2.0]
                (BoxBLref)
                ++(-0.3,0.5)
                -- ++(-0.7,0) node[midway,above] {$R_i$}
            ;
        \end{circuitikz}
        \end{center}

        \begin{align*}
            G &= 10 \log_{10}\frac{P_o}{P_i} \,\unit{\decibel} \\
            &= \parens*{20 \log_{10}\frac{V_o}{V_i} - 10 \log_{10}\frac{R_l}{R_i}} \,\unit{\decibel} \\
            &= \parens*{20 \log_{10}\frac{I_o}{I_i} + 10 \log_{10}\frac{R_l}{R_i}} \,\unit{\decibel}
        \end{align*}

        If $R_l = R_i$ is assumed, then:
        \begin{equation*}
            G
            = 20 \log_{10}\frac{V_o}{V_i} \,\unit{\decibel}
            = 20 \log_{10}\frac{I_o}{I_i} \,\unit{\decibel}
        \end{equation*}

        \Todo{This section will need to be verified, and I need to find more information on when the ``$20 \log_{10}$" form is used.}

    \end{CheatsheetEntryFrame}

\end{multicols}

