\subsection{Complex Analysis}%
\label{sub:complex-analysis}

\begin{multicols}{2}

    \begin{CheatsheetEntryFrame}

        \CheatsheetEntryTitle{Imaginary Unit}
        \begin{equation*}
            i^2 = -1
        \end{equation*}
        %\begin{tabularx}{\textwidth}{CC}
        %    $i^2 = -1$ & $\underset{\substack{\text{(alternative notation for}\\\text{electrical engineering}\\\text{and some other fields)}}}{j^2 = -1}$ \\
        %\end{tabularx}

        Alternative notation for electrical engineering (and some other fields):
        \begin{equation*}
            j^2 = -1
        \end{equation*}

        \CheatsheetEntryTitle{Complex Number Representation}

        The same complex number $z$ can be written:
        \begin{alignat*}{2}
            z
                &= a + ib &
                    \qquad & \text{(Rectangular Form)} \\
                &= r(\cos{\theta} + i \sin{\theta}) &
                    \qquad & \text{(Polar Form)} \\
                &= r e^{i \theta} &
                    \qquad & \text{(Exponential Form)} \\
                &= r \phase{\theta} &
                    \qquad & \text{(Steinmetz Notation)} \\
                &\Exn{= r \cis{\theta},} &
                    \qquad & \Exn{\text{(cis Form) [rarely used]}}
        \end{alignat*}%
        where $a, b, r, \theta \in \mathbb{R}$.

        %The same complex number $z$ can be written:\\[\abovedisplayskip]
        %\begin{center}
        %\begin{tabular}{lc}
        %    Rectangular Form & $a + ib$ \\
        %    Polar Form & $r(\cos{\theta} + i \sin{\theta})$ \\
        %    Exponential Form & $r e^{i \theta}$ \\
        %    Steinmetz Notation \phantom{x} & $r \phase{\theta}$ \\
        %    \Exn{$\cis$ Form {\scriptsize (rarely used)}} & \Exn{$r \cis{\theta}$} \\
        %\end{tabular}
        %\end{center}

        Components:
        \begin{alignat*}{3}
            & \text{Real Part:}            & \MyRe{(z)}  &= a      &&= r \cos{\theta} \phantom{\frac{}{a}} \\
            & \text{Imaginary Part:} \quad & \MyIm{(z)}  &= b      &&= r \sin{\theta} \phantom{\frac{b}{a}} \\
            & \text{Modulus:}              & \abs*{z}    &= r      &&= \sqrt{a^2 + b^2} \phantom{\frac{b}{a}} \\
            & \text{Argument:}             & \arg{(z)}   &= \theta &&= \tan^{-1}{\parens*{\frac{b}{a}}}
        \end{alignat*}

        \CheatsheetEntryTitle{Euler's Formula}
        \begin{equation*}
            e^{i \theta} = \cos{\theta} + i \sin{\theta}, \qquad \forall \theta \in \mathbb{R}
        \end{equation*}

    \end{CheatsheetEntryFrame}

    \begin{CheatsheetEntryFrame}

        \CheatsheetEntryTitle{Complex Conjugate}
        \begin{equation*}
            \complexconjugate{a + ib} = a - ib
        \end{equation*}

        Alternative notation for physics and engineering:
        \begin{equation*}
            \parens*{a + ib}^* = a - ib
        \end{equation*}

        \CheatsheetEntryTitle{Complex Conjugate: $z \complexconjugate{z}$}
        \begin{gather*}
            z \complexconjugate{z} = \abs*{z}^2 \\
            \parens*{a + ib} \complexconjugate{\parens*{a + ib}} = \parens*{a + ib} \parens*{a - ib} = a^2 + b^2
        \end{gather*}
        
    \end{CheatsheetEntryFrame}

    \MulticolsBreak

    \MulticolsPhantomPlaceholder

\end{multicols}

