\subsection{Elementary Algebra and Combinatorics}%
\label{sub:algebra}

\begin{multicols}{2}

    %% These are the Exponentiation and Log Identities sections in the original separated form.
    %
    %\begin{CheatsheetEntryFrame}
    %
    %    \CheatsheetEntryTitle{Exponentiation Identities}
    %    \begin{align*}
    %        a^x a^y &= a^{x+y} \\
    %        a^x b^x &= \parens*{ab}^x \\
    %        \parens*{a^x}^y &= a^{xy} \\
    %        a^0 &= 1
    %        %
    %        \\[\abovedisplayskip]
    %        %
    %        a^{-x} &= \frac{1}{a^x} \\
    %        a^{x-y} &= \frac{a^x}{a^y} \\
    %        \parens*{\frac{a}{b}}^x &= \frac{a^x}{b^x}
    %        %
    %        \\[\abovedisplayskip]
    %        %
    %        a^{\frac{1}{2}} &= \sqrt{a} \\
    %        a^{\frac{1}{x}} &= \sqrt[x]{a} \\
    %        a^{\frac{x}{y}} &= \parens*{\sqrt[y]{a}}^x
    %    \end{align*}
    %
    %\end{CheatsheetEntryFrame}
    %
    %\begin{CheatsheetEntryFrame}
    %
    %    \CheatsheetEntryTitle{Log Identities}
    %
    %    \CheatsheetSmallEquationTitle{(Definition)}
    %    \begin{equation*}
    %        x = a^y \quad\iff\quad \log_a{x} = y
    %    \end{equation*}
    %
    %    \CheatsheetSmallEquationTitle{(Inverses)}
    %    \begin{equation*}
    %        \log_a{a^x} = a^{\log_a{x}} = x
    %    \end{equation*}
    %
    %    \CheatsheetSmallEquationTitle{(Change of Base Law)}
    %    \begin{equation*}
    %        \log_a{x} = \frac{\log_u{x}}{\log_u{a}}
    %    \end{equation*}
    %
    %    \vspace*{-\abovedisplayskip}
    %    \begin{align*}
    %        \log_a{xy} &= \log_a{x} + \log_a{y} \\
    %        \log_a{\parens*{\frac{x}{y}}} &= \log_a{x} - \log_a{y} \\
    %        \log_a{x^n} &= n \log_a{x}
    %        %
    %        \\[\abovedisplayskip]
    %        %
    %        \log_a{1} &= 0 \\
    %        \log_a{a} &= 1
    %        %
    %        \\[\abovedisplayskip]
    %        %
    %        \log_a{\parens*{\frac{1}{x}}} &= - \log_a{x}
    %        %
    %        \\[\abovedisplayskip]
    %        %
    %        \log_a{\parens*{\frac{1}{a}}} &= -1 \\
    %        \log_a{\sqrt{a}} &= \frac{1}{2} \\
    %        \log_a{\sqrt[b]{a}} &= \frac{1}{b}
    %    \end{align*}
    %
    %\end{CheatsheetEntryFrame}

    \begin{CheatsheetEntryFrame}

        \CheatsheetEntryTitle{Factorial}
        \begin{gather*}
            n!
            = \prod_{k = 1}^n{k}
            = 1 \times 2 \times 3 \times \dots \times (n-2) \times (n-1) \times n,
            \\
            \forall n \in \mathbb{Z} : n > 0.
            \qquad\qquad
            0! = 1
        \end{gather*}

    \end{CheatsheetEntryFrame}

    \begin{CheatsheetEntryFrame}

        \CheatsheetEntryTitle{Exponentiation and Logarithm Identities}

        % Hacks to ensure uniform height
        \renewcommand{\W}{\displaystyle} % Simple
        \newcommand{\X}{\displaystyle \vphantom{\parens*{\sqrt[\frac{I}{I}]{I}}}} % No fractions
        \newcommand{\Y}{\displaystyle \vphantom{\frac{I}{I}}} % Simple fractions
        \newcommand{\Z}{\displaystyle \vphantom{\parens*{\parens*{\frac{I}{I}}^I}}} % Complicated fractions


        \begin{center}
        \begin{tabularx}{\textwidth}{Ccc}

            \hline
            \multicolumn{3}{|l|}{{\scriptsize \textbf{Logarithm Definition}}}
                \\
            \multicolumn{1}{|c}{$\X x = a^y$}
                & $\X \iff$
                & \multicolumn{1}{c|}{$\X \log_a{x} = y$}
                \\
            \hline

            && %%%%%%%%%%%%%%%%%%%%%%%%%%%%%%%%%% GAP
                \\ %%%%%%%%%%%%%%%%%%%%%%%%%%%%%% GAP

            $a^{\log_a{x}} = x$
                & $\X \iff$
                & $\X \log_a{a^x} = x$
                \\

            && %%%%%%%%%%%%%%%%%%%%%%%%%%%%%%%%%% GAP
                \\ %%%%%%%%%%%%%%%%%%%%%%%%%%%%%% GAP

            {} % BLANK
                &
                & $\W \overset{\text{\textbf{Change of Base Law}}}{\boxed{\Z \log_a{x} = \frac{\log_u{x}}{\log_u{a}}}}$
                \\

            && %%%%%%%%%%%%%%%%%%%%%%%%%%%%%%%%%% GAP
                \\ %%%%%%%%%%%%%%%%%%%%%%%%%%%%%% GAP

            $\W
                        \begin{aligned}
                            \X a^0 &= 1 \\
                            \X a^1 &= a \\
                            \Y a^{-1} &= \frac{1}{a}
                        \end{aligned}
            $
                &
                    $\W
                        \begin{gathered}
                            \X \longrightarrow \\
                            \X \longrightarrow \\
                            \Y \longrightarrow
                        \end{gathered}
                    $
                &
                    $\W
                        \begin{aligned}
                            \X \log_a{1} &= 0 \\
                            \X \log_a{a} &= 1 \\
                            \Y \log_a{\frac{1}{a}} &= -1
                        \end{aligned}
                    $
                \\

            && %%%%%%%%%%%%%%%%%%%%%%%%%%%%%%%%%% GAP
                \\ %%%%%%%%%%%%%%%%%%%%%%%%%%%%%% GAP

            $\W
                        \begin{aligned}
                            \X a^x a^y         &= a^{x+y} \\
                            \X a^x b^x         &= \parens*{ab}^x \\
                            \X \parens*{a^x}^y &= a^{xy} \\
                        \end{aligned}
            $
                &
                    $\W
                        \begin{gathered}
                            \X \longrightarrow \\
                            \X \\
                            \X \longrightarrow
                        \end{gathered}
                    $
                &
                    $\W
                        \begin{aligned}
                            \X \log_a{xy} &= \log_a{x} + \log_a{y} \\
                            \X & \\
                            \X \log_a{x^n} &= n \log_a{x}
                        \end{aligned}
                    $
                \\

            && %%%%%%%%%%%%%%%%%%%%%%%%%%%%%%%%%% GAP
                \\ %%%%%%%%%%%%%%%%%%%%%%%%%%%%%% GAP

            $\W
                        \begin{aligned}
                            \Z \frac{1}{a^y}   &= a^{-y} \\
                            \Z \frac{a^x}{a^y} &= a^{x-y} \\
                            \Z \frac{a^x}{b^x} &= \parens*{\frac{a}{b}}^x
                        \end{aligned}
            $
                &
                    $\W
                        \begin{gathered}
                            \Z \\
                            \Z \longrightarrow \\
                            \Z
                        \end{gathered}
                    $
                &
                    $\W
                        \begin{aligned}
                            \Z \log_a{\frac{1}{y}} &= -\log_a{y} \\
                            \Z \log_a{\frac{x}{y}} &= \log_a{x} - \log_a{y} \\
                            \Z &
                        \end{aligned}
                    $
                \\

            && %%%%%%%%%%%%%%%%%%%%%%%%%%%%%%%%%% GAP
                \\ %%%%%%%%%%%%%%%%%%%%%%%%%%%%%% GAP

            $\W
                        \begin{aligned}
                            \Y a^{\frac{1}{2}} &= \sqrt{a} \\
                            \Y a^{\frac{1}{y}} &= \sqrt[y]{a} \\
                            \Y a^{\frac{x}{y}} &= \parens*{\sqrt[y]{a}}^x
                        \end{aligned}
            $
                &
                    $\W
                        \begin{gathered}
                            \Y \longrightarrow \\
                            \Y \longrightarrow \\
                            \Y \longrightarrow
                        \end{gathered}
                    $
                &
                    $\W
                        \begin{alignedat}{2}
                            \Y & \log_a{\sqrt{a}}                &&= \frac{1}{2} \\
                            \Y & \log_a{\sqrt[y]{a}}             &&= \frac{1}{y} \\
                            \Y & \log_a{\parens*{\sqrt[y]{a}}^x} &&= \frac{x}{y}
                        \end{alignedat}
                    $
                \\

        \end{tabularx}
        \end{center}

    \end{CheatsheetEntryFrame}

    \begin{CheatsheetEntryFrame}

        \CheatsheetEntryTitle{Pascal's Triangle}
        \begin{equation*}
            \begin{matrix}
                n = 0
                &&
                1
                &&
                \binom{0}{0}
                \\
                n = 1
                &&
                1 \quad 1
                &&
                \binom{1}{0}
                \:\: \binom{1}{1}
                \\
                n = 2
                &&
                1 \quad 2 \quad 1
                &&
                \binom{2}{0}
                \:\: \binom{2}{1}
                \:\: \binom{2}{2}
                \\
                n = 3
                &&
                1 \quad 3 \quad 3 \quad 1
                &&
                \binom{3}{0}
                \:\: \binom{3}{1}
                \:\: \binom{3}{2}
                \:\: \binom{3}{3}
                \\
                n = 4
                &&
                1 \quad 4 \quad 6 \quad 4 \quad 1
                &&
                \binom{4}{0}
                \:\: \binom{4}{1}
                \:\: \binom{4}{2}
                \:\: \binom{4}{3}
                \:\: \binom{4}{4}
            \end{matrix}
        \end{equation*}

        \CheatsheetEntryTitle{$k$-Permutations of $n$}
        \begin{gather*}
            \MyPermutations{n}{k}
            = \frac{n!}{\parens*{n - k}!}
        \end{gather*}

        \CheatsheetEntryTitle{$k$-Combinations of $n$ / Binomial Coefficient}
        \begin{gather*}
            \MyCombinations{n}{k}
            = \binom{n}{k}
            = \frac{n!}{k! \, \parens*{n - k}!}
        \end{gather*}

    \end{CheatsheetEntryFrame}

    \begin{CheatsheetEntryFrame}

        \CheatsheetEntryTitle{Binomial Theorem}
        \begin{align*}
            \parens*{x + y}^n
            &= \binom{n}{0} x^n y^0
            + \binom{n}{1} x^{n-1} y^1
            + \dots
            + \binom{n}{n} x^0 y^n
            \\
            &= \sum_{k=0}^n \binom{n}{k} x^{n-k} y^k
            = \sum_{k=0}^n \binom{n}{k} x^k y^{n-k}
        \end{align*}

    \end{CheatsheetEntryFrame}

    \begin{CheatsheetEntryFrame}

        \CheatsheetEntryTitle{Quadratic Formula}
        %\begin{gather*}
        %    x = \frac{-b \pm \sqrt{b^2 - 4ac}}{2a} \\
        %    \intertext{Discriminant:}
        %    \Delta = b^2 - 4ac
        %\end{gather*}
        \begin{gather*}
            x = \frac{-b \pm \sqrt{b^2 - 4ac}}{2a}
            \qquad\quad
            \MathOverLabel{\text{\ul{discriminant}}}{\Delta = b^2 - 4ac}
        \end{gather*}

        \CheatsheetEntryTitle{Quadratic Factorization}
        \begin{equation*}
            x^2 + bx + c = \parens*{x+v} \parens*{x+u}
            \qquad\quad
            %\MathOverLabel{\text{\ul{sums and products}}}{ % Too much space. Not worth it.
                \begin{array}{c}
                    b = u+v \\
                    c = uv
                \end{array}
            %}
        \end{equation*}
        \vspace{-2ex} % Not sure why there's so much whitespace

        \CheatsheetEntryTitle{Completing The Square}
        \begin{equation*}
            x^2 + b x + c
            = \parens*{x + \xmemphR{\brackets*{\memphR{\frac{b}{2}}}}}^2
            + c
            - \xmemphR{\brackets*{\memphR{\frac{b}{2}}}^2}
        \end{equation*}

    \end{CheatsheetEntryFrame}

    \begin{CheatsheetEntryFrame}

        \CheatsheetEntryTitle{Quadratic/Cubic Identities}
        \begin{align*}
            a^2 - b^2 &= (a + b) (a - b) \\
            a^3 - b^3 &= (a - b) (a^2 + ab + b^2) \\
            a^3 + b^3 &= (a + b) (a^2 - ab + b^2)
        \end{align*}

    \end{CheatsheetEntryFrame}

\end{multicols}
\newpage

\begin{CheatsheetEntryFrame}

    \CheatsheetEntryTitle{Partial Fraction Decomposition}

    A \textit{proper rational function} can be rewritten as a sum of \textit{partial fractions}.

    For each irreducible factor in the denominator, the partial fractions are as follows:
    \renewcommand{\W}{\displaystyle}
    \begin{equation*}
        \begin{array}{ccccc}
            \substack{\text{\ul{irreducible factor in denominator}}} &&&&
                \substack{\text{\ul{partial fractions}}}
                \\[1ex]
            \W {(\memphR{ax + b})}^k &
                & \Longrightarrow & \quad &
                %\W \sum_{n=1}^k{\frac{A_n}{\memphR{(ax + b)}^n}}
                \W \frac{A_1}{\memphR{ax + b}}
                + \frac{A_2}{\parens*{\memphR{ax + b}}^2}
                + \dots
                + \frac{A_k}{\parens*{\memphR{ax + b}}^k}
                \\[3ex]
            \W {(\memphR{ax^2 + bx + c})}^k &
                & \Longrightarrow & \quad &
                %\W \sum_{n=1}^k{\frac{A_n x + B_n}{\memphR{(ax^2 + bx + c)}^n}}
                \W \frac{A_1 x + B_1}{\memphR{ax^2 + bx + c}}
                + \frac{A_2 x + B_2}{\parens*{\memphR{ax^2 + bx + c}}^2}
                + \dots
                + \frac{A_k x + B_k}{\parens*{\memphR{ax^2 + bx + c}}^k}
        \end{array}
    \end{equation*}

    For \emph{improper rational functions}, you must first convert it to a \emph{proper rational function}.

    \emph{Hint: You can use complex numbers to further reduce some factors. Example: $\parens*{1 + x^2} = \parens*{1 + i x} \parens*{1 - i x}$}

\end{CheatsheetEntryFrame}

\begin{multicols}{2}

    \begin{CheatsheetEntryFrame}

        \CheatsheetEntryTitle{todo}

    \end{CheatsheetEntryFrame}

    \MulticolsBreak

    \MulticolsPhantomPlaceholder

\end{multicols}

