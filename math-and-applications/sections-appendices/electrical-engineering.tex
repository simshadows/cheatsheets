\subsection{Further Discussion: Inductors and Inductive Coupling}%
\label{sub:further-discussion-inductors}

\begin{multicols}{2}

    \begin{CheatsheetEntryFrame}

        \CheatsheetEntryTitle{Single Inductor}

        Assuming a single inductor:
        %% Old Version
        %\begin{alignat*}{2}
        %    v &=
        %        N \frac{\diff{\Phi_B}}{\diff{t}} &
        %        \qquad & \text{Faraday's Law}
        %        \\
        %    &=
        %        N \frac{\diff{\Phi_B}}{\diff{i}} \frac{\diff{i}}{\diff{t}} &
        %        \qquad & \text{flux $\Phi_B$ is produced by current $i$}
        %        \\
        %    &=
        %        L \frac{\diff{i}}{\diff{t}} &
        %        \qquad & \text{inductance is change in flux over change in current}
        %\end{alignat*}
        \vspace{-1ex}%
        \begin{center}
        \begin{circuitikz}
            \draw
                (0,0)
                to[short, i=$i$, o-] (2,0)
                to[cute inductor, v=$v$, name=L] (4,0)
                to[short, -o] (6,0)
            ;
            \begin{scope}[on background layer]
                \draw[myorange!25!white, line width=1.5pt]
                    (L.center) ++(0, 0.40) ++(0, 0.02) circle[x radius=1.5, y radius=0.40] % Upper Big
                    (L.center) ++(0,-0.40) ++(0,-0.02) circle[x radius=1.5, y radius=0.40] % Lower Big
                    (L.center) ++(0, 0.25) ++(0, 0.06) circle[x radius=1.0, y radius=0.25] % Upper Small
                    (L.center) ++(0,-0.25) ++(0,-0.06) circle[x radius=1.0, y radius=0.25] % Lower Small
                ;
                \draw
                    (L.center) ++(1.2,0.95) node {$\phi$}
                ;
            \end{scope}
        \end{circuitikz}
        \end{center}
        \begin{enumerate}[label=(\arabic*)]
            \item By \textit{Faraday's law}, a change in magnetic flux $\phi$ through a coil of $N$ turns induces an EMF of $v$.
            \item Current $i$ produces flux $\phi$, so we introduce $\diff{i}$.
            \item We define inductance $L$ as the flux produced per current. {\scriptsize \textit{($L$ is determined by inductor architecture.)}}
        \end{enumerate}
        \begin{equation*}
            \overbracket{v = N \frac{\diff{\phi}}{\diff{t}}}^{\text{(1)}}
            = \overbracket{N \frac{\diff{\phi}}{\diff{i}} \frac{\diff{i}}{\diff{t}}}^{\text{(2)}}
            = \overbracket{L \frac{\diff{i}}{\diff{t}}}^{\text{(3)}}
        \end{equation*}

        Hence, the current-voltage characteristic for self-induction:
        \begin{equation*}
            v = L \frac{\diff{i}}{\diff{t}}
        \end{equation*}

    \end{CheatsheetEntryFrame}

    \renewcommand{\W}{25} % Controls field line opacity

    \begin{CheatsheetEntryFrameStart}

        \CheatsheetEntryTitle{Two Magnetically Coupled Inductors}

        Consider two inductors $L_1$ and $L_2$ that are coupled:
        \begin{center}
        \begin{circuitikz}
            \draw % Left Side
                (0,0)
                to[short, i=$i_1$, o-] ++(1,0) coordinate (L1Top)
                to[cute inductor, l_=$L_1$, name=L1] ++(0,-2)
                to[short, -o] ++(-1,0)
                to[open, v^<=$v_1$] (0,0)
            ;
            \draw % Right Side
                (L1Top) ++(1,0) coordinate (L2Top)
                to[short, i<=$i_2$, -o] ++(1,0)
                to[open, v^>=$v_2$] ++(0,-2)
                to[short, o-] ++(-1,0)
                to[cute inductor, l_=$L_2$, name=L2] (L2Top)
            ;
            \path
                (L1Top) -- coordinate[midway] (TmpPoint) (L2Top)
                (TmpPoint) ++(0,0.1) coordinate (MArcCenter)
            ;
            \draw[simshadows/style/ee/magneticcouplingarrow]
                (MArcCenter) ++(0:0.5) arc (0:180:0.5)
            ;
            \draw
                (MArcCenter) ++(90:0.5) node[above] {$M$}
            ;
            \draw
                (L1Top) ++(-0.20,-0.35) node[circ] {}
                (L2Top) ++( 0.20,-0.35) node[circ] {}
            ;
        \end{circuitikz}
        \end{center}

        Inductor $L_1$ has $N_1$ turns in its coil. \\[0mm]
        Inductor $L_2$ has $N_2$ turns in its coil.

        Current $\memphR{i_1}$ through $\memphR{L_1}$ produces magnetic flux $\memphR{\phi_1}$: \\[0mm]
        {\scriptsize \textit{($\memphB{\phi_2}$ is similar.)}}

        \TwoColumnsMinipages{
            \begin{circuitikz}
                \draw % Left Side
                    (0,0)
                    to[short, i=$\memphR{i_1}$, o-] ++(1,0) coordinate (L1Top)
                    to[cute inductor, l_=$\memphR{L_1}$, name=L1] ++(0,-2)
                    to[short, -o] ++(-1,0)
                ;
                \draw % Right Side
                    (L1Top) ++(1,0) coordinate (L2Top)

                    (L2Top)
                    to[short, -o] ++(1,0)
                    
                    (L2Top)
                    to[cute inductor, mirror, l=$L_2$, name=L2] ++(0,-2)
                    to[short, -o] ++(1,0)
                ;
                \draw
                    (L1Top) ++(-0.20,-0.35) node[circ] {}
                    (L2Top) ++( 0.20,-0.35) node[circ] {}
                ;
                \begin{scope}[on background layer]
                    \draw[myorange!\W!white, line width=1.5pt]
                        (L1.center) ++(-0.60,0) ++(-0.01,0) circle[x radius=0.60, y radius=2.0] % Left Large
                        (L1.center) ++( 0.60,0) ++( 0.01,0) circle[x radius=0.60, y radius=2.0] % Right Large
                        (L1.center) ++(-0.45,0) ++(-0.03,0) circle[x radius=0.45, y radius=1.5] % Left Medium
                        (L1.center) ++( 0.45,0) ++( 0.03,0) circle[x radius=0.45, y radius=1.5] % Right Medium
                        (L1.center) ++(-0.30,0) ++(-0.05,0) circle[x radius=0.30, y radius=1.0] % Left Small
                        (L1.center) ++( 0.30,0) ++( 0.05,0) circle[x radius=0.30, y radius=1.0] % Right Small
                    ;
                    \draw[myred]
                        (L1.center) ++(1.30,1.70) node {$\mathbf{\phi_1}$}
                    ;
                \end{scope}
            \end{circuitikz}
        }{
            \begin{circuitikz}
                \draw % Left Side
                    (0,0)
                    to[short, o-] ++(1,0) coordinate (L1Top)
                    to[cute inductor, l_=$L_1$, name=L1] ++(0,-2)
                    to[short, -o] ++(-1,0)
                ;
                \draw % Right Side
                    (L1Top) ++(1,0) coordinate (L2Top)

                    (L2Top)
                    to[short, i<=$\memphB{i_2}$, -o] ++(1,0)
                    
                    (L2Top)
                    to[cute inductor, mirror, l=$\memphB{L_2}$, name=L2] ++(0,-2)
                    to[short, -o] ++(1,0)
                ;
                \draw
                    (L1Top) ++(-0.20,-0.35) node[circ] {}
                    (L2Top) ++( 0.20,-0.35) node[circ] {}
                ;
                \begin{scope}[on background layer]
                    \draw[myteal!\W!white, line width=1.5pt]
                        (L2.center) ++(-0.60,0) ++(-0.01,0) circle[x radius=0.60, y radius=2.0] % Left Large
                        (L2.center) ++( 0.60,0) ++( 0.01,0) circle[x radius=0.60, y radius=2.0] % Right Large
                        (L2.center) ++(-0.45,0) ++(-0.03,0) circle[x radius=0.45, y radius=1.5] % Left Medium
                        (L2.center) ++( 0.45,0) ++( 0.03,0) circle[x radius=0.45, y radius=1.5] % Right Medium
                        (L2.center) ++(-0.30,0) ++(-0.05,0) circle[x radius=0.30, y radius=1.0] % Left Small
                        (L2.center) ++( 0.30,0) ++( 0.05,0) circle[x radius=0.30, y radius=1.0] % Right Small
                    ;
                    \draw[myblue]
                        (L2.center) ++(-1.30,1.70) node {$\mathbf{\phi_2}$}
                    ;
                \end{scope}
            \end{circuitikz}
        }

    \end{CheatsheetEntryFrameStart}

    \begin{CheatsheetEntryFrameMid}

        Some of the flux $\memphR{\phi_1}$ also links/passes through $\memphB{L_2}$. \\[0mm]
        We will call this $\memphR{\phi_{12}}$. \\[0mm]
        {\scriptsize \textit{(Read this as ``the flux due to 1 that links 2".)}} \\[0mm]
        {\scriptsize \textit{($\memphB{\phi_{21}}$ is similarly "the flux due to 2 that links 1.)}}

        \TwoColumnsMinipages{
            \begin{circuitikz}
                \draw % Left Side
                    (0,0)
                    to[short, i=$\memphR{i_1}$, o-] ++(1,0) coordinate (L1Top)
                    to[cute inductor, l_=$\memphR{L_1}$, name=L1] ++(0,-2)
                    to[short, -o] ++(-1,0)
                ;
                \draw % Right Side
                    (L1Top) ++(1,0) coordinate (L2Top)

                    (L2Top)
                    to[short, -o] ++(1,0)
                    
                    (L2Top)
                    to[cute inductor, mirror, l=$L_2$, name=L2] ++(0,-2)
                    to[short, -o] ++(1,0)
                ;
                \draw
                    (L1Top) ++(-0.20,-0.35) node[circ] {}
                    (L2Top) ++( 0.20,-0.35) node[circ] {}
                ;
                \begin{scope}[on background layer]
                    \draw[myorange!\W!white, line width=1.5pt]
                        (L1.center) ++( 0.50,0) circle[x radius=0.51, y radius=2.0] % Right Large
                        (L1.center) ++( 0.50,0) circle[x radius=0.44, y radius=1.5] % Right Medium
                        (L1.center) ++( 0.50,0) circle[x radius=0.37, y radius=1.0] % Right Small
                    ;
                    \draw[myred]
                        (L1.center) ++(1.20,1.65) node {$\mathbf{\phi_{12}}$}
                    ;
                \end{scope}
            \end{circuitikz}
        }{
            \begin{circuitikz}
                \draw % Left Side
                    (0,0)
                    to[short, o-] ++(1,0) coordinate (L1Top)
                    to[cute inductor, l_=$L_1$, name=L1] ++(0,-2)
                    to[short, -o] ++(-1,0)
                ;
                \draw % Right Side
                    (L1Top) ++(1,0) coordinate (L2Top)

                    (L2Top)
                    to[short, i<=$\memphB{i_2}$, -o] ++(1,0)
                    
                    (L2Top)
                    to[cute inductor, mirror, l=$\memphB{L_2}$, name=L2] ++(0,-2)
                    to[short, -o] ++(1,0)
                ;
                \draw
                    (L1Top) ++(-0.20,-0.35) node[circ] {}
                    (L2Top) ++( 0.20,-0.35) node[circ] {}
                ;
                \begin{scope}[on background layer]
                    \draw[myteal!\W!white, line width=1.5pt]
                        (L2.center) ++(-0.50,0) circle[x radius=0.51, y radius=2.0] % Right Large
                        (L2.center) ++(-0.50,0) circle[x radius=0.44, y radius=1.5] % Right Medium
                        (L2.center) ++(-0.50,0) circle[x radius=0.37, y radius=1.0] % Right Small
                    ;
                    \draw[myblue]
                    (L2.center) ++(-1.20,1.65) node {$\mathbf{\phi_{21}}$}
                    ;
                \end{scope}
            \end{circuitikz}
        }

        The total flux through each inductor is the sum of:
        \begin{itemize}
            \item flux due to current through itself, and
            \item flux due to current in the other inductor.
        \end{itemize}

        \begin{tabularx}{\textwidth}{CC}
            $\memphR{\phi_{L1}} = \memphR{\phi_1} + \memphB{\phi_{21}}$ &
            $\memphB{\phi_{L2}} = \memphB{\phi_2} + \memphR{\phi_{12}}$ \\
        \end{tabularx}

        By \textit{Faraday's law}, a change in magnetic flux induces an EMF in each coil ($\memphR{v_1}$ and $\memphB{v_2}$).

        %\vspace{\parskip}%
        \TwoColumnsMinipages[0.30]{
            \raggedright
            %EMF $\memphR{v_1}$ induced in coil $\memphR{L_1}$:
            \begin{align*}
                \memphR{v_1}
                    &= \memphR{N_1} \frac{\memphR{\diff{\phi_{L1}}}}{\diff{t}} \\
                    &= \memphR{N_1} \parens*{
                            \frac{\memphR{\diff{\phi_{1}}}}{\diff{t}}
                            + \frac{\memphB{\diff{\phi_{21}}}}{\diff{t}}
                        }\\
                    &= \memphR{N_1} \frac{\memphR{\diff{\phi_{1}}}}{\memphR{\diff{i_1}}} \frac{\memphR{\diff{i_1}}}{\diff{t}}
                        + \memphR{N_1} \frac{\memphB{\diff{\phi_{21}}}}{\memphB{\diff{i_2}}} \frac{\memphB{\diff{i_2}}}{\diff{t}} \\
                    &= \memphR{L_1} \frac{\memphR{\diff{i_1}}}{\diff{t}}
                        + \memphP{M_{12}} \frac{\memphB{\diff{i_2}}}{\diff{t}}
            \end{align*}
        }{
            \phantom{x} % Need to put something on this other minipage
        }

        \vspace{\parskip}%
        \TwoColumnsMinipages[-0.30]{
            \phantom{x} % Need to put something on this other minipage
        }{
            \raggedright
            {\scriptsize \textit{\Exn{SIMILARLY...}}} %\\[0mm]
            %EMF $\memphB{v_2}$ induced in coil $\memphB{L_2}$:
            \begin{align*}
                \memphB{v_2}
                    &= \memphB{N_2} \frac{\diff{\memphB{\phi_{L2}}}}{\diff{t}} \\
                    &= \memphB{N_2} \parens*{
                            \frac{\diff{\memphB{\phi_{2}}}}{\diff{t}}
                            + \frac{\diff{\memphR{\phi_{12}}}}{\diff{t}}
                        }\\
                    &= \memphB{N_2} \frac{\diff{\memphB{\phi_{2}}}}{\diff{\memphB{i_2}}} \frac{\diff{\memphB{i_2}}}{\diff{t}}
                        + \memphB{N_2} \frac{\diff{\memphR{\phi_{12}}}}{\diff{\memphR{i_1}}} \frac{\diff{\memphR{i_1}}}{\diff{t}} \\
                    &= \memphB{L_2} \frac{\diff{\memphB{i_2}}}{\diff{t}}
                        + \memphP{M_{21}} \frac{\diff{\memphR{i_1}}}{\diff{t}}
            \end{align*}
        }

        %$\memphR{L_1}$ and $\memphB{L_2}$ are each inductor's \textit{self-inductance}. \\[0mm]
        $\memphP{M_{12}}$ is the \textit{mutual inductance} of \temphR{coil 1} WRT. \temphB{coil 2}. \\[0mm]
        $\memphP{M_{21}}$ is the \textit{mutual inductance} of \temphB{coil 2} WRT. \temphR{coil 1}.

        \textit{(Note that these equations are only true for this configuration, indicated by the \ul{dot convention} in the circuit diagram. This will be explained later.)}

    \end{CheatsheetEntryFrameMid}

\end{multicols}
\begin{multicols}{2}

    \begin{CheatsheetEntryFrameEnd}

        The final step will be to show that:
        \begin{equation*}
            \memphP{M_{12}} = \memphP{M_{21}} = M
        \end{equation*}

        \Todo{Finish this! I'm feeling a bit lazy right now.}

    \end{CheatsheetEntryFrameEnd}

    \begin{CheatsheetEntryFrame}

        \CheatsheetEntryTitle{Coefficient of Coupling}

        Mutual induction $M$ can be expressed in terms of the coefficient of coupling $k$:
        \begin{equation*}
            M = k \sqrt{L_1 L_2}, \qquad 0 \le k \le 1
        \end{equation*}

        \Todo{Again, I'm feeling a bit lazy to explain the equation.}

    \end{CheatsheetEntryFrame}

    \MulticolsBreak

    \begin{CheatsheetEntryFrame}

        \CheatsheetEntryTitle{Single Inductor in the Frequency Domain}

        \begin{center}
        \begin{circuitikz}
            \draw
                (0,0) to[L, mirror, l_=$L$, i_<=$\mathbf{I}$, v^<=$\mathbf{V}$, o-o] ++(-3,0)
            ;
        \end{circuitikz}
        \end{center}

        Assume current:
        \begin{equation*}
            i = I_m \cos{(\omega t + \theta)}
            \quad \Longrightarrow \quad
            \frac{\diff{i}}{\diff{t}} = - \omega I_m \sin{(\omega t + \theta)}
        \end{equation*}

        Due to self-induction, we get voltage:
        \begin{align*}
            v
                &= L \frac{\diff{i}}{\diff{t}}
                = - \omega L I_m \sin{(\omega t + \theta)} \\
                &= \omega L I_m \cos{(\omega t + \theta + \ang{90})}
        \end{align*}

        Converting $v$ and $i$ to the frequency domain:
        \begin{equation*}
            \mathbf{I} = I_m \phase{\theta}
            \qquad \mathbf{V} = \omega L I_m \phase{\theta + \ang{90}}
        \end{equation*}

        Thus, we can express the impedance:
        \begin{align*}
            \mathbf{Z}
                &= \frac{\mathbf{V}}{\mathbf{I}}
                = \frac{\omega L I_m \phase{\theta + \ang{90}}}{I_m \phase{\theta}}
                = \frac{\omega L e^{j(\theta + \ang{90})}}{e^{j \theta}} \\
                &= \omega L e^{j(\theta + \ang{90}) - j \theta}
                = \omega L e^{j \ang{90}} \\
                &= j \omega L
        \end{align*}

        \CheatsheetEntryExtraSeparation

        \CheatsheetEntryTitle{Coupled Inductors in the Frequency Domain}

        Consider two inductors $L_1$ and $L_2$ that are coupled:
        \begin{center}
        \begin{circuitikz}
            \draw % Left Side
                (0,0)
                to[short, i=$i_1$, o-] ++(1,0) coordinate (L1Top)
                to[L, l_=$L_1$, name=L1] ++(0,-2)
                to[short, -o] ++(-1,0)
                to[open, v^<=$v_1$] (0,0)
            ;
            \draw % Right Side
                (L1Top) ++(1,0) coordinate (L2Top)
                to[short, i<=$i_2$, -o] ++(1,0)
                to[open, v^>=$v_2$] ++(0,-2)
                to[short, o-] ++(-1,0)
                to[L, l_=$L_2$, name=L2] (L2Top)
            ;
            \path
                (L1Top) -- coordinate[midway] (TmpPoint) (L2Top)
                (TmpPoint) ++(0,0.1) coordinate (MArcCenter)
            ;
            \draw[simshadows/style/ee/magneticcouplingarrow]
                (MArcCenter) ++(0:0.5) arc (0:180:0.5)
            ;
            \draw
                (MArcCenter) ++(90:0.5) node[above] {$M$}
            ;
            \draw
                (L1Top) ++(-0.20,-0.35) node[circ] {}
                (L2Top) ++( 0.20,-0.35) node[circ] {}
            ;
        \end{circuitikz}
        \end{center}

        Assume the currents:
        \begin{align*}
            i_1 &= I_{1m} \cos{(\omega t + \theta_1)}
                %\quad \Longrightarrow \quad &&
                %\frac{\diff{i_1}}{\diff{t}} = - \omega I_{1m} \sin{(\omega t + \theta_1)}
                \\
            i_2 &= I_{2m} \cos{(\omega t + \theta_2)}
                %\quad \Longrightarrow \quad &&
                %\frac{\diff{i_2}}{\diff{t}} = - \omega I_{2m} \sin{(\omega t + \theta_2)}
        \end{align*}

        From earlier, we found:
        \begin{align*}
            v_1 &= L_1 \frac{\diff{i_1}}{\diff{t}} + M \frac{\diff{i_2}}{\diff{t}} \\
            v_2 &= L_2 \frac{\diff{i_2}}{\diff{t}} + M \frac{\diff{i_1}}{\diff{t}}
        \end{align*}

        Skipping working, the voltage-current relationships in the frequency domain are:
        \begin{align*}
            \mathbf{V}_1 &= j \omega L_1 \mathbf{I}_1 + j \omega M \mathbf{I}_2 \\
            \mathbf{V}_2 &= j \omega L_2 \mathbf{I}_2 + j \omega M \mathbf{I}_1
        \end{align*}

    \end{CheatsheetEntryFrame}

\end{multicols}
\newpage
\begin{multicols}{2}

    \begin{CheatsheetEntryFrame}

        %\newcommand{\MyReusableFormatting}[2]{
        %    \begin{minipage}[t]{0.41\columnwidth}
        %        \centering
        %        #1
        %    \end{minipage}%
        %    %\vrule
        %    \begin{minipage}[t]{0.15\columnwidth}
        %        \begin{center}
        %        \begin{tikzpicture}[scale=1, transform shape]
        %            \node[
        %                single arrow,
        %                draw,
        %                minimum width=3.6ex,
        %                minimum height=4.5ex,
        %                single arrow head extend=0.9ex
        %            ] {} ;
        %            \draw[<-]
        %                (0,0) -- (0,-1)
        %            ;
        %            \path
        %                (0,0) -- (0,-1)
        %                (0,0) -- (0,1) ++(0,0.5) node[above] {{$M$}}
        %            ;
        %        \end{tikzpicture}
        %        \end{center}
        %    \end{minipage}%
        %    %\vrule
        %    \begin{minipage}[t]{0.41\columnwidth}
        %        \centering
        %        #2
        %    \end{minipage}
        %}

        \CheatsheetEntryTitle{Dot Convention: Only One Current Flowing}

        The dot convention indicates the sign of the EMF induced in one coil by another. This can be modelled using dependent sources. \\[0mm]
        {\scriptsize \textit{(Dot positions are determined by the physical architecture.)}}

        To simplify things, let's first consider the case where \\[0mm]
        {\color{myred} no current flows through $L_2$}, so {\color{myred} $L_1$ experiences no mutual EMF}.

        \medskip

        \textbf{Case 1:} Current flows \ul{into} the dot.

        \phantom{\textbf{Case 1:}} The reference polarity of the induced EMF is \\[0mm]
        \phantom{\textbf{Case 1:}} \ul{positive} at the dotted terminal.

        \begin{center}
        \begin{circuitikz}
            \begin{scope}[shift={(0,0)}]
                \draw % Left Side
                    (0,0)
                    to[short, i=$i_1$, o-] ++(1,0) coordinate (L1Top)
                    to[L, l_=$L_1$, name=L1] ++(0,-2)
                    to[short, -o] ++(-1,0) coordinate (LeftVBottom)
                    %to[open, v^<=$v_1$] (0,0)
                    to[open] (0,0) coordinate (LeftVTop)
                ;
                \draw % Right Side
                    (L1Top) ++(1,0) coordinate (L2Top)
                    to[short, -o, name=I2] ++(1,0) coordinate (RightVTop)
                    %to[open, v^>=$v_2$] ++(0,-2)
                    to[open] ++(0,-2) coordinate (RightVBottom)
                    to[short, o-] ++(-1,0)
                    to[L, l_=$L_2$, name=L2] (L2Top)
                ;
                \path
                    (L1Top) -- coordinate[midway] (TmpPoint) (L2Top)
                    (TmpPoint) ++(0,0.1) coordinate (MArcCenter)
                ;
                \draw[simshadows/style/ee/magneticcouplingarrow]
                    (MArcCenter) ++(0:0.5) arc (0:180:0.5)
                ;
                \draw
                    (MArcCenter) ++(90:0.5) node[above] {$M$}
                ;
                \draw
                    (L1Top) ++(-0.20,-0.35) node[circ] {}
                    (L2Top) ++( 0.20,-0.35) node[circ] {}
                ;
                \draw[myred, thick]
                    (I2) -- ++( 135:0.2)
                    (I2) -- ++(  45:0.2)
                    (I2) -- ++( -45:0.2)
                    (I2) -- ++(-135:0.2)
                    (RightVTop) ++(-0.13,0.12) node[above, align=center, font=\scriptsize] {NO\\CURRENT}
                ;
            \end{scope}
            \begin{scope}[shift={(3.75,-1)}]
                \node[simshadows/GenericGrayBlockArrow] {};
            \end{scope}
            \begin{scope}[shift={(4.5,0.6)}]
                \draw % Left Side
                    (0,0)
                    to[short, i=$i_1$, o-] ++(1,0) coordinate (L1Top)
                    -- ++(0,-0.2)
                    to[L, l_=$L_1$, name=L1] ++(0,-1.5)
                    %to[cV, l_=$\displaystyle M \frac{\diff{i_2}}{\diff{t}}$] ++(0,-1.5)
                    to[short] ++(0,-0.4)
                    to[short, name=DepV1] ++(0,-0.7) % to[short, *-*, name=DepV1] ++(0,-0.7)
                    to[short] ++(0,-0.4)
                    to[short, -o] ++(-1,0) coordinate (LeftVBottom)
                    %to[open, v^<=$v_1$] (0,0)
                    to[open] (0,0) coordinate (LeftVTop)
                ;
                \draw % Right Side
                    (L1Top) ++(1,0) coordinate (L2Top)
                    to[short, -o, name=I2] ++(1,0) coordinate (RightVTop)
                    %to[open, v^>=$v_2$] ++(0,-2)
                    to[open] ++(0,-0.2)
                    to[open] ++(0,-1.5)
                    to[open] ++(0,-1.5) coordinate (RightVBottom)
                    to[short, o-] ++(-1,0)
                    to[cV, invert, l_=$\displaystyle M \frac{\diff{i_1}}{\diff{t}}$, /tikz/circuitikz/bipoles/length=1.2cm] ++(0,1.5)
                    to[L, l_=$L_2$, name=L2] ++(0,1.5)
                    -- (L2Top)
                ;
                \path
                    (L1Top) -- coordinate[midway] (TmpPoint) (L2Top)
                    (TmpPoint) ++(0,0.1) coordinate (MArcCenter)
                ;
                \draw[myred, thick]
                    (I2) -- ++( 135:0.2)
                    (I2) -- ++(  45:0.2)
                    (I2) -- ++( -45:0.2)
                    (I2) -- ++(-135:0.2)
                    (RightVTop) ++(-0.00,0.12) node[above, align=center, font=\scriptsize] {NO\\CURRENT}
                    (DepV1) -- ++( 135:0.2)
                    (DepV1) -- ++(  45:0.2)
                    (DepV1) -- ++( -45:0.2)
                    (DepV1) -- ++(-135:0.2)
                    (DepV1) ++(-0.1,0) node[left, align=center, font=\scriptsize] {NO\\MUTUAL\\EMF}
                ;
            \end{scope}
        \end{circuitikz}
        \end{center}

        \newcommand{\MyReusableFormatting}[4]{%
            \begin{center}
            \begin{circuitikz}
                \begin{scope}[shift={(0,0)}]
                    \draw % Left Side
                        (0,0)
                        to[short, o-, #1] ++(1,0) coordinate (L1Top)
                        to[L, l_=$L_1$, name=L1] ++(0,-2)
                        to[short, -o, #2] ++(-1,0) coordinate (LeftVBottom)
                        %to[open, v^<=$v_1$] (0,0)
                        to[open] (0,0) coordinate (LeftVTop)
                    ;
                    \draw % Right Side
                        (L1Top) ++(1,0) coordinate (L2Top)
                        to[short, -o, name=I2] ++(1,0) coordinate (RightVTop)
                        %to[open, v^>=$v_2$] ++(0,-2)
                        to[open] ++(0,-2) coordinate (RightVBottom)
                        to[short, o-] ++(-1,0)
                        to[L, l_=$L_2$, name=L2] (L2Top)
                    ;
                    \path
                        (L1Top) -- coordinate[midway] (TmpPoint) (L2Top)
                        (TmpPoint) ++(0,0.1) coordinate (MArcCenter)
                    ;
                    \draw[simshadows/style/ee/magneticcouplingarrow]
                        (MArcCenter) ++(0:0.5) arc (0:180:0.5)
                    ;
                    \draw
                        (MArcCenter) ++(90:0.5) node[above] {$M$}
                    ;
                    \draw
                        #4
                    ;
                \end{scope}
                \begin{scope}[shift={(3.75,-1)}]
                    \node[simshadows/GenericGrayBlockArrow] {};
                \end{scope}
                \begin{scope}[shift={(4.5,0.6)}]
                    \draw % Left Side
                        (0,0)
                        to[short, o-, #1] ++(1,0) coordinate (L1Top)
                        -- ++(0,-0.2)
                        to[L, l_=$L_1$, name=L1] ++(0,-1.5)
                        %to[cV, l_=$\displaystyle M \frac{\diff{i_2}}{\diff{t}}$] ++(0,-1.5)
                        to[short] ++(0,-0.4)
                        to[short, name=DepV1] ++(0,-0.7) % to[short, *-*, name=DepV1] ++(0,-0.7)
                        to[short] ++(0,-0.4)
                        to[short, -o, #2] ++(-1,0) coordinate (LeftVBottom)
                        %to[open, v^<=$v_1$] (0,0)
                        to[open] (0,0) coordinate (LeftVTop)
                    ;
                    \draw % Right Side
                        (L1Top) ++(1,0) coordinate (L2Top)
                        to[short, -o, name=I2] ++(1,0) coordinate (RightVTop)
                        %to[open, v^>=$v_2$] ++(0,-2)
                        to[open] ++(0,-0.2)
                        to[open] ++(0,-1.5)
                        to[open] ++(0,-1.5) coordinate (RightVBottom)
                        to[short, o-] ++(-1,0)
                        to[cV, l_=$\displaystyle M \frac{\diff{i_1}}{\diff{t}}$, /tikz/circuitikz/bipoles/length=1.2cm, #3] ++(0,1.5)
                        to[L, l_=$L_2$, name=L2] ++(0,1.5)
                        -- (L2Top)
                    ;
                    \path
                        (L1Top) -- coordinate[midway] (TmpPoint) (L2Top)
                        (TmpPoint) ++(0,0.1) coordinate (MArcCenter)
                    ;
                \end{scope}
            \end{circuitikz}%
            \end{center}
        }

        \MyReusableFormatting{i=$i_1$}{}{}{
            (L1Top)          ++(-0.20,-0.35) node[circ] {}
            (L2Top) ++(0,-2) ++( 0.20, 0.35) node[circ] {}
        }

        \smallskip

        \textbf{Case 2:} Current flows \ul{out of} the dot.

        \phantom{\textbf{Case 2:}} The reference polarity of the induced EMF is \\[0mm]
        \phantom{\textbf{Case 2:}} \ul{negative} at the dotted terminal.

        \MyReusableFormatting{i=$i_1$}{}{}{
            (L1Top) ++(0,-2) ++(-0.20, 0.35) node[circ] {}
            (L2Top)          ++( 0.20,-0.35) node[circ] {}
        }

        \MyReusableFormatting{i=$i_1$}{}{invert}{
            (L1Top) ++(0,-2) ++(-0.20, 0.35) node[circ] {}
            (L2Top) ++(0,-2) ++( 0.20, 0.35) node[circ] {}
        }

        %% Old Version
        %\TwoColumnsMinipages{
        %    \begin{circuitikz}
        %        \draw % Left Side
        %            (0,0)
        %            to[short, i=$i_1$, o-] ++(1,0) coordinate (L1Top)
        %            to[L, l_=$L_1$, name=L1] ++(0,-2)
        %            to[short, -o] ++(-1,0) coordinate (LeftVBottom)
        %            %to[open, v^<=$v_1$] (0,0)
        %            to[open] (0,0) coordinate (LeftVTop)
        %        ;
        %        \draw % Right Side
        %            (L1Top) ++(1,0) coordinate (L2Top)
        %            to[short, i<=$i_2$, -o] ++(1,0) coordinate (RightVTop)
        %            %to[open, v^>=$v_2$] ++(0,-2)
        %            to[open] ++(0,-2) coordinate (RightVBottom)
        %            to[short, o-] ++(-1,0)
        %            to[L, l_=$L_2$, name=L2] (L2Top)
        %        ;
        %        \path
        %            (L1Top) -- coordinate[midway] (TmpPoint) (L2Top)
        %            (TmpPoint) ++(0,0.1) coordinate (MArcCenter)
        %        ;
        %        \draw[mymagneticcouplingarrow]
        %            (MArcCenter) ++(0:0.5) arc (0:180:0.5)
        %        ;
        %        \draw
        %            (MArcCenter) ++(90:0.5) node[above] {$M$}
        %        ;
        %        \draw
        %            (L1Top) ++(-0.20,-0.35) node[circ] {}
        %            (L2Top) ++( 0.20,-0.35) node[circ] {}
        %        ;
        %    \end{circuitikz}
        %}{
        %    \begin{circuitikz}
        %        \draw % Left Side
        %            (0,0)
        %            to[short, i=$i_1$, o-] ++(1,0) coordinate (L1Top)
        %            to[L, l_=$L_1$, name=L1] ++(0,-2)
        %            to[short, -o] ++(-1,0) coordinate (LeftVBottom)
        %            %to[open, v^<=$v_1$] (0,0)
        %            to[open] (0,0) coordinate (LeftVTop)
        %        ;
        %        \draw % Right Side
        %            (L1Top) ++(1,0) coordinate (L2Top)
        %            to[short, i<=$i_2$, -o] ++(1,0) coordinate (RightVTop)
        %            %to[open, v^>=$v_2$] ++(0,-2)
        %            to[open] ++(0,-2) coordinate (RightVBottom)
        %            to[short, o-] ++(-1,0)
        %            to[L, l_=$L_2$, name=L2] (L2Top)
        %        ;
        %        \path
        %            (L1Top) -- coordinate[midway] (TmpPoint) (L2Top)
        %            (TmpPoint) ++(0,0.1) coordinate (MArcCenter)
        %        ;
        %        \draw[mymagneticcouplingarrow]
        %            (MArcCenter) ++(0:0.5) arc (0:180:0.5)
        %        ;
        %        \draw
        %            (MArcCenter) ++(90:0.5) node[above] {$M$}
        %        ;
        %        \draw[mypurple]
        %            (L1Top) ++(0,-2) ++(-0.20, 0.35) node[circ] {}
        %            (L2Top) ++(0,-2) ++( 0.20, 0.35) node[circ] {}
        %        ;
        %    \end{circuitikz}
        %}

        %\begin{center}
        %    \DrawnDownArrow

        %    \begin{circuitikz}
        %        \draw % Left Side
        %            (0,0)
        %            to[short, i=$i_1$, o-] ++(1,0) coordinate (L1Top)
        %            -- ++(0,-0.2)
        %            to[L, l_=$L_1$, name=L1] ++(0,-1.5)
        %            to[cV, l_=$\displaystyle M \frac{\diff{i_2}}{\diff{t}}$] ++(0,-1.5)
        %            to[short, -o] ++(-1,0) coordinate (LeftVBottom)
        %            %to[open, v^<=$v_1$] (0,0)
        %            to[open] (0,0) coordinate (LeftVTop)
        %        ;
        %        \draw % Right Side
        %            (L1Top) ++(1.4,0) coordinate (L2Top)
        %            to[short, i<=$i_2$, -o] ++(1,0) coordinate (RightVTop)
        %            %to[open, v^>=$v_2$] ++(0,-2)
        %            to[open] ++(0,-0.2)
        %            to[open] ++(0,-1.5)
        %            to[open] ++(0,-1.5) coordinate (RightVBottom)
        %            to[short, o-] ++(-1,0)
        %            to[cV, invert, l_=$\displaystyle M \frac{\diff{i_1}}{\diff{t}}$] ++(0,1.5)
        %            to[L, l_=$L_2$, name=L2] ++(0,1.5)
        %            -- (L2Top)
        %        ;
        %        \path
        %            (L1Top) -- coordinate[midway] (TmpPoint) (L2Top)
        %            (TmpPoint) ++(0,0.1) coordinate (MArcCenter)
        %        ;
        %    \end{circuitikz}
        %\end{center}

    \end{CheatsheetEntryFrame}

    %\MulticolsBreak

    \begin{CheatsheetEntryFrame}

        \CheatsheetEntryTitle{Dot Convention: Two Currents Flowing}

        When both currents flow, we apply our single-current model on both currents.

        \bigskip

        \textbf{Case 1:} Currents flowing into \ul{same} side dot sides.

        \phantom{\textbf{Case 1:}} Induced EMF \ul{adds} to self-induction.

        \medskip

        \newcommand{\MyReusableFormatting}[3]{
            \begin{circuitikz}
                \begin{scope}[shift={(0,0)}]
                    \draw % Left Side
                        (0,0)
                        to[short, i=$i_1$, o-] ++(1,0) coordinate (L1Top)
                        to[L, l_=$L_1$, name=L1] ++(0,-2)
                        to[short, -o] ++(-1,0) coordinate (LeftVBottom)
                        %to[open, v^<=$v_1$] (0,0)
                        to[open] (0,0) coordinate (LeftVTop)
                    ;
                    \draw % Right Side
                        (L1Top) ++(1,0) coordinate (L2Top)
                        to[short, i<=$i_2$, -o] ++(1,0) coordinate (RightVTop)
                        %to[open, v^>=$v_2$] ++(0,-2)
                        to[open] ++(0,-2) coordinate (RightVBottom)
                        to[short, o-] ++(-1,0)
                        to[L, l_=$L_2$, name=L2] (L2Top)
                    ;
                    \path
                        (LeftVTop)  -- (LeftVBottom)  node[pos=0.2] {$+$} node[pos=0.5] {$v_1$} node[pos=0.8] {$-$}
                        (RightVTop) -- (RightVBottom) node[pos=0.2] {$+$} node[pos=0.5] {$v_2$} node[pos=0.8] {$-$}
                    ;
                    \path
                        (L1Top) -- coordinate[midway] (TmpPoint) (L2Top)
                        (TmpPoint) ++(0,0.1) coordinate (MArcCenter)
                    ;
                    \draw[simshadows/style/ee/magneticcouplingarrow]
                        (MArcCenter) ++(0:0.5) arc (0:180:0.5)
                    ;
                    \draw
                        (MArcCenter) ++(90:0.5) node[above] {$M$}
                    ;
                    \draw
                        #3
                    ;
                \end{scope}
                \begin{scope}[shift={(#2,-2.70)}]
                    \node[simshadows/GenericGrayBlockArrow, #1] {};
                    \path
                        (0,0) -- (0,-0.65) % Ghetto alignment
                    ;
                \end{scope}
            \end{circuitikz}
        }

        \newcommand{\MyReusableFormattingB}[2]{
            \begin{center}
            \begin{circuitikz}
                \begin{scope}[shift={(0,-3.5)}]
                    \draw % Left Side
                        (0,0)
                        to[short, i=$i_1$, o-] ++(1,0) coordinate (L1Top)
                        -- ++(0,-0.2)
                        to[L, l_=$L_1$, name=L1] ++(0,-1.5)
                        %to[cV, l_=$\displaystyle M \frac{\diff{i_2}}{\diff{t}}$] ++(0,-1.5)
                        to[cV, l_=$\displaystyle M \frac{\diff{i_2}}{\diff{t}}$, /tikz/circuitikz/bipoles/length=1.2cm, #1] ++(0,-1.5)
                        to[short, -o] ++(-1,0) coordinate (LeftVBottom)
                        %to[open, v^<=$v_1$] (0,0)
                        to[open] (0,0) coordinate (LeftVTop)
                    ;
                    \draw % Right Side
                        (L1Top) ++(1,0) coordinate (L2Top)
                        to[short, i<=$i_2$, -o, name=I2] ++(1,0) coordinate (RightVTop)
                        %to[open, v^>=$v_2$] ++(0,-2)
                        to[open] ++(0,-0.2)
                        to[open] ++(0,-1.5)
                        to[open] ++(0,-1.5) coordinate (RightVBottom)
                        to[short, o-] ++(-1,0)
                        to[cV, l_=$\displaystyle M \frac{\diff{i_1}}{\diff{t}}$, /tikz/circuitikz/bipoles/length=1.2cm, #2] ++(0,1.5)
                        to[L, l_=$L_2$, name=L2] ++(0,1.5)
                        -- (L2Top)
                    ;
                    \path
                        (L1Top) -- coordinate[midway] (TmpPoint) (L2Top)
                        (TmpPoint) ++(0,0.1) coordinate (MArcCenter)
                    ;
                \end{scope}
            \end{circuitikz}
            \end{center}
        }

        \TwoColumnsMinipages{
            \MyReusableFormatting{rotate=-90}{1.5}{
                (L1Top) ++(-0.20,-0.35) node[circ] {}
                (L2Top) ++( 0.20,-0.35) node[circ] {}
            }
        }{
            \MyReusableFormatting{rotate=-135}{0.2}{
                (L1Top) ++(0,-2) ++(-0.20, 0.35) node[circ] {}
                (L2Top) ++(0,-2) ++( 0.20, 0.35) node[circ] {}
            }
        }%
        \TwoColumnsMinipages{%
            \MyReusableFormattingB{}{invert}
        }{%
            \begin{gather*}
                \boxed{
                \begin{aligned}
                    v_1 &= L_1 \frac{\diff{i_1}}{\diff{t}} + M \frac{\diff{i_2}}{\diff{t}} \\
                    v_2 &= L_2 \frac{\diff{i_2}}{\diff{t}} + M \frac{\diff{i_1}}{\diff{t}}
                \end{aligned}
                }
                \\
                \boxed{
                \begin{aligned}
                    \mathbf{V}_1 &= j \omega L_1 \mathbf{I}_1 + j \omega M \mathbf{I}_2 \\
                    \mathbf{V}_2 &= j \omega L_2 \mathbf{I}_2 + j \omega M \mathbf{I}_1
                \end{aligned}
                }
            \end{gather*}
        }

        \bigskip

        \textbf{Case 2:} Currents flowing into \ul{different} side dot sides.

        \phantom{\textbf{Case 2:}} Induced EMF \ul{opposes} self-induction.

        \medskip

        \TwoColumnsMinipages{
            \MyReusableFormatting{rotate=-90}{1.5}{
                (L1Top) ++(-0.20,-0.35) node[circ] {}
                (L2Top) ++(0,-2) ++( 0.20, 0.35) node[circ] {}
            }
        }{
            \MyReusableFormatting{rotate=-135}{0.2}{
                (L2Top) ++( 0.20,-0.35) node[circ] {}
                (L1Top) ++(0,-2) ++(-0.20, 0.35) node[circ] {}
            }
        }%
        \TwoColumnsMinipages{%
            \MyReusableFormattingB{invert}{}
        }{%
            \begin{gather*}
                \boxed{
                \begin{aligned}
                    v_1 &= L_1 \frac{\diff{i_1}}{\diff{t}} - M \frac{\diff{i_2}}{\diff{t}} \\
                    v_2 &= L_2 \frac{\diff{i_2}}{\diff{t}} - M \frac{\diff{i_1}}{\diff{t}}
                \end{aligned}
                }
                \\
                \boxed{
                \begin{aligned}
                    \mathbf{V}_1 &= j \omega L_1 \mathbf{I}_1 - j \omega M \mathbf{I}_2 \\
                    \mathbf{V}_2 &= j \omega L_2 \mathbf{I}_2 - j \omega M \mathbf{I}_1
                \end{aligned}
                }
            \end{gather*}
        }

    \end{CheatsheetEntryFrame}
    
\end{multicols}

\begin{multicols}{2}

    \begin{CheatsheetEntryFrame}

        \CheatsheetEntryTitle{Series Connection}

        To aid in reinforcing the ideas, we can also consider what happens when two coupled coils are in series.

        \medskip

        \textbf{Case 1:} Current flows into the \ul{same} side. \\[0mm]
        \phantom{\textbf{Case 1:}} \Exn{\textit{(series-aiding connection)}}

        \phantom{\textbf{Case 1:}} Mutual EMF \ul{adds}.
        \begin{gather*}
            v
                = L_1 \frac{\diff{i}}{\diff{t}} + L_2 \frac{\diff{i}}{\diff{t}} + 2M \frac{\diff{i}}{\diff{t}}
                = \parens*{L_1 + L_2 + 2M} \frac{\diff{i}}{\diff{t}} \\[\abovedisplayskip]
            \MathOverLabel{\text{equivalent total inductance}}{L = L_1 + L_2 + 2M}
        \end{gather*}


        \newcommand{\MyReusableFormatting}[1]{
            \begin{center}
            \begin{circuitikz}
                \draw
                    (0,0)
                    to[short, o-] ++(0.5,0)
                    to[short, i=$i$] ++(0.5,0)
                    to[L, l_=$L_1$, name=L1] ++(2.0,0) coordinate (Mid)
                    to[L, l_=$L_2$, name=L2] ++(2.0,0)
                    to[short, -o] ++(1.0,0)
                ;
                \draw[simshadows/style/ee/magneticcouplingarrow]
                    %(L2.above) ++(-1,0.1) ellipse[x radius=1, y radius=0.5, start angle=0, end angle=180]
                    %(L2.above) ++(0,0.1) arc (0:180:1)
                    (L2.above) ++(0,0.1) -- ++(0,0.2) arc (0:180:1 and 0.4) -- ++(0,-0.2)
                ;
                \draw
                    (Mid |- L2.above) ++(0,0.75) node[above] {$M$}
                ;
                \draw
                    #1
                ;
            \end{circuitikz}
            \end{center}
        }

        \MyReusableFormatting{
            (L1.west) ++( 0.00, 0.40) node[circ] {}
            (L2.west) ++( 0.00, 0.40) node[circ] {}
        }%
        \MyReusableFormatting{
            (L1.east) ++( 0.00, 0.40) node[circ] {}
            (L2.east) ++( 0.00, 0.40) node[circ] {}
        }

        \textbf{Case 2:} Current flows into \ul{different} sides. \\[0mm]
        \phantom{\textbf{Case 2:}} \Exn{\textit{(series-opposing connection)}}

        \phantom{\textbf{Case 2:}} Mutual EMF \ul{opposes}.
        \begin{gather*}
            v
                = L_1 \frac{\diff{i}}{\diff{t}} + L_2 \frac{\diff{i}}{\diff{t}} - 2M \frac{\diff{i}}{\diff{t}}
                = \parens*{L_1 + L_2 - 2M} \frac{\diff{i}}{\diff{t}} \\[\abovedisplayskip]
            \MathOverLabel{\text{equivalent total inductance}}{L = L_1 + L_2 - 2M}
        \end{gather*}

        \MyReusableFormatting{
            (L1.west) ++( 0.00, 0.40) node[circ] {}
            (L2.east) ++( 0.00, 0.40) node[circ] {}
        }%
        \MyReusableFormatting{
            (L1.east) ++( 0.00, 0.40) node[circ] {}
            (L2.west) ++( 0.00, 0.40) node[circ] {}
        }

    \end{CheatsheetEntryFrame}

\end{multicols}

