\subsection{Amplifier Analysis}%
\label{sub:appendices-amplifier-analysis}

\begin{multicols}{2}

    \begin{CheatsheetEntryFrame}
        \CheatsheetEntryTitle{Basic NPN Current Mirror} \CheatsheetEntrySubtitle{(Full Working)}

        \begin{center}
        \begin{circuitikz}
            \draw
                (0,0)
                    \MyVCC{label=right:$V_{CC}$}
                    to[R, l=$R$, i_=$I_\text{REF}$] ++(0,-1.8)
                        coordinate (TMP_A)
                    -- ++(0,-0.5)
                        node [currarrow, label=left:$I_{C1}$, rotate=-90] {}
                        node (Q1) [npn, anchor=C, xscale=-1, label=left:Q1] {}
                (Q1.B)
                    -- ++(0.5,0)
                        coordinate (TMP_B)
                    |- (TMP_A)
                (Q1.E)
                    -- ++(0,-0.1)
                        coordinate (TMP_C)
                    -- ++(0,-0.5)
                    \MyVEE{label=right:$V_{EE}$}

                (TMP_B)
                    -- ++(0.5,0)
                        node (Q2) [npn, anchor=B, label=right:Q2] {}
                (Q2.E)
                    |- (TMP_C)
                (Q2.C)
                        node [currarrow, label=right:$I_o$, rotate=-90] {}
                    -- (Q2.C |- TMP_A)
                        node [ocirc, label=above:C2] {}

                (Q1.B)
                    ++(0.1,0)
                        node [currarrow, label=below:$I_B$, rotate=180] {}
                (Q2.B)
                    ++(-0.1,0)
                        node [currarrow, label=below:$I_B$] {}
            ;
            \draw
                (4.2,-1.9)
                    node [simshadows/GenericGrayBlockArrow] {};
            ;
            \draw
                (6.0,-1.1)
                        node [ocirc, label=above:C2] {}
                    %to[short, i_=$I_o$] ++(0,-1)
                    %    coordinate (TMP_D)
                    to[
                        american controlled current source,
                        l_=$I_o$,
                        v^>=$V_{CE2}$,
                        /tikz/circuitikz/bipoles/length=1.2cm
                    ] ++(0,-1.8)
                        coordinate (TMP_E)
                    %-- ++(0,-0.5)
                    \MyVEE{label=left:$V_{EE}$}
                %(TMP_D)
                %    -- ++(0.9,0)
                %        coordinate (TMP_F)
                %    to[R, l=$R_o$] (TMP_F |- TMP_E)
                %    -- (TMP_E)
            ;
        \end{circuitikz}
        \end{center}
        We assumine that all BJTs are in \emph{forward active}.

        %Resistor equations:
        %\begin{gather*}
        %    I_\text{REF}
        %    = I_{C1} + 2 I_B
        %    = \frac{V_{CC} - V_{EE} - V_{BE}}{R}
        %\end{gather*}

        Transistor \emph{early effect} equations:
        \begin{align*}
            I_{C1}
            &= I_{S1} e^{\brackets{{V_{BE}} \mathbin{/} {V_{T1}}}}
            \parens*{1 + \frac{V_{BE}}{V_{A1}}}
            \\
            I_o
            &= I_{S2} e^{\brackets{{V_{BE}} \mathbin{/} {V_{T2}}}}
            \parens*{1 + \frac{V_{CE2}}{V_{A2}}}
        \end{align*}

        Assuming the BJTs are well-matched:
        \begin{gather*}
            I_{S1} e^{\brackets{{V_{BE}} \mathbin{/} {V_{T1}}}}
            = I_{S2} e^{\brackets{{V_{BE}} \mathbin{/} {V_{T2}}}}
            \\
            V_A \defeq V_{A1} = V_{A2}
            \qquad
            \beta \defeq \beta_1 = \beta_2
            \\
            \Longrightarrow\qquad
            \frac{I_{C1}}{I_o}
            = \frac{V_A + V_{BE}}{V_A + V_{CE2}}
        \end{gather*}

        $I_{C1}$ and $I_B$, ignoring \emph{early effect}:
        \begin{gather*}
            I_{C1}
            = \beta I_B
            = I_\text{REF} - 2 I_B
            \\
            \Longrightarrow\qquad
            I_B = \frac{I_\text{REF}}{\beta + 2}
            \qquad
            I_{C1} = \frac{\beta I_\text{REF}}{\beta + 2}
        \end{gather*}

        Thus, $I_o$ while mostly taking into account \emph{early effect}:
        \begin{gather*}
            I_o
            = \frac{
                \beta I_\text{REF} \parens*{V_A + V_{CE2}}
            }{
                \parens*{\beta + 2} \parens*{V_A + V_{BE}}
            }
        \end{gather*}
        %
        If resistor $R$ creates the reference current:
        \begin{gather*}
            I_\text{REF} = \frac{V_{CC} - V_{EE} - V_{BE}}{R}
        \end{gather*}
        %
        If \emph{early effect} is totally neglected (i.e. $V_A \to \infty$):
        \begin{gather*}
            I_o
            = \frac{
                \beta I_\text{REF}
            }{
                \parens*{\beta + 2}
            }
        \end{gather*}
        %
        If transistor $\beta \to \infty$:
        \begin{gather*}
            I_o = I_\text{REF}
        \end{gather*}

        %\Todo{Improve this analysis? Does it really make sense to use $I_{C1} = \beta_1 I_B$?}

    \end{CheatsheetEntryFrame}

    \begin{CheatsheetEntryFrame}
        \CheatsheetEntryTitle{Basic NMOS Current Mirror} \CheatsheetEntrySubtitle{(Full Working)}

        \begin{center}
        \begin{circuitikz}
            \draw
                (0,0)
                    \MyVCC{label=right:$V_{DD}$}
                    to[R, l=$R$] ++(0,-1.8)
                        coordinate (TMP_A)
                    -- ++(0,-0.5)
                        node [currarrow, label=left:$I_\text{REF}$, rotate=-90] {}
                        node (Q1) [nmos, anchor=D, xscale=-1, label=left:Q1] {}
                (Q1.G)
                    -- ++(0.5,0)
                        coordinate (TMP_B)
                    |- (TMP_A)
                (Q1.S)
                    -- ++(0,-0.1)
                        coordinate (TMP_C)
                    -- ++(0,-0.5)
                    \MyVEE{label=right:$V_{SS}$}

                (TMP_B)
                    -- ++(0.5,0)
                        node (Q2) [nmos, anchor=G, label=right:Q2] {}
                (Q2.S)
                    |- (TMP_C)
                (Q2.D)
                        node [currarrow, label=right:$I_o$, rotate=-90] {}
                    -- (Q2.D |- TMP_A)
                        node [ocirc, label=above:D2] {}
            ;
            \draw
                (4.2,-1.9)
                    node [simshadows/GenericGrayBlockArrow] {};
            ;
            \draw
                (6.0,-1.1)
                        node [ocirc, label=above:D2] {}
                    %to[short, i_=$I_o$] ++(0,-1)
                    %    coordinate (TMP_D)
                    to[
                        american controlled current source,
                        l_=$I_o$,
                        v^>=$V_{DS2}$,
                        /tikz/circuitikz/bipoles/length=1.2cm
                    ] ++(0,-1.8)
                        coordinate (TMP_E)
                    %-- ++(0,-0.5)
                    \MyVEE{label=left:$V_{SS}$}
                %(TMP_D)
                %    -- ++(0.9,0)
                %        coordinate (TMP_F)
                %    to[R, l=$R_o$] (TMP_F |- TMP_E)
                %    -- (TMP_E)
            ;
        \end{circuitikz}
        \end{center}
        We assume that all MOSFETs are in \emph{saturation}.

        Transistor equations including channel modulation:
        \begin{align*}
            I_\text{REF}
            = \frac{1}{2} k_{n1} \frac{W_1}{L_1}
            \parens*{V_{GS} - V_{t1}}^2 \parens*{1 + \lambda_1 V_{GS}}
            \\
            I_o
            = \frac{1}{2} k_{n2} \frac{W_2}{L_2}
            \parens*{V_{GS} - V_{t2}}^2 \parens*{1 + \lambda_2 V_{DS2}}
        \end{align*}

        Assuming the MOSFETs are well-matched:
        \begin{gather*}
            \begin{gathered}
                k_{n1} \frac{W_1}{L_1} = k_{n2} \frac{W_2}{L_2}
                \\
                V_{t1} = V_{t2}
                \\
                \lambda \defeq \lambda_1 = \lambda_2
            \end{gathered}
            \quad\Longrightarrow\quad
            \frac{I_\text{REF}}{I_o}
            = \frac{1 + \lambda V_{DS1}}{1 + \lambda V_{DS2}}
        \end{gather*}

        Assuming $V_{DS1} \to 0$:
        \begin{gather*}
            I_o
            = \parens*{1 + \lambda V_{DS2}} I_\text{REF}
            = \parens*{1 + \lambda V_{DS2}} \frac{V_{DD} - V_{SS}}{R}
        \end{gather*}

        If we neglect channel modulation ($\lambda \to 0$):
        \begin{gather*}
            I_o = I_\text{REF} = \frac{V_{DD} - V_{SS}}{R}
        \end{gather*}

        \Todo{Improve this analysis? What if we don't neglect $V_{DS1}$?}

    \end{CheatsheetEntryFrame}

\end{multicols} % TODO: We shouldn't need to have separate multicols environments.
\begin{multicols}{2}

    \begin{CheatsheetEntryFrame}
        \CheatsheetEntryTitle{General Transform \#1} \CheatsheetEntrySubtitle{(Full Working)}
        \begin{center}
        \begin{circuitikz}[scale=0.9]
            \draw 
                (0,0)
                        node [ocirc, label=above:B] {}
                    -- ++(0.6,0)
                        node [currarrow, label=above:$i_B$] {}
                    -- ++(0.6,0)
                    to[R, l=$r_\pi$, v=$v_{BE}$] ++(0,-2)
                    -- ++(1.5,0)
                        node (TMP_A) [circ, label=right:E] {}
                    to[R, l=$R_E$] ++(0,-2)
                    \MyTLGround{}
                (TMP_A)
                    to[american controlled current source, invert, l_={$
                        \begin{gathered}
                            g_m v_{BE} \\
                            = \beta i_B
                        \end{gathered}
                    $}] ++(0,2)
                    -- ++(1.6,0)
                        node [ocirc, label=above:C] {}
            ;
            \draw
                (5.3,-2)
                    node [simshadows/GenericGrayBlockArrow] {};
            ;
            \draw
                (7.2,0)
                        node [ocirc, label=above:B] {}
                    -- ++(0.8,0)
                        node [currarrow, label=above:$i_B$] {}
                    -- ++(0.8,0)
                        node (TMP_A) [circ, label=above:E] {}
                    to[R, l_=$r_\pi$] ++(0,-2)
                    to[R, l_={$
                        \begin{gathered}
                            R_E \parens*{1 + g_m r_\pi} \\
                            = R_E \parens*{1 + \beta}
                        \end{gathered}
                    $}] ++(0,-2)
                    \MyTLGround{}
            ;
        \end{circuitikz}%
        \end{center}
        Across $r_\pi$ and across $R_E$:
        \begin{gather*}
            v_{BE} = i_B r_\pi
            \\
            v_B - v_{BE}
            = \parens*{
                \frac{v_{BE}}{r_\pi}
                + g_m v_{BE}
            } R_E
        \end{gather*}
        Cancelling $v_{BE}$:
        \begin{gather*}
            v_B - i_B r_\pi
            = i_B r_\pi \parens*{
                \frac{1}{r_\pi}
                + g_m
            } R_E
            \\
            v_B
            = i_B \parens*{
                1
                + g_m r_\pi
            } R_E + i_B r_\pi
            \\
            \frac{v_B}{i_B}
            = \parens*{
                1
                + g_m r_\pi
            } R_E + r_\pi
        \end{gather*}

    \end{CheatsheetEntryFrame}
    
    \begin{CheatsheetEntryFrame}
        \CheatsheetEntryTitle{General Transform \#2} \CheatsheetEntrySubtitle{(Full Working)}
        \begin{center}
        \begin{circuitikz}[scale=0.9]
            \draw 
                (0,0)
                    -- ++(0.7,0)
                        node [currarrow, label=above:$i_B$] {}
                    -- ++(0.7,0)
                        node [circ, label=above:B] {}
                    to[R, l=$R_B$] ++(0,-2)
                    -- ++(1.8,0)
                        coordinate (TMP_A)
                        to[short, i_<=$i_o$] ++(0,-1)
                        node [ocirc, label=right:E] {}
                (TMP_A)
                    to[american controlled current source, invert, l_={$\beta i_B$}] ++(0,2)
                    -- ++(1.4,0)
                        node [ocirc, label=above:C] {}
                (0,0)
                    \MyGround{}
            ;
            \draw
                (5.5,-1)
                    node [simshadows/GenericGrayBlockArrow] {}
            ;
            \draw
                (7,0)
                    -- ++(0,-0.4)
                    \MyTLGround{}
                (7,0)
                    -| ++(1,-0.6)
                    to[R, l_=$\displaystyle \frac{R_B}{1 + \beta}$] ++(0,-1.8)
                        node [currarrow, label=left:$i_o$, rotate=90] {}
                    -- ++(0,-0.6)
                        node [ocirc, label=right:E] {}
            ;
        \end{circuitikz}%
        \end{center}
        Across $R_B$ and at summing currents at node E:
        \begin{gather*}
            v_E = - i_B R_B
            \\
            - i_o = \parens*{1 + \beta} i_B
        \end{gather*}
        %
        Cancelling $i_B$:
        \begin{gather*}
            - i_o = \parens*{1 + \beta} \parens*{- \frac{v_E}{R_B}}
            \\
            \frac{v_E}{i_o}
            = \frac{R_B}{1 + \beta}
        \end{gather*}

    \end{CheatsheetEntryFrame}

    \begin{CheatsheetEntryFrame}
        \CheatsheetEntryTitle{General Transform \#3} \CheatsheetEntrySubtitle{(Full Working)}
        \begin{center}
        \begin{circuitikz}[scale=0.9]
            \draw 
                (0,0)
                        node [ocirc, label=above:B] {}
                    -- ++(0.7,0)
                        node [currarrow, label=above:$i_B$] {}
                    -- ++(0.7,0)
                    to[R, l=$r_\pi$, v=$v_{BE}$] ++(0,-2)
                    -- ++(1.8,0)
                        node (TMP_A) [circ, label=below:E] {}
                (TMP_A)
                    to[american controlled current source, invert, l_={$
                        \begin{gathered}
                            g_m v_{BE} \\
                            = \beta i_B
                        \end{gathered}
                    $}] ++(0,2)
                    -- ++(3,0)
                        node [ocirc, label=right:C] {}
                (TMP_A)
                    -- ++(2.4,0)
                    to[R, l_=$r_o$] ++(0,2)
            ;
        \end{circuitikz}

        \begin{circuitikz}[scale=0.9]
            \draw
                (4,1)
                    node [simshadows/GenericGrayBlockArrow] {};
            ;
            \draw
                (5.5,1)
                        node [ocirc, label=above:X] {}
                    -- ++(0.5,0)
                        node [currarrow, label=above:$i_B$] {}
                    to[R, l_=$r_\pi$] ++(2,0)
                    to[R, l_={$
                        \begin{gathered}
                            r_o \parens*{1 + g_m r_\pi}
                            \\
                            = r_o \parens*{1 + \beta}
                        \end{gathered}
                    $}] ++(2,0)
                        node [ocirc, label=above:C] {}
            ;
        \end{circuitikz}%
        \end{center}
        Across $r_\pi$, across $r_o$, and across E-C:
        \begin{gather*}
            v_{BE} = i_B r_\pi
            \\
            v_{BC} - v_{BE} = \parens*{\frac{v_{BE}}{r_\pi} + g_m v_{BE}} r_o
        \end{gather*}
        %
        Cancelling $v_{BE}$:
        \begin{gather*}
            v_{BC} - i_B r_\pi = i_B r_\pi \parens*{\frac{1}{r_\pi} + g_m} r_o
            \\
            v_{BC} = i_B \parens*{1 + g_m r_\pi} r_o + i_B r_\pi
            \\
            \frac{v_{BC}}{i_B}
            = \parens*{1 + g_m r_\pi} r_o + r_\pi
        \end{gather*}

    \end{CheatsheetEntryFrame}

    %\MulticolsBreak

\end{multicols}

