\subsection{Probability}%
\label{sub:probability}

\begin{multicols}{2}

    \CheatsheetEntryFrame{

        \CheatsheetEntryTitle{Probabilities Arithmetic}
        \begin{gather*}
            P(A^c) = 1 - P(A)
            %
            \\[\abovedisplayskip]
            %
            P(A|B) = \frac{P(A \cap B)}{P(B)}
        \end{gather*}

        \CheatsheetSmallEquationTitle{Addition Rule}
        \begin{equation*}
            P(A \cup B) = P(A) + P(B) - P(A \cap B)
        \end{equation*}
        %My method using multiplication rule:
        %\begin{equation*}
        %    P(A \cup B) = 1 - P \parens*{\parens*{A \cup B}^c} = 1 - P(A^c \cap B^c)
        %\end{equation*}
        %% This was written in a final exam cheatsheet I wrote 5 years ago. I don't remember why I thought it was worthwhile.

        \CheatsheetSmallEquationTitle{Multiplication Rule}
        \begin{equation*}
            P(A \cap B) = P(A|B) P(B) = P(B|A) P(A)
        \end{equation*}

        \CheatsheetSmallEquationTitle{Total Probability Rule}
        \CheatsheetSmallText{If $A_1, \dots, A_n$ partition a sample space and $B$ is an event in the same space, then:}
        \begin{equation*}
            P(B) = \sum_{i=1}^n{P(B|A_i) P(A_i)}
        \end{equation*}

        \CheatsheetSmallEquationTitle{Bayes' Rule}
        \CheatsheetSmallText{If $A_1, \dots, A_n$ partition a sample space and $B$ is an event in the same space, then:}
        \begin{equation*}
            P(A_j|B) = \DisplayFrac{P(B|A_j) P(A_j)}{\sum_{i=1}^n{P(B|A_i) P(A_i)}}
        \end{equation*}

    }
    
\end{multicols}

