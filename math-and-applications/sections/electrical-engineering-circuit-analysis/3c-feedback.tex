\subsubsection{Feedback}


\begin{CheatsheetEntryFrame}

    \begin{minipage}[c]{0.27\textwidth}
        \MinipageInheritDocumentFormatting
        \CheatsheetEntryTitle{Feedback \emph{\footnotesize (generalization)}}
        \begin{gather*}
            A_f
            = \frac{x_o}{x_s}
            = \frac{A_o}{1 + \beta A_o}
            \underbrace{
                \approx \frac{1}{\beta}
                \vphantom{\frac{1}{\displaystyle \sum}}
            }_{\mathclap{\substack{
                \mathstrut \text{if} \\
                \mathstrut \beta A_o \gg 1
            }}}
        \end{gather*}
    \end{minipage}%
    \hfill
    \begin{minipage}[c]{0.73\textwidth}
    \begin{tikzpicture}[node distance=1.5cm]
        \node (amplifier) [MyBlockDiagramRectangle, align=center]
            {Amplifier \\ $A_o$};
        \node (feedback) [MyBlockDiagramRectangle, align=center, below of=amplifier, minimum width=3.5cm]
            {Feedback Network \\ $\beta$};
        \node (summing) [MyBlockDiagramEllipse, align=center, left of=amplifier, scale=0.8, xshift=-3.5cm]
            {Summing \\ Network};
        \node (sampling) [MyBlockDiagramEllipse, align=center, right of=amplifier, scale=0.8, xshift=2.8cm]
            {Sampling \\ Network};

        \node (input) [align=center, left of=summing, left]
            {$x_s$};
        \node (output) [align=center, right of=sampling, right]
            {$\begin{gathered} x_o \\ = A_f x_s \end{gathered}$};

        \draw[MyBlockDiagramArrow]
            (summing) -- (amplifier)
                node[midway, above]{$x_i$}
                node[midway, below]{$= x_s - \beta x_o$}
        ;
        \draw[MyBlockDiagramArrow]
            (amplifier) -- (sampling)
                node[midway, above]{$x_o$}
                node[midway, below]{$= A_o x_i$}
        ;
        \draw[MyBlockDiagramArrow]
            (sampling) -- (feedback -| sampling)
            -- (feedback)
                node[midway, above]{$x_o$}
        ;
        \draw[MyBlockDiagramArrow]
            (feedback) -- (feedback -| summing)
                node[midway, above]{$x_f$}
                node[midway, below]{$= \beta x_o$}
            -- (summing)
        ;

        \draw[MyBlockDiagramArrow] (input) -- (summing);
        \draw[MyBlockDiagramArrow] (sampling) -- (output);
    \end{tikzpicture}
    \end{minipage}

\end{CheatsheetEntryFrame}

\newcommand{\TmpFBSeries}[0]{\textbf{\color{mypurple}series}}
\newcommand{\TmpFBShunt}[0]{\textbf{\color{myblue}shunt}}
\newcommand{\TmpFBVoltage}[0]{\textbf{\color{mygreen}voltage}}
\newcommand{\TmpFBCurrent}[0]{\textbf{\color{myred}current}}

\begin{CheatsheetEntryFrame}

    \CheatsheetEntryTitle{Feedback Topologies}

    \begin{minipage}[c]{0.35\columnwidth}
        \MinipageInheritDocumentFormatting
        \begin{gather*}
            R_{if} =
                \begin{cases}
                    \displaystyle R_i \parens{1 + \beta A_o}
                        & \quad \text{\footnotesize if \emph{series} input}
                        \\
                    \displaystyle R_i \frac{\mathstrut 1}{1 + \beta A_o}
                        & \quad \text{\footnotesize if \emph{shunt} input}
                \end{cases}
        \end{gather*}
        \begin{gather*}
            R_{of} =
                \begin{cases}
                    \displaystyle R_o \parens{1 + \beta A_o}
                        & \quad \text{\footnotesize if \emph{series} output}
                        \\
                    \displaystyle R_o \frac{\mathstrut 1}{1 + \beta A_o}
                        & \quad \text{\footnotesize if \emph{shunt} output}
                \end{cases}
        \end{gather*}
        \begin{gather*}
            \omega_{fH} = \omega_H \parens{1 + \beta A_o}
            \\
            \omega_{fL} = \omega_L \frac{1}{1 + \beta A_o}
        \end{gather*}
        \vspace*{0mm} % TODO: Why does this work?
    \end{minipage}%
    \SoftVLine{}%
    \begin{minipage}[c]{0.63\columnwidth}
        \MinipageInheritDocumentFormatting
        \footnotesize
        Basic topology identification strategy:
        \vspace*{-5mm}
        \begin{center}
        \begin{tabular}{|l|c|c|}
            \cline{2-3}
            \multicolumn{1}{c|}{}
                & \thead{Yes}
                & \thead{No} \\\hline
            Is \myul{$A_o$ input node} the same as \myul{$\beta$ input node}?
                & \emph{Shunt} Input
                & \emph{Series} Input \\\hline
            Is \myul{$A_o$ output node} the same as \myul{$\beta$ output node}?
                & \emph{Shunt} Output
                & \emph{Series} Output \\\hline
        \end{tabular}
        \end{center}

        \bigskip

        %\newcommand{\TmpFBAF}[2]{%
        %    $\displaystyle \MathSumStrut{I} #1$\hspace{3mm}
        %    {\footnotesize \parbox[c]{2.4cm}{#2 \\ amplifier}}%
        %}

        Topologies and their appropriate feedback network two-port parameters:
        \begin{center}
        \begin{tabular}{|c||c|c|c||c|c|}
            \hline
            \thead{Topology}
                & \thead{Input \\ Loading \\ Effect}
                & \thead{$\beta$}
                & \thead{Output \\ Loading \\ Effect}
                & \thead{Input \\ Summing}
                & \thead{Output \\ Sampling}
                %& $A_f$
                \\\hline\hline
            \TmpFBSeries{}-\TmpFBSeries{}
                & $z_{11}$ & $z_{12}$ & $z_{22}$ & \TmpFBVoltage{} & \TmpFBCurrent{}
                %& \TmpFBAF{\frac{i_o}{v_s}}{transconductance}
                \\\hline
            \TmpFBSeries{}-\TmpFBShunt{}
                & $h_{11}$ & $h_{12}$ & $h_{22}$ & \TmpFBVoltage{} & \TmpFBVoltage{}
                %& \TmpFBAF{\frac{v_o}{v_s}}{voltage}
                \\\hline
            \TmpFBShunt{}-\TmpFBShunt{}
                & $y_{11}$ & $y_{12}$ & $y_{22}$ & \TmpFBCurrent{} & \TmpFBVoltage{}
                %& \TmpFBAF{\frac{v_o}{i_s}}{transimpedance}
                \\\hline
            \TmpFBShunt{}-\TmpFBSeries{}
                & $g_{11}$ & $g_{12}$ & $g_{22}$ & \TmpFBCurrent{} & \TmpFBCurrent{}
                %& \TmpFBAF{\frac{i_o}{i_s}}{current}
                \\\hline
        \end{tabular}
        \end{center}
        \bigskip
    \end{minipage}%
    \smallskip
    \SoftHLine

    \newcommand{\TmpFBBaseDrawing}[2]{%
        \begin{minipage}[c]{0.49\columnwidth}
            \MinipageInheritDocumentFormatting
            #1
            \vspace*{-4mm}
            \begin{center}
            \scalebox{0.88}{
            \begin{circuitikz}
                \draw
                    % Box: A
                    (0,0)
                        node[
                            rectangle,
                            minimum width=1.2cm,
                            minimum height=2cm,
                            text centered,
                            draw=black,
                            fill=black!10,
                            anchor=center
                        ] (AmplifierOL1) {$A$}
                    (AmplifierOL1.west) ++(0, 0.6) coordinate (AmplifierOL1_i1)
                    (AmplifierOL1.west) ++(0,-0.6) coordinate (AmplifierOL1_i2)
                    (AmplifierOL1.east) ++(0, 0.6) coordinate (AmplifierOL1_o1)
                    (AmplifierOL1.east) ++(0,-0.6) coordinate (AmplifierOL1_o2)

                    % Box: Feedback Network
                    (0,-3)
                        node[
                            rectangle,
                            minimum width=3cm,
                            minimum height=2cm,
                            text centered,
                            draw=black,
                            %fill=black!10,
                            anchor=center
                        ] (AmplifierFB) {}
                    (AmplifierFB.west) ++(0, 0.6) coordinate (AmplifierFB_i1)
                    (AmplifierFB.west) ++(0,-0.6) coordinate (AmplifierFB_i2)
                    (AmplifierFB.east) ++(0, 0.6) coordinate (AmplifierFB_o1)
                    (AmplifierFB.east) ++(0,-0.6) coordinate (AmplifierFB_o2)
                ;
                #2
            \end{circuitikz}
            }
            \end{center}
            \bigskip
        \end{minipage}%
    }
    \newcommand{\TmpFBInputSeries}[2]{
        \draw
            % Around Box A
            (AmplifierOL1_i1) to[R, l_=$R_s      $] ++(-1.6,0) coordinate (AmplifierOL2_i1)
            (AmplifierOL1_i2) to[R, l_=${#1}_{11}$] ++(-1.6,0) coordinate (AmplifierOL2_i2)

            % Feedback Network Internals
            (AmplifierFB_i1)
                -- ++(0.65,0)
                    coordinate (AmplifierFB_i1_TMP)
                to[american controlled voltage source, l=$\beta {#2}_o$, bipoles/length=1.2cm] (AmplifierFB_i1_TMP |- AmplifierFB_i2)
                -- (AmplifierFB_i2)

            % Outer
            (AmplifierOL2_i2) -- ++(-0.7,0) |- (AmplifierFB_i1)
            (AmplifierOL2_i1) -- ++(-1.7,0)
                to[V, v_=$v_s$] ++(0,-2)
                |- (AmplifierFB_i2)

            % Annotations
            (AmplifierOL2_i1) ++(-0.35,0) coordinate (AmplifierOL2_i1_v_TMP)
            (AmplifierOL2_i2) ++(-0.35,0) coordinate (AmplifierOL2_i2_v_TMP)
            (AmplifierFB_i1)  ++(-0.35,0) coordinate (AmplifierFB_i1_v_TMP)
            (AmplifierFB_i2)  ++(-0.35,0) coordinate (AmplifierFB_i2_v_TMP)

            (AmplifierOL2_i1_v_TMP) to[open, v=$v_i$] (AmplifierOL2_i2_v_TMP)
            (AmplifierFB_i1_v_TMP)  to[open, v=$v_f$] (AmplifierFB_i2_v_TMP)
        ;
    }
    \newcommand{\TmpFBOutputSeries}[1]{
        \draw
            % Around Box A
            (AmplifierOL1_o1) to[R, l^=$R_L      $] ++( 1.6,0) coordinate (AmplifierOL2_o1)
            (AmplifierOL1_o2) to[R, l^=${#1}_{22}$] ++( 1.6,0) coordinate (AmplifierOL2_o2)

            % Feedback Network Internals
            (AmplifierFB_o1)
                -- ++(-0.65,0)
                |- (AmplifierFB_o2)

            % Outer
            (AmplifierOL2_o2) -- ++( 0.7,0) |- (AmplifierFB_o1)
            (AmplifierOL2_o1) -- ++( 1.7,0) |- (AmplifierFB_o2)

            % Annotations
            (AmplifierOL2_o1) ++(0.75,0) node[currarrow, label=above:$i_o$, rotate=180]{}
            (AmplifierFB_o1)  ++(1.00,0) node[currarrow, label=above:$i_o$, rotate=180]{}
        ;
    }
    \newcommand{\TmpFBInputShunt}[2]{
        \draw
            % Around Box A
            (AmplifierOL1_i1)
                -- ++(-0.5,0) coordinate (AmplifierOL1_i1_TMP1)
                -- ++(-1.2,0) coordinate (AmplifierOL1_i1_TMP2)
                -- ++(-0.8,0) coordinate (AmplifierOL2_i1)
            (AmplifierOL1_i2)
                -- (AmplifierOL1_i2 -| AmplifierOL2_i1)
                    coordinate (AmplifierOL2_i2)

            (AmplifierOL1_i1_TMP1)
                to[R, l_=$\displaystyle \frac{1}{{#1}_{11}}$] (AmplifierOL1_i1_TMP1 |- AmplifierOL1_i2)
            (AmplifierOL1_i1_TMP2)
                to[R, l_=$R_s$] (AmplifierOL1_i1_TMP2 |- AmplifierOL1_i2)

            % Feedback Network Internals
            (AmplifierFB_i1)
                -- ++(0.65,0)
                    coordinate (AmplifierFB_i1_TMP)
                to[american controlled current source, l=$\beta {#2}_o$, bipoles/length=1.2cm] (AmplifierFB_i1_TMP |- AmplifierFB_i2)
                -- (AmplifierFB_i2)

            % Outer
            (AmplifierOL2_i1)
                -- ++(-1.2,0)
                    coordinate (AmplifierOL2_i1_TMP)
                to[I, invert, l_=$i_s$] (AmplifierOL2_i1_TMP |- AmplifierOL2_i2)
                -- (AmplifierOL2_i2)

            (AmplifierOL2_i1)
                ++(-0.3,0)
                    node[circ]{}
                to[crossing] ++(0,-2.4)
                |- (AmplifierFB_i1)
            (AmplifierOL2_i2)
                ++(-0.8,0)
                    node[circ]{}
                |- (AmplifierFB_i2)

            %(AmplifierOL2_o1) ++(0.75,0) node[currarrow, label=above:$i_o$, rotate=180]{}
            (AmplifierFB_i1)  ++(-1.00,0) node[currarrow, label=above:$i_f$]{}
        ;
    }
    \newcommand{\TmpFBOutputShunt}[1]{
        \draw
            % Around Box A
            (AmplifierOL1_o1)
                -- ++(0.5,0) coordinate (AmplifierOL1_o1_TMP1)
                -- ++(1.2,0) coordinate (AmplifierOL1_o1_TMP2)
                -- ++(0.8,0) coordinate (AmplifierOL2_o1)
            (AmplifierOL1_o2)
                -- (AmplifierOL1_o2 -| AmplifierOL2_o1)
                    coordinate (AmplifierOL2_o2)

            (AmplifierOL1_o1_TMP1)
                to[R, l^=$\displaystyle \frac{1}{{#1}_{22}}$] (AmplifierOL1_o1_TMP1 |- AmplifierOL1_o2)
            (AmplifierOL1_o1_TMP2)
                to[R, l^=$R_L$] (AmplifierOL1_o1_TMP2 |- AmplifierOL1_o2)

            % Feedback Network Internals
            (AmplifierFB_o1)
                -- ++(-0.65,0)
                    node[ocirc]{}
            (AmplifierFB_o2)
                -- ++(-0.65,0)
                    node[ocirc]{}

            % Outer
                (AmplifierOL2_o1) -- ++(1.2,0) node[ocirc]{} ++(0,-0.1) coordinate (TMP_output1)
            (AmplifierOL2_o2) -- ++(1.2,0)
                node[ocirc]{}
                ++(0,0.1)
                to[open, v<=$v_o$] (TMP_output1)

            (AmplifierOL2_o1)
                ++(0.3,0)
                    node[circ]{}
                to[crossing] ++(0,-2.4)
                |- (AmplifierFB_o1)
            (AmplifierOL2_o2)
                ++(0.8,0)
                    node[circ]{}
                |- (AmplifierFB_o2)

            (AmplifierFB_o1) ++(0.35,0) coordinate (AmplifierFB_o1_v_TMP)
            (AmplifierFB_o2) ++(0.35,0) coordinate (AmplifierFB_o2_v_TMP)

            (AmplifierFB_o1_v_TMP) to[open, v=$v_o$] (AmplifierFB_o2_v_TMP)
        ;
    }
    \newcommand{\TmpFBOLBox}{
        \draw[lightgray, dashed]
            % Box: A_o
            (AmplifierOL2_o1) ++(0, 0.85) coordinate (AmplifierOL2_TR)
            (AmplifierOL2_i2) ++(0,-0.85) coordinate (AmplifierOL2_BL)
            (AmplifierOL2_TR) rectangle (AmplifierOL2_BL)
        ;
        \draw
            % Box: A_o (continued)
            (AmplifierOL2_TR) node[above left] {$A_o$}
        ;
    }

    \TmpFBBaseDrawing{\TmpFBSeries{}-\TmpFBSeries{}}{
        \TmpFBInputSeries{z}{i}
        \TmpFBOutputSeries{z}
        \TmpFBOLBox{}
    }%
    \SoftVLine{}%
    \TmpFBBaseDrawing{\TmpFBShunt{}-\TmpFBShunt{}}{
        \TmpFBInputShunt{y}{v}
        \TmpFBOutputShunt{y}
        \TmpFBOLBox{}
    }%
    \smallskip%
    \SoftHLine%
    \TmpFBBaseDrawing{\TmpFBSeries{}-\TmpFBShunt{}}{
        \TmpFBInputSeries{h}{v}
        \TmpFBOutputShunt{h}
        \TmpFBOLBox{}
    }%
    \SoftVLine{}%
    \TmpFBBaseDrawing{\TmpFBShunt{}-\TmpFBSeries{}}{
        \TmpFBInputShunt{g}{i}
        \TmpFBOutputSeries{g}
        \TmpFBOLBox{}
    }%

\end{CheatsheetEntryFrame}

\newpage

\begin{CheatsheetEntryFrame}

    \CheatsheetEntryTitle{Feedback Amplifier General Analysis Method}
    \begin{enumerate}
        \item Identify the \emph{feedback topology} and \emph{feedback network}.
        \item Choose the appropriate \emph{two-port network parameters} and calculate them.
        \item Draw a small signal equivalent circuit of the open loop amplifier \myul{with loading effects included}.
        \item Calculate $A_o$, $R_i$, and $R_o$ {\footnotesize (via. circuit analysis of the previous step)}.
        \item Calculate $A_f$, $R_{if}$, and $R_{of}$ {\footnotesize (using the results of the previous step)}.
    \end{enumerate}

\end{CheatsheetEntryFrame}

%\begin{multicols}{2}
%\end{multicols}

