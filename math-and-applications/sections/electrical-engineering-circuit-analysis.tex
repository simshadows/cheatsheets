\subsection{DC Analysis}%
\label{sub:dc-analysis}

\begin{multicols}{2}
    
    \begin{CheatsheetEntryFrame}

        \CheatsheetEntryTitle{Current and Voltage} \\[0mm]
        \begin{multicols}{2}
            \begin{equation*}
                i \triangleq \frac{\diff{q}}{\diff{t}}
            \end{equation*}

            \begin{equation*}
                v \triangleq \frac{\diff{w}}{\diff{q}}
            \end{equation*}
        \end{multicols}

        By \textit{conventional current}, current is the flow of $+$ve charge.

        \CheatsheetEntryExtraSeparation

        \CheatsheetEntryTitle{Power}
        \begin{equation*}
            p \triangleq \frac{\diff{w}}{\diff{t}} = \frac{\diff{w}}{\diff{q}} \frac{\diff{q}}{\diff{t}}
            \qquad \Rightarrow \qquad
            p = vi
        \end{equation*} \\[0pt]

        \begin{center}
        \begin{circuitikz}
            \draw
                (0,0) to[short] ++(0.8,0)
                to[short, i=$i$] ++(0.2,0)
                to[generic, name=X, v=$v$] ++(2,0)
                to[short] ++(1,0)
            ;
            \node[single arrow,
                draw,
                minimum height=2.5em,
                minimum width=4em,
                single arrow tip angle=125,
                single arrow head extend=0.15em,
                outer sep=1ex,
                rotate=-90,
                anchor=east]
                (Arrow) at (X.north) {};
            \node at (Arrow.center) {$p$};
            \draw
                (Arrow.north) node[right=2ex, align=left] {
                    By \textit{passive sign convention}, \\
                    power flowing \textit{in} is $+$ve.
                }
                %(Arrow.north |- X.center)
                %++(0,-1) node[right=2ex, align=left] {
                %    So for power to be positive, \\
                %    current flows into the positive terminal.
                %}
            ;
        \end{circuitikz}
        \end{center}

        %Passive sign convention will be assumed. %for all electrical engineering material of this cheatsheet.

        %\CheatsheetEntryExtraSeparation

        \CheatsheetEntryTitle{Energy Absorbed}
        \begin{equation*}
            w = \int_{t_0}^{t}{p(\tau) \,\diff{\tau}}
        \end{equation*}

    \end{CheatsheetEntryFrame}

    \begin{CheatsheetEntryFrame}

        %% Old Version
        %\CheatsheetEntryTitle{Ideal Independent Sources}
        %
        %\begin{center}
        %\begin{circuitikz}
        %    \path 
        %        (0,0) node (D) {}
        %        ++(0,1.2) node (C) {}
        %        ++(0,2.2) node (B) {}
        %        ++(0,1.2) node (A) {}
        %        (A) ++(0,0.65) node (L1) {}
        %        (C) ++(0,0.65) node (L2) {}
        %        (A) ++(2,0) ++(1.7,0) node (Arrow1) {} ++(1.7,0) node (ArrowEdge1) {}
        %        (B -| Arrow1) node (Arrow2) {}
        %        (B -| ArrowEdge1) node (ArrowEdge2) {}
        %        (ArrowEdge1) ++(1.1,0) ++(0,0.4) node (DisL1) {}
        %        (ArrowEdge1) ++(0,-0) node (DisD1) {}
        %        (ArrowEdge2) ++(1.1,0) ++(0,0.4) node (DisL2) {}
        %        (ArrowEdge2) ++(0,-0) node (DisD2) {}
        %    ;
        %    \draw
        %        (A) to[V, invert, o-o] ++(2,0)
        %        (B) to[I, o-o] ++(2,0)
        %        (C) to[cV, invert, o-o] ++(2,0)
        %        (D) to[cI, o-o] ++(2,0)
        %        (L1) ++(1,0) node[above] {Independent Sources}
        %        (L2) ++(1,0) node[above] {Dependent Sources}
        %        (Arrow1) node {$\xRightarrow[(\text{set }V=0)]{\text{source disable}}$}
        %        (Arrow2) node {$\xRightarrow[(\text{set }I=0)]{\text{source disable}}$}
        %        (DisL1) node {Short Circuit}
        %        (DisL2) node {Open Circuit}
        %        (DisD1) to[short, o-*] ++(0.8,0) to[short] ++(0.6,0) to[short, *-o] ++(0.8,0)
        %        (DisD2) to[short, o-o] ++(0.8,0) to[open] ++(0.6,0) to[short, o-o] ++(0.8,0)
        %    ;
        %\end{circuitikz}%
        %\end{center}

        \CheatsheetEntryTitle{Ideal Independent Sources}

        \begin{center}
        \begin{circuitikz}
            \path 
                (0,0) node (B) {}
                ++(0,1.6) node (A) {}
                (A) ++(2,0) ++(1.7,0) node (Arrow1) {} ++(1.7,0) node (ArrowEdge1) {}
                (B -| Arrow1) node (Arrow2) {}
                (B -| ArrowEdge1) node (ArrowEdge2) {}
                (ArrowEdge1) ++(1.1,0) ++(0,0.4) node (DisL1) {}
                (ArrowEdge1) ++(0,-0) node (DisD1) {}
                (ArrowEdge2) ++(1.1,0) ++(0,0.4) node (DisL2) {}
                (ArrowEdge2) ++(0,-0) node (DisD2) {}
            ;
            \draw
                (A) to[V, l=$V$, invert, o-o] ++(2,0)
                (B) to[I, l=$I$, o-o] ++(2,0)
                (Arrow1) node {$\xRightarrow[(\text{set }V=0)]{\text{source disable}}$}
                (Arrow2) node {$\xRightarrow[(\text{set }I=0)]{\text{source disable}}$}
                (DisL1) node {Short Circuit}
                (DisL2) node {Open Circuit}
                (DisD1) to[short, o-*] ++(0.8,0) to[short] ++(0.6,0) to[short, *-o] ++(0.8,0)
                (DisD2) to[short, o-o] ++(0.8,0) to[open] ++(0.6,0) to[short, o-o] ++(0.8,0)
                %(A) ++(2,0) ++(-0.35,0.2) node[above] {$V$}
                %(B) ++(2,0) ++(-0.35,0.2) node[above] {$I$}
            ;
        \end{circuitikz}%
        \end{center}

        \CheatsheetEntryTitle{Ideal Dependent Sources}

        \vspace*{1ex}
        \begin{center}
        \begin{circuitikz}
            \path 
                (0,0) node (C) {}
                %++(0,1.6) node (C) {}
                % Everything below is used to align with the Ideal Independent Sources section
                (C) ++(2,0) ++(1.7,0) node (Arrow1) {} ++(1.7,0) node (ArrowEdge1) {}
                (ArrowEdge1) ++(0,-0) node (DisD1) {}
                    (DisD1) -- ++(0.8,0) -- ++(0.6,0) -- ++(0.8,0)
            ;
            \draw
                (C) to[cV, invert, o-o] ++(2,0)
                ++(0.8,0) to[cI, o-o] ++(2,0)
            ;
        \end{circuitikz}%
        \end{center}

    \end{CheatsheetEntryFrame}

    \begin{CheatsheetEntryFrame}

        \CheatsheetEntryTitle{Maximum Power Transfer}

        \begin{minipage}[c]{0.6\columnwidth}
            \begin{center}
            \begin{circuitikz}
                \path
                    (0,0) coordinate (InsideBL)
                    ++(1.8,0  ) coordinate (InsideBR)
                    ++(0  ,1.5) coordinate (InsideTR)
                ;
                \draw[simshadows/style/softgray]
                    (InsideTR) ++( 0  , 0.8) coordinate (BoxTR)
                    (InsideBL) ++(-1.2,-0.8) coordinate (BoxBL)
                    (BoxBL) rectangle (BoxTR)
                ;
                \draw
                    (InsideBL)
                    to[V, l=$v_{th}$, invert] (InsideBL |- InsideTR)
                    to[R, l=$R_{th}$] (InsideTR)
                    to[short, -o] ++( 0.4,0)
                    to[short]     ++( 0.4,0)
                    to[R, l=$R_L$] ++(0,-1.5)
                    to[short]     ++(-0.4,0)
                    to[short, o-] ++(-0.4,0)
                    to[short]     (InsideBL)
                ;
            \end{circuitikz}
            \end{center}
        \end{minipage}%
        \begin{minipage}[c]{0.4\columnwidth}
            \centering
            Maximum power is transferred to $R_L$ if:
            \begin{equation*}
                R_L = R_{th}
            \end{equation*}
            \phantom{Maximum power is transferred to $R_L$ if:} % For alignment
        \end{minipage}

    \end{CheatsheetEntryFrame}

    % Should naturally break here

    \begin{CheatsheetEntryFrame}

        \newcommand{\MyReusableFormatting}[2]{%
            \begin{myminipage}[t]{0.82\columnwidth}
                \raggedright
                #1
            \end{myminipage}%
            %\hspace{0.06\columnwidth}% TODO: Do something better than this ghetto center-alignment.
            \begin{minipage}[t]{0.16\columnwidth}
                {\color{CheatsheetSepColor} \vrule{}}%
                #2
            \end{minipage}
        }
        \newcommand{\MyReusableFormattingB}{\path (-0.9,0) -- (0,0) (-0.65,-1) -- (0.65,-1);} % Ghetto alignment

        \MyReusableFormatting{%
            \CheatsheetEntryTitle{Ohm's Law}

            For resistance $R$ and conductance $G$: \\[2\parskip]

            \begin{tabularx}{\textwidth}{CcC}
                $\displaystyle v = iR$
                    & $\displaystyle G \triangleq \frac{1}{R}$
                    & $\displaystyle i = vG$\\
            \end{tabularx}
        }{%
            \begin{circuitikz}
                \MyReusableFormattingB
                \draw
                    (0,0) to[short] ++(0,-0.2)
                    to[short, i=$i$] ++(0,-0.1)
                    to[R, l=$R$, v=$v$] ++(0,-1.6)
                    to[short] ++(0,-0.3)
                ;
            \end{circuitikz}%
        }

        \CheatsheetEntryExtraSeparation

        \MyReusableFormatting{%
            \CheatsheetEntryTitle{Ideal Capacitors}
            \begin{equation*}
                \Exn{q = Cv}
                \Exn{\qquad \xRightarrow{\textstyle \phantom{.} \frac{\diff{}}{\diff{t}} \phantom{.}} \qquad}
                i = C \frac{\diff{v}}{\diff{t}}
            \end{equation*}

            %\CheatsheetSmallEquationTitle{Stored charge:}
            %\begin{equation*}
            %    q = Cv
            %\end{equation*}

            \vspace{-1ex}%
            {\footnotesize%
            \begin{tabular}{lcl}
                Steady-state DC       & $\to$ & Open Circuit \\
                On jump-discontinuity & $\to$ & Short Circuit \\[2.0ex]
                At high frequencies   & $\to$ & Short Circuit
            \end{tabular}
            }
        }{%
            \begin{circuitikz}
                \MyReusableFormattingB
                \draw
                    (0,0) to[short] ++(0,-0.2)
                    to[short, i=$i$] ++(0,-0.1)
                    to[C, l=$C$, v=$v$] ++(0,-1.6)
                    to[short] ++(0,-0.3)
                ;
            \end{circuitikz}%
        }

        \CheatsheetEntryExtraSeparation

        \MyReusableFormatting{%
            \CheatsheetEntryTitle{Ideal Inductors}
            \begin{equation*}
                %% As much as I wanted to put this in, it clutters it up so much.
                %\Exn{\MathOverLabel{\text{\ul{Faraday's Law}}}{
                %    \varepsilon = -N \frac{\diff{\Phi_B}}{\diff{t}} %= -N \frac{\diff{\Phi_B}}{\diff{i}} \frac{\diff{i}}{\diff{t}}
                %}}
                %\Exn{\qquad \xRightarrow{\textstyle \phantom{.} \phantom{\frac{\diff{}}{\diff{t}}} \phantom{.}} \qquad}
                %\MathOverLabel{\text{\Exn{\ul{Self-Induction}}}}{
                    v = L \frac{\diff{i}}{\diff{t}}
                %}
            \end{equation*}

            \vspace{-1ex}%
            {\footnotesize%
            \begin{tabular}{lcl}
                Steady-state DC       & $\to$ & Short Circuit \\
                On jump-discontinuity & $\to$ & Open Circuit \\[2.0ex]
                At high frequencies   & $\to$ & Open Circuit
            \end{tabular}
            }
        }{%
            \begin{circuitikz}
                \MyReusableFormattingB
                \draw
                    (0,0) to[short] ++(0,-0.2)
                    to[short, i=$i$] ++(0,-0.1)
                    to[L, l=$L$, v=$v$] ++(0,-1.6)
                    to[short] ++(0,-0.3)
                ;
            \end{circuitikz}%
        }

    \end{CheatsheetEntryFrame}

    \begin{CheatsheetEntryFrame}

        \CheatsheetEntryTitle{Series and Parallel Equivalent} \MarkSimilarToDC

        \vspace{1.5ex}
        \begin{minipage}[c]{0.5\columnwidth}
            \centering
            \scalebox{0.8}{
            \begin{circuitikz}
                \draw
                    (0,0)
                    to[generic, o-] ++(2,0)
                    to[generic, -o] ++(2,0)
                ;
            \end{circuitikz}%
            }
        \end{minipage}%
        \begin{minipage}[c]{0.5\columnwidth}
            \centering
            \scalebox{0.8}{
            \begin{circuitikz}
                \draw
                    (0.5,0) to[short, o-]
                    (1,0) -- (1, 0.4) to[generic] (3, 0.4) -- (3,0)
                    to[short, -o] (3.5,0)
                    (1,0) -- (1,-0.4) to[generic] (3,-0.4) -- (3,0)
                ;
            \end{circuitikz}%
            }
        \end{minipage}

        \vspace*{1.5ex}

        \begin{minipage}[c]{0.5\columnwidth}
            \begin{equation*}
                R_S = \sum{R_i}
            \end{equation*}
        \end{minipage}%
        \begin{minipage}[c]{0.5\columnwidth}
            \begin{equation*}
                \frac{1}{R_P} = \sum{\frac{1}{R_i}}
            \end{equation*}
        \end{minipage}

        \begin{minipage}[c]{0.5\columnwidth}
            \begin{equation*}
                \frac{1}{C_S} = \sum{\frac{1}{C_i}}
            \end{equation*}
        \end{minipage}%
        \begin{minipage}[c]{0.5\columnwidth}
            \begin{equation*}
                C_P = \sum{C_i}
            \end{equation*}
        \end{minipage}

        \begin{minipage}[c]{0.5\columnwidth}
            \begin{equation*}
                L_S = \sum{L_i}
            \end{equation*}
        \end{minipage}%
        \begin{minipage}[c]{0.5\columnwidth}
            \begin{equation*}
                \frac{1}{L_P} = \sum{\frac{1}{L_i}}
            \end{equation*}
        \end{minipage}

    \end{CheatsheetEntryFrame}

    \begin{CheatsheetEntryFrame}

        \CheatsheetEntryTitle{Voltage and Current Division}

        %\vspace*{1ex}
        %\begin{minipage}[c]{0.5\columnwidth}
        %    \centering
        %    \textbf{Voltage Division}
        %\end{minipage}%
        %\begin{minipage}[c]{0.5\columnwidth}
        %    \centering
        %    \textbf{Current Division}
        %\end{minipage}

        \begin{minipage}[c]{0.6\columnwidth}
            \centering
            \scalebox{1}{
            \begin{circuitikz}
                \draw
                    (0,0)
                    to[short, o-] ++(1,0)
                    -- ++(0,-0.2)
                    to[R, l_=$R_1$, v^=${\displaystyle v_1 = \frac{R_1}{R_1+R_2}v}$] ++(0,-1.5)
                    -- ++(0,-0.3)
                    to[R, l_=$R_2$, v^=${\displaystyle v_2 = \frac{R_2}{R_1+R_2}v}$] ++(0,-1.5)
                    -- ++(0,-0.2)
                    to[short, -o] ++(-1,0)
                    (0,0)
                    to[open, v=$v$] (0,-3.7)
                ;
            \end{circuitikz}%
            }
        \end{minipage}%
        \begin{minipage}[c]{0.4\columnwidth}
            \centering
            \scalebox{1}{
            \begin{circuitikz}
                \draw
                    (0,1.65)
                    to[short, i=$i$, o-] ++(1,0)
                    to[short] ++(0,-0.15)
                    to[R, l_=$R_1$, i>_=$i_1$] ++(0,-1.5)
                    %to[short] ++(0,-0.25)
                    to[short, -o] ++(-1,0)
                    (1,1.65)
                    to[short] ++(1,0)
                    to[short] ++(0,-0.15)
                    to[R, l_=$R_2$, i>_=$i_2$] ++(0,-1.5)
                    %to[short] ++(0,-0.25)
                    to[short] ++(-1,0)
                    (1,0) ++(0,-0.1)
                    node[below] {${\displaystyle i_1 = \frac{R_2}{R_1+R_2}i}$} ++(0,-1)
                    node[below] {${\displaystyle i_2 = \frac{R_1}{R_1+R_2}i}$}
                ;
            \end{circuitikz}%
            }
        \end{minipage}

    \end{CheatsheetEntryFrame}

\end{multicols}
\begin{multicols}{2}

    \begin{CheatsheetEntryFrame}

        \CheatsheetEntryTitle{Kirchoff's Current Law}

        The sum of signed currents into a node or closed boundary is zero.

        \begin{minipage}[c]{0.55\columnwidth}%
            \centering
            \begin{circuitikz}
                \path % Defining offsets makes the join point look a lot less awkward
                    (0,0) node (Node) {}
                    ++(-0.00,0) node (Offset2) {}
                    ++(-0.04,0) node (Offset1) {}
                    ++(-0.04,0) node (Offset0) {}
                    (Node)
                    ++(0.00,0) node (Offset3) {}
                    ++(0.04,0) node (Offset4) {}
                    ++(0.04,0) node (Offset5) {}
                ;
                \draw 
                    (Offset0) to[short] (Offset5)
                    ( 180:2) to[short, i>=\phantom{x}, color=myred ] (Offset0)
                    ( 160:2) to[short, i>=\phantom{x}, color=myred ] (Offset1)
                    ( 140:2) to[short, i>=\phantom{x}, color=myred ] (Offset2)
                    (  40:2) to[short, i<=\phantom{x}, color=myblue] (Offset3)
                    (  20:2) to[short, i<=\phantom{x}, color=myblue] (Offset4)
                    (   0:2) to[short, i<=\phantom{x}, color=myblue] (Offset5)
                    %( -20:2) to[short, i<=\phantom{x}] (Node)
                    ( -40:2) to[short, i<=\phantom{x}, color=myblue] (Offset3)
                    (-140:2) to[short, i>=\phantom{x}, color=myred ] (Offset2)
                    %(-160:2) to[short, i>=\phantom{x}] (Node)
                    ( 136:1) ++(0,0.2) node[above] {\color{myred}  $i_{i[1..n]}$}
                    (  44:1) ++(0,0.2) node[above] {\color{myblue} $i_{o[1..m]}$}
                    ( -14:1) node {\color{myblue} $\cdot$}
                    ( -20:1) node {\color{myblue} $\cdot$}
                    ( -26:1) node {\color{myblue} $\cdot$}
                    (-154:1) node {\color{myred}  $\cdot$}
                    (-160:1) node {\color{myred}  $\cdot$}
                    (-166:1) node {\color{myred}  $\cdot$}
                ;
            \end{circuitikz}%
        \end{minipage}%
        %\vrule{}%
        \begin{minipage}[c]{0.45\columnwidth}%
            \centering
            \phantom{x} % Phantom to fix broken alignment
            \begin{equation*}
                \memphR{\sum{i_i}} - \memphB{\sum{i_o}} = 0
            \end{equation*}

            \ExtraNotes{\footnotesize \textit{Assumes charge within the node or closed boundary is \myul{always} constant.}}
        \end{minipage}

        \CheatsheetEntryExtraSeparation

        %\CheatsheetEntryTitle{Nodal Analysis}
        %
        %% TODO: Write this section on Nodal Analysis!
        %
        %\CheatsheetEntryExtraSeparation

        \CheatsheetEntryTitle{Supernode (for Nodal Analysis)}

        A \textit{supernode} is formed from two non-reference nodes connected by a voltage source (dependent or independent).

        \begin{center}
        \begin{circuitikz}
            \path
                (0,0) coordinate (BL)
                ++(2,0) coordinate (BR)
                ++(0,1.4) coordinate (TR)
                ++(-2,0) coordinate (TL)
                (1,0.7) coordinate (Middle)

                (TR) ++(1.25,0.40) coordinate (Line0)
                ++(0,-1.00) coordinate (Line1)
                ++(0,-0.65) coordinate (Line2)
                ++(0,-0.65) coordinate (Line3)
            ;
            \draw[{myorange!30!white}, fill={myyellow!10!white}, line width=1.8pt]
                (Middle) ellipse (1.33 and 1.35)
            ;
            \draw 
                (BL) -- (TL)
                (BR) -- (TR)
                (BL) to[V, color=myblue] (BR)
                (TL) to[generic] (TR)
                %(TL) -- ++(0.4,0)
                %    ++(0,0.2) node (Generic1TL) {}
                %(TR) -- ++(-0.4,0)
                %    ++(0,-0.2) node (Generic1BR) {}
                %(Generic1TL) rectangle (Generic1BR)

                (BL) ++(1,0) ++(0,0.65) node {\color{myblue} $v_s$}

                (TL) -- ++(-0.3, 0  ) coordinate (TLL)
                (TL) -- ++( 0  , 0.3) coordinate (TLT)
                (TR) -- ++( 0  , 0.3) coordinate (TRT)
                (TR) -- ++( 0.3, 0  ) coordinate (TRR)
                (BR) -- ++( 0.3, 0  ) coordinate (BRR)
                (BR) -- ++( 0  ,-0.3) coordinate (BRB)
                (BL) -- ++( 0  ,-0.3) coordinate (BLB)
                (BL) -- ++(-0.3, 0  ) coordinate (BLL)
            ;
            %\draw[mygreen, line width=1.5pt] % Alternative thicker coloured generic
            %    (Generic1TL) rectangle (Generic1BR)
            %;
            \draw[gray, line cap=round, dash pattern=on 0.5mm off 0.5mm]
            %\draw[lightgray]
            %\draw[dotted]
            %\draw[line cap=round, dash pattern=on 0.1mm off 0.5mm]
                (TLL) ++(-0.5mm, 0    ) -- ++(-0.3, 0  )
                (TLT) ++( 0    , 0.5mm) -- ++( 0  , 0.3)
                (TRT) ++( 0    , 0.5mm) -- ++( 0  , 0.3)
                (TRR) ++( 0.5mm, 0    ) -- ++( 0.3, 0  )
                (BRR) ++( 0.5mm, 0    ) -- ++( 0.3, 0  )
                (BRB) ++( 0    ,-0.5mm) -- ++( 0  ,-0.3)
                (BLB) ++( 0    ,-0.5mm) -- ++( 0  ,-0.3)
                (BLL) ++(-0.5mm, 0    ) -- ++(-0.3, 0  )
            ;
            \draw[myblue, fill=myblue]
                (BL) node[circ, color=myblue] {}
                (BL) ++(-0.30,-0.05) node[below] {$v_1$}
            ;
            \draw[myblue, fill=myblue]
                (BR) node[circ, color=myblue] {}
                (BR) ++(0.30,-0.05) node[below] {$v_2$}
            ;
            \draw[extranotecolor] % Text
                (Line0) node[right, align=left] {Components in parallel\\are frequently ignored.}
            ;
            \draw % Text
                (Line1) node[right] {\textbf{Forms two equations:}}
                (Line2) ++(0.7,0) node[left] {1)}
                    ++(-0.1,0) node[right] {KCL on supernode}
                (Line3) ++(0.7,0) node[left] {2)}
                    ++(-0.1,0) node[right] {$v_s = v_1 - v_2$}
            ;
            \path % Get coordinates for bezier curve arrows
                (Line0)
                %++(-0.1,0.23) coordinate (ArrowStart0)
                ++(-0.1,0) coordinate (ArrowStart0)
                ++(-0.8,0) coordinate (ArrowControl0)
                (TR) ++(-0.52,0.18) ++(0.05,0.05) coordinate (ArrowEnd0)

                (Line2)
                ++(0.1,0) coordinate (ArrowStart1)
                ++(-0.5,0) coordinate (ArrowControl1)
                (BR) ++(0.35,0.30) coordinate (ArrowEnd1)

                (Line3)
                ++(0.1,-0.12) coordinate (ArrowStart2)
                ++(-1.15,-0.35) coordinate (ArrowControl2)
                (BL) ++(1,0) ++(0.35,-0.35) coordinate (ArrowEnd2)
            ;
            \draw[mygreen, ->, line width=1.7pt, line cap=round] % Draw bezier curve arrow
                (ArrowStart0) .. controls (ArrowControl0) .. (ArrowEnd0)
            ;
            \draw[myred, ->, line width=1.7pt, line cap=round] % Draw bezier curve arrow
                (ArrowStart1) .. controls (ArrowControl1) .. (ArrowEnd1)
            ;
            \draw[myblue, ->, line width=1.7pt, line cap=round] % Draw bezier curve arrow
                (ArrowStart2) .. controls (ArrowControl2) .. (ArrowEnd2)
            ;
        \end{circuitikz}%
        \end{center}

        % TODO: Can you generalize to >2 non-reference nodes?

    \end{CheatsheetEntryFrame}

    \begin{CheatsheetEntryFrame}

        \CheatsheetEntryTitle{Kirchoff's Voltage Law}

        The sum of signed voltages around any closed path is zero.

        \begin{minipage}[c]{0.65\columnwidth}
            \centering
            \begin{circuitikz}
                \draw 
                    (0,0) node (Origin) {}
                    -- ++(0,0.2) to[generic, v^=$v_1$] ++(0,1.8) -- ++(0,0.2)
                    -- ++(0.2,0) to[generic, v^=$v_2$] ++(1.8,0) -- ++(0.2,0)
                    -- ++(0.2,0) ++(0.5,0) node {$\cdots$} ++(0.5,0) -- ++(0.2,0)
                    %-- ++(0.2,0) to[generic, v^=$v_{n-1}$] ++(1.8,0) -- ++(0.2,0) % Use this line for the (n-1)th component
                    -- ++(0.2,0) % Use this line if no (n-1)th component
                    -- ++(0,-0.2) to[generic, v^=$v_n$] ++(0,-1.8) -- ++(0,-0.2)
                    coordinate (BR)
                    to[short] (Origin)
                ;
                %\path
                %    (BR)
                %    ++(2.3,1) coordinate (Equation)
                %    ++(2.1,0) % Extra tail for alignment
                %;
                %\draw
                %    (Equation) node {$\displaystyle \sum{v_k} = 0$}
                %;
            \end{circuitikz}%
        \end{minipage}%
        %\vrule{}
        \begin{minipage}[c]{0.35\columnwidth}%
            \centering
            \phantom{x} % Phantom to fix broken alignment
            \begin{equation*}
                \sum{v_k} = 0
            \end{equation*}
            \phantom{x} % Phantom to fix broken alignment

            % Write about assumptions made?
            %\ExtraNotes{\footnotesize \textit{Assumes that...}}
        \end{minipage}

        \CheatsheetEntryExtraSeparation

        \CheatsheetEntryTitle{Supermesh (for Mesh Analysis)}

        A \textit{supermesh} is formed from two adjacent meshes with a common current source (dependent or independent).

        %\begin{center}
        \begin{circuitikz}
            \path
                (0,0) coordinate (BL)
                ++( 2.2,0  ) coordinate (BR)
                ++( 0  ,2.2) coordinate (MR)
                ++(-1.1,0  ) coordinate (MM)
                ++(-1.1,0  ) coordinate (ML)
                ++( 0  ,2.2) coordinate (TL)
                ++( 2.2,0  ) coordinate (TR)

                (TR) ++(0.94,-0.60) coordinate (Line0)
                ++(0,-1.30) coordinate (Line1)
                ++(0,-0.65) coordinate (Line2)
                ++(0,-0.65) coordinate (Line3)

                (MM) ++(0,1.3) coordinate (MeshCurrent1)
                (BL) ++(0.9,0.7) coordinate (MeshCurrent2)

                (TL) ++(-0.38, 0.38) coordinate (RectangleTL)
                (BR) ++( 0.38,-0.38) coordinate (RectangleBR)
            ;
            \draw[{myorange!30!white}, fill={myyellow!10!white}, line width=1.8pt, rounded corners=2.5mm]
                (RectangleTL) rectangle (RectangleBR)
            ;
            \draw
                (BL)
                to[generic] (ML)
                to[generic] (TL)
                to[generic] (TR)
                to[generic] (MR)
                to[generic] (BR)
                to[generic] (BL)
            ;
            \draw 
                (MM) to[I, l_={\color{myblue} $i_s$}, color=myblue, /tikz/circuitikz/bipoles/length=1.2cm] (ML)
                (MM) to[generic] ++(0.9,0) -- (MR)

                (TL) -- ++(-0.58, 0   ) coordinate (TLL)
                (TL) -- ++( 0   , 0.58) coordinate (TLT)
                (TR) -- ++( 0   , 0.58) coordinate (TRT)
                (TR) -- ++( 0.58, 0   ) coordinate (TRR)
                (MR) -- ++( 0.58, 0   ) coordinate (MRR)
                (BR) -- ++( 0.58, 0   ) coordinate (BRR)
                (BR) -- ++( 0   ,-0.58) coordinate (BRB)
                (BL) -- ++( 0   ,-0.58) coordinate (BLB)
                (BL) -- ++(-0.58, 0   ) coordinate (BLL)
                (ML) -- ++(-0.58, 0   ) coordinate (MLL)
            ;
            \draw[gray, line cap=round, dash pattern=on 0.5mm off 0.5mm]
            %\draw[lightgray]
            %\draw[dotted]
            %\draw[line cap=round, dash pattern=on 0.1mm off 0.5mm]
                (TLL) ++(-0.5mm, 0    ) -- ++(-0.18, 0   )
                (TLT) ++( 0    , 0.5mm) -- ++( 0   , 0.18)
                (TRT) ++( 0    , 0.5mm) -- ++( 0   , 0.18)
                (TRR) ++( 0.5mm, 0    ) -- ++( 0.18, 0   )
                (MRR) ++( 0.5mm, 0    ) -- ++( 0.18, 0   )
                (BRR) ++( 0.5mm, 0    ) -- ++( 0.18, 0   )
                (BRB) ++( 0    ,-0.5mm) -- ++( 0   ,-0.18)
                (BLB) ++( 0    ,-0.5mm) -- ++( 0   ,-0.18)
                (BLL) ++(-0.5mm, 0    ) -- ++(-0.18, 0   )
                (MLL) ++(-0.5mm, 0    ) -- ++(-0.18, 0   )
            ;
            \draw[extranotecolor] % Text
                (Line0) node[right, align=left] {Components in series\\with current source\\are frequently ignored.}
            ;
            \draw % Text
                (Line1) node[right] {\textbf{Forms two equations:}}
                (Line2) ++(0.7,0) node[left] {1)}
                    ++(-0.1,0) node[right] {KVL around supermesh}
                (Line3) ++(0.7,0) node[left] {2)}
                    ++(-0.1,0) node[right] {$i_s = i_1 + i_2$}
            ;
            \path % Get coordinates for bezier curve arrows
                (Line0)
                ++(-0.1,0) coordinate (ArrowStart0)
                ++(-0.6,0) coordinate (ArrowControl0)
                (MR) ++(-0.5,0.3) coordinate (ArrowEnd0)

                (Line2)
                ++(0.1,0) coordinate (ArrowStart1)
                %++(-0.5,0) coordinate (ArrowControl1)
                (ArrowStart1 -| RectangleBR) ++(0.1,0) coordinate (ArrowEnd1)

                (Line3)
                ++(0.1,-0) coordinate (ArrowStart2)
                ++(-1.15,0) coordinate (ArrowControl2)
                (MM) ++(-0.55,0) ++(0.35,-0.35) coordinate (ArrowEnd2)
            ;
            \draw[mygreen, ->, line width=1.7pt, line cap=round] % Draw bezier curve arrow
                (ArrowStart0) .. controls (ArrowControl0) .. (ArrowEnd0)
            ;
            \draw[myred, ->, line width=1.7pt, line cap=round] % Draw arrow
                (ArrowStart1) -- (ArrowEnd1)
            ;
            \draw[myblue, ->, line width=1.7pt, line cap=round] % Draw bezier curve arrow
                (ArrowStart2) .. controls (ArrowControl2) .. (ArrowEnd2)
            ;
            \draw[myblue]
                (MeshCurrent1) pic {simshadows/ee/meshcw} node {$i_1$}
                (MeshCurrent2) pic {simshadows/ee/meshcc} node {$i_2$}
            ;
        \end{circuitikz}%
        %\end{center}

        % TODO: Can you generalize to >2 meshes?

    \end{CheatsheetEntryFrame}

    \MulticolsBreak

    \begin{CheatsheetEntryFrame}

        \CheatsheetEntryTitle{Source Transformation}

        \newcommand{\MyReusableFormatting}[2]{%
            \begin{center}
            \scalebox{0.94}{%
            \begin{circuitikz}
                \path
                    (0,0)      coordinate (LeftBR)
                    ++(3.3,0)  coordinate (RightBL)
                    ++(0,0.75) coordinate (MidHorizontal)
                ;
                \draw
                    (LeftBR)
                    to[short, o-] ++(-2.0,0)
                    to[#1, l=$v_s$, invert, /tikz/circuitikz/bipoles/length=1.25cm] ++(0,1.5)
                    to[R, l=$R$, -o] ++(2.0,0)
                    ++(0.1,0) node[right] {$a$}
                    (LeftBR) ++(0.1,0) node[right] (LeftEdgeRef) {$b$}
                ;
                \draw
                    (RightBL)
                    to[short, -o] ++(2.0,0)
                    ++(0.1,0) node[right] {$b$}

                    (RightBL) to[#2, name=CS, l=$i_s$, /tikz/circuitikz/bipoles/length=1.25cm] ++(0,1.5)
                    to[short, -o] ++(2.0,0)
                    ++(0.1,0) node[right] {$a$}

                    (RightBL) ++(1.2,0)
                    to[R, l=$R$] ++(0,1.5)
                ;
                \path % Calculates middle point.
                    (CS.north) ++(-0.5,0) coordinate (RightEdgeRef) % Ghetto calculation of right edge ref coordinate
                    ($(LeftEdgeRef.east)!0.5!(RightEdgeRef)$) coordinate (MidVertical)
                    (MidVertical |- MidHorizontal) coordinate (Mid)
                ;
                %\draw % Ghetto version
                %    (Mid) ++(0,0.15) node {\large $\xLeftrightarrow{\text{equivalent}}$}
                %;
                \draw[latex-latex, thick, line width=2.0pt]
                    (Mid) ++(-0.6,0)
                    --    ++( 1.2,0)
                ;
                \draw
                    %(Mid) ++(0,0.05) node[above] {equivalent}
                    %(Mid) ++(0,-0.08) node[below] {iff}
                    %      ++(0,-0.40) node[below] {$v_s = i_s R$}
                    (Mid) ++(0,0.09) node[above] {$v_s = i_s R$}
                    (Mid) ++(0,-0.09) node[below, align=center] {$R \notin \braces*{0, \infty}$}
                ;
            \end{circuitikz}
            }
            \end{center}
        }

        \MyReusableFormatting{V}{I}

        Dependent sources work exactly the same.

        %\MyReusableFormatting{cV}{cI}

        \CheatsheetEntryExtraSeparation

        \CheatsheetEntryTitle{Th\'evenin's Theorem}

        \begin{center}
        \scalebox{0.94}{%
        \begin{circuitikz}
            \path
                (0,0)      coordinate (LeftBR)
                ++(2.9,0)  coordinate (RightBL)
                ++(0,0.75) coordinate (MidHorizontal)
            ;
            \draw
                (LeftBR)
                to[short, o-] ++(-0.4,0)
                ++(0,1.5) coordinate (BoxTRref)
                to[short, -o] ++(0.4,0)
                ++(0.1,0) node[right] {$a$}
                (LeftBR) ++(0.1,0)
                node[right] (LeftEdgeRef) {$b$}
            ;
                \draw[simshadows/style/softgray]
                (BoxTRref)        ++( 0  , 0.3) coordinate (BoxTR)
                (LeftBR) ++(-2,0) ++(-1.1,-0.3) coordinate (BoxBL)
                (BoxBL) rectangle (BoxTR)
            ;
            \draw
                ($(BoxTR)!0.5!(BoxBL)$) coordinate (BoxMid)
                (BoxMid) node[align=center] {arbitrary\\linear network\\of sources\\and resistances}
            ;
            \draw
                (RightBL)
                to[short, -o] ++(2.0,0)
                ++(0.1,0) node[right] {$b$}

                (RightBL) to[V, name=S, l=$v_{th}$, invert, /tikz/circuitikz/bipoles/length=1.25cm] ++(0,1.5)
                to[R, l=$R_{th}$, -o] ++(2.0,0)
                ++(0.1,0) node[right] {$a$}
            ;
            \path % Calculates middle point.
                (S.north) ++(-0.5,0) coordinate (RightEdgeRef) % Ghetto calculation of right edge ref coordinate
                ($(LeftEdgeRef.east)!0.5!(RightEdgeRef)$) coordinate (MidVertical)
                (MidVertical |- MidHorizontal) coordinate (Mid)
            ;
            \draw[-latex, thick, line width=2.5pt]
                (Mid) ++(-0.5,0)
                --    ++( 1.0,0)
            ;
            %\draw
            %    (Mid) ++(0,0.09) node[above] {equivalent}
            %;
        \end{circuitikz}
        }
        \end{center}
        \vspace*{-2ex} % So much unnecessary whitespace.
        \begin{align*}
            v_{th} &= \text{open-circuit voltage at $a$-$b$} \\
            R_{th} &= \text{input resistance (looking into $a$-$b$)}
        \end{align*}

        Finding Th\'evenin resistance $R_{th}$:
        \begin{psmallindent}
            % TODO: Why does \textbu{\textsc{}} not give proper smallcaps here?
            {\footnotesize \textbu{\textsc{Method 1 (no dependent sources in network):}}}
            \begin{enumerate}
                \item DISABLE all independent sources.
                \item Find equivalent resistance looking into $a$-$b$.
            \end{enumerate}

            {\footnotesize \textbu{\textsc{Method 2:}}}
            \begin{enumerate}
                \item KEEP all independent sources.
                \item $\displaystyle R_{th} = \frac{\text{open-circuit voltage at $a$-$b$}}{\text{short-circuit current from $a$ to $b$}} = \frac{v_{th}}{i_N}$
            \end{enumerate}

            {\footnotesize \textbu{\textsc{Method 3:}}}
            \begin{enumerate}
                \item DISABLE all independent sources.
                \item Attach an independent source stimulus (usually $1 \si{\volt}$ or $1 \si{\ampere}$) to $a$-$b$.
                    \scalebox{0.85}{%
                    \begin{circuitikz}
                        \path
                            (0,0) coordinate (LeftBoxBRref)
                            ++(3.8,0) coordinate (RightBoxBRref)

                            (LeftBoxBRref)  ++(-0.7,0  ) ++(-0.3,-0.3) coordinate (LeftBoxBL)
                            (LeftBoxBRref)  ++( 0  ,1.5) ++( 0  , 0.3) coordinate (LeftBoxTR)
                            (RightBoxBRref) ++(-0.7,0  ) ++(-0.3,-0.3) coordinate (RightBoxBL)
                            (RightBoxBRref) ++( 0  ,1.5) ++( 0  , 0.3) coordinate (RightBoxTR)
                        ;
                        \draw
                            (LeftBoxBRref)
                            to[short, -o] ++( 0.4,0) coordinate (LeftTermB)
                            to[short]     ++( 0.4,0)
                            to[V, l_={\large $V_\text{test}$}, invert, /tikz/circuitikz/bipoles/length=1.20cm] ++(0,1.2)
                            to[short, i_={\large $I_\text{response}$}]                                         ++(0,0.1)
                            to[short]                                                                          ++(0,0.2)
                            to[short]     ++(-0.4,0) coordinate (LeftTermA)
                            to[short, o-] ++(-0.4,0)

                            (LeftTermA) ++(0, 0.1) node[above] {\large $a$}
                            (LeftTermB) ++(0,-0.1) node[below] {\large $b$}
                        ;
                        \draw
                            (RightBoxBRref)
                            to[short, -o] ++( 0.4,0) coordinate (RightTermB)
                            to[short]     ++( 0.4,0) coordinate (RightBR)
                            to[I, l_={\large $I_\text{test}$}, /tikz/circuitikz/bipoles/length=1.20cm] ++(0,1.2)
                            to[short]                                                                  ++(0,0.1)
                            to[short]                                                                  ++(0,0.2)
                            to[short]     ++(-0.4,0) coordinate (RightTermA)
                            to[short, o-] ++(-0.4,0)

                            (RightTermA) ++(0, 0.1) node[above] {\large $a$}
                            (RightTermB) ++(0,-0.1) node[below] {\large $b$}
                        ;
                        \draw[simshadows/style/softgray]
                            (LeftBoxBL) rectangle (LeftBoxTR)
                        ;
                        \draw[simshadows/style/softgray]
                            (RightBoxBL) rectangle (RightBoxTR)
                        ;
                        % Everything below will be for the voltage indicator
                        \draw[->, line width=1.3pt]
                            (RightBR)
                            ++(1.7,0)    coordinate (VArrowBot)
                            -- ++(0,1.5) coordinate (VArrowTop)
                        ;
                        \draw[dashed]
                            (VArrowTop) ++(-1.2,0) -- ++(1.7,0)
                            (VArrowBot) ++(-1.2,0) -- ++(1.7,0)
                        ;
                        \draw
                            %($(VArrowTop)!0.5!(VArrowBot)$) ++(0.1,0) node[right] {$V_\text{response}$}
                            (VArrowTop) ++(0,0.1) node[above] {\large $V_\text{response}$}
                        ;
                    \end{circuitikz}
                    }
                \item $\displaystyle R_{th} = \frac{V_{\text{test}}}{I_{\text{response}}} = \frac{V_{\text{response}}}{I_{\text{test}}}$
            \end{enumerate}

        \end{psmallindent}

        \CheatsheetEntryExtraSeparation

        \CheatsheetEntryTitle{Norton's Theorem}

        \begin{center}
        \scalebox{0.94}{%
        \begin{circuitikz}
            \path
                (0,0)      coordinate (LeftBR)
                ++(2.9,0)  coordinate (RightBL)
                ++(0,0.75) coordinate (MidHorizontal)
                -- (0,2.1) % Ghetto way to align similarly to the diagram of Thevenin's Theorem.
            ;
            \draw
                (LeftBR)
                to[short, o-] ++(-0.4,0)
                ++(0,1.5) coordinate (BoxTRref)
                to[short, -o] ++(0.4,0)
                ++(0.1,0) node[right] {$a$}
                (LeftBR) ++(0.1,0)
                node[right] (LeftEdgeRef) {$b$}
            ;
            \draw[simshadows/style/softgray]
                (BoxTRref)        ++( 0  , 0.3) coordinate (BoxTR)
                (LeftBR) ++(-2,0) ++(-1.1,-0.3) coordinate (BoxBL)
                (BoxBL) rectangle (BoxTR)
            ;
            \draw
                ($(BoxTR)!0.5!(BoxBL)$) coordinate (BoxMid)
                (BoxMid) node[align=center] {arbitrary\\linear network\\of sources\\and resistances}
            ;
            \draw
                (RightBL)
                to[short, -o] ++(2.0,0)
                ++(0.1,0) node[right] {$b$}

                (RightBL) to[I, name=S, l=$i_N$, /tikz/circuitikz/bipoles/length=1.25cm] ++(0,1.5)
                to[short, -o] ++(2.0,0)
                ++(0.1,0) node[right] {$a$}

                (RightBL) ++(1.4,0)
                to[R, l=$R_N$] ++(0,1.5)
            ;
            \path % Calculates middle point.
                (S.north) ++(-0.5,0) coordinate (RightEdgeRef) % Ghetto calculation of right edge ref coordinate
                ($(LeftEdgeRef.east)!0.5!(RightEdgeRef)$) coordinate (MidVertical)
                (MidVertical |- MidHorizontal) coordinate (Mid)
            ;
            \draw[-latex, thick, line width=2.5pt]
                (Mid) ++(-0.5,0)
                --    ++( 1.0,0)
            ;
            %\draw
            %    (Mid) ++(0,0.09) node[above] {equivalent}
            %;
        \end{circuitikz}
        }
        \end{center}
        \vspace*{-2ex} % So much unnecessary whitespace.
        \begin{align*}
            i_N &= \text{short-circuit current from $a$ to $b$} \\
            R_N &= R_{th}
        \end{align*}

    \end{CheatsheetEntryFrame}

\end{multicols}
\begin{multicols}{2}

    \begin{CheatsheetEntryFrame}

        \CheatsheetEntryTitle{Superposition Principle (in DC Analysis)}

        In a linear circuit, the total effect of several independent sources acting simultaneously is equal to the sum of each independent source acting alone \textit{(i.e. only one independent source active, with all other independent sources disabled)}.

        \textsc{do not disable dependent sources!}

        \textsc{also, this only works for linear responses, so superposition does not work with power.}

    \end{CheatsheetEntryFrame}

    \begin{CheatsheetEntryFrame}

        \newcommand{\MyReusableFormatting}{
            \path (0,0) -- (0,1.6); % Ghetto alignment
            \path
                (0,0)      coordinate (N)
                (150:1.70) coordinate (A)
                ( 30:1.70) coordinate (B)
                (-90:1.70) coordinate (C)
            ;
            \draw
                (A) ++(150:0.3) node {$a$}
                (B) ++( 30:0.3) node {$b$}
                %(A) ++( 90:0.3) node {$a$} % These require less horizontal space, but look worse
                %(B) ++( 90:0.3) node {$b$}
                (C) ++(-90:0.3) node {$c$}
            ;
        }

        \CheatsheetEntryTitle{$\mathrm{Y}$-$\Delta$ Transform}

        \TwoColumnsTextSeparated{$\Longleftrightarrow$}{
            \begin{circuitikz}
                \MyReusableFormatting
                \draw
                    (N) to[R, l_=$\memphR{R_1}$, name=Z1, color=myred, /tikz/circuitikz/bipoles/thickness=4] (A) node[ocirc] {}
                    (N) to[R, l=$R_2$, name=Z2] (B) node[ocirc] {}
                    (N) to[R, l=$R_3$, name=Z3] (C) node[ocirc] {}
                    (N) node[circ] {}
                    (N) ++(-30:0.3) node {$n$}
                ;
            \end{circuitikz}
        }{
            \begin{circuitikz}
                \MyReusableFormatting
                \draw
                    (A)
                    to[R, l=$R_c$, name=ZB] (B) node[circ] {}
                    to[R, l=$\memphB{R_a}$, name=ZC, color=myblue, /tikz/circuitikz/bipoles/thickness=4] (C) node[circ] {}
                    to[R, l=$R_b$, name=ZA] (A) node[circ] {}
                ;
            \end{circuitikz}
        }
        \TwoColumnsTextSeparated{}{
            \begin{equation*}
                \memphR{R_1} = \frac{R_b R_c}{R_a + R_b + R_c}
            \end{equation*}
        }{
            \begin{equation*}
                \memphB{R_a}
                    = \frac{R_1 R_2 + R_2 R_3 + R_3 R_1}{R_1}
            \end{equation*}
        }

        For $\mathrm{Y}$ and $\Delta$ loads to be balanced:
        \TwoColumnsTextSeparated{\phantom{$\Longleftrightarrow$}}{%
            \begin{equation*}
                R_Y = R_1 = R_2 = R_3
            \end{equation*}
        }{%
            \begin{equation*}
                R_\Delta = R_a = R_b = R_c
            \end{equation*}
        }

        So when our $\mathrm{Y}$ or $\Delta$ circuit is balanced:
        \begin{equation*}
            R_\Delta = 3 R_Y
        \end{equation*}

    \end{CheatsheetEntryFrame}

\end{multicols}

%%%%%%%%%%%%%%%%%%%%%%%%%%%%%%%%%%%%%%%%%%%%%%%%%%%%%%%%%%%%%%%%%%%%%%%%%%%%%%%%%%%%%%%%%%%%%%%%%%%%
%%%%%%%%%%%%%%%%%%%%%%%%%%%%%%%%%%%%%%%%%%%%%%%%%%%%%%%%%%%%%%%%%%%%%%%%%%%%%%%%%%%%%%%%%%%%%%%%%%%%

\newpage
\subsection{Sinusoidal Steady-State Analysis}%
\label{sub:sinusoidal-steady-state-analysis}

\begin{multicols}{2}

    \begin{CheatsheetEntryFrame}

        \CheatsheetEntryTitle{Phasor Domain Conversion}

        \newcommand{\X}{\hphantom{x}} % Adding horizontal space to make the table distribute better and look nicer
        \begin{tabularx}{\textwidth}{ccC}
            {\scriptsize \textbf{Time Domain}}   &                             & {\scriptsize \textbf{Phasor Domain}} \\
            \X $v(t) = V_m \cos{(\omega t + \phi)}$ \X & $\Longleftrightarrow$ & $\mathbf{V} = V_m \phase{\phi}$      \\
            \X $i(t) = I_m \cos{(\omega t + \phi)}$ \X & $\Longleftrightarrow$ & $\mathbf{I} = I_m \phase{\phi}$      \\
        \end{tabularx}

        %\begin{tikzpicture}[scale=0.85, transform shape]
        %    \draw[help lines, lightgray]
        %        (0,-1) grid (4,1)
        %    ;
        %    \draw[-latex, line width=1.5pt]
        %        (0,0) -- (4.4,0) coordinate (LabelX)
        %    ;
        %    \draw[myred, dotted, line width=1.2pt, domain=0:4, samples=100]
        %        plot (\x, {cos(0.5*pi*\x r)}) node[right] {\color{mylightred} $\cos{(\omega t)}$}
        %    ;
        %    \draw[myred, line width=2pt, domain=0:4, samples=100]
        %        plot (\x, {cos(0.5*pi*\x r + 70)}) node[right] {$\cos{(\omega t + \phi)}$}
        %    ;
        %    \draw[latex-latex, line width=1.5pt] % Draw vertical line over the curve cuz it looks nicer
        %        (0,-1.4) -- (0,1.4) coordinate (LabelY)
        %    ;
        %    \draw
        %        (LabelX) ++(0  ,-0.1) node[below] {$t$}
        %        (LabelY) ++(0.1, 0  ) node[right] {$x$}
        %    ;
        %\end{tikzpicture}

        \vspace*{1ex}

        \renewcommand{\W}{60} % Phasor angle.
        % IMPORTANT: We will need to manually calculate WaveActualIntercept.
        %            To make it easy, just input this into WolframAlpha:
        %                2*(arccos(0) + (- YOUR_PHASOR_ANGLE_HERE * pi / 180))/pi + 2
        \begin{center}
        \begin{tikzpicture}[x=1.72cm, y=1.72cm, transform shape]
            \begin{scope}[shift={(0,-1.8)}, rotate=-90]
                \path
                    (0,0) coordinate (WaveOrigin)
                    (3,0) coordinate (WaveReferenceIntercept) % Manually calculated.
                    (2.33333333,0) coordinate (WaveActualIntercept) % Manually calculated.
                    % TODO: Maybe automatically calculate these intercepts next time?
                ;
                %\draw[help lines, lightgray]
                %    (0,-1) grid (4,1)
                %;
                \draw[-latex, line width=1.5pt]
                    (0,0) -- (4.4,0) coordinate (LabelX1)
                ;
                \draw[myred, dashed, line width=1.2pt, domain=0:4, samples=100, line cap=round]
                    plot (\x, {cos(0.5*pi*\x r)}) node[right] {\color{myred} $\bm{V_m \cos{(\omega t)}}$}
                ;
                \draw[myred, line width=2pt, domain=0:4, samples=100, line cap=round]
                    plot (\x, {cos(0.5*pi*\x r + \W)}) node[right] {$\bm{V_m \cos{(\omega t + \phi)}}$}
                    coordinate (WaveEnd)
                ;
                \draw[latex-latex, line width=1.5pt] % Draw vertical line over the curve cuz it looks nicer
                    (0,-1.4) -- (0,1.4) coordinate (LabelY1)
                ;
                \draw
                    (LabelX1) ++(0  ,-0.1) node[below] {$\omega t$}
                    (LabelY1) ++(0.1, 0  ) node[right] {$v(t)$}
                ;
                \draw[mypurple]
                    (WaveOrigin) ++(0, 1) ++(0,0.17) node[left] {$\bm{V_m}$}
                    (WaveOrigin) ++(0,-1) ++(0,0.17) node[left] {$\bm{-V_m}$}
                ;
            \end{scope}
            \begin{scope}[shift={(0,0)}]
                \draw[help lines, verylightgray, line width=1.5pt]
                    (0,0) coordinate (PhasorOrigin) circle[radius=1]
                    (1,0) coordinate (CircleRight)
                    (-1,0) coordinate (CircleLeft)
                ;
                \draw[latex-latex, line width=1.5pt]
                    (-1.4,0) -- (1.4,0) coordinate (LabelX2)
                ;
                \draw[latex-latex, line width=1.5pt]
                    (0,-1.4) -- (0,1.4) coordinate (LabelY2)
                ;
                \draw[-latex, myred, line width=2pt, line cap=round]
                    (PhasorOrigin) -- ++(\W:1) coordinate (PhasorEnd)
                ;
                \draw
                    (LabelX2) ++(-0.1 ,-0.05) node[below] {$\MyRe$}
                    (LabelY2) ++( 0.05,-0.1 ) node[right] {$\MyIm$}
                ;
                \draw[mypurple] (30:0.25) node {$\bm{\phi}$};
                \draw[-latex, mypurple, line width=1.2pt, line cap=round] (0:0.40) arc (0:\W:0.40);
                \draw[myred] (PhasorEnd) ++(\W:0.15) node {$\mathbf{V}$};
                % Directions
                \draw[angle 60 reversed-angle 60, mygreen,  line width=1.7pt, line cap=round] ( 10:1.8) arc ( 10: 40:1.8);
                \draw[angle 60 reversed-angle 60, myblue,   line width=1.7pt, line cap=round] (-10:1.8) arc (-10:-40:1.8);
                \draw[angle 60 reversed-angle 60, mypurple, line width=1.7pt, line cap=round] ( 60:1.8) arc ( 60:120:1.8);
                \draw[mygreen]  ( 20:1.8) node[above right, align=left] {\textbf{leading}\\\textbf{direction}};
                \draw[myblue]   (-20:1.8) node[below right, align=left] {\textbf{lagging}\\\textbf{direction}};
                \draw[mypurple] ( 90:1.8) node[above] {\textbf{angular velocity} $\bm{\omega}$};
                %\draw (0,0) circle[radius=2.5]; % Ghetto alignment
            \end{scope}
            \begin{scope}[on background layer]
                %\draw[myblue, dashed, line width=1.5pt] % Old styling for the phasor connector
                \draw[help lines, verylightgray, line width=1.5pt]
                    (PhasorEnd) -- (PhasorEnd |- WaveOrigin)
                ;
                \draw[help lines, verylightgray, line width=1.5pt]
                    (CircleRight)  -- (CircleRight  |- WaveEnd)
                    (CircleLeft)   -- (CircleLeft   |- WaveEnd)
                    (PhasorOrigin |- WaveEnd) ++(-1.4,0) -- ++(2.8,0)
                    %(PhasorOrigin) -- (PhasorOrigin |- WaveOrigin) % Looks ugly
                ;
                \draw[help lines, {mypurple!30!white}, line width=1.5pt]
                    (WaveActualIntercept)    ++(-1.4,0) -- ++(2.8,0) -- ++(0.5,0) ++(-0.2,-0.00) coordinate (WavePhaseDiffEnd)
                ;
                \draw[help lines, {mypurple!30!white}, dashed, line width=1.5pt]
                    (WaveReferenceIntercept) ++(-1.4,0) -- ++(2.8,0) -- ++(0.5,0) ++(-0.2, 0.00) coordinate (WavePhaseDiffStart)
                ;
            \end{scope}
            \begin{scope}[rotate=-90]
                \draw[-angle 60, mypurple, line width=1.7pt]
                    (WavePhaseDiffStart) -- (WavePhaseDiffEnd)
                ;
                \draw[mypurple]
                    ($(WavePhaseDiffStart)!0.5!(WavePhaseDiffEnd)$) coordinate (WavePhaseDiffMid)
                    (WavePhaseDiffMid) ++(0,0.05) node[above] {$\bm{\phi}$}
                    %(WavePhaseDiffStart) ++(0.05,0) node[right, align=left] {\textbf{phase}} % Ugly
                    %(WavePhaseDiffEnd) ++(-0.05,0) node[left, align=right] {\textbf{phase}} % Also ugly
                    (WavePhaseDiffMid) ++(0,0.35) node[above, align=center] {\textbf{phase}}
                ;
                \draw[angle 60 reversed-angle 60, mygreen, line width=1.7pt, line cap=round]
                    (WavePhaseDiffEnd)   ++(-0.6,0) -- ++(-1,0);
                \draw[angle 60 reversed-angle 60, myblue,  line width=1.7pt, line cap=round]
                    (WavePhaseDiffStart) ++( 0.6,0) -- ++( 1,0);
                \draw[myblue]  (WavePhaseDiffStart) ++( 1.1,0.35) node[above, align=center] {\textbf{lagging}\\\textbf{direction}};
                \draw[mygreen] (WavePhaseDiffEnd)   ++(-1.1,0.35) node[above, align=center] {\textbf{leading}\\\textbf{direction}};
            \end{scope}
        \end{tikzpicture}
        \end{center}

    \end{CheatsheetEntryFrame}

    \begin{CheatsheetEntryFrameExn}

        \CheatsheetEntryTitle{Appendix: Useful Periodic Function Relations}
        \begin{gather*}
            \omega = 2 \pi f = \frac{2 \pi}{T}
            \qquad
            \begin{aligned}
                \cos{\parens*{\omega t}} &= \sin{\parens*{\omega t + \ang{90}}} \\
                \sin{\parens*{\omega t}} &= \cos{\parens*{\omega t - \ang{90}}}
            \end{aligned}
            \\
            F_{\text{avg}} = \frac{1}{T} \int_0^T{f(t) \,\diff{t}}
            \qquad
            F_{\text{rms}} = \sqrt{\frac{1}{T} \int_0^T{\parens*{f(t)}^2 \,\diff{t}}}
        \end{gather*}%

    \end{CheatsheetEntryFrameExn}

    % Should naturally column-break here

    \begin{CheatsheetEntryFrame}

        \newcommand{\MinipagesThreeColumns}[3]{
            \begin{minipage}[c]{0.33\columnwidth}%
                \centering
                #1
            \end{minipage}%
            \begin{minipage}[c]{0.33\columnwidth}%
                \centering
                #2
            \end{minipage}%
            \begin{minipage}[c]{0.33\columnwidth}%
                \centering
                #3
            \end{minipage}%
        }

        \CheatsheetEntryTitle{Resistor, Capacitor, and Inductor}

        \vspace*{1.5ex}
        \MinipagesThreeColumns{
            $\displaystyle \mathbf{Z}_R = R$
        }{
            $\displaystyle \mathbf{Z}_C = \frac{1}{j \omega C}$
        }{
            $\displaystyle \mathbf{Z}_L = j \omega L$
        }

        \newcommand{\MyTikzGhettoAlignmentA}{\path (0,0.6) -- (0,-0.6);} % Ghetto alignment
        \MinipagesThreeColumns{
            \begin{circuitikz}
                \MyTikzGhettoAlignmentA
                \draw (0,0) to[R, o-o] ++(2,0);
            \end{circuitikz}
        }{
            \begin{circuitikz}
                \MyTikzGhettoAlignmentA
                \draw (0,0) to[C, o-o] ++(2,0);
            \end{circuitikz}
        }{
            \begin{circuitikz}
                \MyTikzGhettoAlignmentA
                \draw (0,0) to[L, o-o] ++(2,0);
            \end{circuitikz}
        }%
        \vspace{-2.0ex} % Too much whitespace for some reason. Need to cut it down.
        \newcommand{\MyTikzGhettoAlignmentB}{\path (0,-1.2) -- (0,1.2) (-0.40,0) -- (1.40,0);} % Ghetto alignment
        %\newcommand{\MyTikzGhettoAlignmentB}{\path (-0.40,0) -- (1.40,0);} % Ghetto alignment
        \MinipagesThreeColumns{%
            \begin{tikzpicture}[x=1.0cm, y=1.0cm, transform shape]
                \MyTikzGhettoAlignmentB
                \draw[-stealth, myred,    line width=2.0pt, line cap=round] (0,0) -- ++(0:1.30) coordinate (VEnd);
                \draw[-stealth, white,    line width=2.0pt, line cap=round] (0,0) -- ++(0:1.05);
                \draw[-stealth, mypurple, line width=2.0pt, line cap=round] (0,0) -- ++(0:1.00) coordinate (IEnd);
                \draw[myred]    (VEnd) ++( 0  ,-0.05) node[below] {$\mathbf{V}$};
                \draw[mypurple] (IEnd) ++(-0.1,-0.05) node[below] {$\mathbf{I}$};
            \end{tikzpicture}
        }{%
            \begin{tikzpicture}[x=1.0cm, y=1.0cm, transform shape]
                \MyTikzGhettoAlignmentB
                \draw[line width=1.2pt] (0.2,0) -- (0.2,-0.2) -- (0,-0.2);
                \draw[-stealth, mypurple, line width=2.0pt, line cap=round] (0,0) -- ++(  0:1) coordinate (IEnd);
                \draw[-stealth, myred,    line width=2.0pt, line cap=round] (0,0) -- ++(-90:1) coordinate (VEnd);
                \draw[myred]    (VEnd) ++(0.0, 0   ) node[right] {$\mathbf{V}$};
                \draw[mypurple] (IEnd) ++(0  ,-0.05) node[below] {$\mathbf{I}$};
            \end{tikzpicture}
        }{%
            \begin{tikzpicture}[x=1.0cm, y=1.0cm, transform shape]
                \MyTikzGhettoAlignmentB
                \draw[line width=1.2pt] (0.2,0) -- (0.2,0.2) -- (0,0.2);
                \draw[-stealth, mypurple, line width=2.0pt, line cap=round] (0,0) -- ++( 0:1) coordinate (IEnd);
                \draw[-stealth, myred,    line width=2.0pt, line cap=round] (0,0) -- ++(90:1) coordinate (VEnd);
                \draw[myred]    (VEnd) ++(0.0, 0   ) node[right] {$\mathbf{V}$};
                \draw[mypurple] (IEnd) ++(0  ,-0.05) node[below] {$\mathbf{I}$};
            \end{tikzpicture}
        }%
        \vspace{-1.0ex} % Move closer to the next section

        \newcommand{\MinipagesTwoColumns}[2]{%
            \begin{myminipage}[t]{0.82\columnwidth}
                \raggedright
                #1
            \end{myminipage}%
            %\hspace{0.06\columnwidth}% TODO: Do something better than this ghetto center-alignment.
            \begin{minipage}[t]{0.16\columnwidth}
                {\color{CheatsheetSepColor} \vrule{}}%
                #2
            \end{minipage}
        }

        \MinipagesTwoColumns{%
            \CheatsheetEntryTitle{Ohm's Law} \MarkSimilarToDC

            For impedance $\mathbf{Z}$ and admittance $\mathbf{Y}$: \\[2\parskip]

            \begin{tabularx}{\textwidth}{CcC}
                $\displaystyle \mathbf{V} = \mathbf{I} \mathbf{Z}$
                    & $\displaystyle \mathbf{Y} \triangleq \frac{1}{\mathbf{Z}}$
                    & $\displaystyle \mathbf{I} = \mathbf{V} \mathbf{Y}$\\
            \end{tabularx}
        }{%
            \begin{circuitikz}
                \path (-0.9,0) -- (0,0) (-0.65,-1) -- (0.65,-1); % Ghetto alignment
                \draw
                    (0,0) to[short] ++(0,-0.2)
                    to[short, i=$\mathbf{I}$] ++(0,-0.1)
                    to[generic, l=$\mathbf{Z}$, v=$\mathbf{V}$] ++(0,-1.6)
                    to[short] ++(0,-0.3)
                ;
            \end{circuitikz}%
        }

        %\CheatsheetEntryExtraSeparation

        \CheatsheetEntryTitle{Rectangular Form}
        \begin{alignat*}{3}
            \mathbf{Z} &= R + jX
                &&= {\textstyle \frac{G}{G^2 + B^2} + j \frac{-B}{G^2 + B^2}} % This line is optional
                ,
                %& \qquad R, X &\in \mathbb{R}, % TODO: Should I add these parts in?
                %& \; R &\ge 0,
                & \qquad
                    {\scriptsize
                    \begin{aligned}
                        R & \\
                        X &
                    \end{aligned}
                    }
                    &
                    {\scriptsize
                    \begin{aligned}
                        &= \text{Resistance} \\
                        &= \text{Reactance}
                    \end{aligned}
                    }
                \\[1ex]
            \mathbf{Y} &= G + jB
                &&= {\textstyle \frac{R}{R^2 + X^2} + j \frac{-X}{R^2 + X^2}} % This line is optional
                ,
                %& \qquad G, B &\in \mathbb{R},
                %& \; G &\ge 0,
                & \qquad
                    {\scriptsize
                    \begin{aligned}
                        G & \\
                        B &
                    \end{aligned}
                    }
                    &
                    {\scriptsize
                    \begin{aligned}
                        &= \text{Conductance} \\
                        &= \text{Susceptance}
                    \end{aligned}
                    }
        \end{alignat*}

        % TODO: Figure out a better way to deliver this warning.
        %\textsc{Please note that $R = \frac{1}{G}$ only applies to resistive circuits.}

    \end{CheatsheetEntryFrame}

    \begin{CheatsheetEntryFrame}

        % TODO: This is basically a reuse of the DC Analysis section's version. See if we can just make a macro for it?
        \CheatsheetEntryTitle{Series and Parallel Equivalent}

        \vspace{1.5ex}
        \begin{minipage}[c]{0.5\columnwidth}
            \centering
            \scalebox{0.8}{
            \begin{circuitikz}
                \draw
                    (0,0)
                    to[generic, o-] ++(2,0)
                    to[generic, -o] ++(2,0)
                ;
            \end{circuitikz}%
            }
        \end{minipage}%
        \begin{minipage}[c]{0.5\columnwidth}
            \centering
            \scalebox{0.8}{
            \begin{circuitikz}
                \draw
                    (0.5,0) to[short, o-]
                    (1,0) -- (1, 0.4) to[generic] (3, 0.4) -- (3,0)
                    to[short, -o] (3.5,0)
                    (1,0) -- (1,-0.4) to[generic] (3,-0.4) -- (3,0)
                ;
            \end{circuitikz}%
            }
        \end{minipage}

        \vspace*{1.5ex}

        \begin{minipage}[c]{0.5\columnwidth}
            \begin{equation*}
                \mathbf{Z}_S = \sum{\mathbf{Z}_i}
            \end{equation*}
        \end{minipage}%
        \begin{minipage}[c]{0.5\columnwidth}
            \begin{equation*}
                \frac{1}{\mathbf{Z}_P} = \sum{\frac{1}{\mathbf{Z}_i}}
            \end{equation*}
        \end{minipage}

        \begin{minipage}[c]{0.5\columnwidth}
            \begin{equation*}
                \frac{1}{\mathbf{Y}_S} = \sum{\frac{1}{\mathbf{Y}_i}}
            \end{equation*}
        \end{minipage}%
        \begin{minipage}[c]{0.5\columnwidth}
            \begin{equation*}
                \mathbf{Y}_P = \sum{\mathbf{Y}_i}
            \end{equation*}
        \end{minipage}

    \end{CheatsheetEntryFrame}

    \begin{CheatsheetEntryFrame}

        % TODO: This is basically a reuse of the DC Analysis section's version. See if we can just make a macro for it?
        \CheatsheetEntryTitle{Voltage and Current Division} \MarkSimilarToDC

        \begin{minipage}[c]{0.6\columnwidth}
            \centering
            \scalebox{1}{
            \begin{circuitikz}
                \draw
                    (0,0)
                    to[short, o-] ++(1,0)
                    -- ++(0,-0.2)
                    to[generic, l_=$\mathbf{Z}_1$, v^=${\displaystyle \mathbf{V}_1 = \frac{\mathbf{Z}_1}{\mathbf{Z}_1+\mathbf{Z}_2}\mathbf{V}}$] ++(0,-1.5)
                    -- ++(0,-0.3)
                    to[generic, l_=$\mathbf{Z}_2$, v^=${\displaystyle \mathbf{V}_2 = \frac{\mathbf{Z}_2}{\mathbf{Z}_1+\mathbf{Z}_2}\mathbf{V}}$] ++(0,-1.5)
                    -- ++(0,-0.2)
                    to[short, -o] ++(-1,0)
                    (0,0)
                    to[open, v=$\mathbf{V}$] (0,-3.7)
                ;
            \end{circuitikz}%
            }
        \end{minipage}%
        \begin{minipage}[c]{0.4\columnwidth}
            \centering
            \scalebox{1}{
            \begin{circuitikz}
                \draw
                    (0,1.65)
                    to[short, i=$\mathbf{I}$, o-] ++(1,0)
                    to[short] ++(0,-0.15)
                    to[generic, l_=$\mathbf{Z}_1$, i>_=$\mathbf{I}_1$] ++(0,-1.5)
                    %to[short] ++(0,-0.25)
                    to[short, -o] ++(-1,0)
                    (1,1.65)
                    to[short] ++(1,0)
                    to[short] ++(0,-0.15)
                    to[generic, l_=$\mathbf{Z}_2$, i>_=$\mathbf{I}_2$] ++(0,-1.5)
                    %to[short] ++(0,-0.25)
                    to[short] ++(-1,0)
                    (1,0) ++(0,-0.1)
                    node[below] {${\displaystyle \mathbf{I}_1 = \frac{\mathbf{Z}_2}{\mathbf{Z}_1+\mathbf{Z}_2}\mathbf{I}}$} ++(0,-1)
                    node[below] {${\displaystyle \mathbf{I}_2 = \frac{\mathbf{Z}_1}{\mathbf{Z}_1+\mathbf{Z}_2}\mathbf{I}}$}
                ;
            \end{circuitikz}%
            }
        \end{minipage}

    \end{CheatsheetEntryFrame}

\end{multicols}
\begin{multicols}{2}

    \begin{CheatsheetEntryFrame}

        \newcommand{\MyReusableFormatting}{
            \path (0,0) -- (0,1.6); % Ghetto alignment
            \path
                (0,0)      coordinate (N)
                (150:1.70) coordinate (A)
                ( 30:1.70) coordinate (B)
                (-90:1.70) coordinate (C)
            ;
            \draw
                (A) ++(150:0.3) node {$a$}
                (B) ++( 30:0.3) node {$b$}
                %(A) ++( 90:0.3) node {$a$} % These require less horizontal space, but look worse
                %(B) ++( 90:0.3) node {$b$}
                (C) ++(-90:0.3) node {$c$}
            ;
        }

        \CheatsheetEntryTitle{$\mathrm{Y}$-$\Delta$ Transform} \MarkSimilarToDC

        \TwoColumnsTextSeparated{$\Longleftrightarrow$}{
            \begin{circuitikz}
                \MyReusableFormatting
                \draw
                    (N) to[generic, l_=$\memphR{\mathbf{Z}_1}$, name=Z1, color=myred, /tikz/circuitikz/bipoles/thickness=4] (A) node[ocirc] {}
                    (N) to[generic, l=$\mathbf{Z}_2$, name=Z2] (B) node[ocirc] {}
                    (N) to[generic, l=$\mathbf{Z}_3$, name=Z3] (C) node[ocirc] {}
                    (N) node[circ] {}
                    (N) ++(-30:0.3) node {$n$}
                ;
            \end{circuitikz}
        }{
            \begin{circuitikz}
                \MyReusableFormatting
                \draw
                    (A)
                    to[generic, l=$\mathbf{Z}_c$, name=ZB] (B) node[circ] {}
                    to[generic, l=$\memphB{\mathbf{Z}_a}$, name=ZC, color=myblue, /tikz/circuitikz/bipoles/thickness=4] (C) node[circ] {}
                    to[generic, l=$\mathbf{Z}_b$, name=ZA] (A) node[circ] {}
                ;
            \end{circuitikz}
        }
        \TwoColumnsTextSeparated{}{
            \begin{equation*}
                \memphR{\mathbf{Z}_1} = \frac{\mathbf{Z}_b \mathbf{Z}_c}{\mathbf{Z}_a + \mathbf{Z}_b + \mathbf{Z}_c}
            \end{equation*}
        }{
            \begin{equation*}
                \memphB{\mathbf{Z}_a}
                    = \frac{\mathbf{Z}_1 \mathbf{Z}_2 + \mathbf{Z}_2 \mathbf{Z}_3 + \mathbf{Z}_3 \mathbf{Z}_1}{\mathbf{Z}_1}
            \end{equation*}
        }

        For $\mathrm{Y}$ and $\Delta$ loads to be balanced:
        \TwoColumnsTextSeparated{\phantom{$\Longleftrightarrow$}}{%
            \begin{equation*}
                \mathbf{Z}_Y = \mathbf{Z}_1 = \mathbf{Z}_2 = \mathbf{Z}_3
            \end{equation*}
        }{%
            \begin{equation*}
                \mathbf{Z}_\Delta = \mathbf{Z}_a = \mathbf{Z}_b = \mathbf{Z}_c
            \end{equation*}
        }

        So when our $\mathrm{Y}$ or $\Delta$ circuit is balanced:
        \begin{equation*}
            \mathbf{Z}_\Delta = 3 \mathbf{Z}_Y
        \end{equation*}

    \end{CheatsheetEntryFrame}

    \begin{CheatsheetEntryFrame}

        \CheatsheetEntryTitle{Other resistor circuit time-domain analysis techniques that translate to the frequency domain}
        \begin{itemize}
            \item Superposition Principle
            \item Kirchoff's Current Law and Supernodes
            \item Kitchoff's Voltage Law and Supermeshes
            \item Source Transformation
            \item Th\'evenin and Norton's Theorems
        \end{itemize}

        \Todo{Consider either: 1) making properly-expanded sections for these, or 2) find a way to neatly merge the resistor circuit time-domain analysis sections with their analogous frequency domain techniques.}

    \end{CheatsheetEntryFrame}

    \Todo{Topic on phasor diagram trigonometry? E.g. adding two voltage phasors with cosine rule?}

    \MulticolsBreak

    \MulticolsPhantomPlaceholder

\end{multicols}
\newpage
\begin{multicols}{2}

    \begin{CheatsheetEntryFrame}

        \CheatsheetEntryTitle{Coupled Inductors}

        For $M$ mutual inductance, consider two cases:

        \bigskip

        \textbf{Case 1:} Currents flowing into \ul{same} side dot sides.

        \phantom{\textbf{Case 1:}} Induced EMF \ul{adds} to self-induction.

        \medskip

        \newcommand{\MyReusableFormatting}[3]{
            \begin{circuitikz}
                \begin{scope}[shift={(0,0)}]
                    \draw % Left Side
                        (0,0)
                        to[short, i=$\mathbf{I}_1$, o-] ++(1,0) coordinate (L1Top)
                        to[L, l_=$L_1$, name=L1] ++(0,-2)
                        to[short, -o] ++(-1,0) coordinate (LeftVBottom)
                        %to[open, v^<=$v_1$] (0,0)
                        to[open] (0,0) coordinate (LeftVTop)
                    ;
                    \draw % Right Side
                        (L1Top) ++(1,0) coordinate (L2Top)
                        to[short, i<=$\mathbf{I}_2$, -o] ++(1,0) coordinate (RightVTop)
                        %to[open, v^>=$v_2$] ++(0,-2)
                        to[open] ++(0,-2) coordinate (RightVBottom)
                        to[short, o-] ++(-1,0)
                        to[L, l_=$L_2$, name=L2] (L2Top)
                    ;
                    \path
                        (LeftVTop)  -- (LeftVBottom)  node[pos=0.2] {$+$} node[pos=0.5] {$\mathbf{V}_1$} node[pos=0.8] {$-$}
                        (RightVTop) -- (RightVBottom) node[pos=0.2] {$+$} node[pos=0.5] {$\mathbf{V}_2$} node[pos=0.8] {$-$}
                    ;
                    \path
                        (L1Top) -- coordinate[midway] (TmpPoint) (L2Top)
                        (TmpPoint) ++(0,0.1) coordinate (MArcCenter)
                    ;
                    \draw[simshadows/style/ee/magneticcouplingarrow]
                        (MArcCenter) ++(0:0.5) arc (0:180:0.5)
                    ;
                    \draw
                        (MArcCenter) ++(90:0.5) node[above] {$M$}
                    ;
                    \draw
                        #3
                    ;
                \end{scope}
                \begin{scope}[shift={(#2,-2.70)}]
                    \node[simshadows/GenericGrayBlockArrow, #1] {};
                    \path
                        (0,0) -- (0,-0.65) % Ghetto alignment
                    ;
                \end{scope}
            \end{circuitikz}
        }

        \newcommand{\MyReusableFormattingB}[2]{
            \begin{center}
            \begin{circuitikz}
                \begin{scope}[shift={(0,-3.5)}]
                    \draw % Left Side
                        (0,0)
                        to[short, i=$\mathbf{I}_1$, o-] ++(1,0) coordinate (L1Top)
                        -- ++(0,-0.2)
                        to[L, l_=$L_1$, name=L1] ++(0,-1.5)
                        %to[cV, l_=$\displaystyle M \frac{\diff{i_2}}{\diff{t}}$] ++(0,-1.5)
                        to[cV, l_=$\displaystyle j \omega M \textbf{I}_2$, /tikz/circuitikz/bipoles/length=1.2cm, #1] ++(0,-1.5)
                        to[short, -o] ++(-1,0) coordinate (LeftVBottom)
                        %to[open, v^<=$v_1$] (0,0)
                        to[open] (0,0) coordinate (LeftVTop)
                    ;
                    \draw % Right Side
                        (L1Top) ++(1,0) coordinate (L2Top)
                        to[short, i<=$\mathbf{I}_2$, -o, name=I2] ++(1,0) coordinate (RightVTop)
                        %to[open, v^>=$v_2$] ++(0,-2)
                        to[open] ++(0,-0.2)
                        to[open] ++(0,-1.5)
                        to[open] ++(0,-1.5) coordinate (RightVBottom)
                        to[short, o-] ++(-1,0)
                        to[cV, l_=$\displaystyle j \omega M \textbf{I}_1$, /tikz/circuitikz/bipoles/length=1.2cm, #2] ++(0,1.5)
                        to[L, l_=$L_2$, name=L2] ++(0,1.5)
                        -- (L2Top)
                    ;
                    \path
                        (L1Top) -- coordinate[midway] (TmpPoint) (L2Top)
                        (TmpPoint) ++(0,0.1) coordinate (MArcCenter)
                    ;
                \end{scope}
            \end{circuitikz}
            \end{center}
        }

        \TwoColumnsMinipages{
            \MyReusableFormatting{rotate=-90}{1.5}{
                (L1Top) ++(-0.20,-0.35) node[circ] {}
                (L2Top) ++( 0.20,-0.35) node[circ] {}
            }
        }{
            \MyReusableFormatting{rotate=-135}{0.2}{
                (L1Top) ++(0,-2) ++(-0.20, 0.35) node[circ] {}
                (L2Top) ++(0,-2) ++( 0.20, 0.35) node[circ] {}
            }
        }%
        \TwoColumnsMinipages{%
            \MyReusableFormattingB{}{invert}
        }{%
            \begin{equation*}
                \boxed{
                \begin{aligned}
                    \mathbf{V}_1 &= j \omega L_1 \mathbf{I}_1 + j \omega M \mathbf{I}_2 \\
                    \mathbf{V}_2 &= j \omega L_2 \mathbf{I}_2 + j \omega M \mathbf{I}_1
                \end{aligned}
                }
                %\\
                %\Exn{\boxed{
                %\begin{aligned}
                %    v_1 &= L_1 \frac{\diff{i_1}}{\diff{t}} + M \frac{\diff{i_2}}{\diff{t}} \\
                %    v_2 &= L_2 \frac{\diff{i_2}}{\diff{t}} + M \frac{\diff{i_1}}{\diff{t}}
                %\end{aligned}
                %}}
            \end{equation*}
            \vspace{3.0ex}
        }

        \bigskip
        \smallskip % A bit of additional space to make the section look nice.

        \textbf{Case 2:} Currents flowing into \ul{different} side dot sides.

        \phantom{\textbf{Case 2:}} Induced EMF \ul{opposes} self-induction.

        \medskip

        \TwoColumnsMinipages{
            \MyReusableFormatting{rotate=-90}{1.5}{
                (L1Top) ++(-0.20,-0.35) node[circ] {}
                (L2Top) ++(0,-2) ++( 0.20, 0.35) node[circ] {}
            }
        }{
            \MyReusableFormatting{rotate=-135}{0.2}{
                (L2Top) ++( 0.20,-0.35) node[circ] {}
                (L1Top) ++(0,-2) ++(-0.20, 0.35) node[circ] {}
            }
        }%
        \TwoColumnsMinipages{%
            \MyReusableFormattingB{invert}{}
        }{%
            \begin{equation*}
                \boxed{
                \begin{aligned}
                    \mathbf{V}_1 &= j \omega L_1 \mathbf{I}_1 - j \omega M \mathbf{I}_2 \\
                    \mathbf{V}_2 &= j \omega L_2 \mathbf{I}_2 - j \omega M \mathbf{I}_1
                \end{aligned}
                }
                %\\
                %\Exn{\boxed{
                %\begin{aligned}
                %    v_1 &= L_1 \frac{\diff{i_1}}{\diff{t}} - M \frac{\diff{i_2}}{\diff{t}} \\
                %    v_2 &= L_2 \frac{\diff{i_2}}{\diff{t}} - M \frac{\diff{i_1}}{\diff{t}}
                %\end{aligned}
                %}}
            \end{equation*}
            \vspace{3.0ex}
        }

    \end{CheatsheetEntryFrame}

    \begin{CheatsheetEntryFrame}

        \newcommand{\MyReusableFormatting}[2]{
            \begin{center}
            \begin{circuitikz}
                \begin{scope}[shift={(0,0)}, rotate=0]
                    \draw
                        (0,0)
                        to[short, o-] ++(0,-0.2)
                        %to[short, i=$i$] ++(0,-0.1)
                        to[L, l_=$L_1$, name=L1, /tikz/circuitikz/bipoles/length=1.0cm] ++(0,-1.25) coordinate (Mid)
                        to[L, l_=$L_2$, name=L2, /tikz/circuitikz/bipoles/length=1.0cm] ++(0,-1.25)
                        to[short, -o] ++(0,-0.2)
                    ;
                    \draw[simshadows/style/ee/magneticcouplingarrow]
                        (L2.above) ++(0.1,0) -- ++(0.2,0) arc (-90:90:0.30 and 0.65) -- ++(-0.2,0)
                    ;
                    %\draw
                    %    (Mid -| L2.above) ++(0.55,0) node[right] {$M$}
                    %;
                    \draw
                        #1
                    ;
                \end{scope}
                \begin{scope}[shift={(1.25,0)}, rotate=0]
                    \draw
                        (0,0)
                        to[short, o-] ++(0,-0.2)
                        %to[short, i=$i$] ++(0,-0.1)
                        to[L, l_=$L_1$, name=L1, /tikz/circuitikz/bipoles/length=1.0cm] ++(0,-1.25) coordinate (Mid)
                        to[L, l_=$L_2$, name=L2, /tikz/circuitikz/bipoles/length=1.0cm] ++(0,-1.25)
                        to[short, -o] ++(0,-0.2)
                    ;
                    \draw[simshadows/style/ee/magneticcouplingarrow]
                        (L2.above) ++(0.1,0) -- ++(0.2,0) arc (-90:90:0.30 and 0.65) -- ++(-0.2,0)
                    ;
                    %\draw
                    %    (Mid -| L2.above) ++(0.55,0) node[right] {$M$}
                    %;
                    \draw
                        #2
                    ;
                \end{scope}
            \end{circuitikz}
            \end{center}
        }

        \CheatsheetEntryTitle{Coupled Inductors in Series}

        \TwoColumnsMinipages[0.24]{
            \raggedright
            \textbf{Case 1:} \\[0mm]
            \textbf{Series-Aiding Connection}
            
            Equivalent total inductance:
            \begin{equation*}
                L = L_1 + L_2 + 2M
            \end{equation*}
        }{
            \MyReusableFormatting{
                (L1.west) ++( 0.30, 0.00) node[circ] {}
                (L2.west) ++( 0.30, 0.00) node[circ] {}
            }{
                (L1.east) ++( 0.30, 0.00) node[circ] {}
                (L2.east) ++( 0.30, 0.00) node[circ] {}
            }
        }

        \medskip

        \TwoColumnsMinipages[0.24]{
            \raggedright
            \textbf{Case 2:} \\[0mm]
            \textbf{Series-Opposing Connection}
            
            Equivalent total inductance:
            \begin{equation*}
                L = L_1 + L_2 - 2M
            \end{equation*}
        }{
            \MyReusableFormatting{
                (L1.west) ++( 0.30, 0.00) node[circ] {}
                (L2.east) ++( 0.30, 0.00) node[circ] {}
            }{
                (L1.east) ++( 0.30, 0.00) node[circ] {}
                (L2.west) ++( 0.30, 0.00) node[circ] {}
            }
        }

    \end{CheatsheetEntryFrame}

    \begin{CheatsheetEntryFrame}

        \CheatsheetEntryTitle{Coefficient of Coupling}
        \begin{equation*}
            k = \frac{M}{\sqrt{L_1 L_2}}, \qquad 0 \le k \le 1
        \end{equation*}

        This value represents the fraction in which the flux from each coil links with the other coil:
        \begin{equation*}
            k = \frac{\phi_{12}}{\phi_{11} + \phi_{12}}
            \qquad
            k = \frac{\phi_{21}}{\phi_{22} + \phi_{21}}
        \end{equation*}

    \end{CheatsheetEntryFrame}
    
\end{multicols}
\newpage
\begin{multicols}{2}

    \newcommand{\MyReusableFormatting}[3]{
        \begin{circuitikz}
            \draw % Left Side
                (0,0)
                to[short, #1] ++(1,0) coordinate (L1Top)
                to[L, name=L1] ++(0,-2)
                to[short] ++(-1,0) coordinate (LeftVBottom)
                %to[open, o-o, v^<=$v_1$] (0,0)
                to[open, o-o] (0,0) coordinate (LeftVTop)
            ;
            \draw % Right Side
                (L1Top) ++(1,0) coordinate (L2Top)
                to[short, #2] ++(1,0) coordinate (RightVTop)
                %to[open, o-o, v^>=$v_2$] ++(0,-2)
                to[open, o-o] ++(0,-2) coordinate (RightVBottom)
                to[short] ++(-1,0)
                to[L, name=L2] (L2Top)
            ;
            \path
                (LeftVTop)  -- (LeftVBottom)  node[pos=0.2] {$+$} node[pos=0.5] {$\mathbf{V}_1$} node[pos=0.8] {$-$}
                (RightVTop) -- (RightVBottom) node[pos=0.2] {$+$} node[pos=0.5] {$\mathbf{V}_2$} node[pos=0.8] {$-$}
            ;
            \path
                (L1Top) -- coordinate[midway] (GapTop) (L2Top)
            ;
            \draw
                (GapTop) ++(0,0.1) node[above] {$N_1 {:} N_2$}
            ;
            \draw
                #3
            ;
            \draw
                (1.5,-1) pic {simshadows/ee/ironcore}
            ;
        \end{circuitikz}
    }

    \begin{CheatsheetEntryFrame}

        %\CheatsheetEntryTitle{Real Transformer}

        %\Todo{this.}

        %\CheatsheetEntryExtraSeparation

        \CheatsheetEntryTitle{Ideal Transformer}

        Properties:
        \begin{alignat*}{3}
            & \text{1)} \ && \text{Perfect/unity coupling.} & \qquad % Longest line
                & (k = 1) \\
            & \text{2)} && \text{Very large reactances.} &
                & (L_1, L_2, M \to \infty) \\
            & \text{3)} && \text{Lossless windings.} &
                & (R_1 = R_2 = 0) \\
            & \text{4)} && \text{No core losses.}
        \end{alignat*}

        For $N_1$ and $N_2$ coil windings and $n$ turns ratio, \\[0mm]
        we consider four cases:

        \textbf{Reference Case:} \\[0mm]
        Voltages are \ul{same} dot-side. \\[0mm]
        Currents enter \ul{different} voltage terminals. \\[0mm]
        {\scriptsize (This is the most useful to remember.)}

        \smallskip
        \TwoColumnsMinipages{
            \MyReusableFormatting{i=$\mathbf{I}_1$,}{i>=$\mathbf{I}_2$,}{
                (L1Top) ++(-0.20,-0.35) node[circ] {}
                (L2Top) ++( 0.20,-0.35) node[circ] {}
            }
        }{
            \begin{equation*}
                n = \frac{N_2}{N_1} = \frac{\mathbf{V}_2}{\mathbf{V}_1} = \frac{\mathbf{I}_1}{\mathbf{I}_2}
            \end{equation*}
        }

        \medskip

        \textbf{Alternative Case 1:} \\[0mm]
        Instead, currents enter the \ul{same} voltage terminal.

        \smallskip
        \TwoColumnsMinipages{
            \MyReusableFormatting{i=$\mathbf{I}_1$,}{i<=$\memphR{\mathbf{I}_2}$, color=myred,}{
                (L1Top) ++(-0.20,-0.35) node[circ] {}
                (L2Top) ++( 0.20,-0.35) node[circ] {}
            }
        }{
            \begin{equation*}
                n = \frac{N_2}{N_1} = \frac{\mathbf{V}_2}{\mathbf{V}_1} = \memphR{-\frac{\mathbf{I}_1}{\mathbf{I}_2}}
            \end{equation*}
        }

        \medskip

        \textbf{Alternative Case 2:} \\[0mm]
        Instead, voltages are at \ul{different} dot-sides.

        \smallskip
        \TwoColumnsMinipages{
            \MyReusableFormatting{i=$\mathbf{I}_1$,}{i>=$\mathbf{I}_2$,}{
                (L1Top)          ++(-0.20,-0.35) node[circ, color=myred] {}
                (L2Top) ++(0,-2) ++( 0.20, 0.35) node[circ, color=myred] {}
            }
        }{
            \begin{equation*}
                n = \frac{N_2}{N_1} = \memphR{-\frac{\mathbf{V}_2}{\mathbf{V}_1}} = \memphR{-\frac{\mathbf{I}_1}{\mathbf{I}_2}}
            \end{equation*}
        }

        \medskip

        \textbf{Alternative Case 3:} \\[0mm]
        Voltages are at \ul{different} dot-sides. \\[0mm]
        Currents enter the \ul{same} voltage terminal.

        \smallskip
        \TwoColumnsMinipages{
            \MyReusableFormatting{i=$\mathbf{I}_1$,}{i<=$\memphR{\mathbf{I}_2}$, color=myred,}{
                (L1Top)          ++(-0.20,-0.35) node[circ, color=myred] {}
                (L2Top) ++(0,-2) ++( 0.20, 0.35) node[circ, color=myred] {}
            }
        }{
            \begin{equation*}
                n = \frac{N_2}{N_1} = \memphR{-\frac{\mathbf{V}_2}{\mathbf{V}_1}} = \frac{\mathbf{I}_1}{\mathbf{I}_2}
            \end{equation*}
        }

    \end{CheatsheetEntryFrame}
    
    \MulticolsBreak

    \begin{CheatsheetEntryFrame}

        \CheatsheetEntryTitle{Ideal Autotransformer}

        \Todo{Finish this!}

    \end{CheatsheetEntryFrame}

    \Todo{Consider adding a section on reflected load.}
    
\end{multicols}
\newpage
\begin{multicols}{2}

    \begin{CheatsheetEntryFrame}

        \CheatsheetEntryTitle{Power in the Time Domain}
        
        If we consider an arbitrary load (potentially both resistive and reactive) where:
        \begin{alignat*}{2}
            v(t) &= V_m && \cos{(\omega t + \theta_v)} \qquad \qquad \theta = \theta_v - \theta_i \\
            i(t) &= I_m && \cos{(\omega t + \theta_i)}
        \end{alignat*}
        The instantaneous power can be expressed as:
        \begin{align*}
            p(t)
                &= v(t)\ i(t) \vphantom{\frac{}{2}} \\ % Fraction with empty numerator. Makes for slightly better vertical space.
                &= \frac{1}{2} V_m I_m \cos{(\theta)} \memphR{[\underbrace{\vphantom{\frac{1}{2}} 1 + \cos{(2 \omega t + 2 \theta_v)}}_{\textbf{average value} = 1}]} \\
                &\qquad + \frac{1}{2} V_m I_m \sin{(\theta)} \memphB{\underbrace{\vphantom{\frac{1}{2}} \sin{(2 \omega t + 2 \theta_v)}}_{\textbf{average value} = 0}}
                %% Original version below:
                %&= \underbrace{\frac{1}{2} V_m I_m \cos{(\theta)}}_{\text{constant}}
                %+ \underbrace{\frac{1}{2} V_m I_m \cos{(2 \omega t + 2 \theta_v - \theta)}}_{\text{average value} = 0}
        \end{align*}
        %\Exn{\scriptsize \textit{Rarely applied directly, but useful for understanding and context.}}

    \end{CheatsheetEntryFrame}

    \begin{CheatsheetEntryFrame}

        \CheatsheetEntryTitle{RMS Value}

        The \textit{RMS} (or effective) value of \myul{any periodic voltage/current} is the equivalent value in DC to deliver the same average power to a load.
        \begin{equation*}
            F_\text{rms} =\sqrt{\frac{1}{T} \int_0^T{\parens*{f(t)}^2 \,\diff{t}}} 
        \end{equation*}

        For sinusoidal waveforms:

        \begin{minipage}{0.5\columnwidth}%
            \begin{equation*}
                V_\text{rms} = \frac{V_m}{\sqrt{2}} \approx 0.707 \, V_m
            \end{equation*}
        \end{minipage}%
        \begin{minipage}{0.5\columnwidth}%
            \begin{equation*}
                I_\text{rms} = \frac{I_m}{\sqrt{2}} \approx 0.707 \, I_m
            \end{equation*}
        \end{minipage}%

        %% I'm not 100% sure on this one. Will need to learn more!
        %% TODO: Look into this section!
        %\vspace{\parskip}%
        %For a waveform produced from a sum of waveforms:
        %\begin{gather*}
        %    f(t) = f_1(t) + f_2(t) + \dots + f_n(t) \\
        %    F_\text{rms}^2 = F_{\text{rms}1}^2 + F_{\text{rms}2}^2 + \dots + F_{\text{rms}n}^2 
        %\end{gather*}

        \CheatsheetEntryExtraSeparation

    \end{CheatsheetEntryFrame}

\end{multicols}%
%\vspace{-5ex}% Looks better to remove this excess whitespace, but idk...
\begin{multicols}{2}

    \begin{CheatsheetEntryFrame}

        \CheatsheetEntryTitle{Power Factor (PF) and Power Angle ($\theta$)}

        \vspace{\parskip}
        \begin{tabularx}{\textwidth}{CC}
            $\text{PF} = \cos{\theta}$ &
            $\theta = \theta_v - \theta_i$ \\
        \end{tabularx}%
        \vspace{\parskip}

        $\theta$ can be taken as the current lag (relative to voltage).%\\[0mm]
        %{\scriptsize (relative to the voltage)}

        %\vspace{1.5ex}%
        %\begin{tabularx}{\textwidth}{CC}
        %    $\text{PF} = \cos{\theta}$ & $\theta = \theta_v - \theta_i$ \\
        %\end{tabularx}

        %\vspace{-0.5ex}
        \begin{center}
        \begin{tikzpicture}[x=2.0cm, y=2.0cm, transform shape]
            \path (0,0) coordinate (Origin);
            \draw[-stealth, line width=1.7pt] (0:0.5) arc (0:-45:0.5);
            \draw (-22.5:0.35) node {$\theta$};
            \draw[-stealth, mypurple, line width=2.0pt, line cap=round] (0,0) -- ++(-45:1) coordinate (IEnd);
            \draw[-stealth, myred,    line width=2.0pt, line cap=round] (0,0) -- ++(  0:1) coordinate (VEnd);
            \draw
                (VEnd)
                ++(0.0,0) node[right, color=myred] {
                    $\mathbf{V}$
                }
                %++(1.6,0) node[align=center, color=myred, font=\scriptsize] {
                %    \textbf{(assuming voltage as reference)} %\\
                %    %\textbf{($\theta_v = 0$ assumed)}
                %}

                %(0.5,0.05) node[above, align=center, color=myred, font=\scriptsize] {\textbf{REFERENCE}}
                
                (IEnd)
                ++(0,-0.05) node[below, color=mypurple] {$\mathbf{I}$}
            ;
            \draw[angle 60 reversed-angle 60, mygreen, line width=1.7pt, line cap=round]
                ( 10:1.6) arc ( 10: 35:1.6) node[above] {\textbf{$\bm{-}$ve $\bm{\theta}$}}
            ;
            \draw[angle 60 reversed-angle 60, myblue,  line width=1.7pt, line cap=round]
                (-10:1.6) arc (-10:-35:1.6) node[below] {\textbf{$\bm{+}$ve $\bm{\theta}$}}
            ;
            \draw
                ( 15:1.7) ++(1.1, 0.00) coordinate (LeadLabel)
                node[above, align=left, color=mygreen, font=\scriptsize] {
                    \textbf{current \ul{leads} the voltage} \\
                    \textbf{$\quad \implies$ \ul{leading} power factor} \\
                    \textbf{$\quad \implies$ \ul{capacitive} load}
                }

                %(LeadLabel |- Origin)
                %node[align=center] {
                %    $\theta = \theta_v - \theta_i$ \\%[0.75ex]
                %    {\scriptsize ($\theta$ is the current lag angle.)} %\\
                %    %$\text{PF} = \cos{\theta}$
                %}

                (-15:1.7) ++(1.1,-0.00)
                node[below, align=left, color=myblue, font=\scriptsize] {
                    \textbf{current \ul{lags} the voltage} \\
                    \textbf{$\quad \implies$ \ul{lagging} power factor} \\
                    \textbf{$\quad \implies$ \ul{inductive} load}
                }
            ;
        \end{tikzpicture}
        \end{center}
        \vspace{-1ex}

        %{\scriptsize
        %    \smallskip

        %    \ul{\textit{Lagging}} PF means \ul{current lags} voltage ($+$ve $\theta$; \ul{inductive load}).\\[0mm]
        %    \ul{\textit{Leading}} PF means \ul{current leads} voltage ($-$ve $\theta$; \ul{capacitive load}).

        %    %\smallskip

        %    %PF is frequently stated as a percentage.\\[0mm]
        %    %Example: $90\%$ Lagging means $\cos{\theta} = 0.9$ with current lagging the voltage.
        %}

        %\CheatsheetEntryExtraSeparation

        %\CheatsheetEntryTitle{Power Factor}
        %\begin{equation*}
        %    \text{PF} = \cos{\theta}
        %\end{equation*}

        \CheatsheetEntryTitle{Average Power}
        \begin{equation*}
            %P = \frac{V_m I_m}{2} \cos{\theta} = V_{\text{rms}} I_{\text{rms}} \cos{\theta}
            P = \frac{1}{2} V_m I_m \cos{\theta} = V_{\text{rms}} I_{\text{rms}} \cos{\theta}
        \end{equation*}

        \CheatsheetEntryTitle{Reactive Power}
        \begin{equation*}
            %Q = \frac{V_m I_m}{2} \sin{\theta} = V_{\text{rms}} I_{\text{rms}} \sin{\theta}
            Q = \frac{1}{2} V_m I_m \sin{\theta} = V_{\text{rms}} I_{\text{rms}} \sin{\theta}
        \end{equation*}
        {\scriptsize%
        \begin{tabular}{lcccc}
            Inductive Loads
                & $\implies$
                %& $\phantom{x} \implies \phantom{x}$ % Ghetto Alignment
                & $+$ve $\theta$
                & $\implies$
                %& $\phantom{x} \implies \phantom{x}$ % Ghetto Alignment
                & $+$ve $Q$
                \\
            Capacitive Loads
                & $\implies$
                & $-$ve $\theta$
                & $\implies$
                & $-$ve $Q$
                \\
        \end{tabular}
        }

        \CheatsheetEntryExtraSeparation

        \CheatsheetEntryTitle{Apparent Power}
        \begin{align*}
            S = \frac{1}{2} V_m I_m = V_{\text{rms}} I_{\text{rms}}
        \end{align*}

    \end{CheatsheetEntryFrame}

    \begin{CheatsheetEntryFrame}

        \CheatsheetEntryTitle{Complex Power}
        \begin{align*}
            \mathbf{S}
                &= \frac{1}{2} \mathbf{V} \mathbf{I}^* \\
                %= \mathbf{V}_{\text{rms}}^{\phantom{*}} \mathbf{I}_{\text{rms}}^* % meh, not important
                &= \frac{1}{2} V_m I_m \phase{\theta}
                %= V_{\text{rms}} I_{\text{rms}} \phase{\theta} % Probably not necessary
                = \frac{1}{2} V_m I_m \brackets*{\cos{\theta} + j \sin{\theta}} \\
                &= P + j Q
        \end{align*}
        ($\mathbf{I}^*$ is the complex conjugate of $\mathbf{I}$.)

    \end{CheatsheetEntryFrame}

    \begin{CheatsheetEntryFrame}

        \CheatsheetEntryTitle{Power Triangle}

        \vspace{-3.5ex}%
        \begin{center}
        \begin{tikzpicture}[x=2.0cm, y=2.0cm, transform shape]
            \path
                (0,0) -- (-0.2,0) % Ghetto alignment

                (0,0) coordinate (Origin)
                (30:2.0) coordinate (TopCorner)
                (TopCorner |- Origin) coordinate (BottomCorner)

                (Origin)       -- coordinate[midway] (SMid) (TopCorner)
                (Origin)       -- coordinate[midway] (PMid) (BottomCorner)
                (BottomCorner) -- coordinate[midway] (QMid) (TopCorner)

                (BottomCorner) ++(1.0, 0   ) coordinate (ArrowsMiddle)
                (ArrowsMiddle) ++(0  , 0.25) coordinate (TopArrowStart)
                (ArrowsMiddle) ++(0  ,-0.25) coordinate (BottomArrowStart)
            ;
            \draw[-stealth,                line width=3.0pt, line cap=round] (Origin) -- (TopCorner);
            \draw[-stealth, line width=1.7pt] (0:0.6) arc (0:30:0.6);
            \draw (15:0.45) node {$\theta$};
            \draw[-stealth, color=magenta, line width=3.0pt, line cap=round] (BottomCorner) -- (TopCorner);
            \draw[-stealth, color=blue,    line width=3.0pt, line cap=round] (Origin) -- (BottomCorner);
            \draw[angle 60 reversed-angle 60, mygreen, line width=2.0pt, line cap=round]
                (BottomArrowStart) -- ++(0,-0.8) node[below, align=left] {
                    \textbf{$\bm{-}$ve $\bm{\theta}$} \\
                    \textbf{$\bm{-}$ve $\bm{Q}$}
                }
            ;
            \draw[angle 60 reversed-angle 60, myblue,  line width=2.0pt, line cap=round]
                (TopArrowStart) -- ++(0,0.8) node[above, align=left] {
                    \textbf{$\bm{+}$ve $\bm{\theta}$} \\
                    \textbf{$\bm{+}$ve $\bm{Q}$}
                }
            ;
            \draw[mygreen] (BottomArrowStart) ++(0.7,-0.4) node[align=center] {\textbf{Capacitive}\\\textbf{Load}};
            \draw[myblue]  (TopArrowStart)    ++(0.7, 0.4) node[align=center] {\textbf{Inductive}\\\textbf{Load}};
            \draw
                (SMid)
                ++(120:0.16) node {$\mathbf{S}$}
                ++(120:0.25) node[align=center, rotate=30, font=\scriptsize] {\textsc{Complex Power}} % and Apparent Power?
            ;
            \draw[blue]
                (PMid)
                ++(-90:0.16) node {$P$}
                ++(-90:0.22) node[align=center, rotate=0, font=\scriptsize] {\textsc{Average Power}}
            ;
            \draw[magenta]
                (QMid)
                ++(  0:0.16) node {$Q$}
                ++(  0:0.25) node[align=center, rotate=-90, font=\scriptsize] {\textsc{Reactive Power}}
            ;
        \end{tikzpicture}
        \end{center}

        \vspace{-15ex}
        \begin{minipage}{0.6\columnwidth}
        %\centering
        %Some other useful results:
        \begin{equation*}
            \boxed{
                \hphantom{x} % Ghetto Whitespace
                \begin{aligned}
                    \vphantom{\sqrt{P^2}} \mathbf{S} &= P + jQ \\
                    \vphantom{\sqrt{P^2}} S &= \abs*{\mathbf{S}} = \sqrt{P^2 + Q^2} \\
                    \vphantom{\sqrt{P^2}} P &= \MyRe{\mathbf{S}} = S \cos{\theta} \\
                    \vphantom{\sqrt{P^2}} Q &= \MyIm{\mathbf{S}} = S \sin{\theta}
                \end{aligned}
                \hphantom{x} % Ghetto Whitespace
            }
        \end{equation*}
        \end{minipage}
        %{\color{blue} \vrule{}}% Useful for debugging

    \end{CheatsheetEntryFrame}

\end{multicols}
\begin{multicols}{2}

    \begin{CheatsheetEntryFrame}

        \CheatsheetEntryTitle{Maximum Average Power Transfer}

        \vspace{1ex}%
        \begin{minipage}[c]{0.6\columnwidth}
            \begin{center}
            \begin{circuitikz}
                \path
                    (0,0) coordinate (InsideBL)
                    ++(1.8,0  ) coordinate (InsideBR)
                    ++(0  ,1.5) coordinate (InsideTR)
                ;
                \draw[simshadows/style/softgray]
                    (InsideTR) ++( 0  , 0.8) coordinate (BoxTR)
                    (InsideBL) ++(-1.2,-0.8) coordinate (BoxBL)
                    (BoxBL) rectangle (BoxTR)
                ;
                \draw
                    (InsideBL)
                    to[V, l=$\mathbf{V}_{th}$, invert] (InsideBL |- InsideTR)
                    to[generic, l=$\mathbf{Z}_{th}$] (InsideTR)
                    to[short, -o] ++( 0.4,0)
                    to[short]     ++( 0.4,0)
                    to[generic, l=$\mathbf{Z}_L$] ++(0,-1.5)
                    to[short]     ++(-0.4,0)
                    to[short, o-] ++(-0.4,0)
                    to[short]     (InsideBL)
                ;
            \end{circuitikz}
            \end{center}
        \end{minipage}%
        \begin{minipage}[c]{0.4\columnwidth}
            \centering
            Maximum average power is transferred to $\mathbf{Z}_L$ if:
            \begin{equation*}
                \mathbf{Z}_L = \mathbf{Z}_{th}^*
            \end{equation*}
            %\phantom{Maximum average power is transferred to $\mathbf{Z}_L$ if:} % For alignment
            \phantom{x} \\[0mm]
            \phantom{x} \\[0mm]
            \phantom{x}
        \end{minipage}

        {\centering ($\mathbf{Z}_{th}^*$ is the complex conjugate of $\mathbf{Z}_{th}^*$.)}

    \end{CheatsheetEntryFrame}

    \begin{CheatsheetEntryFrame}

        \CheatsheetEntryTitle{Conservation of AC Power}

        Complex/average/reactive power supplied by sources equals the sum delivered to each individual load:
        \begin{align*}
            \sum{\mathbf{S}_k} &= 0 \\
            \sum{P_k} &= 0 \\
            \sum{Q_k} &= 0
        \end{align*}

        However, this does NOT hold for apparent power.

    \end{CheatsheetEntryFrame}

    \MulticolsBreak

    \begin{CheatsheetEntryFrame}

        \CheatsheetEntryTitle{Power Delivered to a Load Impedance}

        Power delivered to impedance $\mathbf{Z} = R + jX$:
        \vspace{3ex}
        \TwoColumnsMinipages[0.175]{
            \raggedright
            \begin{circuitikz}
                \draw
                    (0,0)
                    -- ++(0,0.8)
                    -- ++(0,0.9)
                    to[V, l=$\mathbf{V}$, invert] ++(0,1.8)
                    -- ++(0,0.9)
                    -- ++(0,0.8)
                    to[short, i=$\mathbf{I}$] ++(2.0,0)
                    -- ++(0,-0.5) coordinate (ZTop)
                    -- ++(0,-0.3)
                    to[R, l=$R$, v=$\mathbf{V}_R$] ++(0,-1.8) coordinate (ZMid)
                    to[generic, l=$jX$, v=$\mathbf{V}_X$] ++(0,-1.8)
                    -- ++(0,-0.3) coordinate (ZBottom)
                    -- ++(0,-0.5)
                    -- (0,0)
                ;
                \begin{scope}[on background layer]
                    \draw[simshadows/style/softgray]
                        (ZTop)    ++( 1.0,0) coordinate (BoxTR)
                        (ZBottom) ++(-1.0,0) coordinate (BoxBL)
                        (BoxBL) rectangle (BoxTR)
                    ;
                    \draw
                        (BoxTR |- ZMid) ++(0.1,0) node[right] {$\mathbf{Z}$}
                    ;
                    % Below is the old version of the impedance label.
                    %\draw
                    %    (BoxTR) ++(0.2,0) coordinate (BoxTRRef)
                    %    (BoxTRRef |- ZBottom) coordinate (BoxBRRef)
                    %;
                    %\draw
                    %    (BoxTRRef)
                    %    -- ++(0.3,0) coordinate (ZLabelHorizontalRef)
                    %    |- (BoxBRRef)

                    %    (ZLabelHorizontalRef |- ZMid) ++(0.1,0) node[right] {$\mathbf{Z}$}
                    %;
                \end{scope}
            \end{circuitikz}
        }{
            \raggedright
            \begin{align*}
                P &= I_{\text{rms}}^2 R \\
                Q &= I_{\text{rms}}^2 X
                %\\[\abovedisplayskip]
            \end{align*}
            \begin{align*}
                P &= \frac{V_{\text{Rrms}}^2}{R} \\
                &= V_{\text{rms}}^2 \frac{R}{R^2 + X^2} \\[2ex]
                Q &= \frac{V_{\text{Xrms}}^2}{X} \\
                &= V_{\text{rms}}^2 \frac{X}{R^2 + X^2}
            \end{align*}
            %\phantom{x} % Actually looks worse with this
        }

        %\CheatsheetEntryExtraSeparation

        %\CheatsheetEntryTitle{Power Delivered to a Load Admittance}
        %% This section is still being worked on.

        %\Exn{\scriptsize \textit{(Hint: These expressions can be found by converting $R$ and $X$ to $G$ and $B$ from the previous section.)}}

        %Power delivered to admittance $\mathbf{Y} = G + jB$:

        %%\TwoColumnsMinipages[0.275]{
        %\TwoColumnsMinipages[0.38]{
        %    \raggedright
        %    \begin{circuitikz}
        %        \draw
        %            (0,0)
        %            -- ++(0,0.3)
        %            to[V, l=$\mathbf{V}$, invert] ++(0,1.8)
        %            -- ++(0,0.3)
        %            to[short, i=$\mathbf{I}$] ++(1,0) coordinate (YTL)
        %            -- ++(1,0) coordinate (GTop)
        %            -- ++(1.2,0)
        %            to[generic, l=$jB$, i>_=$\mathbf{I}_B$] ++(0,-2.4) coordinate (YBR)
        %            -- ++(-1.2,0) coordinate (GBottom)
        %            -- (0,0)

        %            (GTop)
        %            to[generic, l=$G$, i>_=$\mathbf{I}_G$] (GBottom)
        %        ;
        %        \begin{scope}[on background layer]
        %            \draw[lightgray, fill=verylightgray, line width=3.0pt]
        %                (YTL) ++(0, 0.4) coordinate (BoxTL)
        %                (YBR) ++(1,-0.4) coordinate (BoxBR)
        %                (BoxBR) rectangle (BoxTL)
        %            ;
        %            \path
        %                (BoxTL -| BoxBR) -- coordinate[midway] (BoxMR) (BoxBR) % Calculate midright point
        %            ;
        %            \draw
        %                (BoxMR) ++(0.1,0) node[right] {$\mathbf{Y}$}
        %            ;
        %            % Below is the old version of the impedance label.
        %            %\draw
        %            %    (BoxTR) ++(0.2,0) coordinate (BoxTRRef)
        %            %    (BoxTRRef |- ZBottom) coordinate (BoxBRRef)
        %            %;
        %            %\draw
        %            %    (BoxTRRef)
        %            %    -- ++(0.3,0) coordinate (ZLabelHorizontalRef)
        %            %    |- (BoxBRRef)

        %            %    (ZLabelHorizontalRef |- ZMid) ++(0.1,0) node[right] {$\mathbf{Z}$}
        %            %;
        %        \end{scope}
        %    \end{circuitikz}%
        %}{%
        %    \phantom{x}
        %    \begin{align*}
        %        P &= \phantom{-} V_{\text{rms}}^2 G \\
        %        Q &= -V_{\text{rms}}^2 B
        %    \end{align*}
        %    %\phantom{x} \\[0mm] % Ghetto alignment
        %    %\phantom{x} \\[0mm]
        %    %\phantom{x} \\[0mm]
        %    \phantom{x}
        %}%
        %\begin{alignat*}{2}
        %    P &= \frac{I_{\text{Rrms}}^2}{R} & \qquad\quad
        %        Q &= \frac{I_{\text{Xrms}}^2}{X} \\
        %    &= I_{\text{rms}}^2 \frac{G}{G^2 + B^2} & \qquad\quad
        %        &= I_{\text{rms}}^2 \frac{B}{G^2 + B^2}
        %\end{alignat*}

        %%\begin{align*}
        %%    P &= \phantom{-} V_{\text{rms}}^2 G \\
        %%    Q &= -V_{\text{rms}}^2 B
        %%\end{align*}
        %%\begin{align*}
        %%    P &= TODO \\
        %%    &= V_{\text{rms}}^2 \frac{R}{R^2 + X^2} \\[2ex]
        %%    Q &= TODO \\
        %%    &= V_{\text{rms}}^2 \frac{X}{R^2 + X^2}
        %%\end{align*}

        %\CheatsheetEntryExtraSeparation

    \end{CheatsheetEntryFrame}

    \Todo{Power Factor Correction?}

\end{multicols}

%%%%%%%%%%%%%%%%%%%%%%%%%%%%%%%%%%%%%%%%%%%%%%%%%%%%%%%%%%%%%%%%%%%%%%%%%%%%%%%%%%%%%%%%%%%%%%%%%%%%
%%%%%%%%%%%%%%%%%%%%%%%%%%%%%%%%%%%%%%%%%%%%%%%%%%%%%%%%%%%%%%%%%%%%%%%%%%%%%%%%%%%%%%%%%%%%%%%%%%%%

\newpage
\subsection{Frequency Response Analysis}%
\label{sub:freq-response-analysis}

\begin{multicols}{2}

    \begin{CheatsheetEntryFrame}

        \CheatsheetEntryTitle{todo}

        \Todo{Start this!}

    \end{CheatsheetEntryFrame}

\end{multicols}

