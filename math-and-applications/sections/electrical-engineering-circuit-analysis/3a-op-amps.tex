\subsection{Amplifier Analysis}%
\label{sub:amplifier-analysis}

\subsubsection{Operational Amplifiers}

%\begin{multicols}{2}
%
%    \Todo{this}
%
%\end{multicols}

\begin{CheatsheetEntryFrame}

    \CheatsheetEntryTitle{Ideal Operational Amplifier}

    DC-coupled, high-gain, differential-input.

    \begin{center}
    \begin{circuitikz}
        \draw 
            (0,0)
                node[op amp, noinv input up, anchor=-](OA){}
            (OA.+)
                -- ++(-0.5,0)
                    node[ocirc, label=left:$v_+$]{}
                ++(-1,0)
                    node[label=left:{\emph{(non-inverting input)}}]{}
            (OA.-)
                -- ++(-0.5,0)
                    node[ocirc, label=left:$v_-$]{}
                ++(-1,0)
                    node[label=left:{\emph{(inverting input)}}]{}
            (OA.out)
                -- ++(0.5,0)
                    node[ocirc, label=right:$v_o$]{}
            (OA.up)
                -- ++(0,0.8)
                    node[ocirc, label=above:$V_\text{CC}$]{}
            (OA.down)
                -- ++(0,-0.8)
                    node[ocirc, label=below:$V_\text{EE}$]{}
        ;
    \end{circuitikz}
    \end{center}

    Equivalent model:
    \begin{center}
    \begin{circuitikz}
        \draw 
            (0,0)
                    node[ocirc, label=left:$v_+$]{}
                to[short, i=$i_+$] ++(2,0)
                    coordinate(TR1)
                to[R, l=$R_i$, v=$v_i$] ++(0,-2)
                to[short, i<=$i_-$] ++(-2,0)
                    node[ocirc, label=left:$v_-$]{}
            (TR1) ++(2,0)
                to[cvsource, l=$A v_i$] ++(0,-2)
                    node[ground, scale=2]{}
            (TR1) ++(2,0)
                to[R, l=$R_o$] ++(2,0)
                node[ocirc, label=right:$v_o$]{}
        ;
    \end{circuitikz}
    \end{center}

    Properties:
    \begin{enumerate}
        \item $R_i \to \infty$ \ExtraNotes{($R_i = \text{input resistance}$)}
        \item $R_o = 0$ \ExtraNotes{($R_o = \text{output resistance}$)}
        \item zero common-mode gain \ExtraNotes{(or equivalently, infinite common-mode rejection)}
        \item $A \to \infty$ \ExtraNotes{($A = \text{open loop voltage gain}$)}
        \item infinite bandwidth
    \end{enumerate}

    \emph{Golden rules}:
    \begin{enumerate}
        \item In negative feedback mode, $v_+ = v_-$ (``virtual ground")
        \item $i_+ = i_- = 0$
    \end{enumerate}
\end{CheatsheetEntryFrame}

\newpage

\begin{CheatsheetEntryFrame}

    \CheatsheetEntryTitle{Inverting Amplifier \scriptsize (Op-Amp Common Configuration)}

    \begin{center}
    \begin{circuitikz}
        \draw 
            % Op-Amp
            (0,0)
                node[op amp, noinv input down, anchor=out](OA){}
            (OA.+)
                -| ++(-0.4,-0.5)
                    coordinate(GND)
                \MyGround{}
            (OA.-)
                -- ++(-0.4,0)
                    coordinate(A)
            (OA.out)
                -- ++(0.3,0)
                    coordinate(C)
                -- ++(0.8,0)
                    node[ocirc, label=right:$v_o$]{}

            % Left Side
            (A)
                to[R, l_=$R_1$] ++(-3,0)
                    node[ocirc, label=left:$v_i$]{}

            % Feedback
            (C)
                -- ++(0,1.8)
                to[R, l=$R_2$] ++(-2.8,0)
                -| (A)

            % Equation
            (OA.out)
                ++(3,1.6)
                    node[]{$\displaystyle \boxed{\frac{v_o}{v_i} = \frac{-R_2}{R_1}}$}
        ;
    \end{circuitikz}
    \end{center}

    \CheatsheetEntryTitle{Non-Inverting Amplifier \scriptsize (Op-Amp Common Configuration)}

    \begin{center}
    \begin{circuitikz}
        \draw 
            % Op-Amp
            (0,0)
                node[op amp, noinv input down, anchor=out](OA){}
            (OA.+)
                -- ++(-0.4,0)
                    node[ocirc, label=left:$v_i$]{}
                    coordinate(GND)
            (OA.-)
                -- ++(-0.4,0)
                    coordinate(A)
            (OA.out)
                -- ++(0.3,0)
                    coordinate(C)
                -- ++(0.8,0)
                    node[ocirc, label=right:$v_o$]{}

            % Left Side
            (A)
                to[R, l_=$R_1$] ++(-3,0)
                \MyGround{}

            % Feedback
            (C)
                -- ++(0,1.8)
                to[R, l=$R_2$] ++(-2.8,0)
                -| (A)

            % Equation
            (OA.out)
                ++(3,1.6)
                    node[]{$\displaystyle \boxed{\frac{v_o}{v_i} = 1 + \frac{R_2}{R_1}}$}
        ;
    \end{circuitikz}
    \end{center}

    \CheatsheetEntryTitle{Integrator \scriptsize (Op-Amp Common Configuration)}

    \begin{center}
    \begin{circuitikz}
        \draw 
            % Op-Amp
            (0,0)
                node[op amp, noinv input down, anchor=out](OA){}
            (OA.+)
                -| ++(-0.4,-0.5)
                    coordinate(GND)
                \MyGround{}
            (OA.-)
                -- ++(-0.4,0)
                    coordinate(A)
            (OA.out)
                -- ++(0.3,0)
                    coordinate(C)
                -- ++(0.8,0)
                    node[ocirc, label=right:$v_o$]{}

            % Left Side
            (A)
                to[R, l_=$R$] ++(-3,0)
                    node[ocirc, label=left:$v_i$]{}

            % Feedback
            (C)
                -- ++(0,1.8)
                to[C, l=$C$] ++(-2.8,0)
                -| (A)

            % Equation
            (OA.out)
                ++(3,2)
                    node[]{$
                        \displaystyle
                        \boxed{
                            \begin{gathered}
                                \frac{v_i}{R} + C \frac{\diff{v_o}}{\diff{t}} = 0
                                \\
                                v_o = \frac{-1}{CR} \int{v_i \,\diff{t}}
                            \end{gathered}
                        }
                    $}
        ;
    \end{circuitikz}
    \end{center}

    However, this design will saturate very quickly in practical cases since $v_i$ is never perfectly zero.

    To get around this issue, a resistor is added:
    \begin{center}
    \begin{circuitikz}
        \draw 
            % Op-Amp
            (0,0)
                node[op amp, noinv input down, anchor=out](OA){}
            (OA.+)
                -| ++(-0.4,-0.5)
                    coordinate(GND)
                \MyGround{}
            (OA.-)
                -- ++(-0.4,0)
                    coordinate(A)
            (OA.out)
                -- ++(0.3,0)
                    coordinate(C)
                -- ++(0.8,0)
                    node[ocirc, label=right:$v_o$]{}

            % Left Side
            (A)
                to[R, l_=$R$] ++(-3,0)
                    node[ocirc, label=left:$v_i$]{}

            % Feedback
            (C)
                -- ++(0,1.8)
                    coordinate(D)
                to[C, l=$C$] ++(-2.8,0)
                -| (A)

            % Equation
            (OA.out)
                ++(2.6,2)
                    node[]{$
                        \displaystyle
                        \boxed{
                            \begin{gathered}
                                \abs*{\frac{1}{j \omega C}} \gg R_m
                            \end{gathered}
                        }
                    $}
        ;
        \draw[color=red]
            % Extra Resistor
            (D)
                -- ++(0,1)
                to[R, l_=$R_m$] ++(-2.8,0)
                -| (D -| A)
        ;
    \end{circuitikz}
    \end{center}

    \CheatsheetEntryTitle{Differentiator \scriptsize (Op-Amp Common Configuration)}

    \begin{center}
    \begin{circuitikz}
        \draw 
            % Op-Amp
            (0,0)
                node[op amp, noinv input down, anchor=out](OA){}
            (OA.+)
                -| ++(-0.4,-0.5)
                    coordinate(GND)
                \MyGround{}
            (OA.-)
                -- ++(-0.4,0)
                    coordinate(A)
            (OA.out)
                -- ++(0.3,0)
                    coordinate(C)
                -- ++(0.8,0)
                    node[ocirc, label=right:$v_o$]{}

            % Left Side
            (A)
                to[C, l=$C$] ++(-3,0)
                    node[ocirc, label=left:$v_i$]{}

            % Feedback
            (C)
                -- ++(0,1.8)
                to[R, l_=$R$] ++(-2.8,0)
                -| (A)

            % Equation
            (OA.out)
                ++(3,2)
                    node[]{$
                        \displaystyle
                        \boxed{
                            \begin{gathered}
                                \frac{v_o}{R} + C \frac{\diff{v_i}}{\diff{t}} = 0
                                \\
                                v_o = - C R \frac{\diff{v_i}}{\diff{t}}
                            \end{gathered}
                        }
                    $}
        ;
    \end{circuitikz}
    \end{center}

    \CheatsheetEntryTitle{Difference Amplifier \scriptsize (Op-Amp Common Configuration)}

    \begin{center}
    \begin{circuitikz}
        \draw 
            % Op-Amp
            (0,0)
                node[op amp, noinv input down, anchor=out](OA){}
            (OA.+)
                -- ++(-0.4,0)
                    coordinate(D)
                to[R, l_=$R_2$] ++(0,-1.8)
                    coordinate(GND)
                \MyGround{}
            (OA.-)
                -- ++(-0.4,0)
                    coordinate(A)
            (OA.out)
                -- ++(0.3,0)
                    coordinate(C)
                -- ++(0.8,0)
                    node[ocirc, label=right:$v_o$]{}

            % Left Side
            (A)
                to[R, l_=$R_1$] ++(-3,0)
                    node[ocirc, label=left:$v_2$]{}
            (D)
                to[R, l_=$R_1$] ++(-3,0)
                    node[ocirc, label=left:$v_1$]{}

            % Feedback
            (C)
                -- ++(0,1.8)
                to[R, l=$R_2$] ++(-2.8,0)
                -| (A)

            % Equation
            (OA.out)
                ++(3,1.6)
                    node[]{$\displaystyle \boxed{v_o = \frac{R_2}{R_1} \parens{v_1 - v_2}}$}
        ;
    \end{circuitikz}
    \end{center}

    However, the input impedance can be too low for practical use.

    To solve this, we use an \emph{instrumentational amplifier} instead.
    \bigskip

    \CheatsheetEntryTitle{Instrumentational Amplifier \scriptsize (Op-Amp Common Configuration)}

    \Todo{Write this section up. (It's not important quite yet for the assignment, so I'm holding off for a while.)}

\end{CheatsheetEntryFrame}

