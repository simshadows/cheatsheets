\subsection{Non-Linear Circuits}%
\label{sub:amplifier-analysis}

%\subsubsection{Non-Linear Circuits}

\begin{multicols}{2}

    \begin{CheatsheetEntryFrame}
        \CheatsheetEntryTitle{Differentiator}

        \begin{center}
        \begin{circuitikz}
            \draw 
                % Op-Amp
                (0,0)
                    node[op amp, noinv input down, anchor=out](OA){}
                (OA.+)
                    -| ++(-0.4,-0.5)
                        coordinate(GND)
                    \MyGround{}
                (OA.-)
                    -- ++(-0.4,0)
                        coordinate(A)
                (OA.out)
                    -- ++(0.3,0)
                        coordinate(C)
                    -- ++(0.8,0)
                        node[ocirc, label=right:$v_o$]{}

                % Left Side
                (A)
                    to[C, red, color=red, l_=$C$] ++(-3,0)
                        node[ocirc, label=left:$v_i$]{}

                % Feedback
                (C)
                    -- ++(0,1.8)
                    to[R, blue, color=blue, l_=$R$] ++(-2.8,0)
                    -| (A)

                % Equation
                (OA.+)
                    ++(-2.6,0.2)
                        node[below]{$
                            \displaystyle
                            \boxed{
                            \begin{gathered}
                                \frac{v_o}{\color{blue} R} + {\color{red} C} \frac{\diff{v_i}}{\diff{t}} = 0
                                \\
                                v_o = - {\color{red} C} {\color{blue} R} \frac{\diff{v_i}}{\diff{t}}
                            \end{gathered}
                            }
                        $}
            ;
        \end{circuitikz}
        \end{center}
    \end{CheatsheetEntryFrame}

    \begin{CheatsheetEntryFrame}
        \CheatsheetEntryTitle{Integrator}

        \begin{center}
        \begin{circuitikz}
            \draw 
                % Op-Amp
                (0,0)
                    node[op amp, noinv input down, anchor=out](OA){}
                (OA.+)
                    -| ++(-0.4,-0.5)
                        coordinate(GND)
                    \MyGround{}
                (OA.-)
                    -- ++(-0.4,0)
                        coordinate(A)
                (OA.out)
                    -- ++(0.3,0)
                        coordinate(C)
                    -- ++(0.8,0)
                        node[ocirc, label=right:$v_o$]{}

                % Left Side
                (A)
                    to[R, red, color=red, l_=$R$] ++(-3,0)
                        node[ocirc, label=left:$v_i$]{}

                % Feedback
                (C)
                    -- ++(0,1.8)
                        coordinate(D)
                    to[C, blue, color=blue, l=$C$] ++(-2.8,0)
                    -| (A)

                %% Equation
                %(OA.out)
                %    ++(2.6,2)
                %        node[]{$
                %            \displaystyle
                %            \boxed{
                %                \begin{gathered}
                %                    \abs*{\frac{1}{j \omega C}} \gg R_m
                %                \end{gathered}
                %            }
                %        $}
            ;
            \draw[darkgreen, color=darkgreen]
                % Extra Resistor
                (D)
                    -- ++(0,1)
                    to[R, l_=$R_m$] ++(-2.8,0)
                    -| (D -| A)
            ;
        \end{circuitikz}
        \end{center}

        Without {\color{darkgreen}$R_m$}:
        \begin{gather*}
            \boxed{
                \begin{gathered}
                    \frac{v_i}{\color{red} R}
                    + {\color{blue} C} \frac{\diff{v_o}}{\diff{t}} = 0
                    \\
                    v_o = \frac{-1}{\color{blue} C \color{red} R} \int{v_i \,\diff{t}}
                \end{gathered}
            }
        \end{gather*}

        {\footnotesize
            {\color{darkgreen}$R_m$} is \myul{optional}, but without it, practical implementations quickly saturate (since $v_i$ is never perfectly zero).
        }

        {\footnotesize
            To select {\color{darkgreen}$R_m$}:
        }
        \begin{gather*}
            \boxed{
                \begin{gathered}
                    \abs*{\frac{1}{j \omega \color{blue} C}} \gg {\color{darkgreen} R_m}
                \end{gathered}
            }
        \end{gather*}

        {\footnotesize \Todo{I'm not 100\% sure how $R_m$ works.} }
    \end{CheatsheetEntryFrame}

    \MulticolsBreak

    \begin{CheatsheetEntryFrame}
        \CheatsheetEntryTitle{Multivibrators}

        Three major categories:
        \begin{itemize}
            \item \textbf{Bistable:} Two stable states
            \item \textbf{Monostable:} One stable state {\footnotesize (pulse generator)}
            \item \textbf{Astable:} No stable states  {\footnotesize (waveform generator)}
        \end{itemize}
    \end{CheatsheetEntryFrame}

    \begin{CheatsheetEntryFrame}
        \CheatsheetEntryTitle{Inverting Schmitt Trigger}\\[0mm]
        \emph{\footnotesize (Bistable Multivibrator)}
        \begin{center}
        \begin{circuitikz}
            \draw 
                % Op-Amp
                (0,0)
                    node[op amp, noinv input up, anchor=out](OA){}
                (OA.-)
                    -- ++(-0.4,0)
                        node[ocirc, label=left:$v_i$]{}
                        coordinate(GND)
                (OA.+)
                    -- ++(-0.4,0)
                        coordinate(A)
                (OA.out)
                    -- ++(0.3,0)
                        coordinate(C)
                    -- ++(0.8,0)
                        node[ocirc, label=right:$v_o$]{}

                % Left Side
                (A)
                    to[R, red, color=red, l_=$R_1$] ++(-3,0)
                    \MyGround{}

                % Feedback
                (C)
                    -- ++(0,1.8)
                    to[R, blue, color=blue, l=$R_2$] ++(-2.8,0)
                    -| (A)

                %% TODO: Replace this with a proper equation?
                %% Equation
                %(OA.-)
                %    ++(-2.6,-0.2)
                %        node[below]{$
                %            \displaystyle
                %            \boxed{\frac{v_o}{v_i} = 1 + \frac{\color{blue} R_2}{\color{red} R_1}}
                %        $}
            ;
        \end{circuitikz}
        \end{center}

        \Todo{Add more notes?}
    \end{CheatsheetEntryFrame}

    \begin{CheatsheetEntryFrame}
        \CheatsheetEntryTitle{Non-Inverting Schmitt Trigger}\\[0mm]
        \emph{\footnotesize (Bistable Multivibrator)}
        \begin{center}
        \begin{circuitikz}
            \draw 
                % Op-Amp
                (0,0)
                    node[op amp, noinv input up, anchor=out](OA){}
                (OA.-)
                    -- ++(-0.4,0)
                    \MyGround{}
                        coordinate(GND)
                (OA.+)
                    -- ++(-0.4,0)
                        coordinate(A)
                (OA.out)
                    -- ++(0.3,0)
                        coordinate(C)
                    -- ++(0.8,0)
                        node[ocirc, label=right:$v_o$]{}

                % Left Side
                (A)
                    to[R, red, color=red, l_=$R_1$] ++(-3,0)
                        node[ocirc, label=left:$v_i$]{}

                % Feedback
                (C)
                    -- ++(0,1.8)
                    to[R, blue, color=blue, l=$R_2$] ++(-2.8,0)
                    -| (A)

                %% TODO: Replace this with a proper equation?
                %% Equation
                %(OA.-)
                %    ++(-2.6,-0.2)
                %        node[below]{$
                %            \displaystyle
                %            \boxed{\frac{v_o}{v_i} = 1 + \frac{\color{blue} R_2}{\color{red} R_1}}
                %        $}
            ;
        \end{circuitikz}
        \end{center}

        \Todo{Add more notes?}
    \end{CheatsheetEntryFrame}

    \MulticolsCleanEnd
\end{multicols}

