\subsection{Sinusoidal Steady-State Analysis}%
\label{sub:sinusoidal-steady-state-analysis}

\subsubsection{Introduction}

\begin{multicols}{2}

    \begin{CheatsheetEntryFrame}

        \CheatsheetEntryTitle{Phasor Domain Conversion}

        \newcommand{\X}{\hphantom{x}} % Adding horizontal space to make the table distribute better and look nicer
        \begin{tabularx}{\textwidth}{ccC}
            {\scriptsize \textbf{Time Domain}}   &                             & {\scriptsize \textbf{Phasor Domain}} \\
            \X $v(t) = V_m \cos{(\omega t + \phi)}$ \X & $\Longleftrightarrow$ & $\mathbf{V} = V_m \phase{\phi}$      \\
            \X $i(t) = I_m \cos{(\omega t + \phi)}$ \X & $\Longleftrightarrow$ & $\mathbf{I} = I_m \phase{\phi}$      \\
        \end{tabularx}

        %\begin{tikzpicture}[scale=0.85, transform shape]
        %    \draw[help lines, lightgray]
        %        (0,-1) grid (4,1)
        %    ;
        %    \draw[-latex, line width=1.5pt]
        %        (0,0) -- (4.4,0) coordinate (LabelX)
        %    ;
        %    \draw[myred, dotted, line width=1.2pt, domain=0:4, samples=100]
        %        plot (\x, {cos(0.5*pi*\x r)}) node[right] {\color{mylightred} $\cos{(\omega t)}$}
        %    ;
        %    \draw[myred, line width=2pt, domain=0:4, samples=100]
        %        plot (\x, {cos(0.5*pi*\x r + 70)}) node[right] {$\cos{(\omega t + \phi)}$}
        %    ;
        %    \draw[latex-latex, line width=1.5pt] % Draw vertical line over the curve cuz it looks nicer
        %        (0,-1.4) -- (0,1.4) coordinate (LabelY)
        %    ;
        %    \draw
        %        (LabelX) ++(0  ,-0.1) node[below] {$t$}
        %        (LabelY) ++(0.1, 0  ) node[right] {$x$}
        %    ;
        %\end{tikzpicture}

        \vspace*{1ex}

        \renewcommand{\W}{60} % Phasor angle.
        % IMPORTANT: We will need to manually calculate WaveActualIntercept.
        %            To make it easy, just input this into WolframAlpha:
        %                2*(arccos(0) + (- YOUR_PHASOR_ANGLE_HERE * pi / 180))/pi + 2
        \begin{center}
        \begin{tikzpicture}[x=1.72cm, y=1.72cm, transform shape]
            \begin{scope}[shift={(0,-1.8)}, rotate=-90]
                \path
                    (0,0) coordinate (WaveOrigin)
                    (3,0) coordinate (WaveReferenceIntercept) % Manually calculated.
                    (2.33333333,0) coordinate (WaveActualIntercept) % Manually calculated.
                    % TODO: Maybe automatically calculate these intercepts next time?
                ;
                %\draw[help lines, lightgray]
                %    (0,-1) grid (4,1)
                %;
                \draw[-latex, line width=1.5pt]
                    (0,0) -- (4.4,0) coordinate (LabelX1)
                ;
                \draw[myred, dashed, line width=1.2pt, domain=0:4, samples=100, line cap=round]
                    plot (\x, {cos(0.5*pi*\x r)}) node[right] {\color{myred} $\bm{V_m \cos{(\omega t)}}$}
                ;
                \draw[myred, line width=2pt, domain=0:4, samples=100, line cap=round]
                    plot (\x, {cos(0.5*pi*\x r + \W)}) node[right] {$\bm{V_m \cos{(\omega t + \phi)}}$}
                    coordinate (WaveEnd)
                ;
                \draw[latex-latex, line width=1.5pt] % Draw vertical line over the curve cuz it looks nicer
                    (0,-1.4) -- (0,1.4) coordinate (LabelY1)
                ;
                \draw
                    (LabelX1) ++(0  ,-0.1) node[below] {$\omega t$}
                    (LabelY1) ++(0.1, 0  ) node[right] {$v(t)$}
                ;
                \draw[mypurple]
                    (WaveOrigin) ++(0, 1) ++(0,0.17) node[left] {$\bm{V_m}$}
                    (WaveOrigin) ++(0,-1) ++(0,0.17) node[left] {$\bm{-V_m}$}
                ;
            \end{scope}
            \begin{scope}[shift={(0,0)}]
                \draw[help lines, verylightgray, line width=1.5pt]
                    (0,0) coordinate (PhasorOrigin) circle[radius=1]
                    (1,0) coordinate (CircleRight)
                    (-1,0) coordinate (CircleLeft)
                ;
                \draw[latex-latex, line width=1.5pt]
                    (-1.4,0) -- (1.4,0) coordinate (LabelX2)
                ;
                \draw[latex-latex, line width=1.5pt]
                    (0,-1.4) -- (0,1.4) coordinate (LabelY2)
                ;
                \draw[-latex, myred, line width=2pt, line cap=round]
                    (PhasorOrigin) -- ++(\W:1) coordinate (PhasorEnd)
                ;
                \draw
                    (LabelX2) ++(-0.1 ,-0.05) node[below] {$\MyRe$}
                    (LabelY2) ++( 0.05,-0.1 ) node[right] {$\MyIm$}
                ;
                \draw[mypurple] (30:0.25) node {$\bm{\phi}$};
                \draw[-latex, mypurple, line width=1.2pt, line cap=round] (0:0.40) arc (0:\W:0.40);
                \draw[myred] (PhasorEnd) ++(\W:0.15) node {$\mathbf{V}$};
                % Directions
                \draw[angle 60 reversed-angle 60, mygreen,  line width=1.7pt, line cap=round] ( 10:1.8) arc ( 10: 40:1.8);
                \draw[angle 60 reversed-angle 60, myblue,   line width=1.7pt, line cap=round] (-10:1.8) arc (-10:-40:1.8);
                \draw[angle 60 reversed-angle 60, mypurple, line width=1.7pt, line cap=round] ( 60:1.8) arc ( 60:120:1.8);
                \draw[mygreen]  ( 20:1.8) node[above right, align=left] {\textbf{leading}\\\textbf{direction}};
                \draw[myblue]   (-20:1.8) node[below right, align=left] {\textbf{lagging}\\\textbf{direction}};
                \draw[mypurple] ( 90:1.8) node[above] {\textbf{angular velocity} $\bm{\omega}$};
                %\draw (0,0) circle[radius=2.5]; % Ghetto alignment
            \end{scope}
            \begin{scope}[on background layer]
                %\draw[myblue, dashed, line width=1.5pt] % Old styling for the phasor connector
                \draw[help lines, verylightgray, line width=1.5pt]
                    (PhasorEnd) -- (PhasorEnd |- WaveOrigin)
                ;
                \draw[help lines, verylightgray, line width=1.5pt]
                    (CircleRight)  -- (CircleRight  |- WaveEnd)
                    (CircleLeft)   -- (CircleLeft   |- WaveEnd)
                    (PhasorOrigin |- WaveEnd) ++(-1.4,0) -- ++(2.8,0)
                    %(PhasorOrigin) -- (PhasorOrigin |- WaveOrigin) % Looks ugly
                ;
                \draw[help lines, {mypurple!30!white}, line width=1.5pt]
                    (WaveActualIntercept)    ++(-1.4,0) -- ++(2.8,0) -- ++(0.5,0) ++(-0.2,-0.00) coordinate (WavePhaseDiffEnd)
                ;
                \draw[help lines, {mypurple!30!white}, dashed, line width=1.5pt]
                    (WaveReferenceIntercept) ++(-1.4,0) -- ++(2.8,0) -- ++(0.5,0) ++(-0.2, 0.00) coordinate (WavePhaseDiffStart)
                ;
            \end{scope}
            \begin{scope}[rotate=-90]
                \draw[-angle 60, mypurple, line width=1.7pt]
                    (WavePhaseDiffStart) -- (WavePhaseDiffEnd)
                ;
                \draw[mypurple]
                    ($(WavePhaseDiffStart)!0.5!(WavePhaseDiffEnd)$) coordinate (WavePhaseDiffMid)
                    (WavePhaseDiffMid) ++(0,0.05) node[above] {$\bm{\phi}$}
                    %(WavePhaseDiffStart) ++(0.05,0) node[right, align=left] {\textbf{phase}} % Ugly
                    %(WavePhaseDiffEnd) ++(-0.05,0) node[left, align=right] {\textbf{phase}} % Also ugly
                    (WavePhaseDiffMid) ++(0,0.35) node[above, align=center] {\textbf{phase}}
                ;
                \draw[angle 60 reversed-angle 60, mygreen, line width=1.7pt, line cap=round]
                    (WavePhaseDiffEnd)   ++(-0.6,0) -- ++(-1,0);
                \draw[angle 60 reversed-angle 60, myblue,  line width=1.7pt, line cap=round]
                    (WavePhaseDiffStart) ++( 0.6,0) -- ++( 1,0);
                \draw[myblue]  (WavePhaseDiffStart) ++( 1.1,0.35) node[above, align=center] {\textbf{lagging}\\\textbf{direction}};
                \draw[mygreen] (WavePhaseDiffEnd)   ++(-1.1,0.35) node[above, align=center] {\textbf{leading}\\\textbf{direction}};
            \end{scope}
        \end{tikzpicture}
        \end{center}

    \end{CheatsheetEntryFrame}

    \begin{CheatsheetEntryFrameExn}

        \CheatsheetEntryTitle{Appendix: Useful Periodic Function Relations}
        \begin{gather*}
            \omega = 2 \pi f = \frac{2 \pi}{T}
            \qquad
            \begin{aligned}
                \cos{\parens*{\omega t}} &= \sin{\parens*{\omega t + \ang{90}}} \\
                \sin{\parens*{\omega t}} &= \cos{\parens*{\omega t - \ang{90}}}
            \end{aligned}
            \\
            F_{\text{avg}} = \frac{1}{T} \int_0^T{f(t) \,\diff{t}}
            \qquad
            F_{\text{rms}} = \sqrt{\frac{1}{T} \int_0^T{\parens*{f(t)}^2 \,\diff{t}}}
        \end{gather*}%

    \end{CheatsheetEntryFrameExn}

    % Should naturally column-break here

    \begin{CheatsheetEntryFrame}

        \newcommand{\MinipagesThreeColumns}[3]{
            \begin{minipage}[c]{0.33\columnwidth}%
                \centering
                #1
            \end{minipage}%
            \begin{minipage}[c]{0.33\columnwidth}%
                \centering
                #2
            \end{minipage}%
            \begin{minipage}[c]{0.33\columnwidth}%
                \centering
                #3
            \end{minipage}%
        }

        \CheatsheetEntryTitle{Resistor, Capacitor, and Inductor}

        \vspace*{1.5ex}
        \MinipagesThreeColumns{
            $\displaystyle \mathbf{Z}_R = R$
        }{
            $\displaystyle \mathbf{Z}_C = \frac{1}{j \omega C}$
        }{
            $\displaystyle \mathbf{Z}_L = j \omega L$
        }

        \newcommand{\MyTikzGhettoAlignmentA}{\path (0,0.6) -- (0,-0.6);} % Ghetto alignment
        \MinipagesThreeColumns{
            \begin{circuitikz}
                \MyTikzGhettoAlignmentA
                \draw (0,0) to[R, o-o] ++(2,0);
            \end{circuitikz}
        }{
            \begin{circuitikz}
                \MyTikzGhettoAlignmentA
                \draw (0,0) to[C, o-o] ++(2,0);
            \end{circuitikz}
        }{
            \begin{circuitikz}
                \MyTikzGhettoAlignmentA
                \draw (0,0) to[L, o-o] ++(2,0);
            \end{circuitikz}
        }%
        \vspace{-2.0ex} % Too much whitespace for some reason. Need to cut it down.
        \newcommand{\MyTikzGhettoAlignmentB}{\path (0,-1.2) -- (0,1.2) (-0.40,0) -- (1.40,0);} % Ghetto alignment
        %\newcommand{\MyTikzGhettoAlignmentB}{\path (-0.40,0) -- (1.40,0);} % Ghetto alignment
        \MinipagesThreeColumns{%
            \begin{tikzpicture}[x=1.0cm, y=1.0cm, transform shape]
                \MyTikzGhettoAlignmentB
                \draw[-stealth, myred,    line width=2.0pt, line cap=round] (0,0) -- ++(0:1.30) coordinate (VEnd);
                \draw[-stealth, white,    line width=2.0pt, line cap=round] (0,0) -- ++(0:1.05);
                \draw[-stealth, mypurple, line width=2.0pt, line cap=round] (0,0) -- ++(0:1.00) coordinate (IEnd);
                \draw[myred]    (VEnd) ++( 0  ,-0.05) node[below] {$\mathbf{V}$};
                \draw[mypurple] (IEnd) ++(-0.1,-0.05) node[below] {$\mathbf{I}$};
            \end{tikzpicture}
        }{%
            \begin{tikzpicture}[x=1.0cm, y=1.0cm, transform shape]
                \MyTikzGhettoAlignmentB
                \draw[line width=1.2pt] (0.2,0) -- (0.2,-0.2) -- (0,-0.2);
                \draw[-stealth, mypurple, line width=2.0pt, line cap=round] (0,0) -- ++(  0:1) coordinate (IEnd);
                \draw[-stealth, myred,    line width=2.0pt, line cap=round] (0,0) -- ++(-90:1) coordinate (VEnd);
                \draw[myred]    (VEnd) ++(0.0, 0   ) node[right] {$\mathbf{V}$};
                \draw[mypurple] (IEnd) ++(0  ,-0.05) node[below] {$\mathbf{I}$};
            \end{tikzpicture}
        }{%
            \begin{tikzpicture}[x=1.0cm, y=1.0cm, transform shape]
                \MyTikzGhettoAlignmentB
                \draw[line width=1.2pt] (0.2,0) -- (0.2,0.2) -- (0,0.2);
                \draw[-stealth, mypurple, line width=2.0pt, line cap=round] (0,0) -- ++( 0:1) coordinate (IEnd);
                \draw[-stealth, myred,    line width=2.0pt, line cap=round] (0,0) -- ++(90:1) coordinate (VEnd);
                \draw[myred]    (VEnd) ++(0.0, 0   ) node[right] {$\mathbf{V}$};
                \draw[mypurple] (IEnd) ++(0  ,-0.05) node[below] {$\mathbf{I}$};
            \end{tikzpicture}
        }%
        \vspace{-1.0ex} % Move closer to the next section

        \newcommand{\MinipagesTwoColumns}[2]{%
            \begin{myminipage}[t]{0.82\columnwidth}
                \raggedright
                #1
            \end{myminipage}%
            %\hspace{0.06\columnwidth}% TODO: Do something better than this ghetto center-alignment.
            \begin{minipage}[t]{0.16\columnwidth}
                {\color{CheatsheetSepColor} \vrule{}}%
                #2
            \end{minipage}
        }

        \MinipagesTwoColumns{%
            \CheatsheetEntryTitle{Ohm's Law} \MarkSimilarToDC

            For impedance $\mathbf{Z}$ and admittance $\mathbf{Y}$: \\[2\parskip]

            \begin{tabularx}{\textwidth}{CcC}
                $\displaystyle \mathbf{V} = \mathbf{I} \mathbf{Z}$
                    & $\displaystyle \mathbf{Y} \triangleq \frac{1}{\mathbf{Z}}$
                    & $\displaystyle \mathbf{I} = \mathbf{V} \mathbf{Y}$\\
            \end{tabularx}
        }{%
            \begin{circuitikz}
                \path (-0.9,0) -- (0,0) (-0.65,-1) -- (0.65,-1); % Ghetto alignment
                \draw
                    (0,0) to[short] ++(0,-0.2)
                    to[short, i=$\mathbf{I}$] ++(0,-0.1)
                    to[generic, l=$\mathbf{Z}$, v=$\mathbf{V}$] ++(0,-1.6)
                    to[short] ++(0,-0.3)
                ;
            \end{circuitikz}%
        }

        %\CheatsheetEntryExtraSeparation

        \CheatsheetEntryTitle{Rectangular Form}
        \begin{alignat*}{3}
            \mathbf{Z} &= R + jX
                &&= {\textstyle \frac{G}{G^2 + B^2} + j \frac{-B}{G^2 + B^2}} % This line is optional
                ,
                %& \qquad R, X &\in \mathbb{R}, % TODO: Should I add these parts in?
                %& \; R &\ge 0,
                & \qquad
                    {\scriptsize
                    \begin{aligned}
                        R & \\
                        X &
                    \end{aligned}
                    }
                    &
                    {\scriptsize
                    \begin{aligned}
                        &= \text{Resistance} \\
                        &= \text{Reactance}
                    \end{aligned}
                    }
                \\[1ex]
            \mathbf{Y} &= G + jB
                &&= {\textstyle \frac{R}{R^2 + X^2} + j \frac{-X}{R^2 + X^2}} % This line is optional
                ,
                %& \qquad G, B &\in \mathbb{R},
                %& \; G &\ge 0,
                & \qquad
                    {\scriptsize
                    \begin{aligned}
                        G & \\
                        B &
                    \end{aligned}
                    }
                    &
                    {\scriptsize
                    \begin{aligned}
                        &= \text{Conductance} \\
                        &= \text{Susceptance}
                    \end{aligned}
                    }
        \end{alignat*}

        % TODO: Figure out a better way to deliver this warning.
        %\textsc{Please note that $R = \frac{1}{G}$ only applies to resistive circuits.}

    \end{CheatsheetEntryFrame}

    \begin{CheatsheetEntryFrame}

        % TODO: This is basically a reuse of the DC Analysis section's version. See if we can just make a macro for it?
        \CheatsheetEntryTitle{Series and Parallel Equivalent} \MarkSimilarToDC

        \vspace{1.5ex}
        \begin{minipage}[c]{0.5\columnwidth}
            \centering
            \scalebox{0.8}{
            \begin{circuitikz}
                \draw
                    (0,0)
                    to[generic, o-] ++(2,0)
                    to[generic, -o] ++(2,0)
                ;
            \end{circuitikz}%
            }
        \end{minipage}%
        \begin{minipage}[c]{0.5\columnwidth}
            \centering
            \scalebox{0.8}{
            \begin{circuitikz}
                \draw
                    (0.5,0) to[short, o-]
                    (1,0) -- (1, 0.4) to[generic] (3, 0.4) -- (3,0)
                    to[short, -o] (3.5,0)
                    (1,0) -- (1,-0.4) to[generic] (3,-0.4) -- (3,0)
                ;
            \end{circuitikz}%
            }
        \end{minipage}

        \vspace*{1.5ex}

        \begin{minipage}[c]{0.5\columnwidth}
            \begin{equation*}
                \mathbf{Z}_S = \sum{\mathbf{Z}_i}
            \end{equation*}
        \end{minipage}%
        \begin{minipage}[c]{0.5\columnwidth}
            \begin{equation*}
                \frac{1}{\mathbf{Z}_P} = \sum{\frac{1}{\mathbf{Z}_i}}
            \end{equation*}
        \end{minipage}

        \begin{minipage}[c]{0.5\columnwidth}
            \begin{equation*}
                \frac{1}{\mathbf{Y}_S} = \sum{\frac{1}{\mathbf{Y}_i}}
            \end{equation*}
        \end{minipage}%
        \begin{minipage}[c]{0.5\columnwidth}
            \begin{equation*}
                \mathbf{Y}_P = \sum{\mathbf{Y}_i}
            \end{equation*}
        \end{minipage}

    \end{CheatsheetEntryFrame}

    \begin{CheatsheetEntryFrame}

        % TODO: This is basically a reuse of the DC Analysis section's version. See if we can just make a macro for it?
        \CheatsheetEntryTitle{Voltage and Current Division} \MarkSimilarToDC

        \begin{minipage}[c]{0.6\columnwidth}
            \centering
            \scalebox{1}{
            \begin{circuitikz}
                \draw
                    (0,0)
                    to[short, o-] ++(1,0)
                    -- ++(0,-0.2)
                    to[generic, l_=$\mathbf{Z}_1$, v^=${\displaystyle \mathbf{V}_1 = \frac{\mathbf{Z}_1}{\mathbf{Z}_1+\mathbf{Z}_2}\mathbf{V}}$] ++(0,-1.5)
                    -- ++(0,-0.3)
                    to[generic, l_=$\mathbf{Z}_2$, v^=${\displaystyle \mathbf{V}_2 = \frac{\mathbf{Z}_2}{\mathbf{Z}_1+\mathbf{Z}_2}\mathbf{V}}$] ++(0,-1.5)
                    -- ++(0,-0.2)
                    to[short, -o] ++(-1,0)
                    (0,0)
                    to[open, v=$\mathbf{V}$] (0,-3.7)
                ;
            \end{circuitikz}%
            }
        \end{minipage}%
        \begin{minipage}[c]{0.4\columnwidth}
            \centering
            \scalebox{1}{
            \begin{circuitikz}
                \draw
                    (0,1.65)
                    to[short, i=$\mathbf{I}$, o-] ++(1,0)
                    to[short] ++(0,-0.15)
                    to[generic, l_=$\mathbf{Z}_1$, i>_=$\mathbf{I}_1$] ++(0,-1.5)
                    %to[short] ++(0,-0.25)
                    to[short, -o] ++(-1,0)
                    (1,1.65)
                    to[short] ++(1,0)
                    to[short] ++(0,-0.15)
                    to[generic, l_=$\mathbf{Z}_2$, i>_=$\mathbf{I}_2$] ++(0,-1.5)
                    %to[short] ++(0,-0.25)
                    to[short] ++(-1,0)
                    (1,0) ++(0,-0.1)
                    node[below] {${\displaystyle \mathbf{I}_1 = \frac{\mathbf{Z}_2}{\mathbf{Z}_1+\mathbf{Z}_2}\mathbf{I}}$} ++(0,-1)
                    node[below] {${\displaystyle \mathbf{I}_2 = \frac{\mathbf{Z}_1}{\mathbf{Z}_1+\mathbf{Z}_2}\mathbf{I}}$}
                ;
            \end{circuitikz}%
            }
        \end{minipage}

    \end{CheatsheetEntryFrame}

\end{multicols}
\begin{multicols}{2}

    \begin{CheatsheetEntryFrame}

        \newcommand{\MyReusableFormatting}{
            \path (0,0) -- (0,1.6); % Ghetto alignment
            \path
                (0,0)      coordinate (N)
                (150:1.70) coordinate (A)
                ( 30:1.70) coordinate (B)
                (-90:1.70) coordinate (C)
            ;
            \draw
                (A) ++(150:0.3) node {$a$}
                (B) ++( 30:0.3) node {$b$}
                %(A) ++( 90:0.3) node {$a$} % These require less horizontal space, but look worse
                %(B) ++( 90:0.3) node {$b$}
                (C) ++(-90:0.3) node {$c$}
            ;
        }

        \CheatsheetEntryTitle{$\mathrm{Y}$-$\Delta$ Transform} \MarkSimilarToDC

        \TwoColumnsTextSeparated{$\Longleftrightarrow$}{
            \begin{circuitikz}
                \MyReusableFormatting
                \draw
                    (N) to[generic, l_=$\memphR{\mathbf{Z}_1}$, name=Z1, color=myred, /tikz/circuitikz/bipoles/thickness=4] (A) node[ocirc] {}
                    (N) to[generic, l=$\mathbf{Z}_2$, name=Z2] (B) node[ocirc] {}
                    (N) to[generic, l=$\mathbf{Z}_3$, name=Z3] (C) node[ocirc] {}
                    (N) node[circ] {}
                    (N) ++(-30:0.3) node {$n$}
                ;
            \end{circuitikz}
        }{
            \begin{circuitikz}
                \MyReusableFormatting
                \draw
                    (A)
                    to[generic, l=$\mathbf{Z}_c$, name=ZB] (B) node[circ] {}
                    to[generic, l=$\memphB{\mathbf{Z}_a}$, name=ZC, color=myblue, /tikz/circuitikz/bipoles/thickness=4] (C) node[circ] {}
                    to[generic, l=$\mathbf{Z}_b$, name=ZA] (A) node[circ] {}
                ;
            \end{circuitikz}
        }
        \TwoColumnsTextSeparated{}{
            \begin{equation*}
                \memphR{\mathbf{Z}_1} = \frac{\mathbf{Z}_b \mathbf{Z}_c}{\mathbf{Z}_a + \mathbf{Z}_b + \mathbf{Z}_c}
            \end{equation*}
        }{
            \begin{equation*}
                \memphB{\mathbf{Z}_a}
                    = \frac{\mathbf{Z}_1 \mathbf{Z}_2 + \mathbf{Z}_2 \mathbf{Z}_3 + \mathbf{Z}_3 \mathbf{Z}_1}{\mathbf{Z}_1}
            \end{equation*}
        }
        \medskip

        For $\mathrm{Y}$ and $\Delta$ loads to be balanced:
        \begin{equation*}
            \memphR{\mathbf{Z}_1} = \memphR{\mathbf{Z}_2} = \memphR{\mathbf{Z}_3}
            = \memphB{\mathbf{Z}_a} = \memphB{\mathbf{Z}_b} = \memphB{\mathbf{Z}_c}
        \end{equation*}

    \end{CheatsheetEntryFrame}

    \begin{CheatsheetEntryFrame}

        \CheatsheetEntryTitle{Other resistor circuit time-domain analysis techniques that translate to the frequency domain}
        \begin{itemize}
            \item Superposition Principle
            \item Kirchoff's Current Law and Supernodes
            \item Kitchoff's Voltage Law and Supermeshes
            \item Source Transformation
            \item Th\'evenin and Norton's Theorems
        \end{itemize}

        \Todo{Consider either: 1) making properly-expanded sections for these, or 2) find a way to neatly merge the resistor circuit time-domain analysis sections with their analogous frequency domain techniques.}

    \end{CheatsheetEntryFrame}

    \Todo{Topic on phasor diagram trigonometry? E.g. adding two voltage phasors with cosine rule?}

    \MulticolsBreak

    \MulticolsPhantomPlaceholder

\end{multicols}

