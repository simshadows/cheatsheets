\subsection{DC Analysis}%
\label{sub:dc-analysis}

\begin{multicols}{2}
    
    \CheatsheetEntryFrame{

        \CheatsheetEntryTitle{Current and Voltage} \\[0mm]
        \begin{multicols}{2}
            \begin{equation*}
                i \triangleq \frac{\diff{q}}{\diff{t}}
            \end{equation*}

            \begin{equation*}
                v \triangleq \frac{\diff{w}}{\diff{q}}
            \end{equation*}
        \end{multicols}

        By \textit{conventional current}, current is the flow of $+$ve charge.

        \CheatsheetEntryExtraSeparation

        \CheatsheetEntryTitle{Power}
        \begin{equation*}
            p \triangleq \frac{\diff{w}}{\diff{t}} = \frac{\diff{w}}{\diff{q}} \frac{\diff{q}}{\diff{t}}
            \qquad \Rightarrow \qquad
            p = vi
        \end{equation*} \\[0pt]

        \begin{figure}[H]\centering
        \begin{circuitikz}
            \draw
                (0,0) to[short] ++(0.8,0)
                to[short, i=$i$] ++(0.2,0)
                to[generic, name=X, v=$v$] ++(2,0)
                to[short] ++(1,0)
            ;
            \node[single arrow,
                draw,
                minimum height=2.5em,
                minimum width=4em,
                single arrow tip angle=125,
                single arrow head extend=0.15em,
                outer sep=1ex,
                rotate=-90,
                anchor=east]
                (Arrow) at (X.north) {};
            \node at (Arrow.center) {$p$};
            \draw
                (Arrow.north) node[right=2ex, align=left] {
                    By \textit{passive sign convention}, \\
                    power flowing \textit{in} is $+$ve.
                }
                %(Arrow.north |- X.center)
                %++(0,-1) node[right=2ex, align=left] {
                %    So for power to be positive, \\
                %    current flows into the positive terminal.
                %}
            ;
        \end{circuitikz}
        %\caption{caption}
        %\label{fig:labelname}
        \end{figure}

        %Passive sign convention will be assumed. %for all electrical engineering material of this cheatsheet.

        %\CheatsheetEntryExtraSeparation

        \CheatsheetEntryTitle{Energy Absorbed}
        \begin{equation*}
            w = \int_{t_0}^{t}{p(\tau) \,\diff{\tau}}
        \end{equation*}

    }

    \CheatsheetEntryFrame{

        %% Old Version
        %\CheatsheetEntryTitle{Ideal Independent Sources}
        %
        %\begin{center}
        %\begin{circuitikz}
        %    \path 
        %        (0,0) node (D) {}
        %        ++(0,1.2) node (C) {}
        %        ++(0,2.2) node (B) {}
        %        ++(0,1.2) node (A) {}
        %        (A) ++(0,0.65) node (L1) {}
        %        (C) ++(0,0.65) node (L2) {}
        %        (A) ++(2,0) ++(1.7,0) node (Arrow1) {} ++(1.7,0) node (ArrowEdge1) {}
        %        (B -| Arrow1) node (Arrow2) {}
        %        (B -| ArrowEdge1) node (ArrowEdge2) {}
        %        (ArrowEdge1) ++(1.1,0) ++(0,0.4) node (DisL1) {}
        %        (ArrowEdge1) ++(0,-0) node (DisD1) {}
        %        (ArrowEdge2) ++(1.1,0) ++(0,0.4) node (DisL2) {}
        %        (ArrowEdge2) ++(0,-0) node (DisD2) {}
        %    ;
        %    \draw
        %        (A) to[V, invert, o-o] ++(2,0)
        %        (B) to[I, o-o] ++(2,0)
        %        (C) to[cV, invert, o-o] ++(2,0)
        %        (D) to[cI, o-o] ++(2,0)
        %        (L1) ++(1,0) node[above] {Independent Sources}
        %        (L2) ++(1,0) node[above] {Dependent Sources}
        %        (Arrow1) node {$\xRightarrow[(\text{set }V=0)]{\text{source disable}}$}
        %        (Arrow2) node {$\xRightarrow[(\text{set }I=0)]{\text{source disable}}$}
        %        (DisL1) node {Short Circuit}
        %        (DisL2) node {Open Circuit}
        %        (DisD1) to[short, o-*] ++(0.8,0) to[short] ++(0.6,0) to[short, *-o] ++(0.8,0)
        %        (DisD2) to[short, o-o] ++(0.8,0) to[open] ++(0.6,0) to[short, o-o] ++(0.8,0)
        %    ;
        %\end{circuitikz}%
        %\end{center}

        \CheatsheetEntryTitle{Ideal Independent Sources}

        \begin{center}
        \begin{circuitikz}
            \path 
                (0,0) node (B) {}
                ++(0,1.6) node (A) {}
                (A) ++(2,0) ++(1.7,0) node (Arrow1) {} ++(1.7,0) node (ArrowEdge1) {}
                (B -| Arrow1) node (Arrow2) {}
                (B -| ArrowEdge1) node (ArrowEdge2) {}
                (ArrowEdge1) ++(1.1,0) ++(0,0.4) node (DisL1) {}
                (ArrowEdge1) ++(0,-0) node (DisD1) {}
                (ArrowEdge2) ++(1.1,0) ++(0,0.4) node (DisL2) {}
                (ArrowEdge2) ++(0,-0) node (DisD2) {}
            ;
            \draw
                (A) to[V, l=$V$, invert, o-o] ++(2,0)
                (B) to[I, l=$I$, o-o] ++(2,0)
                (Arrow1) node {$\xRightarrow[(\text{set }V=0)]{\text{source disable}}$}
                (Arrow2) node {$\xRightarrow[(\text{set }I=0)]{\text{source disable}}$}
                (DisL1) node {Short Circuit}
                (DisL2) node {Open Circuit}
                (DisD1) to[short, o-*] ++(0.8,0) to[short] ++(0.6,0) to[short, *-o] ++(0.8,0)
                (DisD2) to[short, o-o] ++(0.8,0) to[open] ++(0.6,0) to[short, o-o] ++(0.8,0)
                %(A) ++(2,0) ++(-0.35,0.2) node[above] {$V$}
                %(B) ++(2,0) ++(-0.35,0.2) node[above] {$I$}
            ;
        \end{circuitikz}%
        \end{center}

        \CheatsheetEntryTitle{Ideal Dependent Sources}

        \vspace*{1ex}
        \begin{center}
        \begin{circuitikz}
            \path 
                (0,0) node (C) {}
                %++(0,1.6) node (C) {}
                % Everything below is used to align with the Ideal Independent Sources section
                (C) ++(2,0) ++(1.7,0) node (Arrow1) {} ++(1.7,0) node (ArrowEdge1) {}
                (ArrowEdge1) ++(0,-0) node (DisD1) {}
                    (DisD1) -- ++(0.8,0) -- ++(0.6,0) -- ++(0.8,0)
            ;
            \draw
                (C) to[cV, invert, o-o] ++(2,0)
                ++(0.8,0) to[cI, o-o] ++(2,0)
            ;
        \end{circuitikz}%
        \end{center}

    }

    \CheatsheetEntryFrame{

        \CheatsheetEntryTitle{Maximum Power Transfer}

        \begin{minipage}[c]{0.6\columnwidth}
            \begin{center}
            \begin{circuitikz}
                \path
                    (0,0) coordinate (InsideBL)
                    ++(1.8,0  ) coordinate (InsideBR)
                    ++(0  ,1.5) coordinate (InsideTR)
                ;
                \draw[lightgray, fill=verylightgray, line width=3.0pt]
                    (InsideTR) ++( 0  , 0.8) coordinate (BoxTR)
                    (InsideBL) ++(-1.2,-0.8) coordinate (BoxBL)
                    (BoxBL) rectangle (BoxTR)
                ;
                \draw
                    (InsideBL)
                    to[V, l=$v_{th}$, invert] (InsideBL |- InsideTR)
                    to[R, l=$R_{th}$] (InsideTR)
                    to[short, -o] ++( 0.4,0)
                    to[short]     ++( 0.4,0)
                    to[R, l=$R_L$] ++(0,-1.5)
                    to[short]     ++(-0.4,0)
                    to[short, o-] ++(-0.4,0)
                    to[short]     (InsideBL)
                ;
            \end{circuitikz}
            \end{center}
        \end{minipage}%
        \begin{minipage}[c]{0.4\columnwidth}
            \centering
            Maximum power is transferred to $R_L$ if:
            \begin{equation*}
                R_L = R_{th}
            \end{equation*}
            \phantom{Maximum power is transferred to $R_L$ if:} % For alignment
        \end{minipage}

    }

    \MulticolsBreak

    \CheatsheetEntryFrame{

        \renewcommand{\MyReusableFormatting}[2]{%
            \begin{minipage}[c]{0.78\columnwidth}
                #1
            \end{minipage}%
            {\color{CheatsheetSepColor} \vrule{}}%
            %\hspace{0.06\columnwidth}% TODO: Do something better than this ghetto center-alignment.
            \begin{minipage}[c]{0.2\columnwidth}
                #2
            \end{minipage}
        }

        \CheatsheetEntryTitle{Ohm's Law}

        \MyReusableFormatting{%
            For resistance $R$:
            \begin{equation*}
                v = iR
            \end{equation*}

            For conductance $G$:
            \begin{equation*}
                G \triangleq \frac{1}{R}
                \qquad \Rightarrow \qquad
                i = vG
            \end{equation*}
        }{%
            \begin{circuitikz}
                \path (-1,0) -- (0,0); % Ghetto alignment
                \draw
                    (0,0) to[short] ++(0,-0.2)
                    to[short, i=$i$] ++(0,-0.1)
                    to[R, l=$R$, v=$v$] ++(0,-1.6)
                    to[short] ++(0,-0.3)
                ;
            \end{circuitikz}%
        }

        \CheatsheetEntryExtraSeparation

        \CheatsheetEntryTitle{Ideal Capacitors}

        \MyReusableFormatting{%
            \begin{equation*}
                q = Cv
                \qquad \xRightarrow{\textstyle \phantom{.} \frac{\diff{}}{\diff{t}} \phantom{.}} \qquad
                i = C \frac{\diff{v}}{\diff{t}}
            \end{equation*}

            %\CheatsheetSmallEquationTitle{Stored charge:}
            %\begin{equation*}
            %    q = Cv
            %\end{equation*}

            \vspace*{2ex}
            {\footnotesize%
            \begin{tabular}{lcl}
                Steady-state DC       & $\to$ & Open Circuit \\
                On jump-discontinuity & $\to$ & Short Circuit
            \end{tabular}
            }
        }{%
            \begin{circuitikz}
                \path (-1,0) -- (0,0); % Ghetto alignment
                \draw
                    (0,0) to[short] ++(0,-0.2)
                    to[short, i=$i$] ++(0,-0.1)
                    to[C, l=$C$, v=$v$] ++(0,-1.6)
                    to[short] ++(0,-0.3)
                ;
            \end{circuitikz}%
        }

        \CheatsheetEntryExtraSeparation

        \CheatsheetEntryTitle{Ideal Inductors}

        \MyReusableFormatting{%
            \begin{equation*}
                v = L \frac{\diff{i}}{\diff{t}}
            \end{equation*}

            \vspace*{2ex}
            {\footnotesize%
            \begin{tabular}{lcl}
                Steady-state DC       & $\to$ & Short Circuit \\
                On jump-discontinuity & $\to$ & Open Circuit
            \end{tabular}
            }
        }{%
            \begin{circuitikz}
                \path (-1,0) -- (0,0); % Ghetto alignment
                \draw
                    (0,0) to[short] ++(0,-0.2)
                    to[short, i=$i$] ++(0,-0.1)
                    to[L, l=$L$, v=$v$] ++(0,-1.6)
                    to[short] ++(0,-0.3)
                ;
            \end{circuitikz}%
        }

    }

    \CheatsheetEntryFrame{

        \CheatsheetEntryTitle{Series and Parallel Equivalent}

        \vspace{1.5ex}
        \begin{minipage}[c]{0.5\columnwidth}
            \centering
            \scalebox{0.8}{
            \begin{circuitikz}
                \draw
                    (0,0)
                    to[generic, o-] ++(2,0)
                    to[generic, -o] ++(2,0)
                ;
            \end{circuitikz}%
            }
        \end{minipage}%
        \begin{minipage}[c]{0.5\columnwidth}
            \centering
            \scalebox{0.8}{
            \begin{circuitikz}
                \draw
                    (0.5,0) to[short, o-]
                    (1,0) -- (1, 0.4) to[generic] (3, 0.4) -- (3,0)
                    to[short, -o] (3.5,0)
                    (1,0) -- (1,-0.4) to[generic] (3,-0.4) -- (3,0)
                ;
            \end{circuitikz}%
            }
        \end{minipage}

        \vspace*{1.5ex}

        \begin{minipage}[c]{0.5\columnwidth}
            \begin{equation*}
                R_S = \sum{R_i}
            \end{equation*}
        \end{minipage}%
        \begin{minipage}[c]{0.5\columnwidth}
            \begin{equation*}
                \frac{1}{R_P} = \sum{\frac{1}{R_i}}
            \end{equation*}
        \end{minipage}

        \begin{minipage}[c]{0.5\columnwidth}
            \begin{equation*}
                \frac{1}{C_S} = \sum{\frac{1}{C_i}}
            \end{equation*}
        \end{minipage}%
        \begin{minipage}[c]{0.5\columnwidth}
            \begin{equation*}
                C_P = \sum{C_i}
            \end{equation*}
        \end{minipage}

        \begin{minipage}[c]{0.5\columnwidth}
            \begin{equation*}
                L_S = \sum{L_i}
            \end{equation*}
        \end{minipage}%
        \begin{minipage}[c]{0.5\columnwidth}
            \begin{equation*}
                \frac{1}{L_P} = \sum{\frac{1}{L_i}}
            \end{equation*}
        \end{minipage}

    }

    \CheatsheetEntryFrame{

        \CheatsheetEntryTitle{Voltage and Current Division}

        %\vspace*{1ex}
        %\begin{minipage}[c]{0.5\columnwidth}
        %    \centering
        %    \textbf{Voltage Division}
        %\end{minipage}%
        %\begin{minipage}[c]{0.5\columnwidth}
        %    \centering
        %    \textbf{Current Division}
        %\end{minipage}

        \begin{minipage}[c]{0.6\columnwidth}
            \centering
            \scalebox{1}{
            \begin{circuitikz}
                \draw
                    (0,0)
                    to[short, o-] ++(1,0)
                    -- ++(0,-0.2)
                    to[R, l_=$R_1$, v^=${\displaystyle v_1 = \frac{R_1}{R_1+R_2}v}$] ++(0,-1.5)
                    -- ++(0,-0.3)
                    to[R, l_=$R_2$, v^=${\displaystyle v_2 = \frac{R_2}{R_1+R_2}v}$] ++(0,-1.5)
                    -- ++(0,-0.2)
                    to[short, -o] ++(-1,0)
                    (0,0)
                    to[open, v=$v$] (0,-3.7)
                ;
            \end{circuitikz}%
            }
        \end{minipage}%
        \begin{minipage}[c]{0.4\columnwidth}
            \centering
            \scalebox{1}{
            \begin{circuitikz}
                \draw
                    (0,1.65)
                    to[short, i=$i$, o-] ++(1,0)
                    to[short] ++(0,-0.15)
                    to[R, l_=$R_1$, i>_=$i_1$] ++(0,-1.5)
                    %to[short] ++(0,-0.25)
                    to[short, -o] ++(-1,0)
                    (1,1.65)
                    to[short] ++(1,0)
                    to[short] ++(0,-0.15)
                    to[R, l_=$R_2$, i>_=$i_2$] ++(0,-1.5)
                    %to[short] ++(0,-0.25)
                    to[short] ++(-1,0)
                    (1,0) ++(0,-0.1)
                    node[below] {${\displaystyle i_1 = \frac{R_2}{R_1+R_2}i}$} ++(0,-1)
                    node[below] {${\displaystyle i_2 = \frac{R_1}{R_1+R_2}i}$}
                ;
            \end{circuitikz}%
            }
        \end{minipage}

    }

\end{multicols}
\begin{multicols}{2}

    \CheatsheetEntryFrame{

        \CheatsheetEntryTitle{Kirchoff's Current Law}

        The sum of signed currents into a node or closed boundary is zero.

        \begin{minipage}[c]{0.55\columnwidth}%
            \centering
            \begin{circuitikz}
                \path % Defining offsets makes the join point look a lot less awkward
                    (0,0) node (Node) {}
                    ++(-0.00,0) node (Offset2) {}
                    ++(-0.04,0) node (Offset1) {}
                    ++(-0.04,0) node (Offset0) {}
                    (Node)
                    ++(0.00,0) node (Offset3) {}
                    ++(0.04,0) node (Offset4) {}
                    ++(0.04,0) node (Offset5) {}
                ;
                \draw 
                    (Offset0) to[short] (Offset5)
                    ( 180:2) to[short, i>=\phantom{x}, color=myred ] (Offset0)
                    ( 160:2) to[short, i>=\phantom{x}, color=myred ] (Offset1)
                    ( 140:2) to[short, i>=\phantom{x}, color=myred ] (Offset2)
                    (  40:2) to[short, i<=\phantom{x}, color=myblue] (Offset3)
                    (  20:2) to[short, i<=\phantom{x}, color=myblue] (Offset4)
                    (   0:2) to[short, i<=\phantom{x}, color=myblue] (Offset5)
                    %( -20:2) to[short, i<=\phantom{x}] (Node)
                    ( -40:2) to[short, i<=\phantom{x}, color=myblue] (Offset3)
                    (-140:2) to[short, i>=\phantom{x}, color=myred ] (Offset2)
                    %(-160:2) to[short, i>=\phantom{x}] (Node)
                    ( 136:1) ++(0,0.2) node[above] {\color{myred}  $i_{i[1..n]}$}
                    (  44:1) ++(0,0.2) node[above] {\color{myblue} $i_{o[1..m]}$}
                    ( -14:1) node {\color{myblue} $\cdot$}
                    ( -20:1) node {\color{myblue} $\cdot$}
                    ( -26:1) node {\color{myblue} $\cdot$}
                    (-154:1) node {\color{myred}  $\cdot$}
                    (-160:1) node {\color{myred}  $\cdot$}
                    (-166:1) node {\color{myred}  $\cdot$}
                ;
            \end{circuitikz}%
        \end{minipage}%
        %\vrule{}%
        \begin{minipage}[c]{0.45\columnwidth}%
            \centering
            \phantom{x} % Phantom to fix broken alignment
            \begin{equation*}
                \memphR{\sum{i_i}} - \memphB{\sum{i_o}} = 0
            \end{equation*}

            \ExtraNotes{\footnotesize \textit{Assumes charge within the node or closed boundary is \myul{always} constant.}}
        \end{minipage}

        \CheatsheetEntryExtraSeparation

        %\CheatsheetEntryTitle{Nodal Analysis}
        %
        %% TODO: Write this section on Nodal Analysis!
        %
        %\CheatsheetEntryExtraSeparation

        \CheatsheetEntryTitle{Supernode (for Nodal Analysis)}

        A \textit{supernode} is formed from two non-reference nodes connected by a voltage source (dependent or independent).

        \begin{center}
        \begin{circuitikz}
            \path
                (0,0) coordinate (BL)
                ++(2,0) coordinate (BR)
                ++(0,1.4) coordinate (TR)
                ++(-2,0) coordinate (TL)
                (1,0.7) coordinate (Middle)

                (TR) ++(1.25,0.40) coordinate (Line0)
                ++(0,-1.00) coordinate (Line1)
                ++(0,-0.65) coordinate (Line2)
                ++(0,-0.65) coordinate (Line3)
            ;
            \draw[{myorange!30!white}, fill={myyellow!10!white}, line width=1.8pt]
                (Middle) ellipse (1.33 and 1.35)
            ;
            \draw 
                (BL) -- (TL)
                (BR) -- (TR)
                (BL) to[V, color=myblue] (BR)
                (TL) to[generic] (TR)
                %(TL) -- ++(0.4,0)
                %    ++(0,0.2) node (Generic1TL) {}
                %(TR) -- ++(-0.4,0)
                %    ++(0,-0.2) node (Generic1BR) {}
                %(Generic1TL) rectangle (Generic1BR)

                (BL) ++(1,0) ++(0,0.65) node {\color{myblue} $v_s$}

                (TL) -- ++(-0.3, 0  ) coordinate (TLL)
                (TL) -- ++( 0  , 0.3) coordinate (TLT)
                (TR) -- ++( 0  , 0.3) coordinate (TRT)
                (TR) -- ++( 0.3, 0  ) coordinate (TRR)
                (BR) -- ++( 0.3, 0  ) coordinate (BRR)
                (BR) -- ++( 0  ,-0.3) coordinate (BRB)
                (BL) -- ++( 0  ,-0.3) coordinate (BLB)
                (BL) -- ++(-0.3, 0  ) coordinate (BLL)
            ;
            %\draw[mygreen, line width=1.5pt] % Alternative thicker coloured generic
            %    (Generic1TL) rectangle (Generic1BR)
            %;
            \draw[gray, line cap=round, dash pattern=on 0.5mm off 0.5mm]
            %\draw[lightgray]
            %\draw[dotted]
            %\draw[line cap=round, dash pattern=on 0.1mm off 0.5mm]
                (TLL) ++(-0.5mm, 0    ) -- ++(-0.3, 0  )
                (TLT) ++( 0    , 0.5mm) -- ++( 0  , 0.3)
                (TRT) ++( 0    , 0.5mm) -- ++( 0  , 0.3)
                (TRR) ++( 0.5mm, 0    ) -- ++( 0.3, 0  )
                (BRR) ++( 0.5mm, 0    ) -- ++( 0.3, 0  )
                (BRB) ++( 0    ,-0.5mm) -- ++( 0  ,-0.3)
                (BLB) ++( 0    ,-0.5mm) -- ++( 0  ,-0.3)
                (BLL) ++(-0.5mm, 0    ) -- ++(-0.3, 0  )
            ;
            \draw[myblue, fill=myblue]
                (BL) node[circ, color=myblue] {}
                (BL) ++(-0.30,-0.05) node[below] {$v_1$}
            ;
            \draw[myblue, fill=myblue]
                (BR) node[circ, color=myblue] {}
                (BR) ++(0.30,-0.05) node[below] {$v_2$}
            ;
            \draw[extranotecolor] % Text
                (Line0) node[right, align=left] {Components in parallel\\are frequently ignored.}
            ;
            \draw % Text
                (Line1) node[right] {\textbf{Forms two equations:}}
                (Line2) ++(0.7,0) node[left] {1)}
                    ++(-0.1,0) node[right] {KCL on supernode}
                (Line3) ++(0.7,0) node[left] {2)}
                    ++(-0.1,0) node[right] {$v_s = v_1 - v_2$}
            ;
            \path % Get coordinates for bezier curve arrows
                (Line0)
                %++(-0.1,0.23) coordinate (ArrowStart0)
                ++(-0.1,0) coordinate (ArrowStart0)
                ++(-0.8,0) coordinate (ArrowControl0)
                (TR) ++(-0.52,0.18) ++(0.05,0.05) coordinate (ArrowEnd0)

                (Line2)
                ++(0.1,0) coordinate (ArrowStart1)
                ++(-0.5,0) coordinate (ArrowControl1)
                (BR) ++(0.35,0.30) coordinate (ArrowEnd1)

                (Line3)
                ++(0.1,-0.12) coordinate (ArrowStart2)
                ++(-1.15,-0.35) coordinate (ArrowControl2)
                (BL) ++(1,0) ++(0.35,-0.35) coordinate (ArrowEnd2)
            ;
            \draw[mygreen, ->, line width=1.7pt, line cap=round] % Draw bezier curve arrow
                (ArrowStart0) .. controls (ArrowControl0) .. (ArrowEnd0)
            ;
            \draw[myred, ->, line width=1.7pt, line cap=round] % Draw bezier curve arrow
                (ArrowStart1) .. controls (ArrowControl1) .. (ArrowEnd1)
            ;
            \draw[myblue, ->, line width=1.7pt, line cap=round] % Draw bezier curve arrow
                (ArrowStart2) .. controls (ArrowControl2) .. (ArrowEnd2)
            ;
        \end{circuitikz}%
        \end{center}

        % TODO: Can you generalize to >2 non-reference nodes?

    }

    \CheatsheetEntryFrame{

        \CheatsheetEntryTitle{Kirchoff's Voltage Law}

        The sum of signed voltages around any closed path is zero.

        \begin{minipage}[c]{0.65\columnwidth}
            \centering
            \begin{circuitikz}
                \draw 
                    (0,0) node (Origin) {}
                    -- ++(0,0.2) to[generic, v^=$v_1$] ++(0,1.8) -- ++(0,0.2)
                    -- ++(0.2,0) to[generic, v^=$v_2$] ++(1.8,0) -- ++(0.2,0)
                    -- ++(0.2,0) ++(0.5,0) node {$\cdots$} ++(0.5,0) -- ++(0.2,0)
                    %-- ++(0.2,0) to[generic, v^=$v_{n-1}$] ++(1.8,0) -- ++(0.2,0) % Use this line for the (n-1)th component
                    -- ++(0.2,0) % Use this line if no (n-1)th component
                    -- ++(0,-0.2) to[generic, v^=$v_n$] ++(0,-1.8) -- ++(0,-0.2)
                    coordinate (BR)
                    to[short] (Origin)
                ;
                %\path
                %    (BR)
                %    ++(2.3,1) coordinate (Equation)
                %    ++(2.1,0) % Extra tail for alignment
                %;
                %\draw
                %    (Equation) node {$\displaystyle \sum{v_k} = 0$}
                %;
            \end{circuitikz}%
        \end{minipage}%
        %\vrule{}
        \begin{minipage}[c]{0.35\columnwidth}%
            \centering
            \phantom{x} % Phantom to fix broken alignment
            \begin{equation*}
                \sum{v_k} = 0
            \end{equation*}
            \phantom{x} % Phantom to fix broken alignment

            % Write about assumptions made?
            %\ExtraNotes{\footnotesize \textit{Assumes that...}}
        \end{minipage}

        \CheatsheetEntryExtraSeparation

        \CheatsheetEntryTitle{Supermesh (for Mesh Analysis)}

        A \textit{supermesh} is formed from two adjacent meshes with a common current source (dependent or independent).

        %\begin{center}
        \begin{circuitikz}
            \path
                (0,0) coordinate (BL)
                ++( 2.2,0  ) coordinate (BR)
                ++( 0  ,2.2) coordinate (MR)
                ++(-1.1,0  ) coordinate (MM)
                ++(-1.1,0  ) coordinate (ML)
                ++( 0  ,2.2) coordinate (TL)
                ++( 2.2,0  ) coordinate (TR)

                (TR) ++(0.94,-0.60) coordinate (Line0)
                ++(0,-1.30) coordinate (Line1)
                ++(0,-0.65) coordinate (Line2)
                ++(0,-0.65) coordinate (Line3)

                (MM) ++(0,1.3) coordinate (MeshCurrent1)
                (BL) ++(0.9,0.7) coordinate (MeshCurrent2)

                (TL) ++(-0.38, 0.38) coordinate (RectangleTL)
                (BR) ++( 0.38,-0.38) coordinate (RectangleBR)
            ;
            \draw[{myorange!30!white}, fill={myyellow!10!white}, line width=1.8pt, rounded corners=2.5mm]
                (RectangleTL) rectangle (RectangleBR)
            ;
            \draw
                (BL)
                to[generic] (ML)
                to[generic] (TL)
                to[generic] (TR)
                to[generic] (MR)
                to[generic] (BR)
                to[generic] (BL)
            ;
            \draw 
                (MM) to[I, l_={\color{myblue} $i_s$}, color=myblue, /tikz/circuitikz/bipoles/length=1.2cm] (ML)
                (MM) to[generic] ++(0.9,0) -- (MR)

                (TL) -- ++(-0.58, 0   ) coordinate (TLL)
                (TL) -- ++( 0   , 0.58) coordinate (TLT)
                (TR) -- ++( 0   , 0.58) coordinate (TRT)
                (TR) -- ++( 0.58, 0   ) coordinate (TRR)
                (MR) -- ++( 0.58, 0   ) coordinate (MRR)
                (BR) -- ++( 0.58, 0   ) coordinate (BRR)
                (BR) -- ++( 0   ,-0.58) coordinate (BRB)
                (BL) -- ++( 0   ,-0.58) coordinate (BLB)
                (BL) -- ++(-0.58, 0   ) coordinate (BLL)
                (ML) -- ++(-0.58, 0   ) coordinate (MLL)
            ;
            \draw[gray, line cap=round, dash pattern=on 0.5mm off 0.5mm]
            %\draw[lightgray]
            %\draw[dotted]
            %\draw[line cap=round, dash pattern=on 0.1mm off 0.5mm]
                (TLL) ++(-0.5mm, 0    ) -- ++(-0.18, 0   )
                (TLT) ++( 0    , 0.5mm) -- ++( 0   , 0.18)
                (TRT) ++( 0    , 0.5mm) -- ++( 0   , 0.18)
                (TRR) ++( 0.5mm, 0    ) -- ++( 0.18, 0   )
                (MRR) ++( 0.5mm, 0    ) -- ++( 0.18, 0   )
                (BRR) ++( 0.5mm, 0    ) -- ++( 0.18, 0   )
                (BRB) ++( 0    ,-0.5mm) -- ++( 0   ,-0.18)
                (BLB) ++( 0    ,-0.5mm) -- ++( 0   ,-0.18)
                (BLL) ++(-0.5mm, 0    ) -- ++(-0.18, 0   )
                (MLL) ++(-0.5mm, 0    ) -- ++(-0.18, 0   )
            ;
            \draw[extranotecolor] % Text
                (Line0) node[right, align=left] {Components in series\\with current source\\are frequently ignored.}
            ;
            \draw % Text
                (Line1) node[right] {\textbf{Forms two equations:}}
                (Line2) ++(0.7,0) node[left] {1)}
                    ++(-0.1,0) node[right] {KVL around supermesh}
                (Line3) ++(0.7,0) node[left] {2)}
                    ++(-0.1,0) node[right] {$i_s = i_1 + i_2$}
            ;
            \path % Get coordinates for bezier curve arrows
                (Line0)
                ++(-0.1,0) coordinate (ArrowStart0)
                ++(-0.6,0) coordinate (ArrowControl0)
                (MR) ++(-0.5,0.3) coordinate (ArrowEnd0)

                (Line2)
                ++(0.1,0) coordinate (ArrowStart1)
                %++(-0.5,0) coordinate (ArrowControl1)
                (ArrowStart1 -| RectangleBR) ++(0.1,0) coordinate (ArrowEnd1)

                (Line3)
                ++(0.1,-0) coordinate (ArrowStart2)
                ++(-1.15,0) coordinate (ArrowControl2)
                (MM) ++(-0.55,0) ++(0.35,-0.35) coordinate (ArrowEnd2)
            ;
            \draw[mygreen, ->, line width=1.7pt, line cap=round] % Draw bezier curve arrow
                (ArrowStart0) .. controls (ArrowControl0) .. (ArrowEnd0)
            ;
            \draw[myred, ->, line width=1.7pt, line cap=round] % Draw arrow
                (ArrowStart1) -- (ArrowEnd1)
            ;
            \draw[myblue, ->, line width=1.7pt, line cap=round] % Draw bezier curve arrow
                (ArrowStart2) .. controls (ArrowControl2) .. (ArrowEnd2)
            ;
            \draw[myblue]
                (MeshCurrent1) pic {simshadows/MeshCW} node {$i_1$}
                (MeshCurrent2) pic {simshadows/MeshCC} node {$i_2$}
            ;
        \end{circuitikz}%
        %\end{center}

        % TODO: Can you generalize to >2 meshes?

    }

    \MulticolsBreak

    \CheatsheetEntryFrame{

        \CheatsheetEntryTitle{Source Transformation}

        \renewcommand{\MyReusableFormatting}[2]{%
            \begin{center}
            \scalebox{0.94}{%
            \begin{circuitikz}
                \path
                    (0,0)      coordinate (LeftBR)
                    ++(3.3,0)  coordinate (RightBL)
                    ++(0,0.75) coordinate (MidHorizontal)
                ;
                \draw
                    (LeftBR)
                    to[short, o-] ++(-2.0,0)
                    to[#1, l=$v_s$, invert, /tikz/circuitikz/bipoles/length=1.25cm] ++(0,1.5)
                    to[R, l=$R$, -o] ++(2.0,0)
                    ++(0.1,0) node[right] {$a$}
                    (LeftBR) ++(0.1,0) node[right] (LeftEdgeRef) {$b$}
                ;
                \draw
                    (RightBL)
                    to[short, -o] ++(2.0,0)
                    ++(0.1,0) node[right] {$b$}

                    (RightBL) to[#2, name=CS, l=$i_s$, /tikz/circuitikz/bipoles/length=1.25cm] ++(0,1.5)
                    to[short, -o] ++(2.0,0)
                    ++(0.1,0) node[right] {$a$}

                    (RightBL) ++(1.2,0)
                    to[R, l=$R$] ++(0,1.5)
                ;
                \path % Calculates middle point.
                    (CS.north) ++(-0.5,0) coordinate (RightEdgeRef) % Ghetto calculation of right edge ref coordinate
                    ($(LeftEdgeRef.east)!0.5!(RightEdgeRef)$) coordinate (MidVertical)
                    (MidVertical |- MidHorizontal) coordinate (Mid)
                ;
                %\draw % Ghetto version
                %    (Mid) ++(0,0.15) node {\large $\xLeftrightarrow{\text{equivalent}}$}
                %;
                \draw[latex-latex, thick, line width=2.0pt]
                    (Mid) ++(-0.6,0)
                    --    ++( 1.2,0)
                ;
                \draw
                    %(Mid) ++(0,0.05) node[above] {equivalent}
                    %(Mid) ++(0,-0.08) node[below] {iff}
                    %      ++(0,-0.40) node[below] {$v_s = i_s R$}
                    (Mid) ++(0,0.09) node[above] {$v_s = i_s R$}
                    (Mid) ++(0,-0.09) node[below, align=center] {$R \notin \braces*{0, \infty}$}
                ;
            \end{circuitikz}
            }
            \end{center}
        }

        \MyReusableFormatting{V}{I}

        Dependent sources work exactly the same.

        %\MyReusableFormatting{cV}{cI}

        \CheatsheetEntryExtraSeparation

        \CheatsheetEntryTitle{Th\'evenin's Theorem}

        \begin{center}
        \scalebox{0.94}{%
        \begin{circuitikz}
            \path
                (0,0)      coordinate (LeftBR)
                ++(2.9,0)  coordinate (RightBL)
                ++(0,0.75) coordinate (MidHorizontal)
            ;
            \draw
                (LeftBR)
                to[short, o-] ++(-0.4,0)
                ++(0,1.5) coordinate (BoxTRref)
                to[short, -o] ++(0.4,0)
                ++(0.1,0) node[right] {$a$}
                (LeftBR) ++(0.1,0)
                node[right] (LeftEdgeRef) {$b$}
            ;
            \draw[lightgray, fill=verylightgray, line width=3.0pt]
                (BoxTRref)        ++( 0  , 0.3) coordinate (BoxTR)
                (LeftBR) ++(-2,0) ++(-1.1,-0.3) coordinate (BoxBL)
                (BoxBL) rectangle (BoxTR)
            ;
            \draw
                ($(BoxTR)!0.5!(BoxBL)$) coordinate (BoxMid)
                (BoxMid) node[align=center] {arbitrary\\linear network\\of sources\\and resistances}
            ;
            \draw
                (RightBL)
                to[short, -o] ++(2.0,0)
                ++(0.1,0) node[right] {$b$}

                (RightBL) to[V, name=S, l=$v_{th}$, invert, /tikz/circuitikz/bipoles/length=1.25cm] ++(0,1.5)
                to[R, l=$R_{th}$, -o] ++(2.0,0)
                ++(0.1,0) node[right] {$a$}
            ;
            \path % Calculates middle point.
                (S.north) ++(-0.5,0) coordinate (RightEdgeRef) % Ghetto calculation of right edge ref coordinate
                ($(LeftEdgeRef.east)!0.5!(RightEdgeRef)$) coordinate (MidVertical)
                (MidVertical |- MidHorizontal) coordinate (Mid)
            ;
            \draw[-latex, thick, line width=2.5pt]
                (Mid) ++(-0.5,0)
                --    ++( 1.0,0)
            ;
            %\draw
            %    (Mid) ++(0,0.09) node[above] {equivalent}
            %;
        \end{circuitikz}
        }
        \end{center}
        \vspace*{-2ex} % So much unnecessary whitespace.
        \begin{align*}
            v_{th} &= \text{open-circuit voltage at $a$-$b$} \\
            R_{th} &= \text{input resistance (looking into $a$-$b$)}
        \end{align*}

        Finding Th\'evenin resistance $R_{th}$:
        \begin{psmallindent}
            % TODO: Why does \textbu{\textsc{}} not give proper smallcaps here?
            {\footnotesize \textbu{\textsc{Method 1 (no dependent sources in network):}}}
            \begin{enumerate}
                \item DISABLE all independent sources.
                \item Find equivalent resistance looking into $a$-$b$.
            \end{enumerate}

            {\footnotesize \textbu{\textsc{Method 2:}}}
            \begin{enumerate}
                \item KEEP all independent sources.
                \item $\displaystyle R_{th} = \frac{\text{open-circuit voltage at $a$-$b$}}{\text{short-circuit current from $a$ to $b$}} = \frac{v_{th}}{i_N}$
            \end{enumerate}

            {\footnotesize \textbu{\textsc{Method 3:}}}
            \begin{enumerate}
                \item DISABLE all independent sources.
                \item Attach an independent source stimulus (usually $1 \si{\volt}$ or $1 \si{\ampere}$) to $a$-$b$.
                    \scalebox{0.85}{%
                    \begin{circuitikz}
                        \path
                            (0,0) coordinate (LeftBoxBRref)
                            ++(3.8,0) coordinate (RightBoxBRref)

                            (LeftBoxBRref)  ++(-0.7,0  ) ++(-0.3,-0.3) coordinate (LeftBoxBL)
                            (LeftBoxBRref)  ++( 0  ,1.5) ++( 0  , 0.3) coordinate (LeftBoxTR)
                            (RightBoxBRref) ++(-0.7,0  ) ++(-0.3,-0.3) coordinate (RightBoxBL)
                            (RightBoxBRref) ++( 0  ,1.5) ++( 0  , 0.3) coordinate (RightBoxTR)
                        ;
                        \draw
                            (LeftBoxBRref)
                            to[short, -o] ++( 0.4,0) coordinate (LeftTermB)
                            to[short]     ++( 0.4,0)
                            to[V, l_={\large $V_\text{test}$}, invert, /tikz/circuitikz/bipoles/length=1.20cm] ++(0,1.2)
                            to[short, i_={\large $I_\text{response}$}]                                         ++(0,0.1)
                            to[short]                                                                          ++(0,0.2)
                            to[short]     ++(-0.4,0) coordinate (LeftTermA)
                            to[short, o-] ++(-0.4,0)

                            (LeftTermA) ++(0, 0.1) node[above] {\large $a$}
                            (LeftTermB) ++(0,-0.1) node[below] {\large $b$}
                        ;
                        \draw
                            (RightBoxBRref)
                            to[short, -o] ++( 0.4,0) coordinate (RightTermB)
                            to[short]     ++( 0.4,0) coordinate (RightBR)
                            to[I, l_={\large $I_\text{test}$}, /tikz/circuitikz/bipoles/length=1.20cm] ++(0,1.2)
                            to[short]                                                                  ++(0,0.1)
                            to[short]                                                                  ++(0,0.2)
                            to[short]     ++(-0.4,0) coordinate (RightTermA)
                            to[short, o-] ++(-0.4,0)

                            (RightTermA) ++(0, 0.1) node[above] {\large $a$}
                            (RightTermB) ++(0,-0.1) node[below] {\large $b$}
                        ;
                        \draw[lightgray, fill=verylightgray, line width=3.0pt]
                            (LeftBoxBL) rectangle (LeftBoxTR)
                        ;
                        \draw[lightgray, fill=verylightgray, line width=3.0pt]
                            (RightBoxBL) rectangle (RightBoxTR)
                        ;
                        % Everything below will be for the voltage indicator
                        \draw[->, line width=1.3pt]
                            (RightBR)
                            ++(1.7,0)    coordinate (VArrowBot)
                            -- ++(0,1.5) coordinate (VArrowTop)
                        ;
                        \draw[dashed]
                            (VArrowTop) ++(-1.2,0) -- ++(1.7,0)
                            (VArrowBot) ++(-1.2,0) -- ++(1.7,0)
                        ;
                        \draw
                            %($(VArrowTop)!0.5!(VArrowBot)$) ++(0.1,0) node[right] {$V_\text{response}$}
                            (VArrowTop) ++(0,0.1) node[above] {\large $V_\text{response}$}
                        ;
                    \end{circuitikz}
                    }
                \item $\displaystyle R_{th} = \frac{V_{\text{test}}}{I_{\text{response}}} = \frac{V_{\text{response}}}{I_{\text{test}}}$
            \end{enumerate}

        \end{psmallindent}

        \CheatsheetEntryExtraSeparation

        \CheatsheetEntryTitle{Norton's Theorem}

        \begin{center}
        \scalebox{0.94}{%
        \begin{circuitikz}
            \path
                (0,0)      coordinate (LeftBR)
                ++(2.9,0)  coordinate (RightBL)
                ++(0,0.75) coordinate (MidHorizontal)
                -- (0,2.1) % Ghetto way to align similarly to the diagram of Thevenin's Theorem.
            ;
            \draw
                (LeftBR)
                to[short, o-] ++(-0.4,0)
                ++(0,1.5) coordinate (BoxTRref)
                to[short, -o] ++(0.4,0)
                ++(0.1,0) node[right] {$a$}
                (LeftBR) ++(0.1,0)
                node[right] (LeftEdgeRef) {$b$}
            ;
            \draw[lightgray, fill=verylightgray, line width=3.0pt]
                (BoxTRref)        ++( 0  , 0.3) coordinate (BoxTR)
                (LeftBR) ++(-2,0) ++(-1.1,-0.3) coordinate (BoxBL)
                (BoxBL) rectangle (BoxTR)
            ;
            \draw
                ($(BoxTR)!0.5!(BoxBL)$) coordinate (BoxMid)
                (BoxMid) node[align=center] {arbitrary\\linear network\\of sources\\and resistances}
            ;
            \draw
                (RightBL)
                to[short, -o] ++(2.0,0)
                ++(0.1,0) node[right] {$b$}

                (RightBL) to[I, name=S, l=$i_N$, /tikz/circuitikz/bipoles/length=1.25cm] ++(0,1.5)
                to[short, -o] ++(2.0,0)
                ++(0.1,0) node[right] {$a$}

                (RightBL) ++(1.4,0)
                to[R, l=$R_N$] ++(0,1.5)
            ;
            \path % Calculates middle point.
                (S.north) ++(-0.5,0) coordinate (RightEdgeRef) % Ghetto calculation of right edge ref coordinate
                ($(LeftEdgeRef.east)!0.5!(RightEdgeRef)$) coordinate (MidVertical)
                (MidVertical |- MidHorizontal) coordinate (Mid)
            ;
            \draw[-latex, thick, line width=2.5pt]
                (Mid) ++(-0.5,0)
                --    ++( 1.0,0)
            ;
            %\draw
            %    (Mid) ++(0,0.09) node[above] {equivalent}
            %;
        \end{circuitikz}
        }
        \end{center}
        \vspace*{-2ex} % So much unnecessary whitespace.
        \begin{align*}
            i_N &= \text{short-circuit current from $a$ to $b$} \\
            R_N &= R_{th}
        \end{align*}

    }

\end{multicols}
\begin{multicols}{2}

    \CheatsheetEntryFrame{

        \CheatsheetEntryTitle{Superposition Principle (in DC Analysis)}

        In a linear circuit, the total effect of several independent sources acting simultaneously is equal to the sum of each independent source acting alone \textit{(i.e. only one independent source active, with all other independent sources disabled)}.

        \textsc{do not disable dependent sources!}

        \textsc{also, this only works for linear responses, so superposition does not work with power.}

    }

\end{multicols}

%%%%%%%%%%%%%%%%%%%%%%%%%%%%%%%%%%%%%%%%%%%%%%%%%%%%%%%%%%%%%%%%%%%%%%%%%%%%%%%%%%%%%%%%%%%%%%%%%%%%
%%%%%%%%%%%%%%%%%%%%%%%%%%%%%%%%%%%%%%%%%%%%%%%%%%%%%%%%%%%%%%%%%%%%%%%%%%%%%%%%%%%%%%%%%%%%%%%%%%%%

\newpage
\subsection{Sinusoidal Steady-State Analysis}%
\label{sub:sinusoidal-steady-state-analysis}

\begin{multicols}{2}

    \CheatsheetEntryFrame{

        \CheatsheetEntryTitle{Phasor Domain Conversion}

        \renewcommand{\X}{\hphantom{x}} % Adding horizontal space to make the table distribute better and look nicer
        \begin{tabularx}{\textwidth}{ccC}
            {\scriptsize \textbf{Time Domain}}   &                             & {\scriptsize \textbf{Phasor Domain}} \\
            \X $v(t) = V_m \cos{(\omega t + \phi)}$ \X & $\Longleftrightarrow$ & $\mathbf{V} = V_m \angle \phi$       \\
            \X $i(t) = I_m \cos{(\omega t + \phi)}$ \X & $\Longleftrightarrow$ & $\mathbf{I} = I_m \angle \phi$       \\
        \end{tabularx}

        %\begin{tikzpicture}[scale=0.85, transform shape]
        %    \draw[help lines, lightgray]
        %        (0,-1) grid (4,1)
        %    ;
        %    \draw[-latex, line width=1.5pt]
        %        (0,0) -- (4.4,0) coordinate (LabelX)
        %    ;
        %    \draw[myred, dotted, line width=1.2pt, domain=0:4, samples=100]
        %        plot (\x, {cos(0.5*pi*\x r)}) node[right] {\color{mylightred} $\cos{(\omega t)}$}
        %    ;
        %    \draw[myred, line width=2pt, domain=0:4, samples=100]
        %        plot (\x, {cos(0.5*pi*\x r + 70)}) node[right] {$\cos{(\omega t + \phi)}$}
        %    ;
        %    \draw[latex-latex, line width=1.5pt] % Draw vertical line over the curve cuz it looks nicer
        %        (0,-1.4) -- (0,1.4) coordinate (LabelY)
        %    ;
        %    \draw
        %        (LabelX) ++(0  ,-0.1) node[below] {$t$}
        %        (LabelY) ++(0.1, 0  ) node[right] {$x$}
        %    ;
        %\end{tikzpicture}

        \vspace*{1ex}

        \renewcommand{\W}{60} % Phasor angle.
        % IMPORTANT: We will need to manually calculate WaveActualIntercept.
        %            To make it easy, just input this into WolframAlpha:
        %                2*(arccos(0) + (- YOUR_PHASOR_ANGLE_HERE * pi / 180))/pi + 2
        \begin{center}
        \begin{tikzpicture}[x=1.72cm, y=1.72cm, transform shape]
            \begin{scope}[shift={(0,-1.8)}, rotate=-90]
                \path
                    (0,0) coordinate (WaveOrigin)
                    (3,0) coordinate (WaveReferenceIntercept) % Manually calculated.
                    (2.33333333,0) coordinate (WaveActualIntercept) % Manually calculated.
                    % TODO: Maybe automatically calculate these intercepts next time?
                ;
                %\draw[help lines, lightgray]
                %    (0,-1) grid (4,1)
                %;
                \draw[-latex, line width=1.5pt]
                    (0,0) -- (4.4,0) coordinate (LabelX1)
                ;
                \draw[myred, dashed, line width=1.2pt, domain=0:4, samples=100, line cap=round]
                    plot (\x, {cos(0.5*pi*\x r)}) node[right] {\color{myred} $\bm{V_m \cos{(\omega t)}}$}
                ;
                \draw[myred, line width=2pt, domain=0:4, samples=100, line cap=round]
                    plot (\x, {cos(0.5*pi*\x r + \W)}) node[right] {$\bm{V_m \cos{(\omega t + \phi)}}$}
                    coordinate (WaveEnd)
                ;
                \draw[latex-latex, line width=1.5pt] % Draw vertical line over the curve cuz it looks nicer
                    (0,-1.4) -- (0,1.4) coordinate (LabelY1)
                ;
                \draw
                    (LabelX1) ++(0  ,-0.1) node[below] {$\omega t$}
                    (LabelY1) ++(0.1, 0  ) node[right] {$v(t)$}
                ;
                \draw[mypurple]
                    (WaveOrigin) ++(0, 1) ++(0,0.17) node[left] {$\bm{V_m}$}
                    (WaveOrigin) ++(0,-1) ++(0,0.17) node[left] {$\bm{-V_m}$}
                ;
            \end{scope}
            \begin{scope}[shift={(0,0)}]
                \draw[help lines, verylightgray, line width=1.5pt]
                    (0,0) coordinate (PhasorOrigin) circle[radius=1]
                    (1,0) coordinate (CircleRight)
                    (-1,0) coordinate (CircleLeft)
                ;
                \draw[latex-latex, line width=1.5pt]
                    (-1.4,0) -- (1.4,0) coordinate (LabelX2)
                ;
                \draw[latex-latex, line width=1.5pt]
                    (0,-1.4) -- (0,1.4) coordinate (LabelY2)
                ;
                \draw[-latex, myred, line width=2pt, line cap=round]
                    (PhasorOrigin) -- ++(\W:1) coordinate (PhasorEnd)
                ;
                \draw
                    (LabelX2) ++(-0.1 ,-0.05) node[below] {$\MyRe$}
                    (LabelY2) ++( 0.05,-0.1 ) node[right] {$\MyIm$}
                ;
                \draw[mypurple] (30:0.25) node {$\bm{\phi}$};
                \draw[-latex, mypurple, line width=1.2pt, line cap=round] (0:0.40) arc (0:\W:0.40);
                \draw[myred] (PhasorEnd) ++(\W:0.15) node {$\mathbf{V}$};
                % Directions
                \draw[angle 60 reversed-angle 60, mygreen,  line width=1.7pt, line cap=round] ( 15:1.8) arc ( 15: 40:1.8);
                \draw[angle 60 reversed-angle 60, myblue,   line width=1.7pt, line cap=round] (-15:1.8) arc (-15:-40:1.8);
                \draw[angle 60 reversed-angle 60, mypurple, line width=1.7pt, line cap=round] ( 60:1.8) arc ( 60:120:1.8);
                \draw[mygreen]  ( 25:1.8) node[above right, align=left] {\textbf{leading}\\\textbf{direction}};
                \draw[myblue]   (-25:1.8) node[below right, align=left] {\textbf{lagging}\\\textbf{direction}};
                \draw[mypurple] ( 90:1.8) node[above] {\textbf{angular velocity} $\bm{\omega}$};
                %\draw (0,0) circle[radius=2.5]; % Ghetto alignment
            \end{scope}
            \begin{scope}[on background layer]
                %\draw[myblue, dashed, line width=1.5pt] % Old styling for the phasor connector
                \draw[help lines, verylightgray, line width=1.5pt]
                    (PhasorEnd) -- (PhasorEnd |- WaveOrigin)
                ;
                \draw[help lines, verylightgray, line width=1.5pt]
                    (CircleRight)  -- (CircleRight  |- WaveEnd)
                    (CircleLeft)   -- (CircleLeft   |- WaveEnd)
                    (PhasorOrigin |- WaveEnd) ++(-1.4,0) -- ++(2.8,0)
                    %(PhasorOrigin) -- (PhasorOrigin |- WaveOrigin) % Looks ugly
                ;
                \draw[help lines, {mypurple!30!white}, line width=1.5pt]
                    (WaveActualIntercept)    ++(-1.4,0) -- ++(2.8,0) -- ++(0.5,0) ++(-0.2,-0.00) coordinate (WavePhaseDiffEnd)
                ;
                \draw[help lines, {mypurple!30!white}, dashed, line width=1.5pt]
                    (WaveReferenceIntercept) ++(-1.4,0) -- ++(2.8,0) -- ++(0.5,0) ++(-0.2, 0.00) coordinate (WavePhaseDiffStart)
                ;
            \end{scope}
            \begin{scope}[rotate=-90]
                \draw[-angle 60, mypurple, line width=1.7pt]
                    (WavePhaseDiffStart) -- (WavePhaseDiffEnd)
                ;
                \draw[mypurple]
                    ($(WavePhaseDiffStart)!0.5!(WavePhaseDiffEnd)$) coordinate (WavePhaseDiffMid)
                    (WavePhaseDiffMid) ++(0,0.05) node[above] {$\bm{\phi}$}
                    %(WavePhaseDiffStart) ++(0.05,0) node[right, align=left] {\textbf{phase}} % Ugly
                    %(WavePhaseDiffEnd) ++(-0.05,0) node[left, align=right] {\textbf{phase}} % Also ugly
                    (WavePhaseDiffMid) ++(0,0.35) node[above, align=center] {\textbf{phase}}
                ;
                \draw[angle 60 reversed-angle 60, mygreen, line width=1.7pt, line cap=round]
                    (WavePhaseDiffEnd)   ++(-0.6,0) -- ++(-1,0);
                \draw[angle 60 reversed-angle 60, myblue,  line width=1.7pt, line cap=round]
                    (WavePhaseDiffStart) ++( 0.6,0) -- ++( 1,0);
                \draw[myblue]  (WavePhaseDiffStart) ++( 1.1,0.35) node[above, align=center] {\textbf{lagging}\\\textbf{direction}};
                \draw[mygreen] (WavePhaseDiffEnd)   ++(-1.1,0.35) node[above, align=center] {\textbf{leading}\\\textbf{direction}};
            \end{scope}
        \end{tikzpicture}
        \end{center}

    }

    \CheatsheetEntryFrameExn{

        \CheatsheetEntryTitle{Appendix: Useful Periodic Function Relations}
        \begin{gather*}
            \omega = 2 \pi f = \frac{2 \pi}{T}
            \qquad
            \begin{aligned}
                \cos{\parens*{\omega t}} &= \sin{\parens*{\omega t + \ang{90}}} \\
                \sin{\parens*{\omega t}} &= \cos{\parens*{\omega t - \ang{90}}}
            \end{aligned}
            \\
            F_{\text{avg}} = \frac{1}{T} \int_0^T{f(t) \,\diff{t}}
            \qquad
            F_{\text{rms}} = \sqrt{\frac{1}{T} \int_0^T{\parens*{f(t)}^2 \,\diff{t}}}
        \end{gather*}%

    }

    % Should naturally column-break here

    \CheatsheetEntryFrame{

        \renewcommand{\MyReusableFormatting}[3]{
            \begin{minipage}[c]{0.33\columnwidth}%
                \centering
                #1
            \end{minipage}%
            \begin{minipage}[c]{0.33\columnwidth}%
                \centering
                #2
            \end{minipage}%
            \begin{minipage}[c]{0.33\columnwidth}%
                \centering
                #3
            \end{minipage}%
        }

        \CheatsheetEntryTitle{Resistor, Capacitor, and Inductor}

        \vspace*{1.5ex}
        \MyReusableFormatting{
            $\displaystyle \mathbf{Z}_R = R$
        }{
            $\displaystyle \mathbf{Z}_C = \frac{1}{j \omega C}$
        }{
            $\displaystyle \mathbf{Z}_L = j \omega L$
        }

        \renewcommand{\MyReusableFormattingB}{\path (0,0.6) -- (0,-0.6);} % Ghetto alignment
        \MyReusableFormatting{
            \begin{circuitikz}
                \MyReusableFormattingB
                \draw (0,0) to[R, o-o] ++(2,0);
            \end{circuitikz}
        }{
            \begin{circuitikz}
                \MyReusableFormattingB
                \draw (0,0) to[C, o-o] ++(2,0);
            \end{circuitikz}
        }{
            \begin{circuitikz}
                \MyReusableFormattingB
                \draw (0,0) to[L, o-o] ++(2,0);
            \end{circuitikz}
        }%
        \vspace{-2.0ex} % Too much whitespace for some reason. Need to cut it down.
        \renewcommand{\MyReusableFormattingB}{\path (0,-1.2) -- (0,1.2) (-0.40,0) -- (1.40,0);} % Ghetto alignment
        %\renewcommand{\MyReusableFormattingB}{\path (-0.40,0) -- (1.40,0);} % Ghetto alignment
        \MyReusableFormatting{%
            \begin{tikzpicture}[x=1.0cm, y=1.0cm, transform shape]
                \MyReusableFormattingB
                \draw[-stealth, myred,    line width=2.0pt, line cap=round] (0,0) -- ++(0:1.30) coordinate (VEnd);
                \draw[-stealth, white,    line width=2.0pt, line cap=round] (0,0) -- ++(0:1.05);
                \draw[-stealth, mypurple, line width=2.0pt, line cap=round] (0,0) -- ++(0:1.00) coordinate (IEnd);
                \draw[myred]    (VEnd) ++( 0  ,-0.05) node[below] {$\mathbf{V}$};
                \draw[mypurple] (IEnd) ++(-0.1,-0.05) node[below] {$\mathbf{I}$};
            \end{tikzpicture}
        }{%
            \begin{tikzpicture}[x=1.0cm, y=1.0cm, transform shape]
                \MyReusableFormattingB
                \draw[line width=1.2pt] (0.2,0) -- (0.2,-0.2) -- (0,-0.2);
                \draw[-stealth, mypurple, line width=2.0pt, line cap=round] (0,0) -- ++(  0:1) coordinate (IEnd);
                \draw[-stealth, myred,    line width=2.0pt, line cap=round] (0,0) -- ++(-90:1) coordinate (VEnd);
                \draw[myred]    (VEnd) ++(0.0, 0   ) node[right] {$\mathbf{V}$};
                \draw[mypurple] (IEnd) ++(0  ,-0.05) node[below] {$\mathbf{I}$};
            \end{tikzpicture}
        }{%
            \begin{tikzpicture}[x=1.0cm, y=1.0cm, transform shape]
                \MyReusableFormattingB
                \draw[line width=1.2pt] (0.2,0) -- (0.2,0.2) -- (0,0.2);
                \draw[-stealth, mypurple, line width=2.0pt, line cap=round] (0,0) -- ++( 0:1) coordinate (IEnd);
                \draw[-stealth, myred,    line width=2.0pt, line cap=round] (0,0) -- ++(90:1) coordinate (VEnd);
                \draw[myred]    (VEnd) ++(0.0, 0   ) node[right] {$\mathbf{V}$};
                \draw[mypurple] (IEnd) ++(0  ,-0.05) node[below] {$\mathbf{I}$};
            \end{tikzpicture}
        }%
        \vspace{-1.0ex} % Move closer to the next section

        \renewcommand{\MyReusableFormatting}[2]{%
            \begin{minipage}[c]{0.78\columnwidth}
                #1
            \end{minipage}%
            {\color{CheatsheetSepColor} \vrule{}}%
            %\hspace{0.06\columnwidth}% TODO: Do something better than this ghetto center-alignment.
            \begin{minipage}[c]{0.2\columnwidth}
                #2
            \end{minipage}
        }

        \CheatsheetEntryTitle{Ohm's Law}

        \MyReusableFormatting{%
            For impedance $\mathbf{Z}$:
            \begin{equation*}
                \mathbf{V} = \mathbf{I} \mathbf{Z}
            \end{equation*}

            For admittance $\mathbf{Y}$:
            \begin{equation*}
                \mathbf{Y} \triangleq \frac{1}{\mathbf{Z}}
                \qquad \Rightarrow \qquad
                \mathbf{I} = \mathbf{V} \mathbf{Y}
            \end{equation*}
        }{%
            \begin{circuitikz}
                \path (-1,0) -- (0,0); % Ghetto alignment
                \draw
                    (0,0) to[short] ++(0,-0.2)
                    to[short, i=$\mathbf{I}$] ++(0,-0.1)
                    to[generic, l=$\mathbf{Z}$, v=$\mathbf{V}$] ++(0,-1.6)
                    to[short] ++(0,-0.3)
                ;
            \end{circuitikz}%
        }

        \CheatsheetEntryExtraSeparation

        \CheatsheetEntryTitle{Resistance ($R$) and Reactance ($X$)}
        \begin{equation*}
            \mathbf{Z} = R + jX
            ,\qquad R, X \in \mathbb{R}
        \end{equation*}

        \CheatsheetEntryTitle{Conductance ($G$) and Susceptance ($B$)}
        \begin{equation*}
            \mathbf{Y} = G + jB
            ,\qquad G, B \in \mathbb{R}
        \end{equation*}

    }

    \CheatsheetEntryFrame{

        % TODO: This is basically a reuse of the DC Analysis section's version. See if we can just make a macro for it?
        \CheatsheetEntryTitle{Series and Parallel Equivalent}

        \vspace{1.5ex}
        \begin{minipage}[c]{0.5\columnwidth}
            \centering
            \scalebox{0.8}{
            \begin{circuitikz}
                \draw
                    (0,0)
                    to[generic, o-] ++(2,0)
                    to[generic, -o] ++(2,0)
                ;
            \end{circuitikz}%
            }
        \end{minipage}%
        \begin{minipage}[c]{0.5\columnwidth}
            \centering
            \scalebox{0.8}{
            \begin{circuitikz}
                \draw
                    (0.5,0) to[short, o-]
                    (1,0) -- (1, 0.4) to[generic] (3, 0.4) -- (3,0)
                    to[short, -o] (3.5,0)
                    (1,0) -- (1,-0.4) to[generic] (3,-0.4) -- (3,0)
                ;
            \end{circuitikz}%
            }
        \end{minipage}

        \vspace*{1.5ex}

        \begin{minipage}[c]{0.5\columnwidth}
            \begin{equation*}
                \mathbf{Z}_S = \sum{\mathbf{Z}_i}
            \end{equation*}
        \end{minipage}%
        \begin{minipage}[c]{0.5\columnwidth}
            \begin{equation*}
                \frac{1}{\mathbf{Z}_P} = \sum{\frac{1}{\mathbf{Z}_i}}
            \end{equation*}
        \end{minipage}

        \begin{minipage}[c]{0.5\columnwidth}
            \begin{equation*}
                \frac{1}{\mathbf{Y}_S} = \sum{\frac{1}{\mathbf{Y}_i}}
            \end{equation*}
        \end{minipage}%
        \begin{minipage}[c]{0.5\columnwidth}
            \begin{equation*}
                \mathbf{Y}_P = \sum{\mathbf{Y}_i}
            \end{equation*}
        \end{minipage}

    }

    \CheatsheetEntryFrame{

        % TODO: This is basically a reuse of the DC Analysis section's version. See if we can just make a macro for it?
        \CheatsheetEntryTitle{Voltage and Current Division}

        \begin{minipage}[c]{0.6\columnwidth}
            \centering
            \scalebox{1}{
            \begin{circuitikz}
                \draw
                    (0,0)
                    to[short, o-] ++(1,0)
                    -- ++(0,-0.2)
                    to[generic, l_=$\mathbf{Z}_1$, v^=${\displaystyle \mathbf{V}_1 = \frac{\mathbf{Z}_1}{\mathbf{Z}_1+\mathbf{Z}_2}\mathbf{V}}$] ++(0,-1.5)
                    -- ++(0,-0.3)
                    to[generic, l_=$\mathbf{Z}_2$, v^=${\displaystyle \mathbf{V}_2 = \frac{\mathbf{Z}_2}{\mathbf{Z}_1+\mathbf{Z}_2}\mathbf{V}}$] ++(0,-1.5)
                    -- ++(0,-0.2)
                    to[short, -o] ++(-1,0)
                    (0,0)
                    to[open, v=$\mathbf{V}$] (0,-3.7)
                ;
            \end{circuitikz}%
            }
        \end{minipage}%
        \begin{minipage}[c]{0.4\columnwidth}
            \centering
            \scalebox{1}{
            \begin{circuitikz}
                \draw
                    (0,1.65)
                    to[short, i=$\mathbf{I}$, o-] ++(1,0)
                    to[short] ++(0,-0.15)
                    to[generic, l_=$\mathbf{Z}_1$, i>_=$\mathbf{I}_1$] ++(0,-1.5)
                    %to[short] ++(0,-0.25)
                    to[short, -o] ++(-1,0)
                    (1,1.65)
                    to[short] ++(1,0)
                    to[short] ++(0,-0.15)
                    to[generic, l_=$\mathbf{Z}_2$, i>_=$\mathbf{I}_2$] ++(0,-1.5)
                    %to[short] ++(0,-0.25)
                    to[short] ++(-1,0)
                    (1,0) ++(0,-0.1)
                    node[below] {${\displaystyle \mathbf{I}_1 = \frac{\mathbf{Z}_2}{\mathbf{Z}_1+\mathbf{Z}_2}\mathbf{I}}$} ++(0,-1)
                    node[below] {${\displaystyle \mathbf{I}_2 = \frac{\mathbf{Z}_1}{\mathbf{Z}_1+\mathbf{Z}_2}\mathbf{I}}$}
                ;
            \end{circuitikz}%
            }
        \end{minipage}

    }

\end{multicols}
\begin{multicols}{2}

    \CheatsheetEntryFrame{

        \CheatsheetEntryTitle{RMS Value}

        The \textit{RMS} value of \myul{any periodic voltage/current} is the equivalent value in DC to deliver the same average power to a load.
        \begin{equation*}
            F_\text{rms} =\sqrt{\frac{1}{T} \int_0^T{\parens*{f(t)}^2 \,\diff{t}}} 
        \end{equation*}

        For sinusoidal waveforms:

        \begin{minipage}{0.5\columnwidth}%
            \begin{equation*}
                V_\text{rms} = \frac{V_m}{\sqrt{2}} \approx 0.707 V_m
            \end{equation*}
        \end{minipage}%
        \begin{minipage}{0.5\columnwidth}%
            \begin{equation*}
                I_\text{rms} = \frac{I_m}{\sqrt{2}} \approx 0.707 I_m
            \end{equation*}
        \end{minipage}%

        %% I'm not 100% sure on this one. Will need to learn more!
        %% TODO: Look into this section!
        %\vspace{\parskip}%
        %For a waveform produced from a sum of waveforms:
        %\begin{gather*}
        %    f(t) = f_1(t) + f_2(t) + \dots + f_n(t) \\
        %    F_\text{rms}^2 = F_{\text{rms}1}^2 + F_{\text{rms}2}^2 + \dots + F_{\text{rms}n}^2 
        %\end{gather*}

        \CheatsheetEntryExtraSeparation

    }

    \CheatsheetEntryFrame{

        \CheatsheetEntryTitle{Average Power}

        \Todo{Finish this section.}
        \begin{equation*}
            P = I_\text{rms}^2 R
        \end{equation*}

    }

    \MulticolsBreak

    \Todo{Topic on phasor diagram trigonometry? E.g. adding two voltage phasors with cosine rule?}
    
\end{multicols}

%%%%%%%%%%%%%%%%%%%%%%%%%%%%%%%%%%%%%%%%%%%%%%%%%%%%%%%%%%%%%%%%%%%%%%%%%%%%%%%%%%%%%%%%%%%%%%%%%%%%
%%%%%%%%%%%%%%%%%%%%%%%%%%%%%%%%%%%%%%%%%%%%%%%%%%%%%%%%%%%%%%%%%%%%%%%%%%%%%%%%%%%%%%%%%%%%%%%%%%%%

\newpage
\subsection{Digital Circuits and Logic}%
\label{sub:digital-circuits-and-logic}

\begin{multicols}{2}

    \CheatsheetEntryFrame{

        \renewcommand{\MyReusableFormatting}[5]{ % This will be used for content
            %\vspace*{1ex}
            \begin{minipage}[c]{0.20\columnwidth}
                \centering
                #1
            \end{minipage}%
            \begin{minipage}[c]{0.18\columnwidth}
                \centering
                #2
            \end{minipage}%
            %\hspace{0.04\columnwidth}%
            \begin{minipage}[c]{0.32\columnwidth}
                \centering
                \scalebox{0.8}{%
                \begin{tabular}{#3}
                    \HLineA
                    #4
                    \HLineA
                \end{tabular}
                }%
            \end{minipage}%
            \begin{minipage}[c]{0.30\columnwidth}
            \begin{center}
                \begin{circuitikz}
                    \draw
                        #5
                    ;
                \end{circuitikz}
            \end{center}
            \end{minipage}%
            \vspace*{1ex}
        }
        \renewcommand{\MyReusableFormattingB}{ % This will be used for separators
            {\color{lightgray} \hrule{}}
        }
        \renewcommand{\MyReusableFormattingC}{ % This will be used for special separators
            \MyReusableFormattingB%
            \vspace*{2.5pt}%
            \MyReusableFormattingB
        }

        \renewcommand{\X}{{\cellcolor{myred}\color{white}$\bm{0}$}} % Used for logical low
        \renewcommand{\Y}{{\cellcolor{mygreen}\color{white}$\bm{1}$}} % Used for logical high

        \renewcommand{\W}[1]{{\cellcolor{black}\color{white}$\bm{#1}$}} % Used for headings

        \renewcommand{\VRuleA}{\vrule width 3pt}
        \renewcommand{\HLineA}{\noalign{\hrule height 3pt}}

        \CheatsheetEntryTitle{Logic Gates}

        \MyReusableFormatting{NOT {\footnotesize(Inverter)}}{$\overline{A}$}{!{\VRuleA}c!{\VRuleA}c!{\VRuleA}}{
            \W{A} & \W{O} \\ \HLineA
            \X & \Y \\
            \Y & \X \\
        }{
            (0,0) node[american not port, name=G, /tikz/circuitikz/bipoles/length=1.0cm, /tikz/circuitikz/bipoles/not port/circle width=0.35] {}
            (G.in)  -- ++(-0.4,0)
            (G.out) -- ++( 0.4,0)
        }
        \MyReusableFormattingC

        \MyReusableFormatting{AND}{$A \cdot B$}{!{\VRuleA}cc!{\VRuleA}c!{\VRuleA}}{
            \W{A} & \W{B} & \W{O} \\ \HLineA
            \X & \X & \X \\
            \X & \Y & \X \\
            \Y & \X & \X \\
            \Y & \Y & \Y \\
        }{
            (0,0) node[american and port, name=G] {}
            (G.in 1) -- ++(-0.4,0)
            (G.in 2) -- ++(-0.4,0)
            (G.out)  -- ++( 0.4,0)
        }
        \MyReusableFormattingB

        \MyReusableFormatting{OR}{$A + B$}{!{\VRuleA}cc!{\VRuleA}c!{\VRuleA}}{
            \W{A} & \W{B} & \W{O} \\ \HLineA
            \X & \X & \X \\
            \X & \Y & \Y \\
            \Y & \X & \Y \\
            \Y & \Y & \Y \\
        }{
            (0,0) node[american or port, name=G] {}
            (G.in 1) -- ++(-0.4,0)
            (G.in 2) -- ++(-0.4,0)
            (G.out)  -- ++( 0.4,0)
        }
        \MyReusableFormattingB

        \MyReusableFormatting{XOR}{$A \oplus B$}{!{\VRuleA}cc!{\VRuleA}c!{\VRuleA}}{
            \W{A} & \W{B} & \W{O} \\ \HLineA
            \X & \X & \X \\
            \X & \Y & \Y \\
            \Y & \X & \Y \\
            \Y & \Y & \X \\
        }{
                (0,0) node[american xor port, name=G] {}
                (G.in 1) -- ++(-0.4,0)
                (G.in 2) -- ++(-0.4,0)
                (G.out)  -- ++( 0.4,0)
        }

        \MyReusableFormattingC

        \MyReusableFormatting{NAND}{$\overline{A \cdot B}$}{!{\VRuleA}cc!{\VRuleA}c!{\VRuleA}}{
            \W{A} & \W{B} & \W{O} \\ \HLineA
            \X & \X & \Y \\
            \X & \Y & \Y \\
            \Y & \X & \Y \\
            \Y & \Y & \X \\
        }{
            (0,0) node[american nand port, name=G] {}
            (G.in 1) -- ++(-0.4,0)
            (G.in 2) -- ++(-0.4,0)
            (G.out)  -- ++( 0.4,0)
        }
        \MyReusableFormattingB

        \MyReusableFormatting{NOR}{$\overline{A + B}$}{!{\VRuleA}cc!{\VRuleA}c!{\VRuleA}}{
            \W{A} & \W{B} & \W{O} \\ \HLineA
            \X & \X & \Y \\
            \X & \Y & \X \\
            \Y & \X & \X \\
            \Y & \Y & \X \\
        }{
            (0,0) node[american nor port, name=G] {}
            (G.in 1) -- ++(-0.4,0)
            (G.in 2) -- ++(-0.4,0)
            (G.out)  -- ++( 0.4,0)
        }
        \MyReusableFormattingB

        \MyReusableFormatting{XNOR}{$\overline{A \oplus B}$}{!{\VRuleA}cc!{\VRuleA}c!{\VRuleA}}{
            \W{A} & \W{B} & \W{O} \\ \HLineA
            \X & \X & \Y \\
            \X & \Y & \X \\
            \Y & \X & \X \\
            \Y & \Y & \Y \\
        }{
            (0,0) node[american xnor port, name=G] {}
            (G.in 1) -- ++(-0.4,0)
            (G.in 2) -- ++(-0.4,0)
            (G.out)  -- ++( 0.4,0)
        }

    }

    \MulticolsBreak

    \CheatsheetEntryFrame{

        \CheatsheetEntryTitle{Binary Numbers}

        \Todo{This!}

        \CheatsheetEntryExtraSeparation

        \CheatsheetEntryTitle{Hexadecimal and Octal Numbers}

        \Todo{This!}

    }

    \CheatsheetEntryFrame{

        %\CheatsheetEntryTitle{Commutative Laws}
        %\begin{gather*}
        %    A \cdot B = B \cdot A \\
        %    A + B = B + A
        %\end{gather*}

        %\CheatsheetEntryTitle{Associative Laws}
        %\begin{gather*}
        %    (A \cdot B) \cdot C = A \cdot (B \cdot C) = A \cdot B \cdot C \\
        %    (A + B) + C = A + (B + C) = A + B + C
        %\end{gather*}

        \CheatsheetEntryTitle{Distributive Laws}
        \begin{gather*}
            A \cdot (B + C) = (A \cdot B) + (A \cdot C) \\
            A + (B \cdot C) = (A + B) \cdot (A + C)
        \end{gather*}

        \CheatsheetEntryTitle{Absorption Laws}
        \begin{gather*}
            A + (A \cdot B) = A \\
            A \cdot (A + B) = A \\
            (A \cdot B) + (A \cdot \overline{B}) = A \\
            (A + B) \cdot (A + \overline{B}) = A
        \end{gather*}

        \CheatsheetEntryTitle{De Morgan's Theorem}
        \begin{gather*}
            \overline{A + B} = \overline{A} \cdot \overline{B} \\
            \overline{A \cdot B} = \overline{A} + \overline{B}
        \end{gather*}

    }

\end{multicols}

