\documentclass{book}

%\usepackage[margin=2.5cm, bottom=3.0cm]{geometry}
\usepackage[margin=1.6cm, top=2.0cm, bottom=2.5cm]{geometry}
\usepackage{makecell} % cell formatting for tabular environments
%\usepackage{multirow} % multi-row formatting for tabular environments
%\usepackage{booktabs}
%\usepackage{wrapfig} % Very buggy.
\usepackage[table]{xcolor}
\usepackage{subfigure}
\usepackage[thicklines]{cancel}
%\usepackage[
%    type={CC},
%    modifier={by-sa},
%    version={4.0},
%    hyphenation={RaggedRight},
%]{doclicense}

\usepackage{../_latex_includes/sharedpkg}
\usepackage[capitalise, nameinlink]{cleveref}
%\usepackage{svg} % Doesn't work so nicely

\usetikzlibrary{arrows, shapes.arrows, matrix}

\def\TITLE{Sim's Computing Revision Notes}

\setcounter{tocdepth}{2}
\setcounter{secnumdepth}{3}

\raggedbottom

\newcolumntype{L}[1]{>{\raggedright\let\newline\\\arraybackslash\hspace{0pt}}m{#1}}
\renewcommand{\CancelColor}{\color{red}}

% Operations Helpers
\newcommand{\OpSig}[4][]{%
    \def\TempA{#1}%
    \def\TempB{internal}%
    \ifx\TempA\TempB% # TODO: Is there a better way to test for keys?
        {\footnotesize\textsc{(Internal)}} %
    \fi%
    \def\TempA{}%
    \def\TempB{#4}%
    \ifx\TempA\TempB% TODO: Is there a better way to test for empty?
        % #2 is empty
        #2(\textit{#3})
    \else
        % #2 is non-empty
        #2(\textit{#3}) $\to$ \textit{#4}%
    \fi%
}
% Operations Section Helpers
\newcommand{\OpSigSep}{\qquad\qquad\qquad}
\newenvironment{OpSectionSummary}{%
    \begin{center}%
}{%
    \end{center}%
    \medskip%
}

\newcommand{\LoadHelper}[1]{\input{./helpers/#1}}
\newcommand{\LoadSection}[1]{{\input{./sections/#1}}}
\newcommand{\LoadAppendix}[1]{{\input{./sections-appendices/#1}}}

\LoadHelper{general}
\LoadHelper{cheatsheet-helpers}

\LoadHelper{tikz}

\LoadHelper{circuitikz}
\LoadHelper{math-and-cs}
\LoadHelper{minipage}
\LoadHelper{multicols}

\newcommand{\SkipAfterTable}{\bigskip}
\newcommand{\ExampleRef}[1]{\nameref{#1} \ref{#1}}

\begin{document}

\thispagestyle{plain}
\MakeCustomTitle
\bigskip

%\medskip
%text goes here?
%\medskip

%\doclicenseThis%%

{
    \hypersetup{linkcolor=black}
    \tableofcontents
}

\newpage
\mainmatter

\chapter{Fundamentals}%
\label{cha:fundamentals}

    \LoadSection{fundamentals/performance-and-big-o}
    \LoadSection{fundamentals/dynamic-arrays}
    \LoadSection{fundamentals/array-search-algorithms}
    \LoadSection{fundamentals/sorting-algorithms}
    \LoadSection{fundamentals/linked-lists}
    \LoadSection{fundamentals/bsts}
    \newpage
    \LoadSection{fundamentals/binary-heaps/intro}
    \newpage
    \LoadSection{fundamentals/binary-heaps/array-indexing}
    \newpage
    \LoadSection{fundamentals/binary-heaps/push}
    \newpage
    \LoadSection{fundamentals/binary-heaps/pop}
    \newpage
    \LoadSection{fundamentals/binary-heaps/pushpop}
    \newpage
    \LoadSection{fundamentals/binary-heaps/heapify}
    \newpage
    \LoadSection{fundamentals/binary-heaps/further-reading}
    \LoadSection{fundamentals/binary-heaps/references}
    \newpage
    \LoadSection{fundamentals/tries}
    \LoadSection{fundamentals/hash-tables}
    \LoadSection{fundamentals/graphs}

\newpage
\chapter{Divide-and-Conquer Algorithms}%
\label{cha:doc}

    \LoadSection{divide-and-conquer/intro}
    \LoadSection{divide-and-conquer/karatsuba-multiplication}
    \LoadSection{divide-and-conquer/fft}

\newpage
\chapter{Dynamic Programming}
\label{cha:dp}

    \LoadSection{dynamic-programming/intro}
    \LoadSection{dynamic-programming/fibonacci-sequence}
    \LoadSection{dynamic-programming/01-knapsack-problem}

\newpage
\chapter{Standard Implementations}
\label{cha:standard-implementations}

    \LoadSection{standard-implementations/javascript/intro}
    \newpage
    \LoadSection{standard-implementations/javascript/max-heap-implementation}
    \newpage
    \LoadSection{standard-implementations/python}

\appendix
\chapter{Appendices}

\section{Introduction to Fourier Analysis, for Computer Science Students}
\label{sec:appendix--fourier}

    \LoadAppendix{intro-to-fourier-analysis/background}

%\newpage
%\part{Revise}
%
%\Todo{Maybe write a more concise version of everything?}

\end{document}
