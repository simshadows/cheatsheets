\subsection{Differential Equations}%
\label{sub:diff-eq}

\begin{multicols}{2}

    \begin{CheatsheetEntryFrame}

        \CheatsheetEntryTitle{Separable ODEs}

        Write the ODE in the form:
        \begin{equation*}
            h(y) \,\diff{y} = g(x) \,\diff{x}
        \end{equation*}

        To solve, integrate both sides:
        \begin{equation*}
            H(y) = G(x) + C
        \end{equation*}

    \end{CheatsheetEntryFrame}

    \begin{CheatsheetEntryFrame}

        \CheatsheetEntryTitle{First-Order Linear ODEs}

        Write the ODE in the form:
        \begin{equation*}
            \frac{\diff{y}}{\diff{x}} + f(x) \, y = g(x)
        \end{equation*}

        First, define the \textit{integrating factor} $h$:
        \begin{equation*}
            h(x) = e^{\int{f(x) \,\diff{x}}}
            \Exn{
                \qquad \Longrightarrow \qquad
                h'(x) = f(x) \, h(x)
            }
        \end{equation*}

        Multiply the ODE by $h$ and undo the product rule:
        \begin{gather*}
            h(x) \, \frac{\diff{y}}{\diff{x}} + h(x) \, f(x) \, y = g(x) \, h(x) \\
            \frac{\diff{}}{\diff{x}} \parens*{h(x) \, y} = g(x) \, h(x)
        \end{gather*}

        To solve, integrate both sides:
        \begin{equation*}
            h(x) \, y = \int{g(x) \, h(x) \,\diff{x}}
        \end{equation*}

    \end{CheatsheetEntryFrame}

    \MulticolsBreak

    \begin{CheatsheetEntryFrame}

        \CheatsheetEntryTitle{Exact ODEs}

        Write the ODE in the form:
        \begin{equation*}
            F(x, y) + G(x, y) \frac{\diff{y}}{\diff{x}} = 0
        \end{equation*}

        First prove the ODE is \textit{exact} by showing:
        \begin{equation*}
            \frac{\partial{F}}{\partial{y}}
            = \frac{\partial{G}}{\partial{x}}
            \Exn{
                {} = \underbrace{
                    \frac{\partial^2{H}}{\partial{x} \, \partial{y}}
                    = \frac{\partial^2{H}}{\partial{y} \, \partial{x}}
                }_{\substack{\text{equality of}\\\text{mixed partials}}}
            }
        \end{equation*}
        \NegateBelowDisplaySkip

        Thus, our ODE can be rewritten:
        \begin{equation*}
            H_x(x, y) + H_y(x, y) \frac{\diff{y}}{\diff{x}} = 0
        \end{equation*}

        This is can be simplified by undoing the chain rule:
        \begin{gather*}
            \frac{\partial{H}}{\partial{x}} \frac{\diff{x}}{\diff{x}}
            + \frac{\partial{H}}{\partial{y}} \frac{\diff{y}}{\diff{x}}
            = 0 \\
            \frac{\diff{}}{\diff{x}} \parens*{H \parens*{x, y(x)}} = 0
        \end{gather*}

        Therefore, the solution is:
        \begin{equation*}
            H(x, y) = C
        \end{equation*}
        \NegateBelowDisplaySkip

        such that:
        \begin{alignat*}{3}
            H(x, y) &= \int{H_x(x, y) \,\partial{x}} &&= H_1(x, y) &&+ C_1(y) \\
            H(x, y) &= \int{H_y(x, y) \,\partial{y}} &&= H_2(x, y) &&+ C_2(x)
        \end{alignat*}

    \end{CheatsheetEntryFrame}

    \begin{CheatsheetEntryFrame}

        \CheatsheetEntryTitle{Second-Order Linear ODEs}

        \Todo{Do this!}

    \end{CheatsheetEntryFrame}

    \Todo{Consider adding a section on change of variable.}
    
\end{multicols}
\newpage
\begin{multicols}{2}

    \begin{CheatsheetEntryFrame}

        \CheatsheetEntryTitle{Trigonometric Fourier Series}

        For any \emph{periodic function} $f \parens*{t} = f \parens*{t + n T}$:
        \begin{align*}
            f \parens*{t}
            &= a_o
            + \sum_{n=1}^\infty \parens*{
                a_n \cos n \omega_0 t + b_n \sin n \omega_0 t
            }
            \\
            &= a_o
            + \sum_{n=1}^\infty {
                A_n \cos \parens*{n \omega_0 t + \phi_n}
            }
            \qquad\quad \text{\parbox{5em}{\footnotesize{}(Amplitude-Phase Form)}}
        \end{align*}
        \begin{gather*}
            a_0
            = \frac{1}{T} \int_0^T f\parens*{t} \,\diff{t} = \text{average value of $f$}
            \\
            a_n
            = \frac{2}{T} \int_0^T f\parens*{t} \cos n \omega_0 t \,\diff{t}
            \\
            b_n
            = \frac{2}{T} \int_0^T f\parens*{t} \sin n \omega_0 t \,\diff{t}
            \\
            A_n = \sqrt{a_n^2 + b_n^2}
            \qquad\quad
            \phi_n = - \tan^{-1} \frac{b_n}{a_n}
        \end{gather*}

        Fundamental angular frequency:
        \begin{gather*}
            \omega_0 = \frac{2 \pi}{T}
        \end{gather*}
        Each element of the sum is called a \emph{harmonic}.

        Dirichlet conditions (for a convergent Fourier series):
        \begin{itemize}
            \item $f$ must be single-valued everywhere.
            \item $f$ must have a finite number of finite discontinuities, maxima, and minima in any one period.
            \item $\int_{t_0}^{t_0+T} \abs*{f\parens*{t}} \,\diff{t} < \infty$ for any $t_0$.
        \end{itemize}

    \end{CheatsheetEntryFrame}

    \begin{CheatsheetEntryFrame}

        \CheatsheetEntryTitle{Exponential Fourier Series}

        Alternative to the \emph{trigonometric Fourier series}:
        \begin{equation*}
            f\parens*{t}
            = \sum_{n = -\infty}^\infty c_n e^{i n \omega_0 t}
            = c_0 + \sum_{n = 1}^\infty \parens*{
                c_n e^{i n \omega_0 t} + c_{-n} e^{i n \omega_0 t}
            }
        \end{equation*}
        \begin{equation*}
            c_0 = a_0
            \qquad\qquad
            c_n = \frac{1}{T} \int_0^T f\parens*{t} \, e^{-i n \omega_0 t} \,\diff{t}
        \end{equation*}

        Related by:
        \begin{gather*}
            A_n \phase{\phi_n} = a_n - i b_n = 2 c_n
        \end{gather*}

    \end{CheatsheetEntryFrame}

    \begin{CheatsheetEntryFrame}

        \CheatsheetEntryTitle{Fourier Series: Useful Integral Identities}
        \begin{gather*}
            n, m \in \IntegerSet
            \\[0.4em]
            \int_0^T \sin{n \omega_0 t} \,\diff{t} = 0
            \\
            \int_0^T \cos{n \omega_0 t} \,\diff{t} = 0
            \\
            \int_0^T \sin{n \omega_0 t} \, \cos{m \omega_0 t} \,\diff{t} = 0
            \\
            \int_0^T \sin{n \omega_0 t} \, \sin{m \omega_0 t} \,\diff{t} = 0
            , \qquad \parens*{n \ne m}
            \\
            \int_0^T \cos{n \omega_0 t} \, \cos{m \omega_0 t} \,\diff{t} = 0
            , \qquad \parens*{n \ne m}
            \\
            \int_0^T \sin^2{n \omega_0 t} \,\diff{t} = \frac{T}{2}
            \\
            \int_0^T \cos^2{n \omega_0 t} \,\diff{t} = \frac{T}{2}
            \\[1.8em]
            \int t \cos{a t} \,\diff{t}
            = \frac{1}{a^2} \cos{a t} + \frac{1}{a} t \sin{a t} + C
            \\
            \int t \sin{a t} \,\diff{t}
            = \frac{1}{a^2} \sin{a t} - \frac{1}{a} t \cos{a t} + C
        \end{gather*}

    \end{CheatsheetEntryFrame}
    
\end{multicols}
\newpage
\begin{multicols}{2}

    \begin{CheatsheetEntryFrame}

        \CheatsheetEntryTitle{Fourier Transform}

        For function $f$ \emph{(not necessarily periodic)} which we interpret as a periodic function with \emph{infinite period}:
        \begin{gather*}
            F\parens*{\omega}
            = \FourierTransform\brackets*{f\parens*{t}}
            = \int_{-\infty}^\infty f\parens*{t} e^{-i \omega t} \,\diff{t}
            \\
            f\parens*{t}
            = \FourierTransform^{-1}\brackets*{F\parens*{\omega}}
            = \frac{1}{2 \pi}
            \int_{-\infty}^\infty F\parens*{\omega} e^{i \omega t} \,\diff{\omega}
        \end{gather*}

        The Fourier transform $F$ exists when the Fourier integral converges.

    \end{CheatsheetEntryFrame}

    \begin{CheatsheetEntryFrame}

        \CheatsheetEntryTitle{Heaviside Step Function}
        \begin{equation*}
            H\parens*{x}
            \defeq
                \begin{cases}
                    1 &\quad \text{if } x > 0 \\
                    0 &\quad \text{if } x \le 0
                \end{cases}
        \end{equation*}
        Commonly also written as $u\parens*{x}$ or $\theta\parens*{x}$.

        \CheatsheetEntryTitle{Dirac Delta Function}
        \begin{equation*}
            \delta\parens*{x}
            = \frac{\diff{}}{\diff{x}} H\parens*{x}
        \end{equation*}

    \end{CheatsheetEntryFrame}

    \begin{CheatsheetEntryFrame}

        \CheatsheetEntryTitle{Sign/Signum Function}
        \begin{equation*}
            \SignFunction\parens*{x}
            \defeq
            u\parens*{t} - u\parens*{-t}
            =
            \begin{cases}
                -1 &\quad \text{if } x < 0 \\
                0  &\quad \text{if } x = 0 \\
                1  &\quad \text{if } x > 0
            \end{cases}
        \end{equation*}

    \end{CheatsheetEntryFrame}

    \begin{CheatsheetEntryFrame}

        \CheatsheetEntryTitle{Convolution Integral}
        \begin{equation*}
            h\parens*{t} * x\parens*{t}
            = \int_{-\infty}^\infty{
                h\parens*{\lambda} x\parens*{t - \lambda} \,\diff{\lambda}
            }
        \end{equation*}

        Useful properties:
        \begin{gather*}
            x\parens*{t} * h\parens*{t} = h\parens*{t} * x\parens*{t}
            \\
            f\parens*{t} * \brackets*{x\parens*{t} + y\parens*{t}}
            = f\parens*{t} * x\parens*{t} + f\parens*{t} * y\parens*{t}
            \\
            f\parens*{t} * \brackets*{x\parens*{t} * y\parens*{t}}
            = \brackets*{f\parens*{t} * x\parens*{t}} * y\parens*{t}
        \end{gather*}
        \begin{align*}
            f\parens*{t} * \delta\parens*{t}
            &= \int_{-\infty}^\infty {
                f\parens*{\lambda} \delta\parens*{t - \lambda} \,\diff{\lambda}
            }
            = f\parens*{t}
            \\
            f\parens*{t} * \delta\parens*{t - t_0}
            &= f\parens*{t - t_0}
            \\
            f\parens*{t} * \delta'\parens*{t}
            &= \int_{-\infty}^\infty {
                f\parens*{\lambda} \delta'\parens*{t - \lambda} \,\diff{\lambda}
            }
            = f'\parens*{t}
            \\
            f\parens*{t} * u\parens*{t}
            &= \int_{-\infty}^\infty {
                f\parens*{\lambda} u\parens*{t - \lambda} \,\diff{\lambda}
            }
            = \int_{-\infty}^t f\parens*{\lambda} \,\diff{\lambda}
        \end{align*}

    \end{CheatsheetEntryFrame}

    \newcommand{\TmpFTArrow}{{\color{mycontrastblue}\xRightarrow{\FourierTransform}}}

    \begin{CheatsheetEntryFrame}

        \CheatsheetEntryTitle{Fourier Transform: Properties}
        \begin{equation*}
            \begin{array}{cccl}
                F\parens*{t}
                &\TmpFTArrow&
                2 \pi f \parens*{-\omega}
                & \text{\tiny\myul{Duality}}
                \\[0.6em]
                a \, f(t) + b \, g(t)
                &\TmpFTArrow&
                a \, F\parens*{\omega} + b \, G\parens*{\omega}
                & \text{\tiny Linearity}
                \\[0.6em]
                {\displaystyle f\parens*{a t}}
                &\TmpFTArrow&
                {\displaystyle \frac{1}{\abs*{a}} F\parens*{\frac{\omega}{a}}}
                & \text{\tiny\parbox{6em}{Time\\Scaling}}
                \\[0.6em]
                f\parens*{t - t_0}
                &\TmpFTArrow&
                e^{-i \omega t_0} F\parens*{\omega}
                & \text{\tiny\parbox{6em}{Time\\Shifting}}
                \\[0.6em]
                f\parens*{t} e^{i \omega_0 t}
                &\TmpFTArrow&
                F\parens*{\omega - \omega_0}
                & \text{\tiny\parbox{6em}{Amplitude\\Modulation\\(Frequency\\Shifting)}}
                \\[0.6em]
                f^{(n)}\parens*{t}
                &\TmpFTArrow&
                \parens*{i \omega}^n F\parens*{\omega}
                & \text{\tiny\parbox{6em}{Time\\Differentiation}}
                \\[0.6em]
                {\displaystyle \int_{-\infty}^t f\parens*{\tau} \,\diff{\tau}}
                &\TmpFTArrow&
                {\displaystyle
                    \frac{F\parens*{\omega}}{i \omega}
                    + \pi F\parens*{0} \, \delta\parens*{\omega}
                }
                & \text{\tiny\parbox{6em}{Time\\Integration}}
                \\[0.6em]
                f\parens*{-t}
                &\TmpFTArrow&
                F\parens*{-\omega}
                = F^*\parens*{-\omega}
                & \text{\tiny Reversal}
                \\[0.6em]
                f\parens*{t} * g\parens*{t}
                &\TmpFTArrow&
                f\parens*{\omega} \, G\parens*{\omega}
                & \text{\tiny\parbox{6em}{Convolution\\In $t$}}
                \\[0.6em]
                {\displaystyle f\parens*{t} \, g\parens*{t}}
                &\TmpFTArrow&
                {\displaystyle
                    \frac{1}{2 \pi} F\parens*{\omega} * G\parens*{\omega}
                }
                & \text{\tiny\parbox{6em}{Convolution\\In $\omega$}}
            \end{array}
        \end{equation*}

    \end{CheatsheetEntryFrame}

    \begin{CheatsheetEntryFrame}

        \CheatsheetEntryTitle{Fourier Transform: Common Pairs}
        \begin{equation*}
            \begin{array}{ccc}
                \delta\parens*{t}
                &\TmpFTArrow&
                1
                \qquad\TmpFTArrow\qquad
                2 \pi \, \delta\parens*{\omega}
                \\[1em]
                u\parens*{t}
                &\TmpFTArrow&
                \displaystyle
                \pi \, \delta\parens*{\omega} + \frac{1}{j \omega}
                \\[1em]
                u\parens*{t + \tau} - u\parens*{t - \tau}
                &\TmpFTArrow&
                \displaystyle
                2 \frac{\sin{\omega \tau}}{\omega}
                \\[1em]
                \abs*{t}
                &\TmpFTArrow&
                \displaystyle
                \frac{- 2}{\omega^2}
                \\[1em]
                \SignFunction\parens*{t}
                &\TmpFTArrow&
                \displaystyle
                \frac{2}{i \omega}
                \\[1em]
                e^{-a t} \, u\parens*{t}
                &\TmpFTArrow&
                \displaystyle
                \frac{1}{a + i \omega}
                \\[1em]
                e^{a t} \, u\parens*{-t}
                &\TmpFTArrow&
                \displaystyle
                \frac{1}{a - i \omega}
                \\[1em]
                t^n e^{-a t} \, u\parens*{t}
                &\TmpFTArrow&
                \displaystyle
                \frac{n!}{\parens*{a + i \omega}^{n + 1}}
                \\[1em]
                e^{-a \abs*{t}}
                &\TmpFTArrow&
                \displaystyle
                \frac{2 a}{a^2 + \omega^2}
                \\[1em]
                e^{i \omega_0 t}
                &\TmpFTArrow&
                2 \pi \, \delta\parens*{\omega - \omega_0}
                \\[1em]
                \sin{\omega_0 t}
                &\TmpFTArrow&
                i \pi \brackets*{
                    \delta\parens*{\omega + \omega_0}
                    - \delta\parens*{\omega - \omega_0}
                }
                \\[1em]
                \cos{\omega_0 t}
                &\TmpFTArrow&
                \pi \brackets*{
                    \delta\parens*{\omega + \omega_0}
                    + \delta\parens*{\omega - \omega_0}
                }
                \\[1em]
                e^{-a t} \sin\parens*{\omega_0 t} \, u\parens*{t}
                &\TmpFTArrow&
                \displaystyle
                \frac{\omega_0}{\parens*{a + i \omega}^2 + \omega_0^2}
                \\[1em]
                e^{-a t} \cos\parens*{\omega_0 t} \, u\parens*{t}
                &\TmpFTArrow&
                \displaystyle
                \frac{a + i \omega}{\parens*{a + i \omega}^2 + \omega_0^2}
            \end{array}
        \end{equation*}

    \end{CheatsheetEntryFrame}
    
\end{multicols}
\newpage
\begin{multicols}{2}

    \begin{CheatsheetEntryFrame}

        \CheatsheetEntryTitle{Discrete Fourier Transform (DFT)}

        Instead of a continuous function over time, suppose we instead discretely sample it, denoted $x_n$. The DFT of $x_n$ is:
        \begin{align*}
            X_n
            &= \sum_{k=0}^{N-1} x_k e^{-i \frac{2 \pi}{N} n k}
            \\
            &= \sum_{k=0}^{N-1} x_k \brackets*{
                \cos\parens*{\frac{2 \pi}{N} k n}
                - i \sin\parens*{\frac{2 \pi}{N} k n}
            }
        \end{align*}

        {\footnotesize
            Side-note: Fast Fourier Transform (FFT) is a family of algorithms used to calculate the DFT.
        }

        \Todo{What are some better definitions? Can values be complex?}

    \end{CheatsheetEntryFrame}
    
\end{multicols}
\newpage
\begin{multicols}{2}
    
    \begin{CheatsheetEntryFrameStart}

        \CheatsheetEntryTitle{Laplace Transform}

        For function $f$ and some complex variable $s = \sigma + i \omega$:
        \begin{gather*}
            \LaplaceTransform\brackets*{f\parens*{t}}
            = F\parens*{s}
            = \int_{0^-}^\infty f\parens*{t} e^{-s t} \,\diff{t}
        \end{gather*}

        \myul{Useful properties:}
        \begin{center}
            Linearity
            \begin{gather*}
                \LaplaceTransform\brackets*{a_1 f_1(t) + a_2 f_2(t)}
                = a_1 F_1\parens*{s} + a_2 F_2\parens*{s}
            \end{gather*}

            Scaling
            \begin{gather*}
                \LaplaceTransform\brackets*{f\parens*{a t}}
                = \frac{1}{a} F\parens*{\frac{s}{a}}
            \end{gather*}

            Time Shifting
            \begin{gather*}
                \LaplaceTransform\brackets*{f\parens*{t - a} u\parens*{t - a}}
                = e^{-a s} F\parens*{s}
            \end{gather*}

            Frequency Shifting/Translation
            \begin{gather*}
                \LaplaceTransform\brackets*{e^{-a t} f\parens*{t} u\parens*{t}}
                = F\parens*{s + a}
            \end{gather*}

            Time Differentiation
            \begin{gather*}
                \LaplaceTransform\brackets*{f'\parens*{t}}
                = s F\parens*{s} - f\parens*{0^-}
                \\
                \LaplaceTransform\brackets*{f''\parens*{t}}
                = s^2 F\parens*{s} - s f\parens*{0^-} - f'\parens*{0^-}
            \end{gather*}
            \begin{align*}
                \LaplaceTransform\brackets*{f^{\parens*{n}}\parens*{t}}
                &= s^n F\parens*{s}
                - s^{n-1} f\parens*{0^-}
                - s^{n-2} f'\parens*{0^-}
                \\
                &\phantom{=} - \cdots
                - s^0 f^{\parens*{n-1}}\parens*{0^-}
            \end{align*}

            Time Integration
            \begin{gather*}
                \LaplaceTransform\brackets*{\int_0^t f\parens*{x} \,\diff{x}}
                = \frac{1}{s} F\parens*{s}
            \end{gather*}

            Frequency Differentiation
            \begin{gather*}
                \LaplaceTransform\brackets*{t^n f\parens*{t}}
                = \parens*{-1}^n F^{\parens*{n}}\parens*{s}
            \end{gather*}
        \end{center}

    \end{CheatsheetEntryFrameStart}

    \begin{CheatsheetEntryFrameEnd}

        \begin{center}
            Time Periodicity (\Todo{write better notes?})
            \begin{gather*}
                F\parens*{s}
                = \frac{F_1\parens*{s}}{1 - e^{-T s}}
            \end{gather*}

            Initial/Final Values
            \begin{gather*}
                f\parens*{0}
                = \lim_{s \to \infty} s F\parens*{s}
                \qquad\quad
                f\parens*{\infty}
                = \lim_{s \to 0} s F\parens*{s}
            \end{gather*}

            Convolution (\Todo{better notes and properties of the convolution?})
            \begin{gather*}
                F_1\parens*{s} F_2\parens*{s}
                = \LaplaceTransform\brackets*{f_1\parens*{t} * f_2\parens*{t}}
            \end{gather*}
        \end{center}

    \end{CheatsheetEntryFrameEnd}
    
\end{multicols}

