\subsection{Amplifier Analysis}%
\label{sub:amplifier-analysis}

\subsubsection{Operational Amplifiers}

\begin{multicols}{2}

    \begin{CheatsheetEntryFrame}

        \CheatsheetEntryTitle{Ideal Opamp}
        \vspace*{-8mm}
        \begin{center}
        \begin{circuitikz}
            \tikzset{/tikz/circuitikz/bipoles/length=1.2cm}
            \draw[thick] % Opamp Shape
                (0,0) coordinate (OA_BL)
                    ++(0,1.1) coordinate (OA_INV)
                (4.9,4.9) coordinate (OA_TR)
                (OA_BL -| OA_TR) coordinate (OA_BR)
                (OA_BL |- OA_TR) coordinate (OA_TL)
                    ++(0,-1.1) coordinate (OA_NONINV)

                (OA_TL)
                    -- (OA_BL)
                    -- ($(OA_BR)!0.5!(OA_TR)$)
                        coordinate (OA_OUT)
                    -- (OA_TL)
            ;
            \draw
                (OA_NONINV) ++(0.5,0) coordinate (TMP_A)
                (OA_NONINV)
                    ++(-1.2,0)
                        node (TMP_B) [ocirc] {}
                    -- (TMP_A)
                    to[R, l=$R_i$] (OA_INV -| TMP_A)
                    -- (TMP_B |- OA_INV)
                        node [ocirc] {}
                    to[open, v^<=$v_i$] (TMP_B)
                    
                (TMP_A)                node [above] {\Large $+$}
                (TMP_A |- OA_INV)      node [below] {\Large $-$}
                (OA_NONINV) ++(-0.6,0) node [currarrow, label=above:$i_+$] {}
                (OA_INV)    ++(-0.6,0) node [currarrow, label=below:$i_-$] {}

                (OA_OUT)
                    ++(0.6,0)
                        node [ocirc, label=right:$v_o$] {}
                    -- ++(-1.5,0)
                    to[R, l_=$R_o$] ++(-1.1,0)
                    -- ++(-0.1,0)
                    to[
                        american controlled voltage source,
                        l_=$A v_i$,
                        /tikz/circuitikz/bipoles/length=1.4cm
                    ] ++(-1.1,0)
                    -| ++(-0.2,-0.2)
                    \MyGround{}
                    
            ;
            \draw[stealth-, ultra thick, extranotecolor, color=extranotecolor]
                (OA_INV)
                    ++(-0.6,-0.9)
                    -- ++(-0.6,-0.5)
                    node[below, align=center, font={\footnotesize \sc}]{Inverting\\Input}
            ;
            \draw[stealth-, ultra thick, extranotecolor, color=extranotecolor]
                (OA_NONINV)
                    ++(-0.6,0.9)
                    -- ++(-0.6,0.5)
                    node[above, align=center, font={\footnotesize \sc}]{Non-Inverting\\Input}
            ;
            \draw
                (OA_OUT)
                    ++(-2.5,3.9)
                    node[below left]{$\displaystyle \boxed{v_o = A v_i}$}
                (OA_OUT)
                    ++(-2.0,4.5)
                    node [below right] {$\displaystyle \boxed{
                        \begin{aligned}
                            R_i &\to \infty \\
                            R_o &\to 0 \\
                            A &\to \infty \\
                            \text{CMRR} &\to \infty \\
                            \text{Bandwidth} &\to \infty
                        \end{aligned}
                    }$}
                %(OA_OUT)
                %    ++(-2.7,-2.8)
                %    node [above] {\small \myul{\textbf{Golden Rule \#1}}}
                %    node [below] {$\displaystyle \boxed{i_+ = i_- = 0}$}
                (OA_OUT)
                    ++(-0.7,-1.7)
                    node [above] {\small \myul{\textbf{Golden Rules}}}
                    node [below] {$\displaystyle
                        %\boxed{
                        \begin{gathered}
                            \boxed{
                            \begin{gathered}
                                v_+ = v_- \\
                                \textit{\footnotesize ``virtual ground"} \\[-1.2mm]
                                \textit{\footnotesize (only in \myul{negative feedback})}
                            \end{gathered}
                            } \\
                            \boxed{i_+ = i_- = 0}
                        \end{gathered}
                        %}
                    $}
            ;
        \end{circuitikz}
        \end{center}
        \vspace*{-1.5mm}
    \end{CheatsheetEntryFrame}

    \begin{CheatsheetEntryFrame}

        \CheatsheetEntryTitle{Non-Ideal Opamp Model} \CheatsheetEntrySubtitle{(simple version)}
        \begin{minipage}[c]{0.38\columnwidth}
            \MinipageInheritDocumentFormatting
            \begin{gather*}
                \begin{aligned}
                    %\overbrace{
                    I_\text{B} &= \frac{I_{\text{B}+} + I_{\text{B}-}}{2}
                    %}^\text{input bias current}
                    \\[1mm]
                    %\underbrace{
                    I_\text{io} &= I_{\text{B}+} - I_{\text{B}-}
                    %}_\text{input offset current}
                \end{aligned}
                \\[4mm]
                \begin{aligned}
                    I_{\text{B}+} &= I_\text{B} + \frac{1}{2} I_\text{io}
                    \\[1mm]
                    I_{\text{B}-} &= I_\text{B} - \frac{1}{2} I_\text{io}
                \end{aligned}
            \end{gather*}
        \end{minipage}%
        \begin{minipage}[c]{0.62\columnwidth}%
            \MinipageInheritDocumentFormatting
            \smallskip
            \begin{center}
            \begin{circuitikz}
                \draw 
                    (0,0)
                        node[
                            op amp,
                            anchor=-%,
                            %/tikz/circuitikz/bipoles/length=1.3cm
                        ](OA){}
                    (OA.-)
                        -- ++(-0.4,0)
                        -- ++(-1.6,0)
                            node[ocirc]{}
                    (OA.+)
                        -- ++(-0.4,0)
                        to[
                            V,
                            invert,
                            color=red,
                            l=$\color{red} V_\text{io}$,
                            /tikz/circuitikz/bipoles/length=1.2cm
                        ] ++(-1.6,0)
                            node[ocirc]{}
                    (OA.out)
                        -- ++(0.2,0)
                            node[ocirc]{}
                ;
                \draw[red, color=red]
                    (OA.-)
                        ++(-0.25,0)
                        node[currarrow, label=above:$I_{\text{B}-}$]{}
                    (OA.+)
                        ++(-0.25,0)
                        node[currarrow, label=above:$I_{\text{B}+}$]{}
                ;
            \end{circuitikz}
            \end{center}
            \vspace*{-4mm}%
            \Exn{\small%
                \begin{align*}
                    I_\text{B} &= \text{input bias current} \\[-0.5mm]
                    I_\text{io} &= \text{input offset current} \\[-0.5mm]
                    V_\text{io} &= \text{input offset voltage}
                \end{align*}%
            }%
            \vspace*{-8mm}%
        \end{minipage}
        \vspace*{1mm}

        \begin{center}
            {\footnotesize Worst-case output offset voltage of a circuit:}
            \vspace*{-2mm} % TODO: How do we make it so we don't have to do this?
            \begin{gather*}
                V_\text{os}
                = \abs*{
                    V_\text{os} \parens{I_{\text{B}+}}
                    + V_\text{os} \parens{I_{\text{B}-}}
                }
                + \abs*{
                    V_\text{os} \parens{V_\text{io}}
                }
            \end{gather*}
        \end{center}
    \end{CheatsheetEntryFrame}

    \begin{CheatsheetEntryFrame}

        \CheatsheetEntryTitle{Opamp Large Signal Characteristics}

        \begin{enumerate}
            \item Output Saturation
            \item Output Current Limit
            \item Slew Rate Limit
        \end{enumerate}
        %\begin{enumerate}
        %    \setcounter{enumi}{1}
        %    \item Output Current Limit
        %\end{enumerate}
        %\begin{enumerate}
        %    \setcounter{enumi}{2}
        %    \item Slew Rate Limit
        %\end{enumerate}

        \Todo{Expand on this?}
    \end{CheatsheetEntryFrame}

    \Todo{Add notes on frequency response?}

    \MulticolsBreak

    \begin{CheatsheetEntryFrame}

        \CheatsheetEntryTitle{Common Opamp Configurations}
        \smallskip

        \myul{Unity-Gain Buffer}

        \begin{center}
        \begin{circuitikz}
            \tikzset{/tikz/circuitikz/bipoles/length=1.3cm}
            \draw 
                % Op-Amp
                (0,0)
                    node[op amp, noinv input down, anchor=out](OA){}
                (OA.+)
                    -- ++(-0.3,0)
                        node[ocirc, label=left:$v_i$]{}
                        coordinate(GND)
                (OA.-)
                    -- ++(-0.3,0)
                        coordinate(TMP_A)
                (OA.out)
                    -- ++(0.2,0)
                        coordinate(TMP_B)
                    -- ++(0.6,0)
                        node[ocirc, label=right:$v_o$]{}

                % Feedback
                (TMP_B)
                    -- ++(0,1.5)
                    -| (TMP_A)

                % Equation
                (OA.out)
                    ++(-4.2,0.7)
                        node[]{$\displaystyle \boxed{v_o = v_i}$}
            ;
        \end{circuitikz}
        \end{center}

        \bigskip
        \SoftHLine
        \smallskip

        \myul{Inverting Amplifier}
        \vspace*{-3mm}
        \begin{center}
        %\scalebox{0.8}{
        \begin{circuitikz}
            \tikzset{/tikz/circuitikz/bipoles/length=1.3cm}
            \draw 
                % Op-Amp
                (0,0)
                    node[op amp, noinv input down, anchor=out](OA){}
                (OA.+)
                    -- ++(-0.3,0)
                    \MyGround{}
                (OA.-)
                    -- ++(-0.3,0)
                        coordinate(TMP_A)
                    to[R, l_=$R_1$, red, color=red] ++(-2.2,0)
                        node[ocirc, label=left:$v_i$]{}
                (OA.out)
                    -- ++(0.2,0)
                        coordinate(TMP_B)
                    -- ++(0.6,0)
                        node[ocirc, label=right:$v_o$]{}

                % Feedback
                (TMP_B)
                    -- ++(0,1.6)
                        coordinate (TMP_C)
                    to[R, l=$R_2$, blue, color=blue] (TMP_C -| TMP_A)
                    -- (TMP_A)

                % Equation
                (OA.+)
                    ++(-2.6,0.2)
                        node[below]{$
                            \displaystyle
                            \boxed{\frac{v_o}{v_i} = \frac{- \color{blue} R_2}{ \color{red} R_1}}
                        $}
            ;
        \end{circuitikz}
        %}
        \end{center}

        \bigskip
        \SoftHLine
        \smallskip

        \myul{Non-Inverting Amplifier}
        \vspace*{-3mm}
        \begin{center}
        \begin{circuitikz}
            \tikzset{/tikz/circuitikz/bipoles/length=1.3cm}
            \draw 
                % Op-Amp
                (0,0)
                    node[op amp, noinv input down, anchor=out](OA){}
                (OA.+)
                    -- ++(-0.3,0)
                        node[ocirc, label=left:$v_i$]{}
                (OA.-)
                    -- ++(-0.3,0)
                        coordinate(TMP_A)
                    to[R, l_=$R_1$, red, color=red] ++(-2.2,0)
                    \MyGround{}
                (OA.out)
                    -- ++(0.2,0)
                        coordinate(TMP_B)
                    -- ++(0.6,0)
                        node[ocirc, label=right:$v_o$]{}

                % Feedback
                (TMP_B)
                    -- ++(0,1.6)
                        coordinate (TMP_C)
                    to[R, blue, color=blue, l=$R_2$] (TMP_C -| TMP_A)
                    -- (TMP_A)

                % Equation
                (OA.+)
                    ++(-2.6,-0.2)
                        node[below]{$
                            \displaystyle
                            \boxed{\frac{v_o}{v_i} = 1 + \frac{\color{blue} R_2}{\color{red} R_1}}
                        $}
            ;
        \end{circuitikz}
        \end{center}

        \bigskip
        \SoftHLine
        \smallskip

        \myul{Difference Amplifier}
        \vspace*{-4mm}
        \begin{center}
        \begin{circuitikz}
            \tikzset{/tikz/circuitikz/bipoles/length=1.3cm}
            \draw 
                % Op-Amp
                (0,0)
                    node[op amp, noinv input down, anchor=out](OA){}
                (OA.+)
                    -- ++(-0.3,0)
                        coordinate (TMP_D)
                    to[R, l_=$R_1$, red, color=red] ++(-2.2,0)
                        node[ocirc, label=left:$v_1$]{}
                (OA.-)
                    -- ++(-0.3,0)
                        coordinate (TMP_A)
                    to[R, l_=$R_1$, red, color=red] ++(-2.2,0)
                        node[ocirc, label=left:$v_2$]{}
                (OA.out)
                    -- ++(0.2,0)
                        coordinate(TMP_B)
                    -- ++(0.6,0)
                        node[ocirc, label=right:$v_o$]{}

                (TMP_D)
                    to[R, blue, color=blue, l_=$R_2$] ++(0,-1.6)
                    -- ++(0,-0.2)
                    \MyTLGround{}

                % Feedback
                (TMP_B)
                    -- ++(0,1.6)
                        coordinate (TMP_C)
                    to[R, blue, color=blue, l=$R_2$] (TMP_C -| TMP_A)
                    -- (TMP_A)

                % Equation
                (OA.out)
                    ++(-5.1,-1.8)
                        node[]{$
                            \boxed{
                                v_o
                                = \frac{\color{blue} R_2}{\color{red} R_1}
                                \parens{v_1 - v_2}
                            }
                        $}

                % Note
                (OA.out)
                    ++(-2.0,-1.1)
                        node[below right, align=left, font={\footnotesize\em}]{
                            (In practice, this circuit \\
                            has a fairly low input \\
                            impedance, making it \\
                            unsuitable for some \\
                            practical applications.)
                        }
            ;
        \end{circuitikz}
        \end{center}

        \bigskip
        \SoftHLine
        \smallskip

        \Todo{Write instrumentational and summing amplifiers?}

    \end{CheatsheetEntryFrame}

    \MulticolsCleanEnd
\end{multicols}

