\subsection{More Common Circuit Configurations}%
\label{sub:more-common-circuit-configurations}

\subsubsection{Transistor Amplifier Analysis}

\begin{multicols}{2}
    \begin{CheatsheetEntryFrame}

        \CheatsheetEntryTitle{Basic BJT Configurations}

        Three basic configurations:
        \begin{itemize}
            \item \textbf{CE (Common Emitter)}
            \begin{itemize}
                \item Structure:
                \begin{itemize}
                    \item Base = Input
                    \item Collector = Output
                    \item \myul{Emitter = Common}
                \end{itemize}
                \item Typical Characteristics:
                \begin{itemize}
                    \item Low input impedance (but can be improved with an emitter resistor).
                    \item Inverting amplifier.
                \end{itemize}
            \end{itemize}
            \item \textbf{CB (Common Base)}
            \begin{itemize}
                \item Structure:
                \begin{itemize}
                    \item Emitter = Input
                    \item Collector = Output
                    \item \myul{Base = Common}
                \end{itemize}
                \item Typical Characteristics:
                \begin{itemize}
                    \item Low input impedance.
                    \item Not suitable for voltage amplification, but very good at converting current to voltage.
                    \item Non-inverting amplifier.
                \end{itemize}
            \end{itemize}
            \item \textbf{CC (Common Collector)}
            \begin{itemize}
                \item Structure:
                \begin{itemize}
                    \item Base = Input
                    \item Emitter = Output
                    \item \myul{Collector = Common}
                \end{itemize}
                \item Typical Characteristics:
                \begin{itemize}
                    \item High input impedance.
                    \item Low output impedance.
                    \item Suitable for voltage transfer.
                    \item Low voltage gain.
                    %\item Non-inverting amplifier.
                \end{itemize}
            \end{itemize}
        \end{itemize}

        \Todo{Turn this into better notes?}

    \end{CheatsheetEntryFrame}

    \begin{CheatsheetEntryFrame}

        \CheatsheetEntryTitle{Basic MOSFET Configurations}

        Three basic configurations:
        \begin{itemize}
            \item \textbf{CS (Common Source)}
            \begin{itemize}
                \item Structure:
                \begin{itemize}
                    \item Gate = Input
                    \item Drain = Output
                    \item \myul{Source = Common}
                \end{itemize}
                \item Typical Characteristics:
                \begin{itemize}
                    \item Infinite input impedance.
                    \item Very low output impedance.
                    \item Has Miller effect, limiting $f_H$.
                \end{itemize}
            \end{itemize}
            \item \textbf{CG (Common Gate)}
            \begin{itemize}
                \item Structure:
                \begin{itemize}
                    \item Source = Input
                    \item Drain = Output
                    \item \myul{Gate = Common}
                \end{itemize}
                \item Typical Characteristics:
                \begin{itemize}
                    \item Low input impedance.
                    \item No Miller effect.
                \end{itemize}
            \end{itemize}
            \item \textbf{CD (Common Drain)}
            \begin{itemize}
                \item Structure:
                \begin{itemize}
                    \item Gate = Input
                    \item Source = Output
                    \item \myul{Drain = Common}
                \end{itemize}
                \item Typical Characteristics:
                \begin{itemize}
                    \item Low voltage gain.
                    \item Infinite input impedance.
                    \item Low output impedance.
                    \item No Miller effect (because gain is very low).
                \end{itemize}
            \end{itemize}
        \end{itemize}

        \Todo{Turn this into better notes?}

    \end{CheatsheetEntryFrame}

\end{multicols}

\newpage

\newcommand{\TmpCircuitTransformArrow}[2]{%
    \begin{tikzpicture}[scale=1, transform shape]
        %\node[simshadows/GenericGrayBlockArrow] {};
        \draw
            (0,0)
                node [simshadows/GenericGrayBlockArrow] {};
        ;
        % Ghetto Spacing. TODO: Improve this?
        \path (0,0) -- ++(0,#1);
        \path (0,0) -- ++(#2,0);
        \path (0,0) -- ++(-#2,0);
    \end{tikzpicture}%
}

\begin{CheatsheetEntryFrame}
    \CheatsheetEntryTitle{Useful Transforms}

    \begin{center}
    \begin{circuitikz}
        \draw 
            (0,0)
                    node [ocirc, label=above:B] {}
                -- ++(1,0)
                    node [currarrow, label=above:$i_B$] {}
                -- ++(1,0)
                to[R, l=$r_\pi$, v=$v_{BE}$] ++(0,-2)
                -- ++(2,0)
                    node (TMP_A) [circ, label=right:E] {}
                to[R, l=$R_E$] ++(0,-2)
                \MyGround{}
            (TMP_A)
                to[american controlled current source, invert, l_={$
                        \begin{gathered}
                            g_m v_{BE} \\
                            = \beta i_B
                        \end{gathered}
                    $}] ++(0,2)
                -- ++(2,0)
                    node [ocirc, label=above:C] {}
        ;
    \end{circuitikz}%
    \TmpCircuitTransformArrow{-2.6}{2}%
    \begin{circuitikz}
        \draw 
            (0,0)
                    node [ocirc, label=above:B] {}
                -- ++(1,0)
                    node [currarrow, label=above:$i_B$] {}
                -- ++(1,0)
                to[R, l=$r_\pi$] ++(0,-2)
                %to[R, l=$r_\pi$, v=$v_{BE}$] ++(0,-2) % TODO: Is this version better?
                to[R, l={$
                    \begin{gathered}
                        R_E \parens{1 + g_m r_\pi} \\
                        = R_E \parens{1 + \beta}
                    \end{gathered}
                $}] ++(0,-2)
                \MyGround{}
        ;
    \end{circuitikz}
    \end{center}
    \bigskip

    \begin{center}
    \begin{circuitikz}
        \draw 
            (0,0)
                -- ++(0.7,0)
                    node [currarrow, label=above:$i_B$] {}
                -- ++(0.7,0)
                    node [circ, label=above:B] {}
                to[R, l=$R_B$] ++(0,-2)
                -- ++(2,0)
                    node (TMP_A) [circ, label=right:E] {}
                -- ++(0,-0.5)
                    coordinate (TMP_B)
                to[R, l=$R_E$] ++(0,-2)
                \MyGround{}
            (TMP_A)
                to[american controlled current source, invert, l_={$\beta i_B$}] ++(0,2)
                -- ++(2,0)
                    node [ocirc, label=above:C] {}
            (TMP_B)
                -- ++(2,0)
                    node [ocirc, label=above:X] {}
            (0,0)
                \MyGround{}
        ;
    \end{circuitikz}%
    \TmpCircuitTransformArrow{-2.6}{2}%
    \begin{circuitikz}
        \draw 
            (0,0)
                    node [ocirc, label=above:X] {}
                -- ++(-2,0)
                    coordinate (TMP_A)
                -- ++(-2,0)
                to[R, l=$R_E$] ++(0,-2)
                \MyGround{}
            (TMP_A)
                to[R, l=$\displaystyle \frac{R_B}{1 + \beta}$] ++(0,-2)
                \MyGround{}
        ;
        % Ghetto alignment. TODO: Improve this!
        \path
            (0,0) -- ++(0,-4)
        ;
    \end{circuitikz}
    \end{center}
\end{CheatsheetEntryFrame}

