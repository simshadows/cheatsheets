\subsection{Amplifier Analysis}%
\label{sub:amplifier-analysis}

\subsubsection{Operational Amplifiers}

\begin{CheatsheetEntryFrame}

    \CheatsheetEntryTitle{Ideal Opamp}

    \begin{multicols}{2}

        \begin{center}
            \phantom{\myul{\textbf{\footnotesize Schematic Symbol}}}

            \begin{circuitikz}
                \draw 
                    (0,0)
                        node[op amp, anchor=-](OA){}
                    (OA.+)
                        -- ++(-0.5,0)
                            node[ocirc, label=left:$v_+$]{}
                    (OA.-)
                        -- ++(-0.5,0)
                            node[ocirc, label=left:$v_-$]{}
                    (OA.out)
                        -- ++(0.2,0)
                            node[ocirc, label=right:$v_o$]{}
                    %% Biasing is probably not needed here.
                    %(OA.up)
                    %    -- ++(0,0.8)
                    %        node[ocirc, label=above:$V_\text{CC}$]{}
                    %(OA.down)
                    %    -- ++(0,-0.8)
                    %        node[ocirc, label=below:$V_\text{EE}$]{}
                    (OA.out)
                        ++(1.2,1.4)
                        node[]{$\displaystyle \boxed{v_o = A \parens{v_+ - v_-}}$}
                ;
                \draw[stealth-, ultra thick, extranotecolor, color=extranotecolor]
                    (OA.-)
                        ++(0,0.4)
                        -- ++(-0.7,0.5)
                        %node[label={above:{\footnotesize \textsc{Inverting Input}}}]{}
                        node[above, align=center, font={\footnotesize \sc}]{Inverting\\Input}
                ;
                \draw[stealth-, ultra thick, extranotecolor, color=extranotecolor]
                    (OA.+)
                        ++(0,-0.4)
                        -- ++(-0.7,-0.5)
                        %node[label={below:[align=left]{\footnotesize \textsc{Non-Inverting\\Input}}}]{}
                        node[below, align=center, font={\footnotesize \sc}]{Non-Inverting\\Input}
                ;
            \end{circuitikz}
        \end{center}

        \MulticolsBreak

        \begin{center}
            \myul{\textbf{\footnotesize Equivalent Model}}

            \begin{circuitikz}
                \draw 
                    (0,0)
                            node[ocirc, label=left:$v_+$]{}
                        to[short, i=$i_+$] ++(2,0)
                            coordinate(TR1)
                        to[R, l=$R_i$, v=$v_i$] ++(0,-2)
                        to[short, i<=$i_-$] ++(-2,0)
                            node[ocirc, label=left:$v_-$]{}
                    (TR1) ++(2,0)
                        to[cvsource, l=$A v_i$] ++(0,-2)
                            node[ground, scale=2]{}
                    (TR1) ++(2,0)
                        to[R, l=$R_o$] ++(2,0)
                        node[ocirc, label=right:$v_o$]{}
                ;
            \end{circuitikz}
        \end{center}

        \MulticolsCleanEnd
    \end{multicols}%
    \MulticolsReduceVspaceAfter%
    \begin{multicols}{2}
        DC-coupled, high-gain differential-input amplifier.

        \myul{\textbf{Golden rules}}:
        \begin{enumerate}
            \item $v_+ = v_-$ \\ %{\footnotesize (\emph{virtual ground})}
                \emph{\footnotesize (``virtual ground"; appies in \myul{negative feedback} only)}
            \item $i_+ = i_- = 0$
        \end{enumerate}

        %\emph{(Golden rule \#1 is also known as ``virtual ground".)}

        \MulticolsBreak

        \myul{\textbf{Some properties}}:
        \begin{enumerate}
            \item $R_i \to \infty$
            \item $R_o = 0$
            \item $A \to \infty$
            \item infinite CMRR \Exn{(common-mode rejection ratio)}
            \item infinite bandwidth
        \end{enumerate}

        \MulticolsCleanEnd
    \end{multicols}
    \MulticolsReduceVspaceAfter%
\end{CheatsheetEntryFrame}

\begin{multicols}{2}
    \begin{CheatsheetEntryFrame}
        \CheatsheetEntryTitle{Opamp DC Imperfection Modeling}
        \bigskip

        \begin{center}
            \begin{circuitikz}
                \draw 
                    (0,0)
                        node[op amp, anchor=-](OA){}
                    (OA.-)
                        -- ++(-0.5,0)
                        -- ++(-2,0)
                            node[ocirc]{}
                    (OA.+)
                        -- ++(-0.5,0)
                        to[V, invert, color=red, l=$\color{red} V_\text{io}$] ++(-2,0)
                            node[ocirc]{}
                    (OA.out)
                        -- ++(0.2,0)
                            node[ocirc]{}
                ;
                \draw[red, color=red]
                    (OA.-)
                        ++(-0.25,0)
                        node[currarrow, label=above:$I_{\text{B}-}$]{}
                    (OA.+)
                        ++(-0.25,0)
                        node[currarrow, label=above:$I_{\text{B}+}$]{}
                ;
            \end{circuitikz}
        \end{center}

        Currents usually specified in terms of $I_\text{B}$ and $I_\text{io}$:
        \begin{gather*}
            \begin{gathered}
                %\overbrace{
                I_\text{B} = \frac{I_{\text{B}+} + I_{\text{B}-}}{2}
                %}^\text{input bias current}
                \\
                %\underbrace{
                I_\text{io} = I_{\text{B}+} - I_{\text{B}-}
                %}_\text{input offset current}
            \end{gathered}
            \qquad
            \quad
            \begin{aligned}
                I_{\text{B}+} &= I_\text{B} + \frac{1}{2} I_\text{io} \\
                I_{\text{B}-} &= I_\text{B} - \frac{1}{2} I_\text{io}
            \end{aligned}
        \end{gather*}

        Calculate worst-case output offset voltage of a circuit using superposition principle:
        \begin{gather*}
            V_\text{os}
            = \abs*{
                V_\text{os} \parens{I_{\text{B}+}}
                + V_\text{os} \parens{I_{\text{B}-}}
            }
            + \abs*{
                V_\text{os} \parens{V_\text{io}}
            }
        \end{gather*}

        \Exn{
            Terminology:
            \begin{pindent}
                $I_\text{B} = \text{input bias current}$ \\
                $I_\text{io} = \text{input offset current}$ \\
                $V_\text{io} = \text{input offset voltage}$
            \end{pindent}
        }
    \end{CheatsheetEntryFrame}

    \begin{CheatsheetEntryFrame}
        \CheatsheetEntryTitle{Opamp Large Signal Characteristics}

        \begin{enumerate}
            \item Output Saturation
            \item Output Current Limit
            \item Slew Rate Limit
        \end{enumerate}
        %\begin{enumerate}
        %    \setcounter{enumi}{1}
        %    \item Output Current Limit
        %\end{enumerate}
        %\begin{enumerate}
        %    \setcounter{enumi}{2}
        %    \item Slew Rate Limit
        %\end{enumerate}

        \Todo{Expand on this?}
    \end{CheatsheetEntryFrame}

    \Todo{Add notes on frequency response?}

\end{multicols}

