\section{0-1 Knapsack Problem}%
\label{sec:0-1-knapsack-problem}


\subsubsection{Problem Statement}

You have a bag with a maximum weight carrying limit, and a set of items. Each item has a weight and dollar value. Find a subset of items with the largest dollar value possible that can fit within in the bag's weight limit.

There is only one copy of each item available for you to take. Each item is either taken in its entirety, or not taken at all. You cannot ``partially take" an item.


\subsection{Example}
\label{sub:0-1-knapsack-problem--example}

Suppose your bag can only carry up to $\qty{15}{\kilo\gram}$ in total. The possible items we can choose from are shown in \cref{tab:0-1-knapsack-problem--example-items}.

\begin{table}[H]
    \centering
    \caption{The set of items we must choose from.}
    \label{tab:0-1-knapsack-problem--example-items}
    \begin{tabular}{l||c|c}
        & weight & value \\ \hline\hline
        Item 1 & $\qty{12}{\kilo\gram}$ & $\SI{4}[\$]{}$ \\ \hline
        Item 2 & $\qty{4}{\kilo\gram}$ & $\SI{10}[\$]{}$ \\ \hline
        Item 3 & $\qty{2}{\kilo\gram}$ & $\SI{2}[\$]{}$ \\ \hline
        Item 4 & $\qty{1}{\kilo\gram}$ & $\SI{2}[\$]{}$ \\ \hline
        Item 5 & $\qty{1}{\kilo\gram}$ & $\SI{1}[\$]{}$ \\ \hline
    \end{tabular}
\end{table}

The optimal solution is to take all items except Item 1. The total value of these items is $\SI{10}[\$]{} + \SI{2}[\$]{} + \SI{2}[\$]{} + \SI{1}[\$]{} = \SI{15}[\$]{}$. The total weight of these items is $\qty{4}{\kilo\gram} + \qty{2}{\kilo\gram} + \qty{1}{\kilo\gram} + \qty{1}{\kilo\gram} = \qty{8}{\kilo\gram}$, which is clearly within the bag's $\qty{15}{\kilo\gram}$ limit.

If we took Item 1, we will only have $\qty{3}{\kilo\gram}$ remaining. We wouldn't be able to take Item 2, which is a significantly valuable item in the list. The best we can do is to take Item 3 and Item 5, giving us a total of $\SI{4}[\$]{} + \SI{2}[\$]{} + \SI{1}[\$]{} = \SI{7}[\$]{}$.


\subsection{Naive Brute Force Solution}

We can try enumerating every possible subset of items:
\begin{center}
\scriptsize
\begin{tabular}{r||c|c|c|c|c||r|r}
    {}
        & \multicolumn{5}{c||}{Items}
        &
        & \\
    \#
        & 1
        & 2
        & 3
        & 4
        & 5
        & Weight
        & Value \\ \hline\hline
    $1$
        &
        &
        &
        &
        & \hphantom{Take}
        & $\qty{0}{\kilo\gram}$
        & $\SI{0}[\$]{}$ \\ \hline
    $2$
        & Take
        &
        &
        &
        &
        & $\qty{12}{\kilo\gram}$
        & $\SI{4}[\$]{}$ \\ \hline
    $3$
        &
        & Take
        &
        &
        &
        & $\qty{4}{\kilo\gram}$
        & $\SI{10}[\$]{}$ \\ \hline
    $4$
        & Take
        & Take
        &
        &
        &
        & $\qty{16}{\kilo\gram}$
        & $\SI{14}[\$]{}$ \\ \hline
    $5$
        &
        &
        & Take
        &
        &
        & $\qty{2}{\kilo\gram}$
        & $\SI{2}[\$]{}$ \\ \hline
    $6$
        & Take
        &
        & Take
        &
        &
        & $\qty{14}{\kilo\gram}$
        & $\SI{6}[\$]{}$ \\ \hline
    $7$
        &
        & Take
        & Take
        &
        &
        & $\qty{6}{\kilo\gram}$
        & $\SI{12}[\$]{}$ \\ \hline
    $8$
        & Take
        & Take
        & Take
        &
        &
        & $\qty{18}{\kilo\gram}$
        & $\SI{16}[\$]{}$ \\ \hline
    $9$
        &
        &
        &
        & Take
        &
        & $\qty{1}{\kilo\gram}$
        & $\SI{2}[\$]{}$ \\ \hline
    $10$
        & Take
        &
        &
        & Take
        &
        & $\qty{13}{\kilo\gram}$
        & $\SI{6}[\$]{}$ \\ \hline
    $11$
        &
        & Take
        &
        & Take
        &
        & $\qty{5}{\kilo\gram}$
        & $\SI{12}[\$]{}$ \\ \hline
    $12$
        & Take
        & Take
        &
        & Take
        &
        & $\qty{17}{\kilo\gram}$
        & $\SI{16}[\$]{}$ \\ \hline
    $13$
        &
        &
        & Take
        & Take
        &
        & $\qty{3}{\kilo\gram}$
        & $\SI{4}[\$]{}$ \\ \hline
    $14$
        & Take
        &
        & Take
        & Take
        &
        & $\qty{15}{\kilo\gram}$
        & $\SI{8}[\$]{}$ \\ \hline
    $15$
        &
        & Take
        & Take
        & Take
        &
        & $\qty{7}{\kilo\gram}$
        & $\SI{14}[\$]{}$ \\ \hline
    $16$
        & Take
        & Take
        & Take
        & Take
        &
        & $\qty{19}{\kilo\gram}$
        & $\SI{18}[\$]{}$ \\ \hline
\end{tabular}
\qquad\quad
\begin{tabular}{r||c|c|c|c|c||r|r}
    {}
        & \multicolumn{5}{c||}{Items}
        &
        & \\
    \#
        & 1
        & 2
        & 3
        & 4
        & 5
        & Weight
        & Value \\ \hline\hline
    $17$
        &
        &
        &
        &
        & Take
        & $\qty{1}{\kilo\gram}$
        & $\SI{1}[\$]{}$ \\ \hline
    $18$
        & Take
        &
        &
        &
        & Take
        & $\qty{13}{\kilo\gram}$
        & $\SI{5}[\$]{}$ \\ \hline
    $19$
        &
        & Take
        &
        &
        & Take
        & $\qty{5}{\kilo\gram}$
        & $\SI{11}[\$]{}$ \\ \hline
    $20$
        & Take
        & Take
        &
        &
        & Take
        & $\qty{17}{\kilo\gram}$
        & $\SI{15}[\$]{}$ \\ \hline
    $21$
        &
        &
        & Take
        &
        & Take
        & $\qty{3}{\kilo\gram}$
        & $\SI{3}[\$]{}$ \\ \hline
    $22$
        & Take
        &
        & Take
        &
        & Take
        & $\qty{15}{\kilo\gram}$
        & $\SI{7}[\$]{}$ \\ \hline
    $23$
        &
        & Take
        & Take
        &
        & Take
        & $\qty{7}{\kilo\gram}$
        & $\SI{13}[\$]{}$ \\ \hline
    $24$
        & Take
        & Take
        & Take
        &
        & Take
        & $\qty{19}{\kilo\gram}$
        & $\SI{17}[\$]{}$ \\ \hline
    $25$
        &
        &
        &
        & Take
        & Take
        & $\qty{2}{\kilo\gram}$
        & $\SI{3}[\$]{}$ \\ \hline
    $26$
        & Take
        &
        &
        & Take
        & Take
        & $\qty{14}{\kilo\gram}$
        & $\SI{7}[\$]{}$ \\ \hline
    $27$
        &
        & Take
        &
        & Take
        & Take
        & $\qty{6}{\kilo\gram}$
        & $\SI{13}[\$]{}$ \\ \hline
    $28$
        & Take
        & Take
        &
        & Take
        & Take
        & $\qty{18}{\kilo\gram}$
        & $\SI{17}[\$]{}$ \\ \hline
    $29$
        &
        &
        & Take
        & Take
        & Take
        & $\qty{4}{\kilo\gram}$
        & $\SI{5}[\$]{}$ \\ \hline
    $30$
        & Take
        &
        & Take
        & Take
        & Take
        & $\qty{16}{\kilo\gram}$
        & $\SI{9}[\$]{}$ \\ \hline
    $31$
        &
        & Take
        & Take
        & Take
        & Take
        & $\qty{8}{\kilo\gram}$
        & $\SI{15}[\$]{}$ \\ \hline
    $32$
        & Take
        & Take
        & Take
        & Take
        & Take
        & $\qty{20}{\kilo\gram}$
        & $\SI{19}[\$]{}$ \\ \hline
\end{tabular}
\end{center}
\medskip

By rejecting all subsets above the weight limit ($\qty{15}{\kilo\gram}$) and getting the highest-value subset, we can clearly see that this will lead to an optimal solution. However, the runtime complexity is $O\parens*{2^n}$.


%\subsection{Identifying Optimal Substructure}
%
%Starting with an empty selection of items, we build an array of decisions for 
%
%In the base case, a bag with zero carrying capacity will optimally carry an empty set of items totaling $\SI{0}[\$]{}$.


\subsection{Bottom-Up Tabulation Algorithm}

Let:
\begin{itemize}
    \item $W = {}$the bag's carrying weight limit.
    \item $N = {}$the number of items we can choose from.
    \item $\braces*{S_1, S_2, \dots, S_I} = {}$the set of all items we can choose from. \emph{(This intentionally uses one-based indexing.)}
    \item $w_i = {}$the weight of item $S_i$.
    \item $v_i = {}$the dollar value of item $S_i$.
\end{itemize}

For this solution, we will assume integer weight values, $W \ge 0$, and $w_i \le W$ for all possible $w_i$.

Our main data structure will be a table $m \brackets*{i, w}$ with size $\parens*{N + 1} \times \parens*{W + 1}$. For example:

\begin{table}[H]
    \centering
    \caption{An initial table $m \brackets*{i, w}$ for \ExampleRef{ssub:0-1-knapsack-problem--example}.}
    \label{tab:0-1-knapsack-problem--ds}
    \begin{tabular}{cr|rrrrrrrrrrrrrrrr|}
        {}
            &
            & \multicolumn{16}{l|}{$w$} \\
        {}
            &
            & $\phantom{0}0$ & $\phantom{0}1$ & $\phantom{0}2$ & $\phantom{0}3$
            & $\phantom{0}4$ & $\phantom{0}5$ & $\phantom{0}6$ & $\phantom{0}7$
            & $\phantom{0}8$ & $\phantom{0}9$
            & $10$ & $11$ & $12$ &$13$ & $14$ & $15$ \\ \hline
        $i$
            & $0$
            & $0$ & $0$ & $0$ & $0$ & $0$ & $0$ & $0$ & $0$
            & $0$ & $0$ & $0$ & $0$ & $0$ & $0$ & $0$ & $0$ \\
        {}
            & $1$
            & & & & & & & & & & & & & & & & \\
        {}
            & $2$
            & & & & & & & & & & & & & & & & \\
        {}
            & $3$
            & & & & & & & & & & & & & & & & \\
        {}
            & $4$
            & & & & & & & & & & & & & & & & \\
        {}
            & $5$
            & & & & & & & & & & & & & & & & \\ \hline
    \end{tabular}
\end{table}

$i$ corresponds to the subset of items we have currently selected from. For \ExampleRef{ssub:0-1-knapsack-problem--example}:
\begin{itemize}
    \item $i = 0$ means we have selected from an empty set $\{\}$.
    \item $i = 1$ means we have selected from $\braces*{S_1}$.
    \item $i = 2$ means we have selected from $\braces*{S_1, S_2}$.
    \item $i = 3$ means we have selected from $\braces*{S_1, S_2, S_3}$.
    \item $i = 4$ means we have selected from $\braces*{S_1, S_2, S_3, S_4}$.
    \item $i = 5$ means we have selected from $\braces*{S_1, S_2, S_3, S_4, S_5}$.
\end{itemize}

$w$ corresponds to hypothetical weight limits up until the weight limit $W$. For example:
\begin{itemize}
    \item $w = 0$ corresponds to a bag of weight limit $\qty{0}{\kilo\gram}$,
    \item $w = 1$ corresponds to a bag of weight limit $\qty{1}{\kilo\gram}$,
    \item $w = 2$ corresponds to a bag of weight limit $\qty{2}{\kilo\gram}$, etc.
\end{itemize}

Each cell of $m\brackets*{i, w}$ is the highest total dollar value possible for the given $i$ and $w$.

To fill the remaining cells, we must apply an optimal substructure. This can be expressed simply as:
\begin{gather}
    m \brackets*{0, w} = 0
        \label{eq:0-1-knapsack-problem--opt-substructure-1} \\
    m \brackets*{i, w} = \max\parens*{
        m \brackets*{i - 1, w - w_i} + v_i, \,
        m \brackets*{i - 1, w}
    }
        \label{eq:0-1-knapsack-problem--opt-substructure-2}
\end{gather}

\eqref{eq:0-1-knapsack-problem--opt-substructure-1} states the trivial case where choosing from zero items will always optimally total zero value. This lets us prefill zeroes in the first row as shown in \cref{tab:0-1-knapsack-problem--ds}.

\eqref{eq:0-1-knapsack-problem--opt-substructure-2} is a recursive equation based around a yes/no decision on whether or not to include an item in the bag. Importantly, this equation only references the previous row $i - 1$, meaning that we must calculate each row in order starting from $i = 1$. Without concerning ourselves with the details of the calculation, let's fill out this first row $i = 1$:
\begin{center}
    \begin{tabular}{cr|rrrrrrrrrrrrrrrr|}
        {}
            &
            & \multicolumn{16}{l|}{$w$} \\
        {}
            &
            & $\phantom{0}0$ & $\phantom{0}1$ & $\phantom{0}2$ & $\phantom{0}3$
            & $\phantom{0}4$ & $\phantom{0}5$ & $\phantom{0}6$ & $\phantom{0}7$
            & $\phantom{0}8$ & $\phantom{0}9$
            & $10$ & $11$ & $12$ &$13$ & $14$ & $15$ \\ \hline
        $i$
            & $0$
            & $0$ & $0$ & $0$ & $0$ & $0$ & $0$ & $0$ & $0$
            & $0$ & $0$ & $0$ & $0$ & $0$ & $0$ & $0$ & $0$ \\
        {}
            & $1$
            & $0$ & $0$ & $0$ & $0$ & $0$ & $0$ & $0$ & $0$
            & $0$ & $0$ & $0$ & $0$ & $4$ & $4$ & $4$ & $4$ \\
        {}
            & $2$
            & & & & & & & & & & & & & & & & \\
        {}
            & $3$
            & & & & & & & & & & & & & & & & \\
        {}
            & $4$
            & & & & & & & & & & & & & & & & \\
        {}
            & $5$
            & & & & & & & & & & & & & & & & \\ \hline
    \end{tabular}
\end{center}
\SkipAfterTable

%As an example, let's look at the blue-shaded cell at $i = 1$ and $w = 14$. Applying equation \eqref{eq:0-1-knapsack-problem--opt-substructure-2}, we get:
%\begin{align*}
%    \colorbox{blue!30}{$m \brackets*{1, 14}$} &= \max\parens*{
%        m \brackets*{i - 1, w - w_i} + v_i, \,
%        m \brackets*{i - 1, w}
%    }
%    \\
%    &= \max\parens*{
%        m \brackets*{1 - 1, 14 - 12} + 4, \,
%        m \brackets*{1 - 1, 14}
%    }
%    \\
%    &= \max\parens*{
%        m \brackets*{0, 14 - 12} + 4, \,
%        m \brackets*{0, 14}
%    }
%    %\label{eq:0-1-knapsack-problem--example1} %\\
%    \\
%    &= \max\parens*{
%        \colorbox{red!30}{$m \brackets*{0, 2}$} + 4, \,
%        \colorbox{red!30}{$m \brackets*{0, 14}$}
%    }
%    \\
%    &= \max\parens*{
%        0 + 4, \,
%        0
%    }
%    \\
%    &= 4
%\end{align*}
%
%\myul{Don't worry too much about the details of the above calculation for now}, but the key thing to observe here is that to calculate $m \brackets*{1, 14}$, we must read off the values of the red-shaded cells ($m \brackets*{0, 2}$ and $m \brackets*{0, 14}$), both of which are in the previous row.
%
%Discussing how we calculated the $i = 1$ row isn't particularly interesting, and neither is discussing the $i = 2$ row:

Using the results of row $i = 1$, we can now calculate row $i = 2$:
\begin{center}
    \begin{tabular}{cr|rrrrrrrrrrrrrrrr|}
        {}
            &
            & \multicolumn{16}{l|}{$w$} \\
        {}
            &
            & $\phantom{0}0$ & $\phantom{0}1$ & $\phantom{0}2$ & $\phantom{0}3$
            & $\phantom{0}4$ & $\phantom{0}5$ & $\phantom{0}6$ & $\phantom{0}7$
            & $\phantom{0}8$ & $\phantom{0}9$
            & $10$ & $11$ & $12$ &$13$ & $14$ & $15$ \\ \hline
        $i$
            & $0$
            & $0$ & $0$ & $0$ & $0$ & $0$ & $0$ & $0$ & $0$
            & $0$ & $0$ & $0$ & $0$ & $0$ & $0$ & $0$ & $0$ \\
        {}
            & $1$
            & $0$ & $0$ & $0$ & $0$ & $0$ & $0$ & $0$ & $0$
            & $0$ & $0$ & $0$ & $0$ & $4$ & $4$ & $4$ & $4$ \\
        {}
            & $2$
            & $0$ & $0$ & $0$ & $0$ & $10$ & $10$ & $10$ & $10$
            & $10$ & $10$ & $10$ & $10$ & $10$ & $10$ & $10$ & $10$ \\
        {}
            & $3$
            & & & & & & & & & & & & & & & & \\
        {}
            & $4$
            & & & & & & & & & & & & & & & & \\
        {}
            & $5$
            & & & & & & & & & & & & & & & & \\ \hline
    \end{tabular}
\end{center}
\SkipAfterTable

%Similarly, we see that to calculate each cell of the row $i = 2$, we reference the previous row:
%\begin{align*}
%    \colorbox{blue!30}{$m \brackets*{2, 7}$} &= \max\parens*{
%        m \brackets*{2 - 1, 7 - 4} + 10, \,
%        m \brackets*{2 - 1, 7}
%    }
%    \\
%    &= \max\parens*{
%        \colorbox{red!30}{$m \brackets*{1, 3}$} + 10, \,
%        \colorbox{red!30}{$m \brackets*{1, 7}$}
%    }
%    \\
%    &= 10
%\end{align*}

...and so on.

To understand how equation \eqref{eq:0-1-knapsack-problem--opt-substructure-2} works and how we've been calculating these rows, let's use the $i = 3$ row as an example since this is where things get more interesting:
\begin{center}
    \begin{tabular}{cr|rrrrrrrrrrrrrrrr|}
        {}
            &
            & \multicolumn{16}{l|}{$w$} \\
        {}
            &
            & $\phantom{0}0$ & $\phantom{0}1$ & $\phantom{0}2$ & $\phantom{0}3$
            & $\phantom{0}4$ & $\phantom{0}5$ & $\phantom{0}6$ & $\phantom{0}7$
            & $\phantom{0}8$ & $\phantom{0}9$
            & $10$ & $11$ & $12$ &$13$ & $14$ & $15$ \\ \hline
        $i$
            & $0$
            & $0$ & $0$ & $0$ & $0$ & $0$ & $0$ & $0$ & $0$
            & $0$ & $0$ & $0$ & $0$ & $0$ & $0$ & $0$ & $0$ \\
        {}
            & $1$
            & $0$ & $0$ & $0$ & $0$ & $0$ & $0$ & $0$ & $0$
            & $0$ & $0$ & $0$ & $0$ & $4$ & $4$ & $4$ & $4$ \\
        {}
            & $2$
            & $0$ & $0$ & $0$ & \cellcolor{red!30}$0$ & $10$
            & \cellcolor{red!30}$10$ & $10$ & $10$
            & $10$ & $10$ & $10$ & $10$ & $10$ & $10$ & $10$ & $10$ \\
        {}
            & $3$
            & $0$ & $0$ & $2$ & $2$ & $10$ & \cellcolor{blue!30}$10$
            & $12$ & $12$
            & $12$ & $12$ & $12$ & $12$ & $12$ & $12$ & $12$ & $12$ \\
        {}
            & $4$
            & & & & & & & & & & & & & & & & \\
        {}
            & $5$
            & & & & & & & & & & & & & & & & \\ \hline
    \end{tabular}
\end{center}
\SkipAfterTable

Let's take a closer look at how \colorbox{blue!30}{$m \brackets*{3, 5}$} is calculated. $m \brackets*{3, 5}$ is the highest dollar value of a bag with a carrying capacity of $\qty{5}{\kilo\gram}$, selecting only from the set of items $\braces*{S_1, S_2, S_3}$. If we apply equation \eqref{eq:0-1-knapsack-problem--opt-substructure-2}, we get:
\medskip
\begin{gather*}
    w_3 = \qty{2}{\kilo\gram}
    \qquad\qquad
    v_3 = \SI{2}[\$]{}
\end{gather*}
\begin{align}
    \colorbox{blue!30}{$m \brackets*{3, 5}$} &= \max\parens*{
        m \brackets*{i - 1, w - w_3} + v_3, \,
        m \brackets*{i - 1, w}
    }
    \notag \\
    &= \max\parens*{
        m \brackets*{3 - 1, 5 - w_3} + v_3, \,
        m \brackets*{3 - 1, 5}
    }
    \notag \\
    &= \max\parens*{
        m \brackets*{2, 5 - w_3} + v_3, \,
        m \brackets*{2, 5}
    }
    \notag \\
    &= \max\parens*{
        m \brackets*{2, 5 - 2} + 2, \,
        m \brackets*{2, 5}
    }
    \notag \\
    &= \max{(
        \phantom{.}
        \underbrace{
            \colorbox{red!30}{$m \brackets*{2, 3}$} + 2
        }_{\mathclap{\text{Option 1}}}
        \,
        ,
        \phantom{.}
        \underbrace{
            \colorbox{red!30}{$m \brackets*{2, 5}$}
        }_{\mathclap{\text{Option 2}}}
        \phantom{.}
    )}
    \label{eq:0-1-knapsack-problem--example1}
\end{align}

The last line \eqref{eq:0-1-knapsack-problem--example1} reads that we choose the best dollar value of two options:
\begin{pindent}
\begin{description}
    \item[Option 1:] \emph{We put item $S_3$ in the bag.} To calculate this option, we are essentially taking the empty $\qty{5}{\kilo\gram}$ bag, putting item $S_3$ into it, and then asking for the best dollar value that we can put \myul{in the space that remains in the bag} using only the previous items $\braces*{S_1, S_2}$. The weight of item $S_3$ is $w_3 = \qty{2}{\kilo\gram}$, so the remaining space is $w - w_3 = 5 - 2 = \qty{3}{\kilo\gram}$.
        %We calculate this dollar value by first finding the \myul{carrying capacity that will remain} after we account for item $S_3$. $S_3$ has a weight of $w_3 = \qty{2}{\kilo\gram}$, so the capacity that remains is $5 - 2 = \qty{3}{\kilo\gram}$. Thus, we calculate $m \brackets*{2, 3}$, which is the highest dollar value of a bag with this $\qty{3}{\kilo\gram}$
        %, which is $5 - w_3 = 5 - 2 = \qty{3}{\kilo\gram}$. We then find the optimal dollar value for filling this remaining $\qty{2}{\kilo\gram}$ if we choose from an empty set $\braces*{}$. Add the dollar value of $S_1$.
    \item[Option 2:] \emph{We don't put item $S_3$ in the bag.} To calculate this option, we ask what the highest dollar value of a bag with a carrying capacity of the same $\qty{5}{\kilo\gram}$, except we select only from items $\braces*{S_1, S_2}$.
\end{description}
\end{pindent}

Continuing on from \eqref{eq:0-1-knapsack-problem--example1}:
\begin{align*}
    \colorbox{blue!30}{$m \brackets*{3, 5}$} &= \max\parens*{0 + 2, \, 10}
    \\
    &= \SI{10}[\$]{}
\end{align*}

This matches what is written in the cell.

Continuing until the whole table is complete, we get:
\begin{center}
    \begin{tabular}{cr|rrrrrrrrrrrrrrrr|}
        {}
            &
            & \multicolumn{16}{l|}{$w$} \\
        {}
            &
            & $\phantom{0}0$ & $\phantom{0}1$ & $\phantom{0}2$ & $\phantom{0}3$
            & $\phantom{0}4$ & $\phantom{0}5$ & $\phantom{0}6$ & $\phantom{0}7$
            & $\phantom{0}8$ & $\phantom{0}9$
            & $10$ & $11$ & $12$ &$13$ & $14$ & $15$ \\ \hline
        $i$
            & $0$
            & $0$ & $0$ & $0$ & $0$ & $0$ & $0$ & $0$ & $0$
            & $0$ & $0$ & $0$ & $0$ & $0$ & $0$ & $0$ & $0$ \\
        {}
            & $1$
            & $0$ & $0$ & $0$ & $0$ & $0$ & $0$ & $0$ & $0$
            & $0$ & $0$ & $0$ & $0$ & $4$ & $4$ & $4$ & $4$ \\
        {}
            & $2$
            & $0$ & $0$ & $0$ & $0$ & $10$ & $10$ & $10$ & $10$
            & $10$ & $10$ & $10$ & $10$ & $10$ & $10$ & $10$ & $10$ \\
        {}
            & $3$
            & $0$ & $0$ & $2$ & $2$ & $10$ & $10$ & $12$ & $12$
            & $12$ & $12$ & $12$ & $12$ & $12$ & $12$ & $12$ & $12$ \\
        {}
            & $4$
            & $0$ & $2$ & $2$ & $4$ & $10$ & $12$ & $12$ & $14$
            & $14$ & $14$ & $14$ & $14$ & $14$ & $14$ & $14$ & $14$ \\
        {}
            & $5$
            & $0$ & $2$ & $3$ & $4$ & $10$ & $12$ & $13$ & $14$
            & $15$ & $15$ & $15$ & $15$ & $15$ & $15$ & $15$
            & \cellcolor{green!30}$15$ \\ \hline
    \end{tabular}
\end{center}
\SkipAfterTable

Thus, the best dollar value that we can put in a $\qty{15}{\kilo\gram}$ bag using items $\braces*{S_1, S_2, S_3, S_4, S_5}$ is $\colorbox{green!30}{$m \brackets*{5, 15}$} = \SI{15}[\$]{}$.

To read back an optimal set of items, we must \emph{backtrack} starting from \colorbox{green!30}{$m \brackets*{5, 15}$}. The idea is to first look directly backwards to see if the weight is the same, otherwise we subtract the value and the weight and check the appropriate cell. If done correctly, we touch the following cells:
\begin{table}[H]
    \centering
    \caption{Backtracking through the table for \ExampleRef{ssub:0-1-knapsack-problem--example}.}
    \label{tab:0-1-knapsack-problem--backtrack-example}
    \begin{tabular}{ccccccr|rrrrrrrrrrrrrrrr|}
        {} & {} & {} & {} & {} & {}
            &
            & \multicolumn{16}{l|}{$w$} \\
        {} & {} & {} & {} & {} & {}
            &
            & $\phantom{0}0$ & $\phantom{0}1$ & $\phantom{0}2$ & $\phantom{0}3$
            & $\phantom{0}4$ & $\phantom{0}5$ & $\phantom{0}6$ & $\phantom{0}7$
            & $\phantom{0}8$ & $\phantom{0}9$
            & $10$ & $11$ & $12$ &$13$ & $14$ & $15$ \\ \hline
        {} & {} & {} & {} & {} & $i$
            & $0$
            & $0$ & $0$ & $0$ & $0$ & $0$ & $0$ & $0$ & \cellcolor{red!30}$0$
            & $0$ & $0$ & $0$ & $0$ & $0$ & $0$ & $0$ & $0$ \\
        $S_1$ & $\to$ & $\qty{12}{\kilo\gram}$ & $\SI{4}[\$]{}$ & {} & {}
            & $1$
            & $0$ & $0$ & $0$ & $0$ & $0$ & $0$ & $0$ & \cellcolor{red!30}$0$
            & $0$ & $0$ & $0$ & $0$ & $4$ & $4$ & $4$ & $4$ \\
        $S_2$ & $\to$ & $\qty{4}{\kilo\gram}$ & $\SI{10}[\$]{}$ & {} & {}
            & $2$
            & $0$ & $0$ & $0$ & $0$ & $10$ & $10$ & $10$ & $10$
            & $10$ & $10$ & $10$ & \cellcolor{red!30}$10$
            & $10$ & $10$ & $10$ & $10$ \\
        $S_3$ & $\to$ & $\qty{2}{\kilo\gram}$ & $\SI{2}[\$]{}$ & {} & {}
            & $3$
            & $0$ & $0$ & $2$ & $2$ & $10$ & $10$ & $12$ & $12$
            & $12$ & $12$ & $12$ & $12$ & $12$
            & \cellcolor{red!30}$12$ & $12$ & $12$ \\
        $S_4$ & $\to$ & $\qty{1}{\kilo\gram}$ & $\SI{2}[\$]{}$ & {} & {}
            & $4$
            & $0$ & $2$ & $2$ & $4$ & $10$ & $12$ & $12$ & $14$
            & $14$ & $14$ & $14$ & $14$ & $14$ & $14$
            & \cellcolor{red!30}$14$ & $14$ \\
        $S_5$ & $\to$ & $\qty{1}{\kilo\gram}$ & $\SI{1}[\$]{}$ & {} & {}
            & $5$
            & $0$ & $2$ & $3$ & $4$ & $10$ & $12$ & $13$ & $14$
            & $15$ & $15$ & $15$ & $15$ & $15$ & $15$ & $15$
            & \cellcolor{red!30}$15$ \\ \hline
    \end{tabular}
\end{table}

To understand how it works, let's start from the beginning at \colorbox{green!30}{$m \brackets*{5, 15}$}:
\begin{center}
    \begin{tabular}{ccccccr|rrrrrrrrrrrrrrrr|}
        {} & {} & {} & {} & {} & {}
            &
            & \multicolumn{16}{l|}{$w$} \\
        {} & {} & {} & {} & {} & {}
            &
            & $\phantom{0}0$ & $\phantom{0}1$ & $\phantom{0}2$ & $\phantom{0}3$
            & $\phantom{0}4$ & $\phantom{0}5$ & $\phantom{0}6$ & $\phantom{0}7$
            & $\phantom{0}8$ & $\phantom{0}9$
            & $10$ & $11$ & $12$ &$13$ & $14$ & $15$ \\ \hline
        {} & {} & {} & {} & {} & $i$
            & $0$
            & $0$ & $0$ & $0$ & $0$ & $0$ & $0$ & $0$ & $0$
            & $0$ & $0$ & $0$ & $0$ & $0$ & $0$ & $0$ & $0$ \\
        $S_1$ & $\to$ & $\qty{12}{\kilo\gram}$ & $\SI{4}[\$]{}$ & {} & {}
            & $1$
            & $0$ & $0$ & $0$ & $0$ & $0$ & $0$ & $0$ & $0$
            & $0$ & $0$ & $0$ & $0$ & $4$ & $4$ & $4$ & $4$ \\
        $S_2$ & $\to$ & $\qty{4}{\kilo\gram}$ & $\SI{10}[\$]{}$ & {} & {}
            & $2$
            & $0$ & $0$ & $0$ & $0$ & $10$ & $10$ & $10$ & $10$
            & $10$ & $10$ & $10$ & $10$
            & $10$ & $10$ & $10$ & $10$ \\
        $S_3$ & $\to$ & $\qty{2}{\kilo\gram}$ & $\SI{2}[\$]{}$ & {} & {}
            & $3$
            & $0$ & $0$ & $2$ & $2$ & $10$ & $10$ & $12$ & $12$
            & $12$ & $12$ & $12$ & $12$ & $12$
            & $12$ & $12$ & $12$ \\
        $S_4$ & $\to$ & $\qty{1}{\kilo\gram}$ & $\SI{2}[\$]{}$ & {} & {}
            & $4$
            & $0$ & $2$ & $2$ & $4$ & $10$ & $12$ & $12$ & $14$
            & $14$ & $14$ & $14$ & $14$ & $14$ & $14$
            & \cellcolor{red!30}$14$ & \cellcolor{blue!30}$14$ \\
        $S_5$ & $\to$ & $\qty{1}{\kilo\gram}$ & $\SI{1}[\$]{}$ & {} & {}
            & $5$
            & $0$ & $2$ & $3$ & $4$ & $10$ & $12$ & $13$ & $14$
            & $15$ & $15$ & $15$ & $15$ & $15$ & $15$ & $15$
            & \cellcolor{green!30}$15$ \\ \hline
    \end{tabular}
\end{center}
\SkipAfterTable

Directly behind \colorbox{green!30}{$m \brackets*{5, 15}$} is \colorbox{blue!30}{$m \brackets*{4, 15}$}. Here, we check for one of two possible branches
\begin{pindent}
    \begin{description}
        \item[Branch 1:] If $\colorbox{green!30}{$m \brackets*{5, 15}$} \ne \colorbox{blue!30}{$m \brackets*{4, 15}$}$, then we know that \emph{our optimal set does contain item $S_5$}. This is because the optimal set's composition \myul{must have changed} when we introduced the new item $S_5$.
        \item[Branch 2:] If $\colorbox{green!30}{$m \brackets*{5, 15}$} = \colorbox{blue!30}{$m \brackets*{4, 15}$}$, then we say that \emph{our optimal set does not necessarily contain item $S_5$}. This is because the optimal set's composition \myul{doesn't need to have changed} when we introduced the new item $S_5$.
    \end{description}
\end{pindent}

{\footnotesize\emph{Side note: Branch 2 is carefully worded to say that it's possible but uncertain that item $S_5$ can be included in the optimal set, but that we know for certain that it can be left out, hence we ``say" that the optimal set does not necessarily contain $S_5$. It simplifies our discussion to skip over checking if we can include $S_5$.}}

In this example, we see that $\colorbox{green!30}{$m \brackets*{5, 15}$} \ne \colorbox{blue!30}{$m \brackets*{4, 15}$}$, so we take item $S_5$ as part of the optimal set.

Now, depending on the branch, we will visit a new cell. This is simply the corresponding cell as described by equation \eqref{eq:0-1-knapsack-problem--opt-substructure-2}:
\begin{pindent}
    \begin{description}
        \item[If we took Branch 1:] We visit $m \brackets*{i - 1, w - w_i}$. In this case, we visit \colorbox{red!30}{$m \brackets*{4, 14}$}.
        \item[If we took Branch 2:] We visit $m \brackets*{i - 1, w}$. In this case, we visit \colorbox{blue!30}{$m \brackets*{4, 15}$}.
    \end{description}
\end{pindent}

This process can repeat until we arrive at the $i = 0$ row.

Reading off of \cref{tab:0-1-knapsack-problem--backtrack-example}, we:
\begin{itemize}
    \item took Branch 1 at $m \brackets*{5, 15}$,
    \item took Branch 1 at $m \brackets*{4, 14}$,
    \item took Branch 1 at $m \brackets*{3, 13}$,
    \item took Branch 1 at $m \brackets*{2, 11}$, and
    \item took Branch 2 at $m \brackets*{1, 7}$.
\end{itemize}

Therefore, our optimal set is $\braces*{S_5, S_4, S_3, S_2}$, which has a total value of $\SI{15}[\$]{}$.

This tabulation algorithm runs in $O\parens*{NW}$ time and space. Further minor improvements to the algorithm are possible and is left as an exercise for the reader.


\subsection{Naive Top-Down Recursion Algorithm}

\Todo{this?}


\subsection{Top-Down Memoization Algorithm}

\Todo{this?}


\subsection{References}
\begin{itemize}
    \item \href{https://en.wikipedia.org/wiki/Knapsack_problem}{\textbu{Wikipedia}}: The example and the mathematical notation was taken from here.
    \item \href{https://practice.geeksforgeeks.org/problems/0-1-knapsack-problem0945/1}{\textbu{GeeksforGeeks}}: Online practice problem.%, used for verifying my solution.
\end{itemize}

