%% Not needed yet
%\def\meshcw#1#2#3#4{
%    \draw[thick, stealth-, #3] (#1) node{#4}  ++(-70:#2) arc (-70:190:#2);
%}
%\def\meshcc#1#2#3#4{
%    \draw[thick, -stealth, #3] (#1) node{#4}  ++(-10:#2) arc (-10:250:#2);
%}
%
%\def\meshright#1#2#3{
%    \draw[thick, -stealth, #2]
%        (#1)
%            ++(0,-0.1)
%                node[label=above:{#3}]{}
%            ++(-0.4,0.1)
%            -- ++(0.8,0)
%    ;
%}

% Draws a slight offset and a well-scaled ground symbol.
\newcommand{\MyGround}[1]{%
    -- ++(0,-0.1)
        node[ground, scale=2, #1]{}
}
\newcommand{\MyShorterTailedGround}[1]{% dreadful name...
    node[ground, scale=2, #1]{}
}
\newcommand{\MyTLGround}[1]{%
    node[tlground, scale=2, #1]{}
}

%% Not needed yet
%% Borrowing the tground symbol for VCC and VEE
%\newcommand{\MyVCC}[1]{%
%    node[tground, #1]{}
%}
%\newcommand{\MyVEE}[1]{%
%    node[tground, #1]{}
%}
%
%% (Shortcuts for op-amp VCC/VEE)
%\newcommand{\MyOpAmpVCC}[1]{%
%    -- ++(0,0.6)
%        node[tground, label={[label distance=1mm]right:{#1}}]{}
%}
%\newcommand{\MyOpAmpVEE}[1]{%
%    -- ++(0,-0.6)
%        node[tground, label={[label distance=1mm]right:{#1}}]{}
%}

