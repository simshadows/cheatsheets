\subsection{Linear Algebra}%
\label{sub:linear-algebra}

\begin{multicols}{2}

    \begin{CheatsheetEntryFrame}

        \CheatsheetEntryTitle{Dot Product} \CheatsheetEntrySubtitle{(or Scalar Product)}

        In $\mathbb{R}^n$:
        \begin{equation*}
            \mathbf{a} \cdot \mathbf{b} = a_1 b_1 + \cdots + a_n b_n = \sum_{k=0}^n{a_k b_k}
        \end{equation*}

        In $\mathbb{R}^2$ or $\mathbb{R}^3$:
        \begin{equation*}
            \mathbf{a} \cdot \mathbf{b} = \abs*{\mathbf{a}} \abs*{\mathbf{b}} \cos{\theta}
            ,\quad \theta \in [0, \pi]
        \end{equation*}

        Useful geometric properties:
        \begin{itemize}
            \item $\mathbf{a} \cdot \mathbf{a} = \abs*{a}^2$, and hence $\abs*{\mathbf{a}} = \sqrt{\mathbf{a} \cdot \mathbf{a}}$.
            \item Vectors $\mathbf{a}, \mathbf{b} \in \mathbb{R}^n$ are orthogonal if $\mathbf{a} \cdot \mathbf{b} = 0$.
        \end{itemize}

    \end{CheatsheetEntryFrame}

    \begin{CheatsheetEntryFrame}

        \CheatsheetEntryTitle{Cross Product} \CheatsheetEntrySubtitle{(or Vector Product)}

        The \textit{cross product} is only defined in $\mathbb{R}^3$.
        \begin{equation*}
            \begin{pmatrix}
                a_1 \\
                a_2 \\
                a_3
            \end{pmatrix}
            \times
            \begin{pmatrix}
                b_1 \\
                b_2 \\
                b_3
            \end{pmatrix}
            =
            \begin{pmatrix}
                a_2 b_3 - a_3 b_2 \\
                a_3 b_1 - a_1 b_3 \\
                a_1 b_2 - a_2 b_1
            \end{pmatrix}
        \end{equation*}

        Calculation using determinants:
        \begin{gather*}
                \begin{pmatrix}
                    a_1 \\
                    a_2 \\
                    a_3
                \end{pmatrix}
                \times
                \begin{pmatrix}
                    b_1 \\
                    b_2 \\
                    b_3
                \end{pmatrix}
                =
                \begin{vmatrix}
                    \mathbf{e}_1 & \mathbf{e}_2 & \mathbf{e}_3 \\
                    a_1          & a_2          & a_3          \\
                    b_1          & b_2          & b_3
                \end{vmatrix}
            \\
                =
                \mathbf{e}_1
                \begin{vmatrix}
                    \memphR{a_2} & \memphB{a_3} \\
                    \memphB{b_2} & \memphR{b_3}
                \end{vmatrix}
                -
                \mathbf{e}_2
                \begin{vmatrix}
                    \memphR{a_1} & \memphB{a_3} \\
                    \memphB{b_1} & \memphR{b_3}
                \end{vmatrix}
                +
                \mathbf{e}_3
                \begin{vmatrix}
                    \memphR{a_1} & \memphB{a_2} \\
                    \memphB{b_1} & \memphR{b_2}
                \end{vmatrix}
            \\
                = \mathbf{e}_1 \parens*{\memphR{a_2 b_3} - \memphB{a_3 b_2}}
                - \mathbf{e}_2 \parens*{\memphR{a_1 b_3} - \memphB{a_3 b_1}}
                + \mathbf{e}_3 \parens*{\memphR{a_1 b_2} - \memphB{a_2 b_1}}
        \end{gather*}

        Useful geometric properties:
        \begin{itemize}
            \item Vector $\mathbf{a} \times \mathbf{b}$ is orthogonal to $\mathbf{a}$ and $\mathbf{b}$.
            \item $\abs*{\mathbf{a} \times \mathbf{b}} = \abs*{\mathbf{a}} \abs*{\mathbf{b}} \sin{\theta} = \text{area of a parallelogram}$
        \end{itemize}

    \end{CheatsheetEntryFrame}

    \MulticolsBreak

    \begin{CheatsheetEntryFrame}

        \CheatsheetEntryTitle{Cramer's Rule}

        Consider the following linear system with $n \times n$ invertible matrix $A$:
        \begin{equation*}
            A \mathbf{x} = \mathbf{b}
            ,\qquad A \in M_{nn},\ \mathbf{x} \in \mathbb{R}^n,\ \mathbf{b} \in \mathbb{R}^n
        \end{equation*}

        The system has a unique solution:
        \begin{equation*}
            x_k = \frac{\det{\parens*{B_k}}}{\det{\parens*{A}}}
            ,\qquad \forall k = 1, \dots, n,
        \end{equation*}
        where $B_k$ is the matrix obtained from $A$ by replacing the $\Nth{k}{th}$ column with the vector $\mathbf{b}$.

        \CheatsheetEntryExtraSeparation

        \CheatsheetEntryTitle{Cramer's Rule ($2 \times 2$ Matrix)}
        \begin{equation*}
            \begin{dcases}
                \xmemphB{a_{11}} x + \xmemphB{a_{12}} y = \xmemphR{b_1} \\
                \xmemphB{a_{21}} x + \xmemphB{a_{22}} y = \xmemphR{b_2}
            \end{dcases}
            \quad \Rightarrow \quad
                \xmemphB{
                    \begin{bmatrix}
                        a_{11} & a_{12} \\
                        a_{21} & a_{22}
                    \end{bmatrix}
                }
                \begin{bmatrix}
                    x \\
                    y
                \end{bmatrix}
                =
                \xmemphR{
                    \begin{bmatrix}
                        b_1 \\
                        b_2
                    \end{bmatrix}
                }
        \end{equation*}
        \begin{equation*}
                x
                    = \frac{
                        \xmemphB{
                            \begin{vmatrix}
                                \xmemphR{b_1} & a_{12} \\
                                \xmemphR{b_2} & a_{22}
                            \end{vmatrix}
                        }
                    }{
                        \xmemphB{
                            \begin{vmatrix}
                                a_{11} & a_{12} \\
                                a_{21} & a_{22}
                            \end{vmatrix}
                        }
                    }
                    ,\qquad
                y
                    = \frac{
                        \xmemphB{
                            \begin{vmatrix}
                                a_{11} & \xmemphR{b_1} \\
                                a_{21} & \xmemphR{b_2}
                            \end{vmatrix}
                        }
                    }{
                        \xmemphB{
                            \begin{vmatrix}
                                a_{11} & a_{12} \\
                                a_{21} & a_{22}
                            \end{vmatrix}
                        }
                    }
        \end{equation*}

        \CheatsheetEntryTitle{Cramer's Rule ($3 \times 3$ Matrix)}
        \begin{equation*}
            \begin{dcases}
                \xmemphB{a_{11}} x + \xmemphB{a_{12}} y + \xmemphB{a_{13}} z = \xmemphR{b_1} \\
                \xmemphB{a_{21}} x + \xmemphB{a_{22}} y + \xmemphB{a_{23}} z = \xmemphR{b_2} \\
                \xmemphB{a_{31}} x + \xmemphB{a_{32}} y + \xmemphB{a_{33}} z = \xmemphR{b_3}
            \end{dcases}
        \end{equation*}
        \begin{equation*}
                \xmemphB{
                    \begin{bmatrix}
                        a_{11} & a_{12} & a_{13} \\
                        a_{21} & a_{22} & a_{23} \\
                        a_{31} & a_{32} & a_{33}
                    \end{bmatrix}
                }
                \begin{bmatrix}
                    x \\
                    y \\
                    z
                \end{bmatrix}
                =
                \xmemphR{
                    \begin{bmatrix}
                        b_1 \\
                        b_2 \\
                        b_3
                    \end{bmatrix}
                }
        \end{equation*}
        \begin{equation*}
                x
                    = \frac{
                        \xmemphB{
                            \begin{vmatrix}
                                \xmemphR{b_1} & a_{12} & a_{13} \\
                                \xmemphR{b_2} & a_{22} & a_{23} \\
                                \xmemphR{b_3} & a_{32} & a_{33}
                            \end{vmatrix}
                        }
                    }{
                        \xmemphB{
                            \begin{vmatrix}
                                a_{11} & a_{12} & a_{13} \\
                                a_{21} & a_{22} & a_{23} \\
                                a_{31} & a_{32} & a_{33}
                            \end{vmatrix}
                        }
                    }
                    ,\qquad
                y
                    = \frac{
                        \xmemphB{
                            \begin{vmatrix}
                                a_{11} & \xmemphR{b_1} & a_{13} \\
                                a_{21} & \xmemphR{b_2} & a_{23} \\
                                a_{31} & \xmemphR{b_3} & a_{33}
                            \end{vmatrix}
                        }
                    }{
                        \xmemphB{
                            \begin{vmatrix}
                                a_{11} & a_{12} & a_{13} \\
                                a_{21} & a_{22} & a_{23} \\
                                a_{31} & a_{32} & a_{33}
                            \end{vmatrix}
                        }
                    }
                    ,
        \end{equation*}
        \begin{equation*}
                z
                    = \frac{
                        \xmemphB{
                            \begin{vmatrix}
                                a_{11} & a_{12} & \xmemphR{b_1} \\
                                a_{21} & a_{22} & \xmemphR{b_2} \\
                                a_{31} & a_{32} & \xmemphR{b_3}
                            \end{vmatrix}
                        }
                    }{
                        \xmemphB{
                            \begin{vmatrix}
                                a_{11} & a_{12} & a_{13} \\
                                a_{21} & a_{22} & a_{23} \\
                                a_{31} & a_{32} & a_{33}
                            \end{vmatrix}
                        }
                    }
        \end{equation*}

    \end{CheatsheetEntryFrame}

\end{multicols}

