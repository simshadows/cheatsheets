\section{Basic Definitions}%
\label{sec:graphs--basic-definitions}

A graph $G$ consists of two finite sets: a nonempty set $V \parens*{G}$ of \emph{vertices}, and a set $E \parens*{G}$ of \emph{edges}, where each edge is associated with a set consisting of either one or two vertices called its \emph{endpoints}.

\begin{figure}[H]
    \centering
    \begin{tikzpicture}[
        line width=1.6pt,
        draw=black,
        -%{Stealth[length=2.4mm, width=2.8mm]}
    ]
        %\tikzstyle{every node}=[
        %    font={\large\bfseries}
        %]

        \node[GraphNodeRed, draw=black]
            (NODE0) at (0,0) {$v_2$};
        \node[GraphNodeRed, draw=black]
            (NODE1) at (-1,2.5) {$v_3$};
        \node[GraphNodeBlue, draw=black]
            (NODE2) at (-4,0.8) {$v_0$};
        \node[GraphNodePurple, draw=black]
            (NODE3) at (-2.2,-1) {$v_1$};
        \node[GraphNodeYellow, draw=black]
            (NODE4) at (-4.5, -1.3) {$v_4$};

        \draw
            (NODE0)
            edge[
                line width=2.2pt,
                draw=myred
            ]
            node[midway, right, xshift=1mm, yshift=1mm] {$e_4$}
            (NODE1)
        ;
        \draw
            (NODE2)
            edge[
                loop left,
                min distance=14mm,
                line width=2.2pt,
                draw=myblue
            ]
            node[midway, above, xshift=4mm, yshift=2mm] {$e_5$}
            (NODE2)
        ;
        \draw
            (NODE1)
            edge[
                bend right,
                line width=2.2pt,
                draw=mygreen
            ]
            node[midway, left, yshift=2mm] {$e_0$}
            (NODE2)
        ;
        \draw
            (NODE1)
            edge[
                bend left,
                line width=2.2pt,
                draw=mygreen
            ]
            node[midway, left, yshift=2mm] {$e_1$}
            (NODE2)
        ;

        \draw
            (NODE2)
            edge[
                line width=2.2pt,
                draw=mypurple
            ]
            node[midway, right, xshift=1mm, yshift=1mm] {$e_2$}
            (NODE3)
        ;
        \draw
            (NODE3)
            edge[
                line width=2.2pt,
                draw=mypurple
            ]
            node[midway, above, xshift=-0.5mm, yshift=0.5mm] {$e_3$}
            (NODE0)
        ;

        \node[right] at (1.5,0.7) {\parbox{10cm}{
            $\braces*{v_0, v_1, v_2, v_3, v_4}$ are \myul{\emph{vertices}}.
            \\[1mm]
            $\braces*{e_0, e_1, e_2, e_3, e_4, e_5}$ are \myul{\emph{edges}}.
            \\[1mm]
            \textbf{\color{myred} Edge $\boldsymbol{e_4}$ is \myul{\emph{incident}} on each \myul{\emph{endpoint}} $\boldsymbol{v_2}$ and $\boldsymbol{v_3}$.}
            \\[1mm]
            \textbf{\color{myred} Vertices $\boldsymbol{v_2}$ and $\boldsymbol{v_3}$ are \myul{\emph{adjacent}} (connected by an edge).}
            \\[1mm]
            Edges \myul{\emph{connect}} their endpoints.
            \\[1mm]
            \textbf{\color{mypurple} Edges $\boldsymbol{e_2}$ and $\boldsymbol{e_3}$ are \myul{\emph{adjacent}} (shared endpoint).}
            \\[1mm]
            \textbf{\color{mygreen!85!black} Edges $\boldsymbol{e_0}$ and $\boldsymbol{e_1}$ are \myul{\emph{parallel}} (same pair of endpoints).}
            \\[1mm]
            \textbf{\color{myblue!85!black} Edge $\boldsymbol{e_5}$ is a \myul{\emph{loop}} (only one endpoint).}
            \\[1mm]
            \textbf{\color{myblue!85!black} Vertex $\boldsymbol{v_0}$ is a \myul{\emph{adjacent} \emph{to} \emph{itself}} (endpoint of a loop).}
            \\[1mm]
            \textbf{\color{myyellow!75!black} Vertex $\boldsymbol{v_4}$ is \myul{\emph{isolated}} (no incident edges).}
        }};
    \end{tikzpicture}
    %% Old version
    %\begin{tikzpicture}[
    %    line width=1.6pt,
    %    draw=black,
    %    -%{Stealth[length=2.4mm, width=2.8mm]}
    %]
    %    \tikzstyle{every node}=[
    %        font={\large\bfseries}
    %    ]

    %    \node[GraphNodeGreen, mygreen, draw=black, minimum size=6mm]
    %        (NODE0) at (0,0) {};
    %    \node[GraphNodeGreen, mygreen, draw=black, minimum size=6mm]
    %        (NODE1) at (-1,2.5) {};
    %    \node[GraphNodeGreen, mygreen, draw=black, minimum size=6mm]
    %        (NODE2) at (-4,0.5) {};
    %    \node[GraphNodePurple, mypurple, draw=black, minimum size=6mm]
    %        (NODE4) at (-2,-0.7) {};

    %    \draw
    %        (NODE0)
    %        edge[
    %            line width=2.2pt,
    %            myyellow!75!black
    %        ]
    %        coordinate[at end] (edgeend)
    %        coordinate[at start] (edgestart)
    %        node[midway, right, xshift=1mm, yshift=1mm] {Edge}
    %        (NODE1)
    %    ;
    %    \draw
    %        (NODE0)
    %        edge[
    %            loop below,
    %            min distance=14mm,
    %            line width=2.2pt,
    %            myred
    %        ]
    %        node[midway, right, xshift=1mm, xshift=1mm] {Loop}
    %        (NODE0)
    %    ;
    %    \draw
    %        (NODE1)
    %        edge[
    %            bend right,
    %            line width=2.2pt,
    %            myblue
    %        ]
    %        node[midway, left, yshift=2mm] {Parallel Edges}
    %        (NODE2)
    %    ;
    %    \draw
    %        (NODE1)
    %        edge[
    %            bend left,
    %            line width=2.2pt,
    %            myblue
    %        ]
    %        (NODE2)
    %    ;
    %    %\draw (NODE2) edge (NODE3);
    %    %\draw (NODE3) edge (NODE0);

    %    \node[right, mygreen] at (NODE0.east) {Vertex};
    %    \node[left, mypurple] at (NODE4.west) {Isolated Vertex};

    %    \draw
    %        (edgestart)
    %        --
    %        ++(2,1.5)
    %        node[right]{Incidence}
    %        --
    %        (edgeend)
    %    ;
    %\end{tikzpicture}
    \caption{Basic parts of a \emph{graph}.}
    \label{fig:graphs--basic-definitions--1}
\end{figure}

Instead of $E \parens*{G}$, a \emph{directed graph} (or \emph{digraph}) has a set $D \parens*{G}$ of \emph{directed edges}. A \emph{mixed graph} can contain both directed and undirected edges.

A \emph{weighted graph} has valued edges. A \emph{multigraph} may have parallel edges. A \emph{simple graph} has no loops or parallel edges.

\begin{figure}[H]
    \centering
    \subfigure[%
        Directed graph.
    ]{%
        \begin{tikzpicture}[
            line width=1.6pt,
            draw=black,
            >=Stealth
        ]
            \node[GraphNodeSmall]
                (NODE0) at (0,0) {};
            \node[GraphNodeSmall]
                (NODE1) at (-0.4,1.8) {};
            \node[GraphNodeSmall]
                (NODE2) at (-3.5,1.5) {};
            \node[GraphNodeSmall]
                (NODE3) at (-2.5,-1) {};

            \draw (NODE0) edge[line width=2.2pt, ->] (NODE1);
            \draw (NODE0) edge[line width=2.2pt, ->] (NODE2);
            \draw (NODE0) edge[line width=2.2pt, <-] (NODE3);
            \draw (NODE2.275) edge[line width=2.2pt, -{>[right]}] (NODE3.135);
            \draw (NODE2.315) edge[line width=2.2pt, {<[right]}-] (NODE3.095);
            \draw
                (NODE2)
                edge[
                    line width=2.2pt,
                    ->,
                    loop left,
                    min distance=14mm
                ]
                (NODE2)
            ;
        \end{tikzpicture}
        \label{fig:graphs--basic-definitions--directed-graph}
    }
    \subfigure[%
        Undirected multigraph.
    ]{%
        \begin{tikzpicture}[
            line width=1.6pt,
            draw=black,
            >=Stealth
        ]
            \node[GraphNodeSmall]
                (NODE0) at (0,0) {};
            \node[GraphNodeSmall]
                (NODE1) at (-0.4,1.8) {};
            \node[GraphNodeSmall]
                (NODE2) at (-3.5,1.5) {};
            \node[GraphNodeSmall]
                (NODE3) at (-2.5,-1) {};

            \draw (NODE0) edge[line width=2.2pt] (NODE1);
            \draw (NODE0) edge[line width=2.2pt] (NODE2);
            \draw (NODE0) edge[line width=2.2pt, bend right=22] (NODE2);
            \draw (NODE0) edge[line width=2.2pt, bend left=22] (NODE2);
            \draw (NODE0) edge[line width=2.2pt] (NODE3);
            \draw (NODE2) edge[line width=2.2pt] (NODE3);
            \draw (NODE2) edge[line width=2.2pt, bend right=22] (NODE3);
            \draw
                (NODE2)
                edge[
                    line width=2.2pt,
                    loop left,
                    min distance=14mm
                ]
                (NODE2)
            ;
        \end{tikzpicture}
        \label{fig:graphs--basic-definitions--undirected-graph}
    }
    \subfigure[%
        Weighted undirected graph.
    ]{%
        \begin{tikzpicture}[
            line width=1.6pt,
            draw=black,
            >=Stealth
        ]
            \node[GraphNodeSmall]
                (NODE0) at (0,0) {};
            \node[GraphNodeSmall]
                (NODE1) at (-0.4,1.8) {};
            \node[GraphNodeSmall]
                (NODE2) at (-3.5,1.5) {};
            \node[GraphNodeSmall]
                (NODE3) at (-2.5,-1) {};

            \draw
                (NODE0)
                edge[line width=2.2pt]
                node[midway, right, xshift=1mm, yshift=1mm] {$4$}
                (NODE1)
            ;
            \draw
                (NODE0)
                edge[line width=2.2pt]
                node[midway, right, xshift=-1.5mm, yshift=3mm] {$12$}
                (NODE2)
            ;
            \draw
                (NODE0)
                edge[line width=2.2pt]
                node[midway, right, xshift=-4mm, yshift=2.5mm] {$8$}
                (NODE3)
            ;
            \draw
                (NODE2)
                edge[line width=2.2pt]
                node[midway, left, xshift=-0.5mm, yshift=-1.5mm] {$-4$}
                (NODE3)
            ;
            \draw
                (NODE2)
                edge[line width=2.2pt, loop left, min distance=14mm]
                node[very near end, above, xshift=-1.5mm, yshift=0mm] {$25$}
                (NODE2)
            ;
        \end{tikzpicture}
        \label{fig:graphs--basic-definitions--weighted-graph}
    }
    \caption{}%
    \label{fig:graphs--basic-definitions--digraph-multigraph-compare}
\end{figure}

A \emph{list} is a graph with only one \emph{path}. A \emph{tree} is a graph with exactly one path between each pair of vertices.

\begin{figure}[H]
    \centering
    \subfigure[%
        A tree.
    ]{%
        \begin{tikzpicture}[
            line width=1.6pt,
            draw=black,
            >=Stealth
        ]
            \node[GraphNodeSmall] (NODE0) at (0,0) {};
            \node[GraphNodeSmall] (NODE1) at (1,1) {};
            \node[GraphNodeSmall] (NODE11) at (0.4,2) {};
            \node[GraphNodeSmall] (NODE12) at (2.2,1.4) {};
            \node[GraphNodeSmall] (NODE111) at (-0.8,1.4) {};
            \node[GraphNodeSmall] (NODE121) at (1.8,2.3) {};
            \node[GraphNodeSmall] (NODE122) at (1.8,0.3) {};
            \node[GraphNodeSmall] (NODE2) at (-2,0.8) {};
            \node[GraphNodeSmall] (NODE21) at (-1.5,2.5) {};
            \node[GraphNodeSmall] (NODE22) at (-3,1.8) {};
            \node[GraphNodeSmall] (NODE23) at (-3.5,0.2) {};
            \node[GraphNodeSmall] (NODE24) at (-2,-0.4) {};

            \draw
                (NODE0) edge[line width=2.6pt] (NODE1)
                (NODE0) edge[line width=2.6pt] (NODE2)
                (NODE1) edge[line width=2.6pt] (NODE11)
                    (NODE11) edge[line width=2.6pt] (NODE111)
                (NODE1) edge[line width=2.6pt] (NODE12)
                    (NODE12) edge[line width=2.6pt] (NODE121)
                    (NODE12) edge[line width=2.6pt] (NODE122)
                (NODE2) edge[line width=2.6pt] (NODE21)
                (NODE2) edge[line width=2.6pt] (NODE22)
                (NODE2) edge[line width=2.6pt] (NODE23)
                (NODE2) edge[line width=2.6pt] (NODE24)
            ;
        \end{tikzpicture}
        \label{fig:graphs--basic-definitions--tree}
    }
    \qquad\qquad
    \subfigure[%
        A list.
    ]{%
        \begin{tikzpicture}[
            line width=1.6pt,
            draw=black,
            >=Stealth
        ]
            \node[GraphNodeSmall] (NODE0) at (0,0) {};
            \node[GraphNodeSmall] (NODE1) at (0.4,1.2) {};
            \node[GraphNodeSmall] (NODE2) at (-1.4,-0.2) {};
            \node[GraphNodeSmall] (NODE3) at (-1.4,1.0) {};
            \node[GraphNodeSmall] (NODE4) at (-0.7,2.0) {};
            \node[GraphNodeSmall] (NODE5) at (0.4,2.4) {};
            \node[GraphNodeSmall] (NODE6) at (1.4,2.0) {};
            \node[GraphNodeSmall] (NODE7) at (1.2,0.3) {};
            \node[GraphNodeSmall] (NODE8) at (2.4,1.4) {};
            \node[GraphNodeSmall] (NODE9) at (2.2,2.7) {};

            \draw
                (NODE0) edge[line width=2.6pt] (NODE1)
                (NODE1) edge[line width=2.6pt] (NODE2)
                (NODE2) edge[line width=2.6pt] (NODE3)
                (NODE3) edge[line width=2.6pt] (NODE4)
                (NODE4) edge[line width=2.6pt] (NODE5)
                (NODE5) edge[line width=2.6pt] (NODE6)
                (NODE6) edge[line width=2.6pt] (NODE7)
                (NODE7) edge[line width=2.6pt] (NODE8)
                (NODE8) edge[line width=2.6pt] (NODE9)
            ;
        \end{tikzpicture}
        \label{fig:graphs--basic-definitions--list}
    }
    \caption{}%
    \label{fig:graphs--basic-definitions--tree-list-compare}
\end{figure}

\textit{(Many more graph variations exist, but an exhaustive list is beyond the scope of a quick reviewer.)}

A \emph{walk} is an alternating sequence of \myul{adjacent} vertices and edges. A walk with only one vertex (and no edges) is a \emph{trivial walk}. A \emph{closed} walk starts and ends on the same vertex.

A \emph{trail} is a walk with no repeated edges. A \emph{circuit} is a closed trail with at least one edge.

A \emph{path} is a walk with no repeated edges or vertices. A \emph{simple circuit} is a closed path except it starts and ends on the same vertex, and has at least one edge.

\Todo{Diagram for path? Also, we should rework these walk/trail/path and related definitions.}

\Todo{Define \emph{complete graphs} $K_n$. Make diagrams for them. (Epp, Page 633)}

\Todo{Define \emph{complete bipartite graphs} $K_{m,n}$. Make diagrams for them. (Epp, Page 633)}

\Todo{Define \emph{subgraph}. (Epp, Page 634)}

Two vertices are \emph{connected} iff there is walk between them. A graph is \emph{connected} iff there is a walk between every pair of its vertices.

A graph $H$ is a \emph{connected component} of its supergraph $G$ iff $H$ is connected, and no connected subgraph of $G$ is a supergraph of $H$ that also contains vertices and edges not in $H$. \Todo{Can we simplify this definition? Also, we should draw diagrams for this.}

The \emph{degree} of a vertex (denoted $\mathrm{deg}\parens*{v}$ for the degree of vertex $v$) is the number of edges incident on it, with loops counted twice. The \emph{total degree} of a graph is the sum of the degrees of all vertices in the graph.

