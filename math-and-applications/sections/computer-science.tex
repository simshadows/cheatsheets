\subsection{Databases: The Relational Model, Relational Algebra, and some SQL}%
\label{sub:relational-algebra}

\begin{multicols}{2}

    % TODO: Find a way to work this into this cheatsheet.
    %\begin{CheatsheetEntryFrameNew}

    %    \CheatsheetEntryTitle{Relations}

    %    The term \textit{relation} can refer to either:
    %    \begin{itemize}
    %        \item a \textit{relation schema}, or
    %        \item a \textit{relation instance}.
    %    \end{itemize}

    %    Intended usage should be clear from context.

    %\end{CheatsheetEntryFrameNew}

    \begin{CheatsheetEntryFrame}

        \CheatsheetEntryTitle{Relation Schemas}

        Relation schemas are \ul{unordered sets} of \textit{attributes}.

        Notation for relation $R$ and attributes $A, B, C, \dots$:
        \begin{equation*}
            R(A, B, C, \dots)
        \end{equation*}
        %% I want a visually annotated version like this below.
        %% Problem is that the relation name annotation looks really awkward.
        %\newcommand{\X}{\rule{0mm}{1.1em}}
        %\begin{equation*}
        %    {
        %        \color{mycontrastred}
        %        \overbrace{\color{black} \X \hspace{1em} R}^{\mathclap{\substack{\textbf{\textit{relation}}\\\textbf{\textit{name}}}}}
        %    }
        %    ({
        %        \color {mycontrastpurple}
        %        \overbrace{\color{black} \X A, B, C, \dots}^{\textbf{\textit{attributes}}}
        %    })
        %\end{equation*}

        Practical example:
        \begin{center}
            \ttd{Accounts(branch, accountno, balance)}
            %% The original visually annotated version.
            %% No longer used since it's already described elsewhere.

            %\newcommand{\X}{\rule{0mm}{1.1em}}

            %$\color{mycontrastred} \overbrace{\color{black} \X \texttt{Accounts}}^{\substack{\textbf{\textit{relation}}\\\textbf{\textit{name}}}}$\texttt{(}$\color {mycontrastpurple} \overbrace{\color{black} \X \texttt{branch, accountno, balance}}^{\textbf{\textit{attributes}}}$\texttt{)}
        \end{center}

        \SqlSubsectionAddSeparation
        \begin{SqlSubsection}{SQL Create/Remove Relation (Examples)}
            \begin{CheatsheetSubsectionLst}
                CREATE TABLE Accounts (
                    branch      VARCHAR(20),
                    accountno   CHAR(7)       PRIMARY KEY,
                    balance     NUMERIC       NOT NULL,

                    CONSTRAINT accountnoformat CHECK(
                        accountno ~ '[A-Z]-[0-9]{5}'
                    )
                );
            \end{CheatsheetSubsectionLst}

            \medskip
            {\footnotesize Note: The \texttt{\textasciitilde} regex operator is specific to \textit{PostgreSQL}.}
            %% Would be nice if an explanation of syntax could be worked in...
            %\medskip
            %Syntax:
            %\begin{CheatsheetSubsectionLst}
            %    CREATE TABLE <tablename> (
            %        <attribute> <data type> <constraints>,
            %        <attribute> <data type> <constraints>,
            %        <...>
            %        <table-level constraint>,
            %        <table-level constraint>,
            %        <...>
            %    );
            %\end{CheatsheetSubsectionLst}
        \end{SqlSubsection}
        \SqlSubsectionReduceSkip
        \begin{SqlSubsection}{SQL Remove Relation (Example)}%
            \begin{CheatsheetSubsectionLst}
                DROP TABLE Accounts;
            \end{CheatsheetSubsectionLst}
        \end{SqlSubsection}

    \end{CheatsheetEntryFrame}

    \begin{CheatsheetEntryFrame}

        \CheatsheetEntryTitle{Attributes}

        Attributes are labels for parts of a schema.

        The \textit{domain} of an attribute is the set of values that can be associated with the attribute.

        For attribute $A$, this is denoted:
        \begin{equation*}
            \dom{(A)}
        \end{equation*}

        \SqlSubsectionReduceSkip
        \begin{SqlSubsection}{Some Common SQL Data Types and Constraints}
            \newcommand{\SqlOptionalPart}[1]{{\color{SqlSubsectionTextColorGrayed}(p,s)}}
            \texttt{INTEGER, SMALLINT, BIGINT, NUMERIC\SqlOptionalPart{(p,s)},}\\[0mm]
            \texttt{REAL, DOUBLE PRECISION, FLOAT\SqlOptionalPart{(p)},}\\[0mm]
            \texttt{CHAR(n), VARCHAR(n), CLOB,}\\[0mm]
            \texttt{BOOLEAN, BINARY(n), VARBINARY(n), BLOB,}\\[0mm]
            \texttt{DATE, TIME, TIMESTAMP, INTERVAL.}
        \end{SqlSubsection}
        %% Would be nice to add something like this, but I can't really fit it in while also having it be sufficiently useful.
        %\SqlSubsectionReduceSkip
        %\begin{SqlSubsection}{Some Common SQL Constraints}
        %    \SqlSubsectionTableRemoveSpace
        %    \begin{tabularx}{\textwidth}{X|X}
        %        Column Constraint & Table Constraint \\ \hline
        %        \texttt{NOT NULL} & \texttt{NOT NULL (col)} \\
        %        \texttt{PRIMARY KEY} & \texttt{PRIMARY KEY (col)} \\
        %        \texttt{UNIQUE} & \texttt{UNIQUE (col, ...)} \\
        %        \texttt{REFERENCES rel(col)} & \texttt{FOREIGN KEY (col) REFERENCES rel(col)}
        %    \end{tabularx}
        %    \SqlSubsectionTableRemoveSpace
        %\end{SqlSubsection}

    \end{CheatsheetEntryFrame}

    %\Todo{Discuss the domain of the whole schema?}

    %\MulticolsBreak

    \begin{CheatsheetEntryFrame}

        \CheatsheetEntryTitle{Relation Instances}

        Relation instances are \ul{unordered sets} of \textit{tuples}.

        Tuples are mappings from schema $R$ to domain $\dom{(R)}$.

        Practical examples:

        \begin{center}
            % Old version
            %\begin{tabularx}{\textwidth}{rXXrr}
            %\begin{tabular}{|lllll|}
            %    \multicolumn{5}{l}{\ttd{Flights}}
            %        \\ \hline
            %    \multicolumn{1}{|l}{\ttd{number}}
            %        & \multicolumn{1}{l}{\ttd{from}}
            %        & \multicolumn{1}{l}{\ttd{to}}
            %        & \multicolumn{1}{l}{\ttd{depart}}
            %        & \multicolumn{1}{l|}{\ttd{arrive}}
            %        \\ \hline\hline
            %    \ttd{QF401}
            %        & \ttd{SYD}
            %        & \ttd{MEL}
            %        & \ttd{06:00}
            %        & \ttd{07:35}
            %        \\
            %    \ttd{JQ603}
            %        & \ttd{SYD}
            %        & \ttd{MEL}
            %        & \ttd{06:40}
            %        & \ttd{08:15}
            %        \\
            %    \ttd{VA999}
            %        & \ttd{SYD}
            %        & \ttd{BNE}
            %        & \ttd{21:30}
            %        & \ttd{22:00}
            %        \\
            %    \ttd{TT206}
            %        & \ttd{MEL}
            %        & \ttd{SYD}
            %        & \ttd{06:40}
            %        & \ttd{08:15}
            %        \\ \hline
            %\end{tabular}

            \newcommand{\Y}{0.5}
            \newcommand{\Yhalf}{0.25}
            \newcommand{\MyTableCell}[3]{(R#1 -| C#2) node[right, align=left, font=\small] {\vphantom{$M_I^{I^x}$}\texttt{#3}}}
            \begin{tikzpicture}[scale=1, transform shape]
                \path
                    (0,0)      coordinate (LeftEdge)
                    ++(0.06,0) coordinate (C1)
                    ++(2.18,0) coordinate (C2)
                    ++(1.60,0) coordinate (C3)
                    ++(2.30,0) coordinate (RightEdge)
                    ++(0.15,0) coordinate (RightOS1)
                    ++(0.40,0) coordinate (RightOS2)
                ;
                \begin{scope}[shift={(0,0)}]
                    \path
                        (0,0)         coordinate (Top)
                        ++(0,-\Yhalf) coordinate (RA)
                        ++(0,-\Yhalf) coordinate (Bottom)

                        (Top)
                        ++(0, 0.060) % Slight extra offset
                        ++(0, \Yhalf) coordinate (RT) % Title row

                        (Top)
                        ++(0,-0.1) coordinate (TopOS1)

                        (Top)
                        ++(0,1.2) coordinate (TopOS2)
                        ++(0,0.6) coordinate (TopOS3)
                    ;
                    \draw
                        (Top -| LeftEdge) rectangle (RightEdge |- Bottom)
                    ;
                    \draw
                        \MyTableCell{T}{1}{\textbf{Customers}}

                        \MyTableCell{A}{1}{customerid}
                        \MyTableCell{A}{2}{name}
                        \MyTableCell{A}{3}{address}
                    ;
                \end{scope}
                \begin{scope}[shift={(0,-0.58)}]
                    \path
                        (0,0)         coordinate (Top)
                        ++(0,-\Yhalf) coordinate (R1)
                        ++(0,-\Y)     coordinate (R2)
                        ++(0,-\Y)     coordinate (R3) coordinate (HorizontalMid)
                        ++(0,-\Y)     coordinate (R4)
                        ++(0,-\Y)     coordinate (R5)
                        ++(0,-\Yhalf) coordinate (Bottom)
                    ;
                    \draw
                        (Top -| LeftEdge) rectangle (RightEdge |- Bottom)
                    ;
                    \draw
                        \MyTableCell{1}{1}{17745042}
                        \MyTableCell{1}{2}{Amy}
                        \MyTableCell{1}{3}{Surry Hills}

                        \MyTableCell{2}{1}{43891675}
                        \MyTableCell{2}{2}{Chris}
                        \MyTableCell{2}{3}{Richmond}

                        \MyTableCell{3}{1}{46227800}
                        \MyTableCell{3}{2}{Josh}
                        \MyTableCell{3}{3}{North Ryde}
                        
                        \MyTableCell{4}{1}{56917294}
                        \MyTableCell{4}{2}{Sam}
                        \MyTableCell{4}{3}{Kensington}

                        \MyTableCell{5}{1}{95272911}
                        \MyTableCell{5}{2}{Vanessa}
                        \MyTableCell{5}{3}{Richmond}
                    ;

                    \draw[mycontrastblue]
                        (HorizontalMid -| RightOS2)
                        ++(0.10,0) coordinate (RowArrowConvg)
                        ++(0.05,0) node[right, align=left, font=\small] {
                            \textbf{\textit{tuples,}} \\
                            \textbf{\textit{rows,}} \\
                            \textbf{\textit{records}}
                        }
                    ;
                    \draw[-stealth, cap=round, line width=2.0pt, mycontrastblue] (R1 -| RightOS2) -- (R1 -| RightOS1);
                    \draw[-stealth, cap=round, line width=2.0pt, mycontrastblue] (R2 -| RightOS2) -- (R2 -| RightOS1);
                    \draw[-stealth, cap=round, line width=2.0pt, mycontrastblue] (R3 -| RightOS2) -- (R3 -| RightOS1);
                    \draw[-stealth, cap=round, line width=2.0pt, mycontrastblue] (R4 -| RightOS2) -- (R4 -| RightOS1);
                    \draw[-stealth, cap=round, line width=2.0pt, mycontrastblue] (R5 -| RightOS2) -- (R5 -| RightOS1);
                    \draw[          cap=round, line width=2.0pt, mycontrastblue] (R1 -| RightOS2) -- (R5 -| RightOS2);
                    %\draw[         cap=round, line width=2.0pt, mycontrastblue] (RowArrowConvg) -- (RowArrowConvg -| RightOS2);

                    \draw[mycontrastpurple]
                        (TopOS2 -| C3)
                        ++(1.8, 0) coordinate (ColArrowConvgRef)
                        ++(0.05, 0) node[right, align=left, font=\small] {
                            \textbf{\textit{attributes,}} \\
                            \textbf{\textit{columns,}} \\
                            \textbf{\textit{fields}}
                        }
                    ;
                    \path
                        (RA -| C1) ++(2.05, 0.30) coordinate (ColArrowEnd1)
                        (RA -| C2) ++(1.05, 0.30) coordinate (ColArrowEnd2)
                        (RA -| C3) ++(0.95, 0.30) coordinate (ColArrowEnd3)
                        (ColArrowConvgRef) ++(0, -0.1) coordinate (ColArrowConvg)
                    ;
                    \draw[-stealth, cap=round, line width=2.0pt, mycontrastpurple] (ColArrowConvg) -- (ColArrowEnd1);
                    \draw[-stealth, cap=round, line width=2.0pt, mycontrastpurple] (ColArrowConvg) -- (ColArrowEnd2);
                    \draw[-stealth, cap=round, line width=2.0pt, mycontrastpurple] (ColArrowConvg) -- (ColArrowEnd3);

                    \draw[mycontrastred]
                        (TopOS3 -| C2)
                        ++(0.6, 0) coordinate (RelArrowConvgRef)
                        ++(0.05, 0) node[right, align=left, font=\small] {
                            \textbf{\textit{relation,}} \\
                            \textbf{\textit{table}}
                        }
                    ;
                    \path
                        (RT -| C1) ++(1.65, 0.15) coordinate (RelArrowEnd)
                        (RelArrowConvgRef) ++(0, -0.4) coordinate (RelArrowConvg)
                    ;
                    \draw[-stealth, cap=round, line width=2.0pt, mycontrastred] (RelArrowConvg) -- (RelArrowEnd);

                    % An attempt was made to make smooth lines...
                    %\draw[-stealth, cap=round, line width=1.5pt]
                    %    plot [smooth] coordinates { (RowArrowConvg) (R1 -| RightOS2) (R1 -| RightOS1) };
                    %\draw[-stealth, cap=round, line width=1.5pt]
                    %    plot [smooth] coordinates { (RowArrowConvg) (R2 -| RightOS2) (R2 -| RightOS1) };
                    %\draw[-stealth, cap=round, line width=1.5pt]
                    %    plot [smooth] coordinates { (RowArrowConvg) (R3 -| RightOS2) (R3 -| RightOS1) };
                    %\draw[-stealth, cap=round, line width=1.5pt]
                    %    plot [smooth] coordinates { (RowArrowConvg) (R4 -| RightOS2) (R4 -| RightOS1) };
                    %\draw[-stealth, cap=round, line width=1.5pt]
                    %    plot [smooth] coordinates { (RowArrowConvg) (R5 -| RightOS2) (R5 -| RightOS1) };
                \end{scope}
            \end{tikzpicture}

            \medskip

            {\small%
            \begin{tabular}{|llr|}
                \multicolumn{3}{l}{\ttd{Accounts}}
                    \\ \hline
                \multicolumn{1}{|l}{\ttd{branch}}
                    & \multicolumn{1}{l}{\ttd{accountno}}
                    & \multicolumn{1}{l|}{\ttd{balance}}
                    \\ \hline\hline
                \ttd{Richmond}
                    & \ttd{A-02772}
                    & \ttd{20.87}
                    \\
                \ttd{Macquarie Park}
                    & \ttd{J-31553}
                    & \ttd{60899.58}
                    \\
                \ttd{Richmond}
                    & \ttd{W-40018}
                    & \ttd{84731.08}
                    \\
                \ttd{Haymarket}
                    & \ttd{A-74884}
                    & \ttd{483.94}
                    \\
                \ttd{Haymarket}
                    & \ttd{P-85953}
                    & \ttd{7294.62}
                    \\ \hline
            \end{tabular}%
            }
            
            \medskip

            {\small%
            \begin{tabular}{|ll|}
                \multicolumn{2}{l}{\ttd{HeldBy}}
                    \\ \hline
                \multicolumn{1}{|l}{\ttd{account}}
                    & \multicolumn{1}{l|}{\ttd{customer}}
                    \\ \hline\hline
                \ttd{A-02772}
                    & \ttd{43891675}
                    \\
                \ttd{A-02772}
                    & \ttd{95272911}
                    \\
                \ttd{J-31553}
                    & \ttd{46227800}
                    \\
                \ttd{W-40018}
                    & \ttd{43891675}
                    \\
                \ttd{W-40018}
                    & \ttd{95272911}
                    \\
                \ttd{A-74884}
                    & \ttd{17745042}
                    \\
                \ttd{P-85953}
                    & \ttd{17745042}
                    \\ \hline
            \end{tabular}%
            }

        \end{center}

    \end{CheatsheetEntryFrame}

\end{multicols}
\newpage
\begin{multicols}{2}
    
    \begin{CheatsheetEntryFrame}

        \CheatsheetEntryTitle{Selection}
        \begin{equation*}
            \relselect_{\text{expr}}{(\text{Rel})}
        \end{equation*}

    \end{CheatsheetEntryFrame}

    \begin{CheatsheetEntryFrame}

        \CheatsheetEntryTitle{Projection}
        \begin{equation*}
            \relproject_{A,B,C}{(\text{Rel})}
        \end{equation*}

    \end{CheatsheetEntryFrame}

    \begin{CheatsheetEntryFrame}

        \CheatsheetEntryTitle{Rename}
        \begin{equation*}
            \relrename_{\text{schema}}{(\text{Rel})}
        \end{equation*}

    \end{CheatsheetEntryFrame}

    \begin{CheatsheetEntryFrame}

        \CheatsheetEntryTitle{Natural Join}
        \begin{equation*}
            \text{Rel}_1 \bowtie \text{Rel}_2
        \end{equation*}

    \end{CheatsheetEntryFrame}

    \begin{CheatsheetEntryFrame}

        \CheatsheetEntryTitle{Theta Join}
        \begin{equation*}
            \text{Rel}_1 \bowtie_C \text{Rel}_2
        \end{equation*}

    \end{CheatsheetEntryFrame}

\end{multicols}
